\newenvironment{styleBeispelfTi}{}{}
\newenvironment{styleBeispelTimes}{}{}
\newenvironment{styleBeispiel}{}{}
\newenvironment{styleBeispielf}{}{}
\newenvironment{styleBeispielfn}{}{}
\newenvironment{styleBeispieli}{}{}
\newenvironment{styleBeispielin}{}{}
\newenvironment{styleLiteratur}{}{}
\newenvironment{styleNormalAbst}{}{}
\newenvironment{styleNormalEA}{}{}
\newenvironment{styleNormalEinzug}{}{}
\newenvironment{stylePCOhneTiteli}{}{}
\newenvironment{styleSeiteorii}{}{}
\newenvironment{styleStandard}{}{}

\renewcommand{\sectref}[1]{Section~\ref{#1}}
\providecommand{\sectgref}[1]{Abschnitt~\ref{#1}}


% cmld: subsection ohne Überschrift
\makeatletter
\providecommand\ssubsection{\@startsection {subsection}{1}{\z@}%
                                   {-3.5ex \@plus -1ex \@minus -.2ex}%
                                   {-1em}%
                                   {\normalfont\large\bfseries}}
\makeatother


% cmld: subsubsection ohne Überschrift
\makeatletter
\providecommand\ssubsubsection{\@startsection {subsubsection}{1}{\z@}%
                                   {-3.5ex \@plus -1ex \@minus -.2ex}%
                                   {-1em}%
                                   {\normalfont\bfseries}}
\makeatother

% cmld: example sources for Chapter 13
\providecommand{\exsourcefive}[3][]{\trans\strut\hfill\citep[#1][#2]{#3}}

% cmld: example sources for Chapter 14
\providecommand{\exsourcefour}[1]{\strut\hfill {#1}}
\providecommand{\exsourcefourn}[2]{\\\strut\hfill {\citep[#1]{#2}}}

% cmld: example sources for Chapter 13
%\providecommand{\exsource}[2]{\trans\strut\hfill(from \citealt[#1]{#2})}
\providecommand{\exsource}[2]{\jambox{(\citealt[#1]{#2})}}

% cmld: Einrückung exe korrigieren (römische Zahlen sind sehr breit.)
%\exewidth{(\hspace*{18pt})}

% cmld: inter-word space in glossed examples
%\glossglue = 0.8em plus 8pt minus 8pt
% cmld: gloss formatting
%\renewcommand{\eachwordtwo}{\small} %if done like this font size in examples in footnotes is wrong
%% St.Mü. \renewcommand{\glossfont}{\small}

% cmld: exe mit großen römischen Zahlen auf der obersten Ebene
\newcounter{fundcount}
\makeatletter
\newcommand{\exI}{\save@counters\refstepcounter{fundcount}\renewcommand{\thexnumi}{\Roman{fundcount}}\@ifnextchar [{\@exI}{\item}\reset@counters}
\makeatother

% cmld: exe mit kleinen römischen Zahlen auf der obersten Ebene (für Fußnoten)
\newcounter{footcount}
\makeatletter
\newcommand{\exF}{\save@counters\refstepcounter{footcount}\renewcommand{\thexnumi}{\roman{footcount}}\@ifnextchar [{\@exF}{\item}\reset@counters}
\makeatother

% cmld: roman numbered footnotes, numbered within each footnote
%% St.Mü. \let\oldfootnote\footnote
%% St.Mü. \renewcommand{\footnote}[1]{\begingroup%
%% St.Mü. \let\exfont\fnexfont%
%% St.Mü. \let\glossfont\fnglossfont%
%% St.Mü. \let\transfont\fntransfont%
%% St.Mü. \let\exnrfont\fnexnrfont%
%% St.Mü. \renewcommand{\thexnumi}{\roman{xnumi}}%
%% St.Mü. \oldfootnote{#1}%
%% St.Mü. \setcounter{footcount}{0}
%% St.Mü. \endgroup}

%% St.Mü. \renewcommand{\fnexfont}{\footnotesize}
%% St.Mü. \renewcommand{\fnglossfont}{\footnotesize}


%\newcommand{\vref}{}
\let\citew\citet

% cmld: fix formatting and positioning of number for exe environments with tables
\def\extab{\ex\leavevmode\vadjust{\vspace{-0.95\baselineskip}}\newline\normalfont}
\def\extabb{\ex\leavevmode\vadjust{\vspace{-0.8\baselineskip}}\newline\normalfont}

% cmld: New column types (for tabularx)
\newcolumntype{Q}{>{\raggedright\arraybackslash}X}
\newcolumntype{C}{>{\centering\arraybackslash}X}
\newcolumntype{P}{>{\raggedright\hspace{0pt}\arraybackslash}p}


% cmld: mssing Ref
\providecommand{\missing}{XY [siehe Fn.~\ref{fn-herausgeber-topo}]}

% cmld: Comments and notes 
% for Luise Dorenbusch (cmld)
 \newcounter{todocountercmld}
 \newcommand{\cmldcomm}[2][]
 {\refstepcounter{todocountercmld}{\todo[size=\scriptsize,
     color=yellow, #1]{\sffamily \textbf{[CMLD~\thetodocountercmld]}
       #2}}}
% internal comments (int)
 \newcounter{todocounterint}
 \newcommand{\intcomm}[2][]
 {\refstepcounter{todocounterint}{\todo[size=\scriptsize, color=white, #1]{\sffamily \textbf{[INTERNAL~\thetodocounterint]} #2}}}

% cmld: too many bad boxes!!
\setlength{\emergencystretch}{2pt}

% cmld: AVMs
\avmfont{\mathit}
\providecommand{\tpv}[1]{\textnormal{\textit{#1}}}
\avmsortfont{\scriptsize\it}
% minimize vertical spacing
\def\avmjvskip{0.2ex}
\newcommand{\tp}[1]{\avmspan{\textit{#1}}}
% types (adaptiert von Stefan Müllers merkmalstruktur.sty)
\newcommand{\onems}[2][]{%
  \mbox{%
    \delimiterfactor=1000 \delimitershortfall=0pt
    \tabcolsep=0pt
    $\left[%
    \begin{tabular}{>{\upshape\scshape}l}
    \if\relax\detokenize{#1}\relax\else
    \raisebox{1mm}{{\small\normalfont\itshape #1}}%
    \\[-1mm]
    \fi
    #2%
    \end{tabular}%
    \right]$%
  }%
  \vspace{1mm}%
}

\newcommand{\onemslo}[2][]{%
  \mbox{%
    \delimiterfactor=1000 \delimitershortfall=0pt
    \tabcolsep=0pt
    $\left[%
    \begin{tabular}{>{\upshape\scshape}l}
    \if\relax\detokenize{#1}\relax\else
    \raisebox{1mm}{{\small\normalfont\itshape #1}}%
    \\[.5mm]
    \fi
    #2%
    \end{tabular}%
    \right.$%
  }%
  \vspace{1mm}%
}

% types und aligniert
\providecommand{\avmtype}[1]{\raisebox{1mm}{{\small\normalfont\itshape #1}}}

% cmld: temporary command for German single quotes
%\providecommand{\textsq}[1]{{\color{red}‚}#1{\color{red}‘}}
\providecommand{\textsq}[1]{‚#1‘}

% cmld: temporary command for English single quotes
%\providecommand{\textsqe}[1]{{\color{red}‘}#1{\color{red}’}}
\providecommand{\textsqe}[1]{‘#1’}



% cmld: Randnummern
\usepackage{marginnote}
\newcounter{randcount}[chapter]
\makeatletter
\def\randnum{\@ifnextchar[{\@with}{\@without}}
\def\@with[#1]{\leavevmode\marginnote[\hfill#1]{#1}}
\def\@without{\refstepcounter{randcount}\leavevmode\marginnote[\hfill\arabic{randcount}]{\arabic{randcount}}}
\makeatother

% funktioniert nicht :(
% \makeatletter
% \@addtoreset{randcount}{chapter}
% \makeatother

% \makeatletter
% \providecommand{\randalph}[1]{\def\@currentlabel{#1}}
% \makeatother

\makeatletter
\providecommand{\exalph}[1]{\def\@currentlabel{#1}}
\makeatother

% cmld: UMBRUCH
% cmld: mainly vertical
\providecommand{\Hack}[1]{#1}
% cmld: mainly horizontal
\providecommand{\hack}[1]{#1}


\def\INS#1{}
\def\INA#1{}
% % allows for margin notes of index entries.

% \newif\ifshowindex \showindexfalse

%  \def\INS#1{\ifshowindex%
%    \index{#1}
%    \marginpar{\mbox{}\raggedright\scriptsize\rm #1}%
%    \else%
%    \index{#1}%
%    \fi}

%  \def\INA#1{\ifshowindex%
%    \index{ZZZ-#1}%
%    \marginpar{\mbox{}\raggedright\scriptsize\rm #1}%
%    \else%
%    \index{ZZZ-#1}%
%    \fi}

\let\oldavm\avm
\def\avm{\oldavm\scshape}

\def\avhr#1{\mbox{\small$\left[\begin{tabular}{@{}l@{}}#1\end{tabular}\right]$}}
\def\avha#1{\small{\[\left[\begin{tabular}{@{}l@{}}#1
\end{tabular}\right]\]}}
\def\avmcirc#1{\mbox{\rlap{\mbox{$\bigcirc$}}{\kern3pt$\scriptstyle #1$\kern3pt}}}\newcommand{\shcite}[1]{\citeyear{#1}}



\newcommand{\sub}[1]{\(_{\mbox{\scriptsize\textrm{#1}}}\)}
\renewcommand{\mathrm}[1]{\text{#1}}

% http://tex.stackexchange.com/questions/8255/vertically-aligning-textsuperscript-and-textsubscript-together
\def\supsub#1#2{\rlap{\textsuperscript{\tiny #1}}\textsubscript{\tiny #2}}

\newcommand{\pollardsag}{Pollard~\& Sag\xspace}


\let\addedcitet\citet


%https://tex.stackexchange.com/questions/75675/how-can-i-force-hyphenation-in-chapter-titles
%\usepackage{ragged2e}
%\renewcommand*{\raggedsection}{\RaggedRight}% default is \raggedright


% biblatex stuff
% get rid of initials for Carl J. Pollard and Carl Pollard in the main text:
\ExecuteBibliographyOptions{uniquename=false}
\DefineBibliographyStrings{german}{%
  andothers = {et al.}
}
%\indexfield{author}

%https://tex.stackexchange.com/questions/402044/getting-the-authors-to-show-in-the-index-while-using-shortauthor-and-biblatex/402067#402067

% If the user provided a shortauthor in the bibtex entry, we use the authentic author (as with the
% authorindex package) if it is defined, otherwise we use the author.
% This gets F/T as shorthand right and puts the guys in the index.

\renewbibmacro*{citeindex}{%
  \ifciteindex
    {\iffieldequalstr{labelnamesource}{shortauthor}
      {\ifnameundef{authauthor}
        {\indexnames{author}}
        {\indexnames{authauthor}}}
      {\iffieldequalstr{labelnamesource}{author}
        {\ifnameundef{authauthor}% if defined use authauthor
          {\indexnames{author}}
          {\indexnames{authauthor}}} % if defined use this field 
        {\iffieldequalstr{labelnamesource}{shorteditor}
          {\ifnameundef{autheditor}
            {\indexnames{editor}}
            {\indexnames{autheditor}}}
          {\indexnames{labelname}}}}}
    {}}




\newcommand{\zb}{z.\,B.\ }
\newcommand{\Zb}{Z.\,B.\ }
\newcommand{\ZB}{Z.\,B.\ }

\newcommand{\dash}{d.\,h.\@\xspace}        % was macht dh sonst??


\newcommand{\inThisVolume}{%
\iflanguage{german}{%
{in diesem Band\xspace}}{%
{in this volume\xspace}}}

\newcommand{\pages}{%
\iflanguage{german}{%
{S.\,}}{%
{pp.\,}}}


\newcommand{\editors}{%
\iflanguage{german}{%
{Hrsg.}}{%
{eds.}}}

\newcommand{\volume}{%
\iflanguage{german}{%
{{Bd.\,}}}{%
{{Vol.\,}}}}

% The \ex definition
\newcounter{smtempcnt}

%% \newcommand{\ex}[1]{\setcounter{smtempcnt}{\value{equation}}%
%% \addtocounter{smtempcnt}{#1}%
%% \arabic{tempcnt}}%

\newcommand{\mex}[1]{\setcounter{smtempcnt}{\value{equation}}%
\addtocounter{smtempcnt}{#1}%
\arabic{smtempcnt}}%




\newcommand{\spacebr}{\hspaceThis{[}}





  % adds lines to both the odd and even page.
% bloddy hell! This is really an alpha package! Do not use the draft option! 07.03.2016
\usepackage{addlines}

%% \let\addlinesold=\addlines
%% % there is one optional argument. Second element in brackets is the default 
%% \renewcommand{\addlines}[1][1]{
%% \todosatz{addlines}
%% \addlinesold[#1]
%% }


% do nothing now
%\let\addlinesold=\addlines
%\renewcommand{\addlines}[1][1]{}

% for addlines to work
\strictpagecheck




% to get the hyperrefs right for page numbers of chapters
\newcounter{mylastpagecount}



\newcommand{\sliste}[1]{%
\mbox{%
$\left\langle\mbox{\upshape\scshape #1}\right\rangle$}%
}

\newcommand{\nliste}[1]{%
$\left\langle\mbox{#1}\right\rangle$%
}


\newcommand{\abar}{\mbox{$\overline{\mbox{A}}$}\xspace}


\usepackage{\stylepath jambox}


% if we have two example environments, we avoid space by ending the first with \zmid
\def\zmid{\z\vspace{-\baselineskip}}
\def\zlmid{\zl\vspace{-\baselineskip}}
%\def\zlmid{\zl}

%https://tex.stackexchange.com/questions/51113/horizontal-equivalent-to-raisebox

% did not work for \ea and so on.
%% \makeatletter
%% \newcommand*{\shifttext}[2]{%
%%   \settowidth{\@tempdima}{#2}%
%%   \makebox[\@tempdima]{\hspace*{#1}#2}%
%% }
%% \makeatother

%% \newcommand{\shiftleft}[2]{\makebox[0pt][r]{\makebox[#1][l]{#2}}}




