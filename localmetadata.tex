%%%%%%%%%%%%%%%%%%%%%%%%%%%%%%%%%%%%%%%%%%%%%%%%%%%%
%%%                                              %%%
%%%                 Metadata                     %%%
%%%          fill in as appropriate              %%%
%%%                                              %%%
%%%%%%%%%%%%%%%%%%%%%%%%%%%%%%%%%%%%%%%%%%%%%%%%%%%%


%\title{Tilman N. Höhle: Beiträge zur deutschen {Grammatik}/\newlineCover Papers on German grammar}  %look no further, you can change those things right here.
%\title{Beiträge zur deutschen {Grammatik}/\newlineCover Papers on German grammar}  %look no further, you can change those things right here.
%\subtitle{Gesammelte Schriften von/\newlineCover Collected papers by\newlineCover  Tilman N. Höhle}%add a subtitle between the braces if you have one
\title{Beiträge zur deutschen {Grammatik}}  %look no further, you can change those things right here.
\subtitle{Gesammelte Schriften\newlineCover{} von Tilman N. Höhle}%add a subtitle between the braces if you have one
\BackTitle{Beiträge zur deutschen {Grammatik}}
\BackBody{%
Der vorliegende Band bietet eine vollständige Sammlung der
veröffentlichten und unveröffentlichten Schriften zur deutschen
Grammatik von Tilman N.\ Höhle. Sie besteht aus zwei Teilen. Den ersten
Teil bildet \emph{Topologische Felder}, ein im Jahr 1983 verfasstes
Manuskript in Buchlänge, das jedoch nicht abgeschlossen  wurde. 
\emph{Topologische Felder} ist eine sorgfältige
Untersuchung der topologischen Eigenschaften deutscher Sätze 
und schließt eine eingehende Diskussion typologischer
Annahmen mit ein. Der zweite Teil umfasst alle
weiteren veröffentlichten und unveröffentlichten Höhleschen Papiere
zur deutschen Grammatik.

Alle hier versammelten Arbeiten hatten weitreichenden Einfluss
auf die deutsche Sprachwissenschaft deskriptiver und theoretischer Prägung, 
insbesondere auf eine spezielle Ausprägung der theoretischen
Sprachwissenschaft, die Head-Driven Phrase Structure Grammar. Die Arbeiten
befassen sich mit den Themen Satzstruktur, Wortstellung,
Koordination, (Verum-)Fokus, Wortstruktur, der Beziehung zwischen
Relativpronomen und Verben in V2, Extraktion, sowie den Grundlagen einer
phonologischen Theorie in constraintbasierter Grammatik.

\bigskip

\noindent
This volume contains the complete collection of published and
unpublished work on German grammar by Tilman N.\ Höhle. It consists of
two parts. The first part is \emph{Topologische
Felder}, a book-length manuscript that was written in 1983 but was
never finished nor published. It is a careful examination of the
topological properties of German sentences, including a discussion of
typological assumptions. The second part assembles all other published and
unpublished papers by Höhle on German grammar.

All of these papers were highly influential in German linguistics, in
theoretical linguistics in general, and in a specific variant of
theoretical linguistics, Head-Driven Phrase Structure Grammar. Topics
covered are clause structure, constituent order, coordination, (verum)
focus, word structure, the relationship between relative pronouns and
verbs in V2, extraction, and the foundations of a theory of phonology
in constraint-based grammar.
}
%\dedication{Change dedication in localmetadata.tex}
\typesetter{Luise Dorenbusch, Luise Hiller, Robert Fritzsche, Sebastian Nordhoff, Stefan Müller}
% \proofreader{Change proofreaders in localmetadata.tex}
% \author{Tilman N. Höhle}
\author{Stefan Müller\and Marga Reis\lastand Frank Richter}
\lsMultiauthorstrue

\renewcommand{\lsSeries}{classics}
\renewcommand{\lsSeriesNumber}{5} %will be assigned when the book enters the proofreading stage
\renewcommand{\lsURL}{http://langsci-press.org/catalog/book/149} % contact the coordinator for the right number
\renewcommand{\lsID}{149} % contact the coordinator for the right number
\BookDOI{10.5281/zenodo.1145680}
\renewcommand{\lsISBNdigital}{978-3-96110-032-3}
\renewcommand{\lsISBNhardcover}{978-3-96110-033-0}

%\renewcommand{\lsCoverTitleFont}[1]{\sffamily\addfontfeatures{Scale=MatchUppercase}\fontsize{42pt}{16.75mm}\selectfont #1}

\AtBeginDocument{
\renewcommand{\lsEditorPrefix}{{\Large Herausgegeben von\\}}
  }
