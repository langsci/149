%% -*- coding:utf-8 -*-

\documentclass[output=paper]{langsci/langscibook}
\author{Tilman N. Höhle}
\title{\texorpdfstring{The \emph{w}-\ldots{} \emph{w}-construction:\newlineCover{} Appositive or scope indicating?}{The w-... w-construction: Appositive or scope indicating?}}
\renewcommand{\lsCollectionPaperFooterTitle}{The \noexpand\emph{w}-\noexpand\ldots{} \noexpand\emph{w}-construction: Appositive or scope indicating?}
%\renewcommand{\lsCollectionPaperFooterTitle}{The w-\ldots w-construction: Appositive or scope indicating?}

\abstract{}
\maketitle
\rohead{\thechapter\hspace{0.5em}The \emph{w}-\ldots{} \emph{w}-construction} % Display short title
\renewcommand*{\thefootnote}{\fnsymbol{footnote}}
\ChapterDOI{10.5281/zenodo.1169683}
\begin{document}
\selectlanguage{english}
\label{chap-w-w}

\nocite{Hoehle1989a,Hoehle1989b}

\setcounter{footnote}{4}

\footnotetext{%
	\emph{Editors’ note:} This contribution was originally published in Lutz, Uli, Müller,
        Gereon \& Arnim von Stechow (eds.). 2000. Wh-\emph{scope marking} (Linguistik
        Aktuell/""Linguistics Today 37), 249--270. Amsterdam: Benjamins. (A previous, largely identical version can be found in Lutz, Uli \& Gereon Müller (eds.). 1996. \emph{Papers on} wh-\emph{scope marking} (Arbeitspapiere des Sonderforschungsbereichs 340, Bericht Nr. 76), 37--58. Universität Stuttgart/Universität Tübingen.) The layout and citation style have been adapted to the format chosen for the present volume.%
}

%\chapter{The \textit{w}-\ldots \textit{w}-construction: Appositive or scope indicating?}
\setcounter{footnote}{0}

\section[Historical background]{Historical background\protect\footnote{%
The text that follows is a reconstruction of talks held in 1989 and 1990. It closely follows \citet{Hoehle1989a}, with a few additional observations taken from \citet{Hoehle1989b}; see the references. (\citealt{Hoehle1990} was mainly an abridged version of \citealt{Hoehle1989b}.) Sections~\ref{sec:12-8}ff. and the notes have been added in February 1996. [A few rough passages have been smoothed down in 1999.] I am grateful to Gereon Müller for encouraging me to orally present major portions of this on December 1, 1995.}}
\label{sec:12-1}

\setcounter{footnote}{0}
\renewcommand*{\thefootnote}{\arabic{footnote}}
%\exewidth{(35)}

In the field of linguistic activities that I have been associated
with, the \textit{was}-~\ldots{}~\textit{w}"=construction was established as a topic of
interest through certain bold remarks made by Thilo Tappe during an
RDGG meeting\footnote{%
	For information about the RDGG (Recent
  Developments in Generative Grammar) interest group, which was
  founded on the initiative of Jan Koster and Craig Thiersch, see
  \citet[ix]{Toman1985Preface}.%
}
in January 1980 (see (\ref{ex:12-16}) below). A variant of
Tappe's idea became widely known through Riemsdijk's correspondence
paper \citep{Riemsdijk1982}. Over the years, informal discussions of the
properties of the construction and aspects of its analysis were taken
up sporadically, partly during RDGG meetings, partly in personal
communications. Luckily, many of the results found their way into \textit{Bausteine} \citep[354ff., 374f., 384f., 393, 400]{Stechow1988}.  Somewhat surprisingly, though, none of these authors felt a
need to defend their assumption that \textit{was} is a scope indicator
against the traditional assumption that the construction is appositive
in nature (see (\ref{ex:12-15})).\footnote{%
	To be sure, \citet{McDaniel1986} did provide
  specific reasons for her analysis, see Section~\ref{sec:12-11}. But her work was
  not generally known here at that time. I came to know it only while
  preparing for \citet{Hoehle1989a}, and made no attempt to do justice to
  its empirical observations and
  theoretical proposals.%
}
But at last, this issue came up during a conference in November 1987 when \'E.\,Kiss presented her view of a similar construction in Hungarian\il{Hungarian} (see (\ref{ex:12-7}) below). Her view met with criticism from more than one side. Some discussants argued for the traditional
view, while I tried to argue for Tappe's idea on the basis of the closely related \textit{w}-P \ldots{} \textit{w}-P constructions (Section~\ref{sec:12-5}). The present article is an attempt to assess the plausibility of each idea.

\section{Variant I: \textit{Was  \ldots{}  w}-P  --  initial observations}
\label{sec:12-2}

\settowidth{\jamwidth}{(10)} 
Consider the unembedded example (\ref{ex:12-1}) and its paraphrases in (\ref{ex:12-2}): 
\begin{exe}
\ex
\label{ex:12-1}
\gll
Was glaubst du, wer Recht hat? \\
what think you who right has \\
\ex
\label{ex:12-2}
\textit{Possible paraphrases}: 
\begin{xlist}
\ex
\label{ex:12-2a}
\gll
Wer, glaubst du, hat Recht? \\
who, think you, has right \\
\ex
\label{ex:12-2b}
\gll
Was glaubst du$\setminus$; wer hat Recht? \\
what think you; who has right \\
\ex
\label{ex:12-2c}
\gll
Was glaubst du hinsichtlich der Frage / darüber, wer Recht hat? \\
what think you wrt. the question {} {there.about} who right has \\
\ex
\label{ex:12-2d}
\gll
Wer glaubst du, daß Recht hat? \\
who think you that right has \\
\end{xlist}
\end{exe}
Given that the paraphrases differ syntactically, it is natural to ask
whether any of them might be structurally related to (\ref{ex:12-1}) in some way.

The analysis of (\ref{ex:12-2a}) is controversial. It is either a parenthetical
construction or an extraction from an embedded F2 clause (i.e., from a
clause with a finite verb in second position). On either analysis,
there is no similarity to (\ref{ex:12-1}).

The analysis of (\ref{ex:12-2b}), again, is not perfectly clear. But the fall of
the \isi{intonation} after \textit{du} and the position of \textit{hat}
between \textit{wer} and \textit{Recht} are best taken as indications
that this is a sequence of two complete clauses, none of which is
embedded in the other. (\ref{ex:12-1}) differs from (\ref{ex:12-2b}) in both respects.

In (\ref{ex:12-2c}), \textit{was} is clearly a direct object of \textit{glaubst},
and the embedded \textit{wh}"=interrogative clause \textit{wer Recht
  hat} is semantically related to \textit{was}, the relation being
mediated by \textit{darüber} or \textit{hinsichtlich der Frage}. One
might imagine that the corresponding components of (\ref{ex:12-1}) stand in a
similar relation.  (This is, in essence, the traditional idea
expressed below in (\ref{ex:12-15}).)

In (\ref{ex:12-2d}), \textit{wer} is extracted from the embedded object
clause. One might imagine that \textit{was} in (\ref{ex:12-1}) functions as
something like a place holder for \textit{wer} with the effect that
the semantic properties and (part of) the structural properties of (\ref{ex:12-1})
are calculated just like they are in (\ref{ex:12-2d}). (This was, in essence,
Tappe's idea expressed below in (\ref{ex:12-16}).)

The construction seen in (\ref{ex:12-1}) is further illustrated in (\ref{ex:12-3a})--(\ref{ex:12-3f}):
\begin{exe}
\ex
\label{ex:12-3}
\begin{xlist}
\ex[]{
\label{ex:12-3a}
\gll
Was meint Karl, wen wir gewählt haben? \\
what thinks Karl whom we elected have\\}
\ex[]{
\label{ex:12-3b}
\gll
Was nimmt man an, wie der Prozeß ausgeht? \\
what assumes one {} how the trial ends\\}
\ex[]{
\label{ex:12-3c}
\gll
Was wird angenommen, wie der Prozeß ausgeht? \\
what becomes assumed how the trial ends\\}
\ex[]{
\label{ex:12-3d}
\gll
Was hat sie gesagt, mit wem er kommen will? \\
what has she said with whom he come wants\\}
\ex[*]{
\label{ex:12-3e}
\gll
Was scheint es, wen Hans geschlagen hat? \\
what seems it whom Hans hit has\\
\exsource{248 (60a)}{McDaniel1986}}
\ex[?]{
\label{ex:12-3f}
\gll
Was scheint dir, wen Hans geschlagen hat? \\
what seems to.you whom Hans hit has\\}
\ex[]{
\label{ex:12-3g}
\gll
Wen scheint es, daß Hans geschlagen hat? \\
whom seems it that Hans hit has\\}
\end{xlist}
\end{exe}    
(\ref{ex:12-3c}) is a passive construction corresponding to (\ref{ex:12-3b}). Hence, if \textit{was} in (\ref{ex:12-3b}) is accusative, \textit{was} in (\ref{ex:12-3c}) is nominative. In
passing, we note that the \textit{was}-construction with
\textit{scheint} in (\ref{ex:12-3e}) (where \textit{es} is obligatory) is
unacceptable, whereas the \textit{was}-construction with \textit{scheint}
plus dative in (\ref{ex:12-3f}) is much better and the extraction in (\ref{ex:12-3g}) is
fine (for speakers who do long extractions).

In (\ref{ex:12-4}), the \textit{was \ldots{}  w}-P construction is embedded in a matrix
that selects interrogative clauses:
\begin{exe}
\ex
\label{ex:12-4}
\gll
Heinz möchte wissen / es ist egal,\\
Heinz wants {know} / it is {no.difference}\\
\begin{xlist}
\ex
\label{ex:12-4a}
\gll
\ldots{}  was du glaubst, wer Recht hat \\
 {}  what you think who right has \\
\ex
\label{ex:12-4b}
\gll
\ldots{}  was Karl meint, wen wir gewählt haben \\
 {}  what Karl thinks whom we elected have \\
\ex
\label{ex:12-4c}
\gll
\ldots{}  was man annimmt, wie der Prozeß ausgeht \\
  {}   what one assumes how the trial ends \\
\ex
\label{ex:12-4d}
\gll
\ldots{}  was angenommen wird, wie der Prozeß ausgeht \\
  {}   what assumed becomes how the trial ends \\
\end{xlist}
\end{exe}
\addlines
Clearly, there is no way to assimilate embedded cases like these to
the structure of the paraphrases (\ref{ex:12-2a}) or (\ref{ex:12-2b}).

The construction can also be iterated:
\begin{exe}
\judgewidth{\%}
\ex
\label{ex:12-5}
\begin{xlist}
\ex[]{
\label{ex:12-5a}
\gll
Was glaubst du, was Karl meint, wen wir gewählt haben? \\
what think you what Karl thinks whom we elected have \\}
\ex[]{
\label{ex:12-5b}
\gll
Es ist egal, was du glaubst, was Karl meint, wen wir gewählt haben \\
it is no.difference what you think what Karl thinks whom we elected have \\}
\ex[\%]{
\label{ex:12-5c}
\gll
Was glaubst du, daß Karl meint, wen wir gewählt haben? \\
what think you that Karl thinks whom we elected have \\}
\ex[\%]{
\label{ex:12-5d}
\gll
Es ist egal, was du glaubst, daß Karl meint, wen wir gewählt haben \\
it is no.difference what you think that Karl thinks whom we elected have \\}
\ex[]{
\label{ex:12-5e}
\gll
Wen glaubst du, daß Karl meint, daß wir gewählt haben? \\
whom think you that Karl thinks that we elected have \\}
\end{xlist}
\end{exe}
In (\ref{ex:12-5a}) and (\ref{ex:12-5b}) \textit{was} occurs twice: this is a natural kind
of expression for many speakers, in particular for those who do not do
long extractions such as (\ref{ex:12-5e}). Many speakers who use both long
extractions and the \textit{was}-construction reject \textsqe{mixed} examples
like (\ref{ex:12-5c}) and (\ref{ex:12-5d}). But there is a minority who find nothing
objectionable with them.
        
Constructions similar to the \textit{was \ldots{}  w}-P construction occur
in a number of languages. Thus, the situation in Frisian seems almost
identical to German\il{German}:
\begin{exe}
\ex
\label{ex:12-6}
\begin{xlist}
\ex
\label{ex:12-6a}
\gll
Wat tinke jo w\^er't Jan wennet? \\
what think you {where that} Jan resides \\
\exsource{99 (3c)}{Hiemstra1986}
\ex
\label{ex:12-6b}
\gll
Wat tinke jo wa't my sjoen hat? \\
what think you {who that} me seen has \\
\exsource{99 (2c)}{Hiemstra1986}
\end{xlist}        
\end{exe}
Note, though, that the \emph{wh}"=phrase in the embedded clause (\textit{w\^er't} and  \textit{wa't}) is
suffixed by  \textit{'t} (\textsqe{that}), in accordance with the general rule for embedded
\emph{wh}"=interrogatives in Frisian.

A large group of speakers of Hungarian\il{Hungarian} use a similar construction, sometimes
referred to as the \textsqe{\textit{mit}-strategy}:
\begin{exe}
\ex
\label{ex:12-7}
\gll
Mit gondolsz hogy mit mondott Vili hogy ki l\'atta J\'anost? \\
what you.think that what said Vili that who saw Janos \\
\exsource{263 (\ref{ex:12-30})}{deMey1986}
\end{exe}
Of this example, Kiss said that ``according to the native speakers'
intuitions, [this] is not a complex sentence but a series of
non"=embedded questions'' \citep[212]{Kiss1988}.  That is, she
suggested for (\ref{ex:12-7}) a structure that might be adequate for (\ref{ex:12-2b}). But
this is incompatible with the complementizer \textit{hogy} (\textsqe{that})
appearing in (\ref{ex:12-7}). It shows up before the \emph{wh}"=expressions
\textit{mit} and \textit{ki} in accordance with the general rule for
embedded \emph{wh}"=interrogatives in Hungarian\il{Hungarian}. Kiss in fact
considers (\ref{ex:12-7}) to be marginal, but this judgement is not universally
shared; cf.\ \citet{Maracz1987}.\footnote{%
	See also \citet[Ch.\,7]{Maracz1989}
  and \citet{Horvath1995} for ample discussion.%
}

In two major variants of Romani\il{Romani} (a Balkan language with Indic
substrate), again, a very similar construction exists:
\begin{exe}
\ex
\label{ex:12-8}
\begin{xlist}
\ex
\label{ex:12-8a}
\gll
So o Dem\`{\i}ri mislinol kas i Ar\`{\i}fa dikhol? \\
what the Demir thinks whom the Arifa sees \\
\exsource{111 (31a)}{McDaniel1986}
\ex
\label{ex:12-8b}
\gll
Na \'{\j}anav so o Dem\`{\i}ri mislinol kas i Ar\`{\i}fa dikhl\^a? \\
not I.know what the Demir thinks whom the Arifa saw \\
\exsource{112 (32b)}{McDaniel1986}
\end{xlist}
\end{exe}

\section{Characteristics of variant I}
\label{sec:12-3}

From embedded constructions as in (\ref{ex:12-4}), (\ref{ex:12-5}) and (\ref{ex:12-8b}), the position of the
finite verb in (\ref{ex:12-1}), (\ref{ex:12-3}) and (\ref{ex:12-6}), \textit{'t} in (\ref{ex:12-6}) and \textit{hogy} in (\ref{ex:12-7}) we can draw some
conclusions:
\begin{exe}
\ex
\label{ex:12-9}
\begin{xlisti}
  \ex The construction is a complex sentence with a constituent clause
  embedded in a matrix clause.
\label{ex:12-9i}
\ex
\label{ex:12-9ii}
\begin{xlista}
\ex
\label{ex:12-9iia}
The matrix clause is formally and semantically a
\emph{wh}"=interrogative clause
\ex\exalph{9iib}{\label{ex:12-9iib}
with \textit{was} occupying the position that is characteristic of \emph{wh}"=interrogative clauses.}
\end{xlista}
\end{xlisti}
\end{exe}
In all cases considered so far, the embedded clause looks like any ordinary
embedded \textit{wh}"=interrogative clause conforming to the rules of the individual
language. This impression is confirmed in (\ref{ex:12-10}):
%\exewidth{(35)}
\begin{exe}
\ex
\label{ex:12-10}
\begin{xlist}
\ex[]{
\label{ex:12-10a}
\gll
Was glaubt sie, auf wessen Hilfe man sich verlassen kann? \\
what thinks she on whose help one self rely can \\}
\ex[*]{
\label{ex:12-10b}
\gll
Was glaubt sie, daß man sich auf wessen Hilfe verlassen kann? \\
what thinks she that one self on whose help rely can \\}
\ex[*]{
\label{ex:12-10c}
\gll
Was glaubt sie, auf wessen Hilfe kann man sich verlassen? \\
what thinks she on whose help can one self rely \\}
\ex[*]{
\label{ex:12-10d}
\gll
Was glaubt sie, auf wessen Hilfe sich verlassen zu können? \\
what thinks she on whose help self rely to can \\}
\ex[*]{
\label{ex:12-10e}
\gll
Was glaubt sie, ob man sich auf dessen Hilfe verlassen kann? \\
what thinks she whether one self on his help rely can \\}
\end{xlist}
\end{exe}
(There must be no fall of \isi{intonation} at the comma.) Although the
matrix predicate \textit{glaub-} can combine with \textit{daß} clauses,
F2 clauses, and infinitival clauses, (\ref{ex:12-10b})--(\ref{ex:12-10d}) are impossible:
(\ref{ex:12-10b}) has no \emph{wh}"=phrase in clause initial position; embedded F2
interrogatives as in (\ref{ex:12-10c}) are disallowed in German\il{German}; and so are
infinitival interrogatives as in (\ref{ex:12-10d}). (\ref{ex:12-10e}) demonstrates that it
is not sufficient for the embedded clause to be interrogative: it must
be a \textit{wh}"=interrogative clause. This is summarized in the third
clause of (\ref{ex:12-9}):
\begin{exe}
\exi{(9)}
\begin{xlist}
\exi{iii.}\exalph{9iii}{\label{ex:12-9iii}
The constituent clause is formally an indirect \emph{wh}"=interrogative clause.}
\end{xlist}
\end{exe}        
All matrix predicates lexically select a non-interrogative complement
clause (in fact, all can combine with a \textit{daß} clause), and many
do not even allow for an interrogative complement, (\ref{ex:12-11}). Predicates
that only select interrogative complements cannot combine with
\textit{was}, (\ref{ex:12-12}).
\addlines
\begin{exe}
\ex
\label{ex:12-11}
\begin{xlist}
\ex[]{\label{ex:12-11a}
\gll
Karl denkt, daß wir diesen Kandidaten gewählt haben \\
Karl thinks that we this candidate elected have \\}
\ex[*]{
\label{ex:12-11b}
\gll
Karl denkt, welchen Kandidaten wir gewählt haben \\
Karl thinks which candidate we elected have \\}
\ex[]{
\label{ex:12-11c}
\gll
Was denkt Karl, welchen Kandidaten wir gewählt haben? \\
what thinks Karl which candidate we elected have \\}
\end{xlist}
\ex
\label{ex:12-12}
\begin{xlist}
\ex[]{
\label{ex:12-12a}
\gll
Karl möchte wissen, wen wir gewählt haben \\
Karl wants know whom we elected have \\}
\ex[*]{
\label{ex:12-12b}
\gll
Karl möchte wissen, daß wir sie gewählt haben \\
Karl wants know that we her elected have \\}
\ex[*]{
\label{ex:12-12c}
\gll
Was möchte Karl wissen, wen wir gewählt haben? \\
what wants Karl know whom we elected have\\}
\end{xlist}
\end{exe}
This is expressed in the fourth clause of (\ref{ex:12-9}):
\begin{exe}
\exi{(9)}
\begin{xlist}
\exi{iv.}\exalph{9iv}
{\label{ex:12-9iv}
The matrix predicate selects a non-interrogative complement clause.}
\end{xlist}
\end{exe}
It is in large part the tension between (\ref{ex:12-9iii}) and (\ref{ex:12-9iv}) that gives
the \textit{was \ldots{}  w}-P construction its strange appearance.

There is, however, a further aspect to selection by the matrix. In all
cases that I am aware of, the matrix can also combine with a \isi{nominal}
expression (\textit{das}, \textit{was}, \ldots{} ) with propositional meaning in
place of the constituent clause, as in (\ref{ex:12-13}), and it can often have
\textit{es} or \textit{das} in combination with the constituent clause, as
in (\ref{ex:12-14}).\footnote{%
  The constructions seen in (\ref{ex:12-13}) and (\ref{ex:12-14}) are not
  confined to matrix predicates that select a  \textit{daß} clause:
\ea
\label{ex:12-fn4i}
\gll Was möchte Karl wissen?\\     
     what wants Karl know \\
\ex
\label{ex:12-fn4ii}
\gll Karl kann das nicht wissen, ob es dort regnet \\  
     Karl can that not   know whether it there rains \\
\ex
\label{ex:12-fn4iii}
\gll Karl hat es immer bedauert, mir vertraut zu haben \\
     Karl has it always regretted to.me trusted to have \\
\z
They correlate with the observations on (\ref{ex:12-3e}) and (\ref{ex:12-3f}):
\ea[*]{\label{ex:12-fn4iv}
\gll Was scheint es?\\
     what seems it \\
}
\ex[?]{
\label{ex:12-fn4v}
\gll Was scheint dir?\\       
     what seems to.you \\
}
\zlast%
}
\begin{exe}
\ex
\label{ex:12-13}
\begin{xlist}
\ex
\label{ex:12-13a}
\gll
Das sagt Hanna \\
that says Hanna \\
\ex
\label{ex:12-13b}
\gll
Was  denkt  Hanna? \\
what thinks Hanna \\
\end{xlist}
\ex
\label{ex:12-14}
\begin{xlist}
\ex
\label{ex:12-14a}
\gll
Das denkt Hanna (nur), daß es dort regnet \\
that thinks Hanna (only) that it there rains \\
\ex
\label{ex:12-14b}
\gll
Hanna hat es oft gesagt, daß es dort regnet \\
Hanna has it often said that it there rains \\
\end{xlist}
\end{exe}
This observation is expressed in the last clause of (\ref{ex:12-9}):
\begin{exe}
\exi{(9)}
\begin{xlist}
\exi{v.}\exalph{9v}{\label{ex:12-9v} The matrix predicate can combine with a \isi{nominal} expression} 
\begin{xlista}
\ex\exalph{9va}{\label{ex:12-9va}
in place of a complement clause,  }
\ex\exalph{9vb}{\label{ex:12-9vb}
or in addition to a constituent clause.}
\end{xlista}
\end{xlist}
\end{exe}
(This applies to German\il{German}. I have not inquired into other languages.) It is the
co-existence of (\ref{ex:12-9iv}) and (\ref{ex:12-9v}) that gives rise to the competition between the
analytic ideas that we will now turn to.

\section{Analytic ideas}
\label{sec:12-4}

In my experience, everyone who is aware of the properties expressed in (\ref{ex:12-9}) but
has not investigated the construction in detail is prone to suggest an
analysis along the lines of (\ref{ex:12-15}). (Thus, I am confident (\ref{ex:12-15}) can be considered
\textit{the} traditional idea although I am not sure that it can be found
anywhere in the traditional literature on German\il{German}.)
\begin{exe}
\ex
\label{ex:12-15}
\textit{Traditional idea: \textsqe{appositive}}: 
 \begin{xlisti}
\ex
\label{ex:12-15i}
\textit{Was} is a complement of the matrix predicate. 
\ex
\label{ex:12-15ii}
The constituent clause is (not a complement but) something like an
apposition elucidating \textit{was}; cf.\ paraphrase type (\ref{ex:12-2c}).
\end{xlisti}
\end{exe}
Therefore, Tappe's suggestion \citep{Tappe1980} was felt to be genuinely intriguing:
\begin{exe}
\ex
\label{ex:12-16}
\textit{Tappe, Riemsdijk idea: \textsqe{scope indicating}}: 
\begin{xlisti}
\ex
\label{ex:12-16i}
\textit{Was} is (not a complement but) a \textsqe{scope marker} that is
\textsqe{base-generated} in \textsc{Comp}; it must be coindexed with a \emph{wh}"=phrase
in the \textsc{Comp} of the constituent clause.\footnote{%
  This assumption is of course only applicable to languages that characterize their
  \textit{wh}"=interrogative clauses by some specific \textsqe{\textsc{Comp} position}. Thus, 
  it is not evident that it is relevant for Hungarian\il{Hungarian}; cf.\ (\ref{ex:12-7}).%
}
\ex
\label{ex:12-16ii}
The constituent clause is a complement of the matrix
predicate; cf.\ paraphrase type (\ref{ex:12-2d}).
\end{xlisti}
\end{exe}
Evidently, both ideas raise quite a number of questions. For instance, while
(\ref{ex:12-15i}) (unlike (\ref{ex:12-16i})) relies on (\ref{ex:12-9v}), the notions of \textsqe{apposition} and
\textsqe{elucidation} in (\ref{ex:12-15ii}) are in need of clarification.

The construction exemplified in (\ref{ex:12-14}) might seem to be an instance of
the relation appealed to in (\ref{ex:12-15ii}). But this impression is
misleading.  Occasionally, \textit{es} and \textit{das} in (\ref{ex:12-14}) are
considered to be associated with no semantic content whatsoever, so
that they do not play any role in the determination of the clause's
meaning. If this is true, \textit{was} in (\ref{ex:12-13}) and in the \textit{was \ldots{}  w}-P construction must be something totally different, as it
obviously contributes to the meaning of the clause. Alternatively,
\textit{es} and \textit{das} in (\ref{ex:12-14}) are often considered to be
cataphors. That is, they contribute importantly to the determination
of the clause's meaning, but identify their content with that of the
embedded clause they are cataphorically related to. Again, the same
cannot be true for \textit{was} in (\ref{ex:12-1}), (\ref{ex:12-3}), etc.: (\ref{ex:12-1}) does not have
the (impossible) meaning \textsqe{(do) you think who is right} that would
result from identifying the content of \textit{was} with that of the
constituent clause. Thus, if (\ref{ex:12-9v}) is relevant at all, its clause
(\ref{ex:12-9va}) is, but (\ref{ex:12-9vb}) cannot play any role for (\ref{ex:12-15}) (or (\ref{ex:12-16})). Put
differently, it does not seem possible to understand both \textit{was}
and the constituent clause in terms of antecedent analytic experience.

Still, some general account might conceivably be developed that
predicts that when a matrix predicate takes \textit{was} as a \isi{nominal}
complement, any clause it combines with must be of a different
semantic type than the matrix ordinarily combines with, in accordance
with (\ref{ex:12-9iii}) and (\ref{ex:12-9iv}). In this way, (\ref{ex:12-10a})--(\ref{ex:12-10d}) could conceivably
be accounted for.\footnote{%
	See \citet{Dayal1994} for an explicit analysis
  of Hindi along these lines.%
}
But then it seems next to impossible to
account for the negative datum (\ref{ex:12-10e}).

Also, it is not clear why (\ref{ex:12-5c}) and (\ref{ex:12-5d}) should not be acceptable to
all speakers who accept long extractions, given that (\ref{ex:12-17}) would be a
possible structure for (\ref{ex:12-5c}):
\begin{exe}
\ex
\label{ex:12-17}
Was\textsubscript{\textit{i}} [glaubst du [daß Karl t\textsubscript{\textit{i}} meint [wen wir gewählt haben]]]?
\end{exe}        
In (\ref{ex:12-16i}), the very concept of a \textsqe{scope marker} is in need of
clarification.  The scope being indicated is obviously the \textsqe{scope} of
interrogativity. But it may be more, perhaps including the scope of a
\emph{wh}"=quantifier and, if so, also the scope of the variable
restriction (thus differing from pure markers of interrogativity such
as \textit{ka} in Japanese). Also, the coindexation is obviously meant
to have similar consequences like coindexation of a long extracted
phrase and its trace(s), so that the complement is not evaluated as an
interrogative clause, in accordance with (\ref{ex:12-9iv}). But how does this
come about?  And how are (\ref{ex:12-10c}) and (\ref{ex:12-10d}) accounted for? (Cf.\ Section~\ref{sec:12-10} on the latter question.)

To appreciate how any reliance on the notion of \textsqe{coindexation} can be
problematic, we may look at a proposal in \citet[106]{Hiemstra1986}.  The
claim there is that (i) \textit{was} and the embedded clause are
coindexed (because they both relate to an object position licensed by
the matrix), and (ii) any clause and its head are coindexed. \citet{Hiemstra1986} takes \textit{was} and the \emph{wh}"=phrase to be situated in
the heads of their clauses and (iii) to be coindexed with the heads.
Alternatively, one may take them to be specifiers of C and (iii') to
be coindexed with C. In any case, by transitivity of coindexation \textit{was} and the \emph{wh}"=phrase end up being coindexed. This seems
like a remarkable result: the coindexation appealed to in (\ref{ex:12-16i}) is
deduced from more general principles, and (\ref{ex:12-15}) and (\ref{ex:12-16}) are seen to
inadequately isolate different aspects of one and the same structural
configuration. In fact, however, transitivity of coindexation in
\citet{Hiemstra1986} is just a mirage arising from equivocations.  There
may be a sensible explication for the coindexation in step (i),
although this is far from evident in light of our discussion of
(\ref{ex:12-9vb}). There may also be some explication for the coindexation of a
clause and its head in step (ii), although this again is not at all
obvious. Spec-head coindexation in step (iii') -- or even coindexation
in step (iii) of a \emph{wh}"=expression and the position it is
situated in -- might be explicable in its own way. But these three
(hypothetical) explications have nothing in common. For example, the
embedded \emph{wh}"=phrase is definitely not an object of the matrix
in the way that \textit{was} or the embedded clause possibly is one;
and the coindexation of embedded and matrix clause that results from
transitivity makes no sense at all. Hence, this tale about
coindexation fails to have the consequence intended by (\ref{ex:12-16i}): it does
not express any sensible relation between \textit{was} and the \emph{wh}"=phrase. It merely serves to obscure distinctions that no
analysis can afford to ignore.  (Of course, \citet{Hiemstra1986} is not
alone in this: abuse of coindexation is ubiquitous in the literature.)

The version of (\ref{ex:12-16i}) in \citet{Riemsdijk1982} more articulately asserts
that \textit{was} and the \emph{wh}"=phrase bear identical \textsqe{scope
indices}, where a scope index ``is a property of the \emph{wh}"=feature'' that is associated with a \emph{wh}"=word and
percolates to the \emph{wh}"=phrase containing that word. Still, the
scope index is of the same kind as other indices used in the
grammar. Therefore, maleficent interactions with several modules of
the grammar must be circumvented by judiciously assigning different
percolation mechanisms and well-formedness conditions on coindexation
to different levels of representation.

Faced with open questions of all kinds, we turn to observations that
might help motivate a choice between (\ref{ex:12-15}) and (\ref{ex:12-16}).

\section{Variant II: \textit{w}-P  \ldots{}  \textit{w}-P}
\label{sec:12-5}

\largerpage
Many (but not all)\footnote{%
	The variation among speakers has no
  obvious dialectal or regional basis.%
}
speakers of German\il{German} use a
construction that looks just like the \textit{was \ldots{}  w}-P
construction, except that it exhibits a copy of the \emph{wh}"=phrase
in place of \textit{was}:

\begin{exe}
\ex
\label{ex:12-18}
\begin{xlist}
\ex[]{
\label{ex:12-18a}
\gll
Wer glaubst du, wer Recht hat? \\
who think you who right has \\}
\ex[]{
\label{ex:12-18b}
\gll
Wen meint Karl, wen wir gewählt haben? \\
whom thinks Karl whom we elected have \\}
\ex[]{
\label{ex:12-18c}
\gll
Wie nimmt man an, wie der Prozeß ausgeht? \\
how assumes one {} how the trial ends \\}
\ex[]{\label{ex:12-18d}
\gll
Wovon denkst du, wovon wir leben? \\
where.of think you where.of we live \\}
\ex[*]{\label{ex:12-18e}
\gll
Auf wen hat sie gesagt, auf wen er warten soll?\\
on whom has she said on whom he wait should \\}
\ex[?]{\label{ex:12-18f}
\gll
Wieviel meint sie, wieviel das kostet? \\
how-much thinks she how.much that costs \\}
\ex[?]{
\label{ex:12-18g}
\gll
Wen scheint es, wen Hans geschlagen hat? \\
whom seems it whom Hans hit has \\
\exsource{247 (59a)}{McDaniel1986}}
\end{xlist}
\end{exe}
It can also be embedded:
\begin{exe}
\ex
\label{ex:12-19}
\gll
Heinz möchte wissen / es ist egal,\\
Heinz wants know {} it is no.difference \\
\begin{xlist}
\ex[ ]{
\label{ex:12-19a}
\gll
\ldots{}  wer du glaubst, wer Recht hat \\
 {}  who you think who right has \\}
\ex[ ]{
\label{ex:12-19b}
\gll
\ldots{}  wen Karl meint, wen wir gewählt haben \\
  {}  whom Karl thinks whom we elected have \\}
\ex[ ]{
\label{ex:12-19c}
\gll
\ldots{}  wie man annimmt, wie der Prozeß ausgeht \\
  {}  how one assumes how the trial ends \\}
\ex[ ]{
\label{ex:12-19d}
\gll
\ldots{}  wovon du denkst, wovon wir leben \\
  {}  where.of you think where.of we live \\}
\ex[?]{
\label{ex:12-19e}
\gll
\ldots{} auf wen sie gesagt hat, auf wen er warten soll \\
  {}  on whom she said has on whom he wait should \\}
\ex[ ]{
\label{ex:12-19f}
\gll
\ldots{}  wieviel sie meint, wieviel das kostet \\
  {}  how.much she thinks how.much that costs \\}
\end{xlist} 
\end{exe}
To my ear, (\ref{ex:12-19e}) and (\ref{ex:12-19f}) seem markedly better than (\ref{ex:12-18e}) and (\ref{ex:12-18f}). Still,
there are strong restrictions on the \emph{wh}"=phrase to be copied:
\pagebreak
\begin{exe}
\ex[ ]{
\label{ex:12-20}
\gll
Heinz möchte wissen / es ist egal, \\
Heinz wants know {} it is no.difference \\}
\begin{xlist}
\ex[*]{
\label{ex:12-20a}
\gll
\ldots{}  welche (Bücher) du glaubst, welche Bücher sie gerne liest \\
  {}  which \hspaceThis{(}books you think which books she gladly reads \\}
\ex[*]{
\label{ex:12-20b}
\gll
\ldots{}  wessen (Hund) du meinst, wessen Hund das ist \\
  {}  whose \hspaceThis{(}dog you think whose dog that is \\}
\ex[*]{
\label{ex:12-20c}
\gll
\ldots{}  wen sie gesagt hat, auf wen er warten soll \\
 {}  whom she said has on whom he wait should \\}
\end{xlist}
\end{exe}
(\ref{ex:12-20a}) and (\ref{ex:12-20b}) show that the \emph{wh}"=word does not combine with an
ordinary noun.\footnote{%
	Ellen Brandner told me Josef Bayer told her
  there are actually speakers who use this kind of example.%
} (\ref{ex:12-20c})
shows that when the embedded \emph{wh}"=phrase is a prepositional
phrase, the full PP must be copied, as in (\ref{ex:12-19e}); just copying its
\isi{nominal} constituent is strictly impossible.

Variant I can be embedded in an exclamative matrix (\ref{ex:12-21}), and variant
II can, too, at least to some extent (\ref{ex:12-22}).
\begin{exe}
\ex
\label{ex:12-21}
\begin{xlist}
\ex
\label{ex:12-21a}
\gll
Du würdest dich wundern, was Heinz meint, wieviel du verdienst \\
you would self be.surprised what Heinz thinks how.much you earn \\
\ex
\label{ex:12-21b}
\gll
Schildern Sie mal, was Karl glaubt, wie das funktionieren soll! \\
describe you.\textsc{honor} \textsc{prtcl} what Karl thinks how that function should \\
\end{xlist}       
\ex
\label{ex:12-22}
\gll
Du würdest dich wundern, wie Heinz meint, wie das funktioniert \\
you would self be.surprised how Heinz thinks how that functions \\
\end{exe}
The copying construction is also known from other languages. It is found in
Frisian:
\begin{exe}
\ex
\label{ex:12-23}
\begin{xlist}
\ex
\label{ex:12-23a}
\gll
W\^er tinke jo w\^er't Jan wennet? \\
where think you {where that} Jan resides \\
\exsource{99 (3b)}{Hiemstra1986}
\ex
\label{ex:12-23b}
\gll
Wa tinke jo wa't my sjoen hat? \\
who think you {who that} me seen has \\
\exsource{99 (2b)}{Hiemstra1986}
\end{xlist}
\end{exe}
And in Afrikaans:
\begin{exe}
\ex
\label{ex:12-24}
\begin{xlist}
\ex
\label{ex:12-24a}
\gll
Waarvoor dink julle waarvoor werk ons? \\
where.for think you where.for work we \\
\exsource{725 (8)}{Plessis1977}
\ex
\label{ex:12-24b}
\gll
Met wie het jy nou weer ges\^e met wie het Sarie gedog met
        wie gaan Jan trou?\\
with who have you now again said with who has Sarie thought with who goes Jan marry \\
\exsource{725 (11)}{Plessis1977}
\ex
\label{ex:12-24c}
\gll
Waaroor dink jy waaroor dink die bure wat / waar stry ons
        die meeste oor? \\
where.about think you where.about think the neighbours what {} where argue we
     the most about \\
\exsource{725 (15)}{Plessis1977}
\end{xlist}
\end{exe}
Note that in Afrikaans, embedded \emph{wh}"=interrogatives need not have the
independent verb in final position: in informal speech, the second position,
as seen in (\ref{ex:12-24}), is preferred \citep[530]{Ponelis1979}. Notice also the
remarkable case of full PP copying combined with preposition stranding in the
lowest clause in (\ref{ex:12-24c}).

One variant of Romani\il{Romani} also makes use of the copying construction:
\begin{exe}
\ex
\label{ex:12-25}
\begin{xlist}
\ex
\label{ex:12-25a}
\gll
Kas misline kas o Dem\`{\i}ri dikhl\^a? (McDaniel~1986:\,182\,(126a))\hspace{-1ex}\\
       whom you.think whom the Demir saw \\
% guckt zu weit raus
%\jambox{\citep[182 (126a)]{McDaniel1986}\hspace{-1ex}}
%\exsource{182 (126a)}{McDaniel1986}
\ex
\label{ex:12-25b}
\gll
Kas izgl\`eda kas o Dem\`{\i}ri mar\'{\j}a? \\
whom it.seems whom the Demir hit \\\exsource{247 (59b)}{McDaniel1986}
\end{xlist}
\end{exe}
As a rough summary, variant II can be characterized as in (\ref{ex:12-26}):
\begin{exe}
\ex
\label{ex:12-26}
The characteristics of variant II are identical to (\ref{ex:12-9}), except for
(\ref{ex:12-9iib}): there is a copy of the \emph{wh}"=phrase, rather than
\textit{was}. The copy (and hence, the \emph{wh}"=phrase) must not
contain a full noun.\footnote{%
	Considering the observation in \citet[247f.]{McDaniel1986} that (\ref{ex:12-18g}) appears to be better than (\ref{ex:12-3e}), variant
  II possibly does not fully comply with (\ref{ex:12-9v}).%
}
\end{exe}
Modifying the aspect of the analysis that is responsible for the form
of the initial \emph{wh}"=expression takes us from an analysis of
variant II to an analysis of variant I (or vice versa). It appears,
thus, that analyses of variants I and II must be closely related.

Obviously, (\ref{ex:12-15}) and (\ref{ex:12-16}) differ markedly with respect to their ability
to accommodate (\ref{ex:12-26}). According to (\ref{ex:12-15ii}), there is a relation between the initial \emph{wh}"=expression and the embedded clause, but no
relation between the initial \emph{wh}"=expression and the embedded \emph{wh}"=phrase. According to (\ref{ex:12-16}), the converse is true. But variant II
is characterized by a specific relation between the initial \emph{wh}"=expression and the embedded \emph{wh}"=phrase. Hence, the
existence of variant II is altogether unexpected upon (\ref{ex:12-15}), but seems
natural upon (\ref{ex:12-16i}).

If the copy in variant II is indeed a \textsqe{scope marker} just like \textit{was} in variant I, (\ref{ex:12-20c}) shows that Tappe's original version of
(\ref{ex:12-16i}) is more correct than Riemsdijk's. According to \citet{Riemsdijk1982},
the \textsqe{scope index} of the PP is identical to the scope index of the
\isi{nominal} embedded in it. Hence, there is no reason why (\ref{ex:12-20c}) should be
any worse than (\ref{ex:12-19e}). But according to Tappe, the initial \emph{wh}"=expression is related to the \emph{wh}"=phrase itself, as in
(\ref{ex:12-19e}), rather than to anything embedded in it.

\section{\textit{Wh}"=phrases in situ}
\label{sec:12-6}

In situ \emph{wh}"=phrases provide another opportunity to study the consequences of
(\ref{ex:12-15}) and (\ref{ex:12-16}):
\begin{exe}
\judgewidth{?*}
\ex
\label{ex:12-27}
\begin{xlist}
\ex[ ]{
\label{ex:12-27a}
\gll
\textsc{was} meint \textsc{wer}, wen wir gewählt haben?\\
what thinks who whom we elected have \\}
\trans\strut\hfill (cf.\ \citealt[153, (79b)]{McDaniel1986})
\exi{}[ ]{
\gll
(= \textsc{wen} meint \textsc{wer}, daß wir gewählt haben?) \\
(= whom thinks who that we elected have)\\}
\ex[?*]{
\label{ex:12-27b}
\gll
\textsc{wer} meint \textsc{was}, wen wir gewählt haben? \\
who thinks what whom we elected have \\}
\end{xlist}
\end{exe}
Most speakers I have consulted agree that (\ref{ex:12-27a}) is fully acceptable or at
least possible. This is expected upon (\ref{ex:12-15i}). It is compatible with (\ref{ex:12-16i}) if
\textit{was} does not just indicate the \textsqe{scope} of interrogativity but (at least) the
scope of a \emph{wh}"=quantifier. Most speakers strongly reject (\ref{ex:12-27b}). This is totally
surprising upon (\ref{ex:12-15i}). Upon (\ref{ex:12-16i}), (\ref{ex:12-27b}) should be absolutely impossible for
all speakers. For some, though, the effect is slightly less strong. The same
results are found with embedding:
\begin{exe}
\judgewidth{?*}
\ex
\label{ex:12-28}
\begin{xlist}
\ex[ ]{
\label{ex:12-28a}
\gll
Es ist egal, \textsc{was} \textsc{ wer} meint, wen wir gewählt haben \\
it is no.difference what who thinks whom we elected have \\}
\ex[?*]{
\label{ex:12-28b}
\gll
Es ist egal, \textsc{wer} \textsc{was} meint, wen wir gewählt haben \\
it is no.difference who what thinks whom we elected have \\}
\end{xlist}
\end{exe}
Observations on echo questions are similar:
\begin{exe}
\ex[?*]{
\label{ex:12-29}
\gll
Karl meint \textsc{was}/, wen wir gewählt haben? \\
Karl thinks what whom we elected have \\}
\end{exe}
Most speakers strongly reject examples like this, which is surprising upon
(\ref{ex:12-15i}). And again, some reject (\ref{ex:12-29}) less vehemently than (\ref{ex:12-16i}) would lead one
to expect.

Judgements are sharp with variant II:
\begin{exe}
\ex
\label{ex:12-30}
\begin{xlist}
\ex[ ]{
\label{ex:12-30a}
\gll
\textsc{wo} meint \textsc{wer}, wo das stattfindet? \\
where thinks who where that place.takes \\  }
\ex[*]{
\label{ex:12-30b}
\gll
\textsc{wer} meint \textsc{wo}, wo das stattfindet? \\
who thinks where where that place.takes\\}
\end{xlist}
\ex
\label{ex:12-31}
\begin{xlist}
\ex[ ]{
\label{ex:12-31a}
\gll
Es ist egal, \textsc{wo} \textsc{wer} meint, wo das stattfindet \\
it is no.difference where who thinks where that place.takes \\}
\ex[*]{
\label{ex:12-31b}
\gll
Es ist egal, \textsc{wer} \textsc{wo} meint, wo das stattfindet \\
it is no.difference who where thinks where that place.takes\\}
\end{xlist}
\end{exe}
For speakers who actively use variant II, (\ref{ex:12-30a}) and (\ref{ex:12-31a}) are fine, but
(\ref{ex:12-30b}) and (\ref{ex:12-31b}) are inconceivable (on the intended reading).

\section{LF movement?}
\label{sec:12-7}

One way to explicate the notion of a \emph{wh}"=scope indicator is to assume \textsqe{LF
movement} of the \emph{wh}"=phrase from the embedded clause to the initial
\emph{wh}"=expression. There are (at least) two problems with this idea: coordination
and matrix negation.

Consider (\ref{ex:12-32a}) and (\ref{ex:12-32b}):
\begin{exe}
\ex
\label{ex:12-32}
\gll
Es ist egal,\\
it is no.difference \\
\begin{xlist}
\ex[ ]{
\label{ex:12-32a}
\gll
\ldots{}  ob sie kommt und wen sie mitbringt \\
{}   whether she comes and whom she with.brings \\}
\ex[*]{
\label{ex:12-32b}
\gll
\ldots{}  was er meint, ob sie kommt und wen sie mitbringt \\
  {}  what he thinks whether she comes and whom she with.brings \\}
\ex[ ]{
\label{ex:12-32c}
\gll
\ldots{}  was er meint, wann sie kommt und wen sie mitbringt \\
{}  what he thinks when she comes and whom she with.brings \\}
\end{xlist} 
\end{exe}
Although a \textit{whether} clause and a \emph{wh}"=clause can in
general be conjoined, as in (\ref{ex:12-32a}), they cannot in the \textit{w- \ldots{}  w}"=construction (\ref{ex:12-32b}). This is just what we would expect on the evidence
of (\ref{ex:12-10e}).  And expectedly, two \textit{wh}"=clauses can be conjoined,
as in (\ref{ex:12-32c}). But what would the result of LF movement look like in
this case? Both \textit{wann} and \textit{wen} would have to move to
the position of \textit{was} -- how are they situated to one another at
LF?\footnote{%
	And in terms of (\ref{ex:12-16i}): what would it mean for
  \textit{was} to be \textsqe{coindexed} with both \textit{wann} and
  \textit{wen}?%
}
Even if one might prefer to leave this question to a
general theory of coordination, it is of no use to ignore it for long.
 
As for negation in the matrix, consider first the examples in (\ref{ex:12-33}): a
\textit{was \ldots{}  w}-P construction in (\ref{ex:12-33a}), a long extraction in
(\ref{ex:12-33b}), a sequence of unembedded clauses (just like (\ref{ex:12-2b})) in (\ref{ex:12-33c}),
and a complex construction like (\ref{ex:12-2c}) in (\ref{ex:12-33d}).
\begin{exe}
\ex
\label{ex:12-33}
\begin{xlist}
\ex
\label{ex:12-33a}
\gll
Was meint jeder, wen Hanna mitbringt?\\
what thinks everybody whom Hanna with.brings \\
\ex
\label{ex:12-33b}
\gll
Wen meint jeder, daß Hanna mitbringt?\\
whom thinks everybody that Hanna with.brings \\
\ex
\label{ex:12-33c}
\gll
Was meint jeder$\setminus$; wen bringt Hanna mit?\\
what thinks everybody whom brings Hanna with \\
\ex
\label{ex:12-33d}
\gll
Was meint jeder hinsichtlich der Frage, wen Hanna mitbringt?\\
what thinks everybody wrt. the question whom Hanna with.brings \\
\end{xlist}
\end{exe}
None of these examples is problematic.\footnote{%
	Also, the bound
  reading of the pronoun in (i) is fully acceptable to many speakers,
  pace \citet[152 (21b)]{Dayal1994}:
\ea
\label{ex:12-fn11i}
\gll
Was glaubt  [jeder Student]\textsubscript{\textit{i}}, mit wem er\textsubscript{\textit{i}} gesprochen hat?\\
what thinks every student\textsubscript{\textit{i}} with whom he\textsubscript{\textit{i}} spoken has \\
\zlast%
}
But when \textit{everybody} is replaced by \emph{nobody}, results are very different, as
\citet[214]{Kiss1988} was the first to observe (for Hungarian\il{Hungarian}):
\begin{exe}
\ex
\label{ex:12-34}
\begin{xlist}
\ex[*]{
\label{ex:12-34a}
\gll
Was meint keiner, wen Hanna mitbringt?\\
what thinks nobody whom Hanna with.brings\\}
\ex[ ]{
\label{ex:12-34b}
\gll
Wen meint keiner, daß Hanna mitbringt?\\
whom thinks nobody that Hanna with.brings\\}
\ex[*]{
\label{ex:12-34c}
\gll
Was meint keiner$\setminus$; wen bringt Hanna mit?\\
what thinks nobody whom brings Hanna with\\}
\ex[ ]{
\label{ex:12-34d}
\gll
Was meint keiner hinsichtlich der Frage, wen Hanna mitbringt?\\
what thinks nobody wrt. the question whom Hanna with.brings\\}
\end{xlist}
\end{exe}
The extraction in (\ref{ex:12-34b}) is possible (if somewhat marginal) for speakers who
do long extractions. In contrast, (\ref{ex:12-34a}) is definitely bad (or impossible, for
some speakers). Notice also that (\ref{ex:12-34d}), which is supposed to be semantically
similar to the analysis of (\ref{ex:12-34a}) upon (\ref{ex:12-15ii}), is possible (in certain
contexts). This appears to indicate that (\ref{ex:12-15}) will not provide a plausible
account for (\ref{ex:12-34a}). But LF movement in accordance with (\ref{ex:12-16i}) does not seem to
provide a plausible account either, for why should it be blocked in (\ref{ex:12-34a})
while S-structure movement is possible in (\ref{ex:12-34b})?

\section{Interpretational dependencies}
\label{sec:12-8}

Originally, the notion of \textsqe{LF movement} was motivated by the observation that
certain interpretational dependencies seem to comply with restrictions that
overt (S"=structural) movement is subject to. Viewed from this perspective,
(\ref{ex:12-34b}) is a genuine problem for an LF movement account of (\ref{ex:12-34a}). Still, there
is a similar blocking effect induced by negation in (\ref{ex:12-35c}), pointed out to me
by Jürgen Pafel (p.c.) in spring 1989:
\begin{exe}
\ex
\label{ex:12-35}
\begin{xlist}
\ex[ ]{
\label{ex:12-35a}
\gll
[Was für Bücher] hat niemand gelesen? \\
what for books has nobody read \\}
\ex[ ]{
\label{ex:12-35b}
\gll
[Was] hat Karla [für Bücher] gelesen? \\
what has Karla for books read \\}
\ex[*]{
\label{ex:12-35c}
\gll
[Was] hat niemand [für Bücher] gelesen? \\
what has nobody for books read \\}
\end{xlist}
\end{exe}
In general, \textit{was} can be detached from an \isi{NP} of the form \textit{was für NP}, as in (\ref{ex:12-35b}). But when negation intervenes between the preposed part \textit{was} and the remnant \textit{für NP}, as in (\ref{ex:12-35c}), the result is bad. Similar observations hold for in situ \emph{wh}"=phrases as in (\ref{ex:12-36}):
\begin{exe}
\ex
\label{ex:12-36}
\begin{xlist}
\ex[ ]{
\label{ex:12-36a}
\gll
Es ist egal, \textsc{wem} Karla \textsc{wen} vorgestellt hat \\
it is no.difference to.whom Karla whom introduced has \\}
\ex[ ]{
\label{ex:12-36b}
\gll
Es ist egal, \textsc{wer} \textsc{wen} niemals betrogen hat \\
it is no.difference who whom never deceived has \\}
\ex[*]{
\label{ex:12-36c}
\gll
Es ist egal, \textsc{wem} niemand \textsc{wen} vorgestellt hat \\
it is no.difference to.whom nobody whom introduced has \\}
\ex[*]{
\label{ex:12-36d}
\gll
Es ist egal, \textsc{wer} niemals \textsc{wen} betrogen hat \\
it is no.difference who never whom deceived has \\}
\end{xlist}
\end{exe}
Thus, (\ref{ex:12-34a}) seems to fall into a pattern such that at S-structure
negation must not intervene beween some interpretationally dependent
expression -- the \emph{wh}"=phrase in (\ref{ex:12-34a}), the remnant in (\ref{ex:12-35c}),
the in situ \emph{wh}"=phrase in (\ref{ex:12-36}) -- and the position it is
dependent on. For thorough empirical and theoretical discussion see
\citet{Beck1993} and \citet{Beck1996}, where a non-traditional notion of \textsqe{LF
(movement)} is motivated.\footnote{%
Contrary to \citet[48]{Beck1996}, but in
  accordance with \citet[11]{Beck1993}, I consider it highly probable that
  at least \textit{für} in (\ref{ex:12-35b},c) must be relevantly related to \textit{was}, since \textit{was für} is something like an idiom.%
}
In broader empirical context, then, (\ref{ex:12-16i}) actually appears to receive
support from (\ref{ex:12-34a}).

\section{Exclamatives}
\label{sec:12-9}

In (\ref{ex:12-21}) and (\ref{ex:12-22}), partly repeated in (\ref{ex:12-38}) below, we have seen that the \textit{w- \ldots{}  w}"=construction can be embedded in an exclamative matrix. This
merits closer inspection.\footnote{%
	I am grateful to Franz d'Avis for useful conversation on this topic.%
}

For present purposes, I consider a predicate to be \textsqe{exclamative} if it
(i) combines with clauses that look like \emph{wh}"=interrogative
clauses but (ii) does not (on the same reading) combine with \textit{whether} clauses and (iii) allows the \emph{wh}"=clause to be
introduced by certain \emph{wh}"=phrases that do not occur in bona
fide interrogative clauses.\footnote{%
	Thus, it is the similarity in
  German\il{German} of exclamative and interrogative predicates with respect to
  their complements that gives rise to our discussion. Note this is
  not universal: Irish, e.g., does not have it, according to
  \citet[99]{McCloskey1979}.%
}
The predicates \textit{wunder-} (\textsqe{be
surprised}) and \textit{schilder-} (\textsqe{describe}) are exclamative in
this sense. Only \textit{wunder-} is illustrated in (\ref{ex:12-37});
but note that exclamative predicates need not in general be \textsqe{emotive}
in any obvious sense.
\begin{exe}
\ex
\label{ex:12-37}
\begin{xlist}
\ex[ ]{
\label{ex:12-37a}
\gll
Sie wundert sich, wieviel du verdienst \\
she is.surprised self how.much you earn \\}
\ex[*]{
\label{ex:12-37b}
\gll
Sie wundert sich, ob du viel verdienst\\
she is.surprised self whether you much earn \\}
\ex[ ]{
\label{ex:12-37c}
\gll
Sie wundert sich, [was für riesige Füße] er hat \\
she is.surprised self {\hphantom{[}what} for huge feet he has \\}
\ex[ ]{
\label{ex:12-37d}
\gll
Sie wundert sich, [wie erfolglos] er ist\\
she is.surprised self {\hphantom{[}how} unsuccessful he is \\}
\ex[ ]{
\label{ex:12-37e}
\gll
Sie wundert sich, [welches Behagen] sie empfindet \\
she is.surprised self {\hphantom{[}which} comfort she senses \\}
\ex[ ]{
\label{ex:12-37f}
\gll
Sie wundert sich, [wie (sehr / wenig)] sich die Stadt verändert hat \\
she is.surprised self {\hphantom{[}how} (very / little) self the city changed has \\}
\ex[ ]{
\label{ex:12-37g}
\gll
Sie wundert sich, was er manchmal schnarcht \\
she is.surprised self what he sometimes snores \\}
\end{xlist}
\end{exe}
\emph{wh}"=phrases like those in (\ref{ex:12-37}c-g) are impossible (or, at
least, infelicitous) in true interrogatives; I will call them
\textsqe{exclamative \emph{wh}"=phrases}. The special properties of exclamative
\emph{wh}"=phrases cannot in general be traced to lexical properties of
some \emph{wh}"=word. Thus, \textit{was für} in (\ref{ex:12-37c}), \textit{wie} in
(\ref{ex:12-37d}), and \textit{welch-} in (\ref{ex:12-37e}) seem to be just the same as in
ordinary \emph{wh}"=interrogative phrases. In these cases, the
exclamative quality of the phrases apparently derives compositionally
from the combination with the other constituents in the \emph{wh}"=phrase. (But adverbial (or ad-adverbial)  \textit{wie} in (\ref{ex:12-37f}) and \textit{was} in (\ref{ex:12-37g}), both meaning \textsqe{how much}, seem to be confined to
exclamatives.)  Absence of \textit{whether} clauses, as in (\ref{ex:12-37b}), is a
necessary but not sufficient condition. There are some classes of
predicates such as \textit{aufzähl-} (\textsqe{enumerate}) that take bona fide \emph{wh}"=interrogative clauses but no  \textit{whether} clauses; see \citet{Schwarz1994}
for thorough discussion. Thus, the correct generalization appears to
be: if a predicate takes a clause with an exclamative \emph{wh}"=phrase,
it also takes a clause with an ordinary \emph{wh}"=phrase, but does not
(on the same reading) take a \textit{whether} clause.

Some examples with the \textit{was \ldots{}  w}-P construction appear in (\ref{ex:12-38}):
\begin{exe}
\ex
\label{ex:12-38}
\begin{xlist}
\ex
\label{ex:12-38a}
\gll
Du würdest dich wundern, was Heinz meint, wieviel du
verdienst\\
you would self be.surprised what Heinz thinks how.much you earn \\\jambox{= (\ref{ex:12-21a})}
\ex
\label{ex:12-38b}
\gll
Schildern Sie mal, was Heinz glaubt, wie das funktionieren
soll! \\
describe you.\textsc{honor} \textsc{prtcl} what Heinz thinks how that
function should \\\jambox{= (\ref{ex:12-21b})}
\ex
\label{ex:12-38c}
\gll
Sie findet es schrecklich, was Heinz sagt, wer alles gekommen ist \\
she finds it awful what Heinz says who all come is\\
\ex
\label{ex:12-38d}
\gll
Er begreift jetzt, was sie denkt, was für Nägel wir brauchen \\
he grasps now what she thinks what for nails we need \\
\end{xlist}
\end{exe}
But examples degrade significantly when the \emph{wh}"=phrase is an
exclamative \emph{wh}"=phrase:
\begin{exe}
\ex
\label{ex:12-39}
\begin{xlist}
\ex
\label{ex:12-39a}
\gll
Sie wundert sich, (?*was er meint) wie sehr sich die Stadt verändert hat \\
she is.surprised self (\phantom{?*}what he thinks) how very self the city changed has \\
\ex
\label{ex:12-39b}
\gll
Schildern Sie mal, (??was Heinz sagt) welches Behagen er empfindet! \\
describe you.\textsc{honor} \textsc{prtcl} (\phantom{??}what Heinz says) which comfort he senses \\
\ex
\label{ex:12-39c}
\gll
Sie findet es schrecklich, (?*was er glaubt) was sie manchmal schnarcht \\
she finds it awful (\phantom{?*}what he thinks) what she sometimes snores \\
\ex
\label{ex:12-39d}
\gll
Er begreift jetzt, (?*was sie denkt) was für winzige Nägel wir brauchen \\
he grasps now (\phantom{?*}what she thinks) what for tiny nails we need \\
\end{xlist}
\end{exe}
\addlines[2]
\enlargethispage{3pt}
On a first look, the material in parentheses may be felt to be
anything between mildly disturbing and thoroughly confusing. The
longer the examples are looked at, the more judgements appear to converge
towards outright rejection.  As can be expected upon this observation,
unembedded counterparts are nothing better, be they interrogative (\ref{ex:12-40})
or exclamative (\ref{ex:12-41}):\footnote{%
	There is one exception:
\ea
\label{ex:12-fn15i}
\gll
was \textsc{denkst} du / \textsc{meinen} Sie / \textsc{glaubt} ihr, was der manchmal schnarcht!\\   
          what think you.\textsc{sg} {} think you.\textsc{honor} {} think you.\textsc{pl} what he sometimes snores \\
   \z    
          This unembedded exclamative \textit{was \ldots{}  w}-P construction
          is extremely restricted along several dimensions. Only verba
          sentiendi are possible matrix predicates (no verba dicendi);
          only functionally second persons appear as their subjects;
          the verb must appear in second position, even though usually
          the final position as in (\ref{ex:12-41}) is possible or even preferred;
          and the meaning is not compositional: the matrix translates
          as \textsqe{you cannot imagine \ldots{}}.%
      }
\begin{exe}
\judgewidth{?*}
\ex
\label{ex:12-40}
\begin{xlist}
\ex[?*]{
\label{ex:12-40a}
\gll
Was meint er, wie sehr sich die Stadt verändert hat? \\
what thinks he how very self the city changed has \\}
\ex[??]{
\label{ex:12-40b}
\gll
Was sagt Heinz, welches Behagen er empfindet? \\
what says Heinz which comfort he senses \\}
\ex[?*]{
\label{ex:12-40c}
\gll
Was glaubt er, was sie manchmal schnarcht? \\
what thinks he what she sometimes snores \\}
\ex[?*]{
\label{ex:12-40d}
\gll
Was denkt sie, was für winzige Nägel wir brauchen? \\
what thinks she what for tiny nails we need \\}
\end{xlist}
\ex
\label{ex:12-41}
\begin{xlist}
\ex
\label{ex:12-41a}
\gll
(?*Was er meint) wie sehr sich die Stadt verändert hat! \\
(\phantom{?*}what he thinks) how very self the city changed has \\
\ex
\label{ex:12-41b}
\gll
(??Was Heinz sagt) welches Behagen er empfindet! \\
(\phantom{??}what Heinz says) which comfort he senses \\
\ex
\label{ex:12-41c}
\gll
(?*Was er glaubt) was sie manchmal schnarcht! \\
(\phantom{?*}what he thinks) what she sometimes snores \\
\ex
\label{ex:12-41d}
\gll
(?*Was sie denkt) was für winzige Nägel wir brauchen! \\
(\phantom{?*}what she thinks) what for tiny nails we need \\
\end{xlist}
\end{exe}
As surprising as these observations are, they seem to demonstrate
that (\ref{ex:12-16i}), as opposed to (\ref{ex:12-15ii}), is correct in that they seem to
reveal a specific dependency between \textit{was} and the \emph{wh}"=phrase
in the embedded clause.

However, this impression might be deceptive. I assume all \emph{wh}"=clauses receive a Hamblin style interpretation. The \emph{wh}"=phrase denotes a set of contextually salient entities (of suitable semantic type), call this the W-Set.  Correspondingly, the
\emph{wh}"=clause denotes a set of propositions, call this the C-Set.
The cardinality of the C-Set depends on the cardinality of the W-Set.
Interrogative and exclamative predicates exert different conditions on
the C-Set. The essence of interrogativity is that there is a possible
choice between different members of (the W-Set, hence) the C-Set. It
appears that ordinary \emph{wh}"=phrases invariably are associated with
a non-trivial W-Set, that is, a set with more than one member. (Hence,
the C-Set of any \emph{wh}"=clause they occur in has more than one
member.) But exclamative predicates are not concerned with the
possibility of choice. Rather, they induce a (speaker's)
presupposition that some member(s) of the C-Set be true. Exclamative
\emph{wh}"=phrases, in turn, appear to always denote a singleton set;
and I suggest that is why they do not occur with an interrogative
matrix. This may be illustrated with a predicate such as  \textit{tell} that can
be exclamative, interrogative or declarative:
\begin{exe}
\ex
\label{ex:12-42}
\begin{xlist}   
\ex
\label{ex:12-42a}
She did not tell me what fool had called her.
\ex
\label{ex:12-42b}
She did not tell me whether this fool or that fool or  \ldots{}  had called her.
\ex
\label{ex:12-42c}
There is/""are some x, x a fool, such that she did not tell me that x had called her.
\ex
\label{ex:12-42d}
She did not tell me what a fool had called her.
\ex
\label{ex:12-42e}
She did not tell me that such a fool had called her.
\ex
\label{ex:12-42f}
There is a certain extraordinary amount \textit{a} such that she did
not tell me that some person who is a fool to degree \textit{a} had called her.
\end{xlist}  
\end{exe}
Here, (\ref{ex:12-42a}) is ambiguous between an interrogative reading, which can be
paraphrased by (\ref{ex:12-42b}), and an exclamative reading, which can be paraphrased by
(\ref{ex:12-42c}). But (\ref{ex:12-42d}) with the exclamative \emph{wh}"=phrase \textit{what a fool} can only be
paraphrased by (\ref{ex:12-42e}). The message in (\ref{ex:12-42d}) is not that there are several
fools such that one (or more) of them has called her, but that some person who
called her is a terrible fool. Thus, a slightly more articulate paraphrase may
look like (\ref{ex:12-42f}).

If considerations along these lines are correct, it may be possible to
explain (\ref{ex:12-39})--(\ref{ex:12-41}) upon (\ref{ex:12-15ii}), i.e., by relying on a relation
between \textit{was} and the embedded clause, rather than its \emph{wh}"=phrase. In any case, the initial \emph{wh}"=expression in a \textit{w- \ldots{} w}"=construction must have properties of an ordinary \emph{wh}"=phrase in that it induces a non"=trivial W-Set whose cardinality is incompatible with that of (the exclamative \emph{wh}"=phrase in) the embedded clause.
This is natural upon (\ref{ex:12-15i}). Upon (\ref{ex:12-16i}), it seems unexpected for a
\textsqe{scope marker} to have a property like this.

\section{On (\ref{ex:12-9iii})}
\label{sec:12-10}

According to (\ref{ex:12-16ii}), the embedded clause is a complement of the
matrix. (\ref{ex:12-16i}) is intended to imply that, semantically, it cannot be an
interrogative clause. How, then, can the empirical generalization
(\ref{ex:12-9iii}) follow from (\ref{ex:12-16})? Specifically, the question is how
to account for (\ref{ex:12-10c}) and (\ref{ex:12-10d}), repeated below.

From Section~\ref{sec:12-8} we know that an in situ \emph{wh}"=phrase in a multiple interrogation
structure is subject to similar restrictions as the \textit{w- \ldots{}  w}"=construction;
cf.\ (\ref{ex:12-36}). Now observe in situ \emph{wh}"=phrases in embedded F2 clauses:
\begin{exe}
\ex
\label{ex:12-43}
\begin{xlist}
\ex[ ]{
\label{ex:12-43a}
\gll
Es ist egal, \textsc{wer} der Meinung war, dort hätte \textsc{wer} gewohnt \\
it is no.difference who of.the opinion was there had who resided \\}
\ex[*]{
\label{ex:12-43b}
\gll
Es ist egal, \textsc{wer} der Meinung war, \textsc{wer} hätte dort gewohnt \\
it is no.difference who of.the opinion was who had there resided \\}
\end{xlist}
\end{exe}
\addlines
Even though (\ref{ex:12-43a}) is not a model of beauty, relating the embedded
postverbal \textit{wer} to the matrix \textit{wer} is possible. The same is
strictly impossible with  \textit{wer} in (\ref{ex:12-43b}). Thus, the preverbal position
in an embedded F2 clause, which can be considered a \textsqe{\textsc{Comp} position} in
the sense of (\ref{ex:12-16i}), cannot be related to a \emph{wh}"=phrase in the
matrix \textsc{Comp}. This fact may be sufficient to account for (\ref{ex:12-44}) (=
(\ref{ex:12-10c})):
\begin{exe}
\ex[*]{
\label{ex:12-44}
\gll
Was  glaubt sie, auf wessen Hilfe kann man sich verlassen? \\
what thinks she on whose help can one self rely \\}
\end{exe}
From (\ref{ex:12-16})'s perspective, the problem with (\ref{ex:12-45}) (= (\ref{ex:12-10d})) is very different:
\begin{exe}
\ex[*]{
\label{ex:12-45}
\gll
Was glaubt sie, auf wessen Hilfe sich verlassen zu können?\\
what thinks she on whose help self rely to can \\}
\end{exe}
There are relative clauses such as (\ref{ex:12-46a}) that involve an initial
infinitival clause, and some speakers accept similar \emph{wh}"=interrogative clauses (\ref{ex:12-46b}); cf.\ \citet{Trissler1991}.
\begin{exe}
\ex
\label{ex:12-46}
\begin{xlist}
\ex
\label{ex:12-46a}
\gll
(Das ist ein Umstand) [[den\textsubscript{\textit{i}} [t\textsubscript{\textit{i}} zu berücksichtigen]] man nicht vergessen sollte] \\
(that is a circumstance) {\hphantom{[]}which} {} to heed one not forget should \\
\ex
\label{ex:12-46b}
\gll
(Sie wollte wissen) [[[welchen Umstand]\textsubscript{\textit{i}} [t\textsubscript{\textit{i}} zu berücksichtigen]] man nicht vergessen sollte] \\
(she wanted know) {\hphantom{[[]}which} circumstance {} to heed one not forget should \\
\end{xlist}
\end{exe}
Infinitival clauses like these are peculiar in that they are
pied-piped relative or \emph{wh}"=interrogative phrases. That is, the
\textsqe{\emph{wh}"=feature} that originates from the relative/""interrogative
word contained in their \textsc{Comp} position cannot rest in that \textsc{Comp} but
percolates up to the infinitival clause.  For some reason, infinitival
clauses in German\il{German} never tolerate a relative/""interrogative phrase in
their \textsc{Comp}. That is, the phrases \textit{welchen Umstand} in (\ref{ex:12-46b})
and \textit{auf wessen Hilfe} in (\ref{ex:12-45}) are not \emph{wh}"=phrases in the
technical sense; only their mother constituents are. Hence, the
infinitival clause in (\ref{ex:12-45}) does not have a \emph{wh}"=phrase in its
\textsc{Comp}, thus violating (\ref{ex:12-16i}).

\section{Relative clause constructions}
\label{sec:12-11}

McDaniel reports on Romani\il{Romani} relative clause constructions (\ref{ex:12-47}) that are
remarkably similar to interrogative \textit{w- \ldots{}  w}"=constructions. She even found a
speaker of German\il{German} who accepted the construction in (\ref{ex:12-48}) (cf.\ \citealt[189, note 8]{McDaniel1986}).\footnote{%
	McDaniel documents and discusses some further kinds of \textsqe{partial \emph{wh}"=movement} in Romani\il{Romani} and in variants of German\il{German} that I have no independent information
about; see \citet{McDaniel1986} and \citet{McDaniel1989}. According to \citet{McDaniel1995}, approximately the same range of constructions can be found
in child English\il{English}.%
}
\begin{exe}
\judgewidth{\%}
\ex
\label{ex:12-47}
\begin{xlist}
\ex
\label{ex:12-47a}
\gll
Ake o \'chavo so mislinav kas i Ar\`{\i}fa dikhl\^a \\
                 here the boy what/""that I.think whom the Arifa saw \\
\exsource{113 (33a)}{McDaniel1986}
\ex
\label{ex:12-47b}
\gll
Ake o \'chavo so mislinav so o Dem\`{\i}ri mangol ka\c{c}a te khel\^av \\
                 here the boy what/that I.think that/what the Demir wants with.whom to I.dance \\
\exsource{135 (59a)}{McDaniel1986}
\end{xlist}
\ex[\%]{
\label{ex:12-48}
\gll
Das  ist der Junge, mit dem ich glaube, mit dem Hans spricht \\
that is  the boy with whom I believe with whom Hans speaks \\
\exsource{182 (125b)}{McDaniel1986}}
\end{exe}
Certain relative clause constructions in Irish evidence the same
structural properties:
\begin{exe}
\ex
\label{ex:12-49}
\gll
(an doras) aL   mheasann sibh [aN bhfuil an eochair ann]\\
\hspaceThis{(}the door C\textsubscript{gap} think you {\hphantom{[}C\textsubscript{pron}} is the key in.it \\
\exsource{19 (49)}{McCloskey1979}
\end{exe}
The particle \textit{aN} introduces clauses containing a resumptive
pronoun; thus, the clause in brackets could be used as a so-called
\textsqe{indirect} relative clause by itself. The particle \textit{aL} usually
introduces clauses containing a gap/""trace (in various extraction
constructions, e.g. in \textsqe{direct} relative clauses). Cf.\ also \citet[44; 168]{McCloskey1979}. But evidently, there is no \isi{NP} or PP gap: the matrix
predicate (\textit{think}) does not combine with a non-propositional
complement \isi{NP}/""PP that could serve as a trace related to the antecedent
\isi{NP} in (\ref{ex:12-49}).  Exactly the same consideration applies to (\ref{ex:12-47}) and
(\ref{ex:12-48}). Hence, the traditional idea (\ref{ex:12-15}) is unable to accommodate
constructions like these.

\sloppy
\printbibliography[heading=subbibliography,notkeyword=this]
\refstepcounter{mylastpagecount}\label{chap-w-w-end}
\end{document}
