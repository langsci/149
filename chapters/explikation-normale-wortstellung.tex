%% -*- coding:utf-8 -*-

\documentclass[output=paper]{langsci/langscibook}
\author{Tilman N. Höhle}
%\title{Explikationen für  \noexpand\glqq{}normale Betonung\noexpand\grqq{} und \noexpand\glqq{}normale Wortstellung\noexpand\grqq{}}
\title{Explikationen für  „normale Betonung” und „normale Wortstellung”}
\abstract{}
\maketitle
%\rohead{\thechapter\hspace{0.5em}Explikationen für  ``normale Betonung'' und ``normale Wortstellung''} % Display short title
\renewcommand*{\thefootnote}{\fnsymbol{footnote}}
\ChapterDOI{10.5281/zenodo.1169667}
\begin{document}
\label{chap-normale-Betonung}
\selectlanguage{german}

\setcounter{section}{-1}

\setcounter{footnote}{4}

\footnotetext{%
	\emph{Anmerkung der Herausgeber:} Diese Arbeit erschien erstmals in Abraham Werner (Hrsg.). 1982. \emph{Satzglieder im Deutschen. Vorschläge zur syntaktischen, semantischen und pragmatischen Fundierung} (Studien zur deutschen Grammatik 15), 75-153. Tübingen: Narr. Der vorliegende Wiederabdruck ist im Wesentlichen textuell unverändert; jedoch haben wir ihn dem einheitlichen Bandformat angepasst, sowie im Interesse der Leserfreundlichkeit die Literaturverweise L1, L2, {\ldots} im Text durch die entsprechenden Autorenverweise ersetzt, die Fußnoten durchgezählt (\dash Fn.\,12a/12b des Originals sind jetzt Fn.\,13/14, die weiteren Fn.\ entsprechend versetzt) und einen Zählfehler (Sprung von Bsp.\,(169) zu (171)) sowie weitere kleine Mängel beseitigt. -- Die als "`\citealt{Hoehle79c}"' verzeichnete Arbeit blieb unrealisiert.%
}

\setcounter{footnote}{0}

\section[Einleitung]{Einleitung\protect\footnote{Dieser Aufsatz ist eine überarbeitete Fassung von \citet{Hoehle79a}. Eine frühere (und kürzere) Fassung der Abschnitte~\ref{sec:3-1} und \ref{sec:3-2} ist als \citet{Hoehle79b} erschienen. (Was ich hier als \textsq{strukturell normal} bezeichne, habe ich in diesen früheren Fassungen \textsq{grammatisch normal} genannt.)\\
\indent Für Diskussionen zu hier behandelten Themen danke ich Manfred Bierwisch, Nomi
Erteschik"=Shir, Hans"=Peder Kromann, Oddleif Leirbukt, Jürgen Le\-nerz und Marga Reis. Le\-nerz und
Reis verdanke ich eine Reihe von spezifischen Anregungen und Hinweisen für Verbesserungen.
}}\label{sec:3-0}

\setcounter{footnote}{0}
\renewcommand*{\thefootnote}{\arabic{footnote}}

\largerpage
In der \isi{Topologie} (Wortstellungslehre) des Deutschen\il{Deutsch} kann man grob zwei Bereiche unterscheiden: Gewisse Satzbestandteile sind strengen topologischen Regularitäten unterworfen; dies gilt besonders für verbale Elemente und \textsq{Konjunktionen}, aber in hohem Maße auch für die Bestandteile von Nominal"=, Präpositional"= und Adjektiv/""Adverbialphrasen. Insofern sind die Fakten, die in diesem Bereich zu beschreiben sind, weitgehend unstrittig. Die Positionsmöglichkeiten verschiedener Elemente -- etwa NPs und PPs --  innerhalb des Mittelfeldes relativ zueinander scheinen dagegen einigermaßen undurchsichtig. In vielen Fällen kann man weder sagen, daß eine bestimmte \isi{Wortfolge} ausgeschlossen oder die einzig mögliche ist, noch, daß zwischen zwei verschiedenen möglichen Wortfolgen völlig \textsq{freie Variation} besteht. Vielmehr wird oft gesagt, eine gewisse \isi{Wortfolge} sei bei \textsq{normaler} \isi{Betonung} unmöglich, sie bedinge eine \textsq{kontrastive} oder \textsq{emphatische} \isi{Betonung}; ähnlich wird nicht selten von \textsq{normaler} \isi{Wortstellung} gesprochen, im Unterschied etwa zu einer \textsq{kontrastiven} oder \textsq{emphatischen} Stellung. Damit ist explizit oder implizit oft die Ansicht verbunden, daß, diese Unterscheidung zwischen \textsq{Normalität} und \textsq{Nicht"=Normalität} (a) innerhalb des syntaktischen Beschreibungssystems als solche zum Ausdruck kommen muß, oder gar (b) daß die \textsq{nicht"=normalen}, \textsq{markierten} Fälle gar nicht zum primären Gegenstandsbereich der Syntax gehören.

Nach meiner Kenntnis gibt es jedoch kaum irgendwo befriedigende Explikationen der Unterscheidung zwischen \textsq{normaler} und \textsq{nicht"=normaler} \isi{Betonung} und \isi{Wortstellung}. Daher fehlen befriedigende Kriterien zu ihrer gegenseitigen Abgrenzung, verschiedene Autoren bewerten gegebene Beispiele verschieden, und die Relevanz einer solchen Unterscheidung ist keineswegs offensichtlich. Damit ist auch der Gegenstandsbereich und mögliche Aufbau der Syntax in diesem Bereich der \isi{Topologie} unklar, und jeder Streit darüber ist müßig, solange diese Begriffe nicht geklärt sind.

Um künftigen Untersuchungen zur \isi{Topologie} in dieser Hinsicht den Weg zu ebnen, versuche ich in diesem Aufsatz zu klären, inwieweit hier überhaupt sprachwissenschaftlich relevante Begriffe vorliegen, die darüber hinaus essentiell sind in dem Sinne, daß sie nicht auf unabhängig gesicherte Begriffe definitorisch zurückgeführt werden können, und im Zusammenhang damit, wie diese Begriffe zu explizieren sind. Dabei wird sich u.\,a.\ zeigen, daß man streng zwischen intuitiven Begriffen von \textsq{stilistischer Normalität/""Unmarkiertheit} und nicht"=intuitiven Begriffen von \textsq{struktureller Normalität/""Unmarkiertheit} unterscheiden muß. Während die intuitiven Normalitätsbegriffe essentiell und offensichtlich relevant sind, sind die strukturellen Normalitätsbegriffe nicht essentiell und von bestenfalls zweifelhafter Relevanz.

\section{Normalbetonung}
\label{sec:3-1}

In den folgenden Abschnitten möchte ich zeigen, daß der Begriff
\textsq{Normalbetonung} der Klärung bedarf (\ref{subsec:3-1-1}); daß geläufige
Klärungsversuche unklar, inkonsistent oder empirisch inadäquat sind
(\ref{subsec:3-1-2}), und daß, eine adäquate Explikation in spezifischer Weise von
pragmatischen Begriffen Gebrauch machen muß (\ref{subsec:3-1-3}).

\subsection{Klärungsbedürftigkeit des Begriffs}
\label{subsec:3-1-1}

Die Extension des Begriffs \textsq{Normalbetonung} ist immer dann
unproblematisch, wenn sie rein formal, etwa durch eine einfache
Nominaldefinition, festgelegt ist. So spricht \zb \citet[521ff]{Bartsch76} von "`standard \isi{intonation} pattern"'; dies liegt immer dann vor, wenn "`the \isi{intonation} peak lies on the noun of the last
term of the sentence"'. Das Problem dabei ist nur, warum eine solche \isi{Betonung} als \textsq{Standard}"= oder \textsq{Normalbetonung}
ausgezeichnet werden soll. Welches empirische Interesse besteht
daran, die Betonungen in \eqref{ex:3-1} im Unterschied zu denen in (\ref{ex:3-2}) als
\textsq{normal} anzusehen (Kursivdruck steht für \isi{Betonung})?
\eal \label{ex:3-1}
\ex
\label{ex:3-1a}
es heißt, daß die Theorie den \textit{Fach}leuten gefallen hat
\ex
\label{ex:3-1b}
~\hphantom{es heißt} daß der Junge dem \textit{Pfarrer} begegnet ist
\ex
\label{ex:3-1c}
~\hphantom{es heißt} daß der \textit{Junge} getanzt hat
\ex
\label{ex:3-1d}
~\hphantom{es heißt} daß der \textit{Junge} kommt
\zl
\eal \label{ex:3-2}
\ex
\label{ex:3-2a}
es heißt, daß die Theorie den Fachleuten ge\textit{fallen} hat
\ex
\label{ex:3-2b}
~\hphantom{es heißt} daß der Junge dem Pfarrer \textit{begegnet} ist
\ex
\label{ex:3-2c}
~\hphantom{es heißt} daß der Junge \textit{getanzt} hat
\ex
\label{ex:3-2d}
~\hphantom{es heißt} daß der Junge \textit{kommt}
\zl
Ähnlich uninformativ sind \zb Ausführungen bei \citet[79ff]{Kiparsky66}. Er\linebreak schreibt zunächst (S.\,79):
\begin{quotation}
  "`Wir berücksichtigen durchaus nur die normale, affektfreie \isi{Betonung}
  von Sätzen. Emphatische oder \isi{kontrastive Betonung} lassen wir
  konsequent beiseite. Entgegen der \isi{Normalbetonung}, die strengen
  Regeln unterliegt, kann diese ein beliebiges Wort des Satzes treffen
  und stellt daher keine besonderen Probleme."'
\end{quotation}
Wie man im Zweifelsfall \textsq{normale} von \textsq{emphatischer}
oder \textsq{kontrastiver} \isi{Betonung} empirisch unterscheiden kann,
führt er nicht aus. Aus seinen für \textsq{Normalbetonung} konzipierten Regeln kann man nur entnehmen, daß bei mehrstelligen Verben die \isi{Betonung} des letzten Substantivs im Allgemeinen \textsq{normal} sein soll. Allerdings möchte er mit seinem Regelwerk auch Fälle wie
(\ref{ex:3-3}) erfassen (S.\,91 (\ref{ex:3-51})):
\ea
\label{ex:3-3}
der Arzt wird den Patienten unter\textit{suchen}
\z
Dies ist überraschend, denn tatsächlich kann sein Regelsystem diese
\isi{Betonung} nicht generieren: Dazu müßte \textit{der Arzt wird den Patienten} 
eine \isi{Konstituente} sein (S.\,92 (53)). Aufgrund seiner Regeln (F2) und
(T1) (S.\,80f) ist dies jedoch ausgeschlossen. Welche empirischen
Gegebenheiten können Kiparsky veranlaßt haben, entgegen seinem eigenen
Regelapparat Betonungen wie in (\ref{ex:3-3}) als \textsq{normal} zu betrachten?
Da keine inhaltliche Bestimmung von \textsq{normaler} vs.\
\textsq{nicht"=normaler} \isi{Betonung} angegeben ist, kann man diese Frage
nicht einmal rational diskutieren.

Wie unklar die Identifizierung der \textsq{Normalbetonung} ist, sieht
man deutlich an den folgenden Zitaten. So schreibt Kirkwood:
\ea \label{ex:3-4}
"`In (3) neither object carries special emphasis. In (4)
\begin{enumerate}
\setlength{\itemindent}{2em}
\item[(3)] Er gab dem Kind das Buch
\item[(4)] Er gab das Buch dem Kind
\end{enumerate}
the dative object carries contrastive stress\ \ldots{}"' \citep[86]{Kirkwood69}
\z
(Ebenso \zb in \citealt{Griesbach1961a} (IV): 85) Dabei nimmt er offensichtlich an, daß jeweils das letzte Nomen voll betont ist. Dies aber ist \zb für Bartsch, wie wir gesehen haben, gerade
ein Musterfall für \textsq{standard \isi{intonation} pattern}. (Andererseits macht sie doch zwischen solchen Beispieltypen eine Unterscheidung, die -- unter einer adäquaten Explikation dieser Begriffe -- der Unterscheidung zwischen normaler und kontrastiver \isi{Betonung} entspricht; vgl.\ \ref{subsubsec:3-1-4-4}). Und für Lenerz haben nicht nur seine parallelen Beispiele (2b) und (3a) \textsq{Normalbetonung}; er schreibt
ausdrücklich:
\ea \label{ex:3-5}
\begin{enumerate}
\setlength{\itemindent}{2em}
\item["`{}(2) a)] Ich habe dem \textit{Kassierer} das Geld gegeben
\item[(2) b)] Ich habe das Geld dem \textit{Kassierer} gegeben
\item[(3) a)] Ich habe dem Kassierer das \textit{Geld} gegeben
\end{enumerate}
In (2) a) ist bei der Abfolge IO DO das IO als \isi{Rhema} durch den normalen Satzakzent hervorgehoben."' \citep[147 (43)]{Lenerz77}
\z
\addlines[2]
Dabei spielt die Unterscheidung von \textsq{normalem} vs.\
\textsq{nicht"=normalem Satzakzent} bei Lenerz' Untersuchungen zu
Wortstellungsregularitäten eine zentrale Rolle; vgl.\ z.\,B.:
\ea
\label{ex:3-6}
"`Dabei ist (5) b) mit normaler \isi{Intonation} zu lesen; bei emphatischer
\isi{Betonung} des DO ist ein entsprechender Satz
\begin{enumerate}
	\setlength{\itemindent}{2em}
\item[(5) b)] ?*Ich schickte an einen Bewerber den \textit{Frage}bogen
\item[c)] Ich schickte an einen Bewerber die UNTERLAGEN (und nicht \\
\item [] den ABLEHNUNGSBESCHEID)
\end{enumerate}
möglich."' \citep[66f]{Lenerz77}
\z
In Abschnitt~\ref{subsubsec:3-1-4-3} von \citet{Lenerz77} weist Lenerz darauf hin, daß
etliche topologische Regularitäten "`nur in Sätzen mit normaler \isi{Intonation} gelten"' (S.\,34) und daß bei \textsq{nicht"=normaler} \isi{Intonation} z.\,T.\ andere Regularitäten gelten; dies ist für ihn ein Grund, "`auch im folgenden Sätze mit Emphase"= oder Kontrastintonation weitgehend aus[zu]klammern"' (S.\,34). Bei der Unterscheidung dieser 3 Akzent"= und Intonationsarten stützt er sich auf \citet{Bierwisch66}, den wir unten kritisieren werden.

\subsection{Verschiedene Explikationsversuche}
\label{subsec:3-1-2}

Wir sehen, daß in vielen Arbeiten zur \isi{Topologie} Annahmen über \textsq{Normalbetonung} eine wichtige Rolle spielen und daß dieser Begriff klärungsbedürftig ist; man sucht nach einer empirisch signifikanten Explikation. Wir wollen kurz einige Explikationsversuche betrachten, die sich da und dort in der Literatur finden.

Gelegentlich hört oder liest man etwas wie (\ref{ex:3-7}):
\ea
\label{ex:3-7}
Ein Satz mit \isi{Normalbetonung} ist in allen Kontexten möglich.
\z
Es ist klar, daß eine Behauptung wie (\ref{ex:3-7}) abwegig ist: Es gibt
überhaupt keinen Satz, der unter Wahrung konversationeller Maximen in beliebigen Kontexten geäußert werden kann. Eine Variante von (\ref{ex:3-7})
ist (\ref{ex:3-8}):
\begin{exe}
\ex \label{ex:3-8}
"`Unmarkierte Sätze: [\ldots{}] Normaler Satzakzent und damit keine
erkennbaren Beschränkungen für Vorgänger- und Folgesätze"'
\citep[38]{Altmann76}\footnote{\label{fn:3-1}%
	Darauf folgt der Zusatz: "`Aus Gründen der leichteren
  Unterscheidbarkeit möchte ich teilidentische Sätze bei den Vorgängersätzen ausschließen''. Falls
  dieser Zusatz inhaltlich relevant sein soll, kann er wohl nur die Aufhebung der zuvor gemachten
  Aussage beinhalten.%
} 
\end{exe}
Selbst wenn man (\ref{ex:3-8}) in möglichst konstruktiver Weise interpretiert,
ist dies keine adäquate Explikation des üblichen Sprachgebrauchs. So
liegt nach Ansicht aller Autoren in (\ref{ex:3-9a}) normale Betonung\is{Normalbetonung} und normale
\isi{Wortstellung} vor; für (\ref{ex:3-9b}) sind die Ansichten geteilt.
\eal \label{ex:3-9}
\ex
\label{ex:3-9a}
er will seinem Freund das \textit{Auto} schenken
\ex
\label{ex:3-9b}
er will das Auto seinem \textit{Freund} schenken
\ex
\label{ex:3-9c}
Karl hat gestern einen \textit{Porsche} gekauft
\zl
Soweit nicht weitergehende kontextuelle Informationen gegeben sind,
ist jedoch nur (\ref{ex:3-9b}), nicht aber (\ref{ex:3-9a}) ein möglicher Nachfolgersatz
für (\ref{ex:3-9c}). (Wir kommen auf das Beispiel in \ref{subsubsec:3-1-4-3} zurück).

Ein anderer geläufiger Explikationsversuch ist (\ref{ex:3-10}):
\ea
\label{ex:3-10}
Nur Sätze mit \isi{Normalbetonung} sind als Textanfänge möglich.
\z
Wenn man "`Textanfänge"' wörtlich nimmt, ist dieses Kriterium unbrauchbar, denn in empirisch
vorfindlichen Textanfängen finden sich Sätze mit allen denkbaren Betonungsmustern; demnach wären
alle Betonungen \textsq{Normalbetonungen}. Vielleicht könnte man eine Theorie über \textsq{normale
  Textanfänge} entwickeln, auf die (\ref{ex:3-10}) sich beziehen könnte; solange eine solche Theorie
nicht gegeben ist, ist (\ref{ex:3-10}) nicht die gesuchte Explikation.\footnote{\label{fn:3-2}%
  Darüber hinaus vermute ich, daß Proponenten einer Theorie normaler Textanfänge ein Beispiel wie (i) nicht als \textsq{normalen Textanfang}
  \ea
  \label{ex:3-fn2}
  dein Karatelehrer hat ihr ein \textit{Auto} geschenkt
  \z
  zulassen würden; der Satz hat aber nach allgemeiner Ansicht normale
  \isi{Betonung}. \textsq{Als normaler Textanfang möglich} wäre also nur eine
  hinreichende, keine notwendige Bedingung für normalbetonte
  Sätze. Vgl.\ Fn.\,\ref{fn:3-25}.

  Abgesehen davon bliebe natürlich fraglich, wieso man gerade die bei
  Textanfängen zu beobachtenden Betonungen als \textsq{normal} in
  irgendeinem relevanten Sinne betrachten sollte. Soweit es eine
  plausible Theorie über \textsq{normale Textanfänge} gibt (cf.\ \citealt{Clark77}),
  ist diese keine spezielle \textsq{Texttheorie}, sondern aus derselben
  pragmatischen Theorie deduzierbar, auf die ich die Explikation von
  \textsq{Normalbetonung} gründen möchte; \dash die Explikation von
  \textsq{Normalbetonung} mit Hilfe \textsq{normaler Textanfänge} wäre
  (selbst wenn sie möglich wäre) überflüssig.%
}

Ein ähnlicher Explikationsversuch ist (\ref{ex:3-11}):
\ea
\label{ex:3-11}
Nur Sätze mit \isi{Normalbetonung} können als Diskursanfänge dienen.
\z
Wenn mit "`Diskurs"' alltägliche Konversation gemeint ist, ist (\ref{ex:3-11})
wiederum keine Explikation des tradierten Begriffs. Unter geeigneten
Umständen kann ein Gespräch \zb ohne weiteres mit der Äußerung von
(\ref{ex:3-12}) eingeleitet werden, obwohl dies nach allgemeiner Ansicht keine
\isi{Normalbetonung} ist; nach (\ref{ex:3-11}) wären alle denkbaren Betonungen
\textsq{normal}.
\ea
\label{ex:3-12}
weißt du schon, daß \textit{Karl} die Scheune angezündet hat?
\z
\addlines
\citet[50ff]{Schmerling76} diskutiert einen Explikationsversuch von Stockwell. Sie gibt folgendes Zitat von ihm:
\begin{quotation}
  "`When utterances are elicited from an informant, certain \isi{intonation}
  patterns regularly occur. These are citation patterns. They can be
  re-elicited from any number of informants with almost perfect
  consistency [\ldots] Such patterns may, for convenience, be labelled
  "`normal"'."' \citep[50]{Schmerling76}
\end{quotation}
Schmerling zeigt, daß es zumindest für eine Reihe einstelliger Verben (solche vom Typ (1d)) äußerst
zweifelhaft ist, ob wirklich konsistente Befragungsergebnisse erzielt würden. Mit Recht weist sie
darauf hin, daß es darüber hinaus ganz unklar ist, inwiefern kontextlosen
Zitatbetonungen\footnote{\label{fn:3-3}%
  Diesen von Stockwell benutzten Ausdruck behalte ich bei, obwohl er inadäquat ist: Die in Rede
  stehenden Intonationsmuster haben mit \textsq{Zitaten} im üblichen Sinne des Wortes nichts zu tun;
  \textsq{elicitation patterns} wäre ein angemessenerer Ausdruck.%
}
irgendein sprachwissenschaftliches Interesse zukommen sollte: Natürliches Sprechen, insbesondere
natürliches Betonen, vollzieht sich in Kontexten. Mit welcher Berechtigung sollte dann ein offenbar
theoretisch wichtiger Begriff wie \textsq{Normalbetonung} auf die Beobachtung völlig unnatürlichen
Verhaltens, nämlich des kontextlosen \textsq{Zitierens}, gegründet werden? Zumindest müßten Gründe
für ein solches Vorgehen explizit diskutiert werden.\footnote{\label{fn:3-4}%
  Schmerling entwickelt eine Hypothese, wie die konsistenten kontextlosen Zitatbetonungen zustande
  kommen. Die Fakten, auf die sie sich dabei beruft, sind aber offenbar komplizierter, als sie
  erkannt hat, cf.\ \citet[956]{Bean78}. -- Unabhängig davon halte ich Schmerlings ad
  hoc"=Hypothesen, die auf einer angeblichen \textsq{Bedeutungslosigkeit} von \textsq{Zitaten},
  \dash isolierten Sätzen ohne natürlichen Kontext beruhen, für prinzipiell falsch, da es eine
  allgemeine \textit{Erklärung} für diese Fälle gibt; vgl.\ \ref{subsubsec:3-1-4-4}.%
}
Die Berufung auf \textsq{Zitate} zur Etablierung von \textsq{Normalbetonung} erscheint tatsächlich so abwegig, daß man vermuten möchte, daß es Stockwell hier eigentlich nicht um die Einführung des Begriffs \textsq{Normalbetonung} geht, sondern daß er einen prätheoretischen Begriff davon bereits hat und \textsq{Zitatbetonungen} lediglich als ein empirisches Korrelat dafür benutzt. In jedem Fall können \textsq{Zitate} keine Explikation von \textsq{Normalbetonung} geben; bestenfalls bieten sie einen (mehr oder weniger zuverlässigen) \textsq{Test}, um normalbetonte Sätze als solche zu erkennen. Wir werden sehen, daß sich unter einer adäquaten Explikation verstehen läßt, was \textsq{Zitatbetonungen} mit \isi{Normalbetonung} zu tun haben.

Es ist bemerkenswert, daß alle diese Explikationsversuche in irgendeiner Weise auf Kontexte abheben. Inhaltlich sind damit, wie wir später sehen werden, solche Explikationsversuche wie (\ref{ex:3-13}) eng verwandt:
\begin{exe}
	\ex \label{ex:3-13}
	Bei \isi{Normalbetonung} wird kein Teil des Satzes besonders hervorgehoben;
bei allen anderen Betonungen ist der betonte Teil hervorgehoben.
\end{exe}
Das Problem bei dieser Formulierung ist, daß nicht klar ist, was unter "`\isi{Hervorhebung}"' zu verstehen ist. Ist \zb in (\ref{ex:3-9b}) \textit{Freund}
\textsq{besonders hervorgehoben}? Kirkwood scheint dieser Meinung zu
sein, vgl.\ (\ref{ex:3-4}); aber inwiefern ist dann \textit{Auto} in (\ref{ex:3-9a}) nicht
\textsq{besonders hervorgehoben}? Immerhin kann man zu (\ref{ex:3-9a}) die
Fortsetzung (\ref{ex:3-14}) bilden,
\ea
\label{ex:3-14}
\ldots{} und nicht die \textit{Melk}maschine
\z
und da möchte man \textit{Auto} und \textit{Melkmaschine} wohl
intuitiv als \textsq{hervorgehoben} betrachten; gleichwohl gilt (\ref{ex:3-9a}) 
gemeinhin als \textsq{normalbetont}. Etwas ähnliches wie (\ref{ex:3-13}) hat
Stockwell offenbar im Sinn (und das stützt die Vermutung, daß er
durchaus eine inhaltliche, von \textsq{Zitaten} unabhängige
Vorstellung von \isi{Normalbetonung} hat). In dem ausgelassenen Teil des
oben angeführten Zitats steht nämlich:
\begin{quotation}
  ,,They [= citation patterns, TNH] are also heard in normal discourse,
  but no sampling or statistical work of any sort has been done to
  indicate whether such patterns are the most frequent ones or not. It
  is true, however, that where they are observed in normal discourse,
  they can be shown by questioning to carry no additional component or
  differential meaning beyond that which is assignable to the
  \isi{segmental} morphemes alone."'\\
\citep[50; \isi{Hervorhebung} von mir]{Schmerling76}
\end{quotation}
Das Problem ist hier natürlich, zu klären, inwiefern nicht"=normal betonte Sätze \textsq{additional components or differential meaning}
mit sich führen, durch die sie sich von normalbetonten Sätzen unterscheiden.

Zu den bisher besprochenen Explikationsversuchen gibt es etliche
Varianten; soweit ich sehe, leiden sie alle unter ähnlichen Defekten
wie die hier zitierten. Ich möchte jetzt zu zwei völlig anders
gearteten Bestimmungen kommen.

\citet[151ff]{Bierwisch66} hält es für "`notwendig, wenigstens
drei Arten der \isi{Hervorhebung} zu unterscheiden"' (S.\,151 Z.\,3). Erstens "`hat
jeder Satz automatisch einen Primärakzent"' (S.\,151 Z.\,5), einen "`normalen
\isi{Hauptakzent} [\ldots{}]"' (S.\,151 Z.\,14-15). Eine inhaltliche Bestimmung gibt er
nicht. Man könnte aber (wenn nicht das \textit{wenigstens} S.\,151 Z.\,3 wäre) versuchen, aufgrund der Bestimmung der beiden anderen \textsq{Hervorhebungsarten} ex negativo auch den \textsq{normalen Hauptakzent} zu bestimmen.
\begin{exe}
	\settowidth\jamwidth{(Bierwisch 1966: 151 Z.\,17-19)}
	\ex {\label{ex:3-15}
"`Ein zweiter Typ der \isi{Hervorhebung}, den wir Kontrast nennen wollen,
ergibt sich, wenn zwei oder mehr Sätze mit parallelen Konstituenten,
die aber mit verschiedenen Morphemen besetzt sind, aufeinander folgen."'} \jambox{\citep[151 Z.\,17-19]{Bierwisch66}}
\end{exe}
\addlines
Sein Beispiel ist (\ref{ex:3-16}) (= \citealt[151 (31c)]{Bierwisch66}); einschlägig wäre wohl auch der Diskurs (\ref{ex:3-17}). Es ist nicht klar, wie Beispiele wie (\ref{ex:3-18}) unter
\ea
\label{ex:3-16}
\textit{Klaus} wohnt in \textit{München} und \textit{ich} wohne in \textit{Berlin}
\z
\eal \label{ex:3-17}
\ex
\label{ex:3-17a}
A: \textit{wer} wohnt \textit{wo}?
\ex
\label{ex:3-17b}
B: \textit{Klaus} wohnt in \textit{München}
\zl
\ea
\label{ex:3-18}
Karl hat dem \textit{Kind} \textit{Haschisch} gegeben
\z
dieser Terminologie zu rubrizieren sind. Ein solcher Satz kann ohne jeden sprachlichen Kontext geäußert werden, setzt also keinen \textsq{Kontrast} zu parallelen, aber \isi{morphologisch} verschiedenen Konstituenten voraus, teilt aber mit (\ref{ex:3-16}) und (\ref{ex:3-17}) die Eigenschaft, daß mehr als 1 \isi{Konstituente} voll betont ist. Nach meinem Eindruck ist
Bierwischs \textsq{kontrastive} \isi{Betonung} überhaupt nur durch diese Eigenschaft von Sätzen mit \textsq{normalem Hauptakzent} unterschieden. Es spricht nichts dagegen, solche Unterschiede
terminologisch auszuzeichnen, aber ich sehe andererseits nicht, welche Fakten dafür sprechen; die einfache Feststellung, daß \textit{n} Hauptakzente gegeben sind, mit \textit{n} = 1 bzw.\ \textit{n} $>$ 1, scheint mir völlig zureichend. (Bierwischs Motiv für die terminologische Unterscheidung
allerdings scheint klar: Er geht davon aus, daß genau 1 \textsq{normaler Hauptakzent} aufgrund von Kiparskys \citep{Kiparsky66} und Heidolphs \citep{Heidolph70} Regeln festgelegt ist, während das für mehr als 1 \textsq{Hauptakzent} nicht gilt. Aber dies ist ein rein formaler Gesichtspunkt ohne inhaltliches Interesse. -- Warum Kiparsky für \textsq{Normalbetonung} genau 1 \isi{Hauptakzent} annimmt, läßt sich allerdings unter einer adäquaten Explikation des Begriffs verstehen, cf.\ \ref{subsubsec:3-1-4-4}.)

Seinen dritten \textsq{Hervorhebungstyp} bezeichnet Bierwisch als
\textsq{Emphase}:
\begin{exe}
	\settowidth\jamwidth{(Bierwisch 1966: 152)}
\ex	{\label{ex:3-19}
"`Bedingung für die Einführung von [einem \textsq{Emphasemorphem}] E
[in einen Satz S] ist ein Satz S$^{\prime}$, der mit S bis auf die
Endkette von [der \textsq{emphatisierten} \isi{Konstituente}] K identisch
ist. K dominiert in S$^{\prime}$ entweder eine andere Morphemkette,
und S ist eine "`paradigmatische Korrektur"' von S$^{\prime}$, oder K
dominiert in S$^{\prime}$ ebenfalls die Morphemkette x und S ist eine
Echofrage. S und S$^{\prime}$ stehen also zueinander im Verhältnis von
Aussage und \isi{Negation} der Aussage, oder von Aussage und
Echofrage."'} \jambox{\citep[152]{Bierwisch66}}
\ex {\label{ex:3-20}
"`Für die Akzentzuordnung bei Emphase ist die Annahme plausibel, daß
nicht nur die emphatisierte \isi{Konstituente} den Primärakzent bekommt,
sondern daß zugleich, anders als bei Kontrastbetonung, alle anderen
Akzente stärker herabgedrückt werden. [\ldots] Weiterhin aber scheint es
plausibel zu sein, diesem \isi{Hauptakzent} einen stärkeren Tonsprung
zuzuordnen als bei nicht"=emphatischen Sätzen mit nur einem
\isi{Hauptakzent}."'} \jambox{\citep[153]{Bierwisch66}}
\end{exe}
Aufgrund der Angaben in (\ref{ex:3-19}) ist eine \textsq{emphatische} \isi{Betonung}
nicht von einer \textsq{kontrastiven} zu unterscheiden. (\ref{ex:3-21a}) kann
  als \textsq{paradigmatische Korrektur} (gegenüber (\ref{ex:3-21b}) mit der
    Einleitung "`nein, \ldots{}"')
\eal \label{ex:3-21}
\ex
\label{ex:3-21a}
\textit{Karl} hat den \textit{Hund} geschlagen
\ex
\label{ex:3-21b}
(höre ich richtig?) \textit{Heinz} hat die \textit{Katze} geschlagen?
\ex
\label{ex:3-21c}
was denn? \textit{Karl} hat den \textit{Hund} geschlagen?
\zl
wie auch als Antwort auf eine \textsq{Echofrage} (\ref{ex:3-21c}) (mit der Einleitung "`ja, in der
Tat, \ldots{}"') gebraucht werden und hätte insofern \textsq{emphatische} \isi{Betonung}. Zugleich enthalten
(\ref{ex:3-21}a,b) ,,parallele Konstituenten, die aber mit verschiedenen Morphemen besetzt sind'';
nach (\ref{ex:3-15}) ist die \isi{Betonung} in (\ref{ex:3-21a}) daher \textsq{kontrastiv}. Man kann daher
allenfalls sagen, daß ein bestimmtes Akzentmuster für verschiedene kommunikative Zwecke eingesetzt
wird: Einmal zur \textsq{Kontrastierung} im Sinne von (\ref{ex:3-15}), einmal als Antwort auf eine
Echofrage, und einmal als \textsq{paradigmatische Korrektur}. Aber inwiefern die verschiedenen
Zwecke, die man mit der Äußerung eines gegebenen Satzes verfolgt, Gegenstand der Grammatik sein
sollten, sehe ich nicht.

Andererseits trifft es nicht zu, daß, wie in (\ref{ex:3-20}) behauptet, eine \textsq{paradigmatische
  Korrektur} mit einer Deakzentuierung der  Konstituenten ohne \textsq{Primärakzent} einhergeht. Es
ist empirisch nicht der Fall, daß \zb (\ref{ex:3-9a}) je nach Verwendungszusammenhang phonetisch
anders realisiert würde, etwa derart, daß bei \textsq{paradigmatischer Korrektur} wie mit der
Fortsetzung (\ref{ex:3-14}) die Betonungsverhältnisse obligatorisch anders wären als bei
Fortsetzungen wie (\ref{ex:3-22}), die keine \textsq{paradigmatischen Korrekturen}
sind:\footnote{\label{fn:3-5}% 
  Die in (\ref{ex:3-20}) angesprochene Deakzentuierung hat Bierwisch ja offensichtlich auch nicht
  empirisch beobachtet; er hält lediglich die "`Annahme"', daß es so etwas gibt, für
  "`plausibel"'.%
}
\begin{exe}
\exi{(9)}{\exalph{9}}
	\begin{xlist}
		\ex	er will seinem Freund das \textit{Auto} schenken
	\end{xlist}
\end{exe}
\eal \label{ex:3-22}
\ex
\label{ex:3-22a}
\ldots{} und nicht etwa nur \textit{Grüße} bestellen lassen
\ex
\label{ex:3-22b}
\ldots{} und \textit{nicht}, wie du behauptest, seinen Geburtstag igno\textit{rier}en
\zl
%\addlines[2]
\largerpage[-1]
Es trifft wohl zu, daß man sprachliche Unterschiede besonders
hervorheben kann; so kann man Betonungsunterschiede dadurch
hervorheben, daß man voll betonte Konstituenten besonders stark betont
und gering betonte Konstituenten \isi{intonatorisch} und akzentuell
besonders wenig hervorhebt. In ähnlicher Weise kann man den
Unterschied zwischen \textit{Pein} und \textit{Bein} dadurch hervorheben, daß man das /p/
in \textit{Pein} besonders stark aspiriert, während man bei dem /b/ in \textit{Bein} die
Öffnung des Verschlusses relativ zum Beginn der Stimmlippenschwingung
besonders lange hinauszögert. Dies sind jedoch ganz allgemeine
performatorische Strategien zur Verdeutlichung sprachlicher Kontraste,
die als solche niemals obligatorisch sind. Da die Deakzentuierung von
(\ref{ex:3-20}) daher nicht obligatorisch und kein Phänomen von systematischem
Interesse ist, sollte sie außer Betracht bleiben. Die unter (\ref{ex:3-19})
genannten Betonungseigenschaften sind von den unter (\ref{ex:3-15}) genannten
nicht zu unterscheiden. Daraus folgt, daß es keinen theoretisch
interessanten Unterschied zwischen \textsq{normaler},
\textsq{kontrastiver} und \textsq{emphatischer} \isi{Betonung} (bei
Bierwisch) gibt.\footnote{\label{fn:3-6}%
  Eine solche Kritik ist natürlich in gewissem Maße ungerecht, da es Bierwisch weniger darum geht,
  optimale Definitionen für nicht"=normale vs.\ normale Betonungen aufzustellen, als darum, für
  intuitiv \textsq{normale} Betonungen Regeln aufzustellen. Entsprechendes gilt für die unter
  (\ref{ex:3-8}), (\ref{ex:3-23}), (\ref{ex:3-28}), (\ref{ex:3-29}) zitierten Äußerungen von Altmann.%
}

Als letzten Explikationsversuch, diesmal für \textsq{kontrastive} (im
Unterschied zu \textsq{normalbetonten}) Äußerungen, betrachten wir
(\ref{ex:3-23}):
\begin{exe}
	\settowidth\jamwidth{(Altmann 1977: 100)}
	\ex {\label{ex:3-23}
Unter kontrastiven Äußerungen sind ,,Äußerungen zu verstehen, die
nicht als Antwort auf eine Wortfrage (im Sinne des Fragesatztestes)
folgen können, sondern dem Typ der Gegenbehauptung zu einer expliziten
deklarativen Äußerung (aber auch zu einer sicheren Annahme)
entsprechen.”}\jambox{\citep[100]{Altmann77}}
\end{exe}
In diesem Zitat geht Altmann offenbar davon aus, daß man nur dann
sinnvoll davon sprechen kann, ein Satz S\sub{\textit{i}} weise
\textsq{Kontrastbetonung} auf, wenn mit der Äußerung von S\sub{\textit{i}} eine
Kontrastierungshandlung vollzogen wird. Es ist aber klar, daß unter
dieser Annahme keine Explikation des traditionellen Sprachgebrauches
möglich ist; (\ref{ex:3-24a}) \zb weist nach üblicher Ausdrucksweise entschieden
\textsq{Kontrastbetonung} auf, muß aber keineswegs als Gegenbehauptung
(etwa zu (\ref{ex:3-24b})) verwendet werden, sondern kann auch (im Sinne
\eal \label{ex:3-24}
\ex
\label{ex:3-24a}
\textit{Karl} hat den Hund geschlagen
\ex
\label{ex:3-24b}
ich glaube, \textit{Heinz} hat den Hund geschlagen
\ex
\label{ex:3-24c}
wer hat den Hund geschlagen?
\zl
des \textsq{Fragesatztests}) als Antwort auf (\ref{ex:3-24c}) dienen. Umgekehrt
kann ein Satz mit \textsq{Normalbetonung} wie (\ref{ex:3-25a}) auch als
Gegenbehauptung
\eal \label{ex:3-25}
\ex
\label{ex:3-25a}
Karl hat den \textit{Hund} geschlagen
\ex
\label{ex:3-25b}
ich glaube, Karl hat die \textit{Katze} geschlagen
\zl
(etwa zu (\ref{ex:3-25b})) verwendet werden. Die Frage, zu welchem Zweck eine
Äußerung getan wird, trägt zur Explikation der tradierten Begriffe
ersichtlich nichts bei.

Die Formulierung in (\ref{ex:3-23}) impliziert offenbar, daß alle Äußerungstypen,
die nicht als Antworten auf Wortfragen im Sinne des \textsq{Fragesatztests} dienen können, \textsq{kontrastiv}, \dash jedenfalls \textsq{nicht"=normal} sind. Auch in dieser Hinsicht
ist (\ref{ex:3-23}) keine Explikation der traditionellen Begriffe. So sind die
Erstglieder von sog.\ exozentrischen Determinativkomposita im Allgemeinen nicht
erfragbar; die a"=Sätze in (\ref{ex:3-26}) und (\ref{ex:3-27}) sind keine Antworten auf die
b"=Sätze:
\eal \label{ex:3-26}
\ex
\label{ex:3-26a}
ich habe das \textit{Groß}maul gesehen
\ex
\label{ex:3-26b}
welches Maul hast du gesehen?
\zl
\eal \label{ex:3-27}
\ex
\label{ex:3-27a}
ich habe einen \textit{Lang}finger erwischt
\ex
\label{ex:3-27b}
was für einen Finger hast du erwischt?
\zl
Gleichwohl sind (\ref{ex:3-26a}), (\ref{ex:3-27a}) zweifellos normal betont. Auch Sätze
mit dem Adverb \textit{übrigens} können nicht als Antworten auf Fragen
verwendet werden; trotzdem wird niemand sagen wollen, sie seien
generell \textsq{nicht-normal} betont.\footnote{\label{fn:3-7}%
  Auch zu Satzfragen, bei denen man gewöhnlich durchaus zwischen \textsq{normaler} und
  \textsq{nicht-normaler} \isi{Betonung} unterscheidet -- vgl.\ (i) vs.\ (ii) -- gibt es keine
  entsprechenden Wortfragen. 
  \ea
  \label{ex:3-fn7i}
  hat Karl den \textit{Hund} geschlagen?
  \ex
  \label{ex:3-fn7ii}
  hat Karl den Hund \textit{geschlagen}?
  \z 
  Aus Gründen wie diesen ist nicht möglich, die Eruierung des Satzfokus
  quasi"=operational vom Ergebnis des \textsq{Fragesatztests} abhängig zu
  machen. Der \textsq{Test} des natürlichen Widerspruchs, wie ihn \citet{Chomsky76a} verwendet,
  liefert teilweise bessere Ergebnisse als der \textsq{Fragesatztest}, ist aber auch nicht auf alle
  relevanten Beispiele anzuwenden. In jedem Fall gilt, daß derartige \textsq{Tests} u.\,U.\ heuristisch
  nützlich sind, aber nicht zur Definition theoretischer Begriffe dienen können: Ein \textsq{Test}
  ist nur dann aussagekräftig, wenn klar ist, inwiefern und warum er zur Abgrenzung
  der fraglichen Begriffe beiträgt. Dies aber setzt eine unabhängige
  Klärung dieser Begriffe voraus. (Umgekehrt ist zu erwarten, daß mit
  Hilfe einer adäquaten Fokustheorie Relationen wie "`natürliche
  Antwort"', "`natürlicher Widerspruch"', "`natürliche Fortsetzung"'
  explizierbar werden; vgl.\ \zb \citet[100]{Chomsky76a};
  \citet{Stechow80a}).%
}

Nach all diesen vergeblichen Explikationsversuchen drängt sich der
Verdacht auf, daß es da in Wahrheit gar nichts zu explizieren gibt;
daß die Rede von \textsq{Normalbetonung} usw.\ eventuell nur eine von
vielen terminologischen Traditionen in der Grammatik ohne
substantiellen Gehalt ist. Eben diesen Schluß zieht Altmann:
\begin{exe}
	\ex \label{ex:3-28}
"`Einerseits konnte gezeigt werden, daß die Annahme eines Normalakzents
für jeden satzwertigen Ausdruck [\ldots] sich zwar auf tiefwurzelnde
Intuitionen stützen kann (die sogar testbar und statistisch erfaßbar
sind), daß sie aber theoretisch nicht haltbar ist: [eine gewisse
\isi{Betonung} B\sub{1}] ist nicht weniger normal bezüglich eines bestimmten
Gebrauchs als [eine andere \isi{Betonung} B\sub{2}]."' \citep[109]{Altmann1978}
\end{exe}
(wobei wohl zu ergänzen wäre: \ldots{} bezüglich eines anderen
Gebrauchs). Dementsprechend spricht er einerseits von
\begin{exe}
		\settowidth\jamwidth{(Altmann 1977: 97)}
	\ex {\label{ex:3-29}
"`Akzentpositionen [\ldots], die von kompetenten Sprechern des
Deutschen\il{Deutsch} automatisch als die "`normalen Akzentpositionen"' bezeichnet
werden"'}\jambox{\citep[97]{Altmann1978}}
\end{exe}
und betont andererseits, "`daß es keinen "`normalen"' Satzakzent für ein
bestimmtes Satzexemplar gibt"' \citep[95]{Altmann1978}.

\addlines
Diese Aussagen sind offensichtlich widersprüchlich. Wenn die Sprecher
wirklich so tiefwurzelnde und automatisch wirksame Intuitionen für
\textsq{Normalbetonung} haben, wie es in (\ref{ex:3-28}) und (\ref{ex:3-29}) behauptet wird,
dann ist eine Theorie, die es nicht erlaubt, diesen Intuitionen
Rechnung zu tragen, selbst nicht haltbar. (Wir kommen auf (\ref{ex:3-28}) in \ref{subsubsec:3-1-4-1} noch einmal zurück.)

Wenn es auch etwas übertrieben ist, von allgemein und automatisch mit eindeutigen Aussagen reagierenden Intuitionen zu reden, glaube ich doch, daß Altmann soweit Recht hat, daß sehr viele Sprecher bei sehr vielen Beispielen ein klares Gefühl haben, daß eine gegebene \isi{Betonung} unauf"|fällig, \textsq{normal} ist, während eine andere \isi{Betonung} auf"|fällig, \textsq{nicht"=normal} ist. Um hervorzuheben, daß es demzufolge ein reales Explikandum gibt und daß dies mit primären Sprecherintuitionen zu tun hat, formuliere ich das als eigenen Explikationsversuch:
\begin{exe}
	\ex\label{ex:3-30}
Ein Satz ist normal betont, wenn die Sprecher diese \isi{Betonung} als
stilistisch normal empfinden; er ist nicht-normal betont, wenn sie
diese \isi{Betonung} als stilistisch nicht"=normal empfinden.
\end{exe}
Damit ist klar, daß \textsq{stilistisch normale Betonung} ein
relevanter Begriff ist, da er -- als unmittelbares und offenbar
systematisches Urteil der Sprecherintuition -- etwa den gleichen Rang
wie Urteile über Akzeptabilität oder Synonymie einnimmt; es ist auch
klar, daß er als intuitives Urteil -- genau so wie der Begriff der
Akzeptabilität oder der der Synonymie -- nicht definitorisch auf
andere Begriffe zurückgeführt werden kann; in diesem Sinne ist er ein
essentieller Begriff. Für sprachwissenschaftliche Zwecke ist mit
dieser Bestimmung natürlich so lange noch wenig gewonnen, wie es
unklar ist, welche Eigenschaften der Sätze ein solches Stilempfinden
beeinflussen. Dies versuche ich im folgenden Abschnitt mit Hilfe des
Begriffs \textsq{Fokus} zu klären.

\largerpage
Erst wenn wir eine inhaltlich gefüllte signifikante Explikation von \textsq{Normalbetonung} haben, können wir beurteilen, in welchem Maße und in welchem Sinne die Unterscheidung
zwischen normaler und nicht"=normaler \isi{Betonung} \zb für die Syntax eventuell belangvoll ist.

\subsection{Fokus, Kontext und Betonung}
\label{subsec:3-1-3}

\subsubsection{Bestimmung von \textsq{Fokus}}
\label{subsubsec:3-1-3-1}

%\largerpage
Ich gebrauche den Ausdruck \textsq{Fokus} in derselben Weise wie
\citet{Chomsky76a}. Nach dieser terminologischen Tradition gilt folgendes für die Sätze in (\ref{ex:3-31}): Der \isi{Fokus} von (\ref{ex:3-31a}) ist \textit{Karl}; das notiere ich so: Fk (\ref{ex:3-31a}) = \textit{Karl}. Für die anderen Sätze gilt: Fk (\ref{ex:3-31b}) = \textit{dem Kind}; Fk (\ref{ex:3-31c}) = \textit{das Buch}; Fk (\ref{ex:3-31d}) = \textit{geschenkt}.
\eal \label{ex:3-31}
\ex
\label{ex:3-31a}
\textit{Karl} hat dem Kind das Buch geschenkt
\ex
\label{ex:3-31b}
Karl hat dem \textit{Kind} das Buch geschenkt
\ex
\label{ex:3-31c}
Karl hat dem Kind das \textit{Buch} geschenkt
\ex
\label{ex:3-31d}
Karl hat dem Kind das Buch \textit{geschenkt}
\zl
Grob gesprochen ist der \isi{Fokus} der Teil des Satzes, mit dem in gewissem
Sinne etwas Neues mitgeteilt wird; der Rest des Satzes -- das
\textsq{Topik} -- wird als bekannt vorausgesetzt.\footnote{\label{fn:3-8}%
  Für einen von \isi{Topik} und \isi{Fokus} verschiedenen Teil \textsq{Transition}, wie er u.\,a.\ \zb in
  \citet{Sgall73} angenommen wird, ist in der hier zu entwickelnden Explikation kein Raum. Ebenfalls
  ungerechtfertigt scheint es mir, neben der Unterscheidung von \isi{Fokus} und \isi{Topik} ähnliche, aber nicht
  identische Unterscheidungen wie \textsq{Thema/""Rhema} und \textsq{psychologisches
    \isi{Subjekt}/""Objekt} einzuführen (die alle mit \textsq{dominance} im Sinne von Erteschik"=Shir
  nichts zu tun haben). Vgl.\ zur Kritik \citet{Pasch78b,Pasch78a}, \citet{Sgall77}.%
}
Eine genauere Formulierung ist in (\ref{ex:3-32}) gegeben:
\begin{exe}
\ex \label{ex:3-32}
\textit{Inhaltliche Bestimmung von \textsq{Fokus}:} \\
Bei einer Äußerung eines Satzes S\sub{\textit{i}} ist jener Teil von S\sub{\textit{i}} der
\isi{Fokus} Fk (S\sub{\textit{i}}), dessen Funktion in S\sub{\textit{i}} nicht aufgrund des relevanten Kontexts bekannt ist. (Die übrigen Teile von S\sub{\textit{i}} bilden das \isi{Topik} Tk (S\sub{\textit{i}}).)
\end{exe}
Dazu einige Erläuterungen.

\addlines[2]
\textsq{Relevanter Kontext} (RK) ist das, was man als das gemeinsame
Vorwissen von Sprecher und Hörer bezeichnen kann und was
\citet{Chomsky76a} -- in einer ungewöhnlichen Verwendung des Worts --
\textsq{Presupposition} nennt. Bei einer Äußerung von (\ref{ex:3-31a}) geht der
Sprecher offensichtlich davon aus, daß es ihm und dem Hörer bekannt
ist, daß jemand dem Kind das Buch geschenkt hat, und daß es dem Hörer
unbekannt ist, daß (er, der Sprecher, glaubt, daß) die Agensrolle von
Karl ausgefüllt worden ist; insofern ist die Funktion von
\textit{Karl} in (\ref{ex:3-31a}) das, was für den Hörer \textsq{neu}
ist. Entsprechend bei den anderen Beispielen; insbesondere nimmt der
Sprecher bei einer Äußerung von (\ref{ex:3-31d}) an, daß aus dem RK bekannt ist,
daß es einen Vorgang gegeben hat, an dem Karl, das Kind und das Buch
in je bestimmter Weise beteiligt waren; daß die Funktion, diesen
Vorgang zu bezeichnen, von \textit{geschenkt} und nicht \zb von
\textit{gestohlen} übernommen wird, ist (nach der Annahme des
Sprechers) nicht aus dem RK bekannt.\footnote{\label{fn:3-9}%
    Häufig (\zb in \citealt[43]{Erteschick79c}) wird angenommen, daß bei Chomsky, anders als in (\ref{ex:3-32}), \textsq{Fokus} als "`phrase containing the \isi{intonation} center"' definiert sei. Dieser Eindruck beruht jedoch auf ungenauer Lektüre. In \citet[89]{Chomsky76a} werden zunächst die
  Ausdrücke "`focus"' und "`presupposition"' eingeführt und mittels
  \textsq{natürlicher Erwiderungen} charakterisiert; erst danach heißt
  es S.\,90 "`\ldots{}{} one might propose that the focus is determined by the
  surface structure, namely, as the phrase containing the \isi{intonation}
  center"'. \textsq{Fokus} ist also in \citet{Chomsky76a} ebenso wie in
  (\ref{ex:3-32}) begriff"|lich unabhängig von Betonungsfaktoren: Er ist aufgrund des
  \textsq{\isi{intonation} center} determiniert, nicht aber definiert. Es
  liegt auf der Hand, daß nur auf diese Weise ein inhaltlich nicht"=trivialer Begriff von \isi{Fokus} zu etablieren ist.

  Erteschik"=Shir benutzt den Begriff \textsq{\isi{dominant} constituent}, der mit \textsq{Fokus} eng verwandt, aber, wie sie (\citealt[443ff]{Erteschick79a}; \citealt[43ff]{Erteschick79c}) betont, nicht
  identisch ist. Der wesentliche Unterschied scheint zu sein, daß ein nach (\ref{ex:3-32}) bestimmter \isi{Fokus} einen Teil enthalten kann, der dem Sprecher/""Hörer \textsq{wichtiger} erscheint als der Rest des \isi{Fokus}. Eine solche Unterscheidung ist intuitiv nicht unplausibel, und Erteschik"=Shir versucht anhand interessanten Materials, ihre theoretische Relevanz nachzuweisen. Wenn ich recht sehe, bedeutet das (wenn sie Recht hat), daß für gewisse Zwecke neben der \isi{Fokus}/""\isi{Topik}"=Unterscheidung weitere semantisch/""pragmatische Unterscheidungen eingeführt werden müssen. Manche Einzelheiten der begriff"|lichen Bestimmung von \textsq{dominance} und ihrer empirischen
  Verwendung scheinen mir jedoch sehr klärungsbedürftig. Überflüssig wird der Fokusbegriff, wie es
  scheint, dadurch jedenfalls nicht.%
}

In (\ref{ex:3-32}) ist \textsq{Fokus} für Äußerungen von Sätzen bestimmt
worden. Da wir künftig meistens~-- wie schon bei (\ref{ex:3-31})~-- nicht
Äußerungen, sondern Sätze betrachten, benötigen wir den Begriff
\textsq{möglicher Fokus}:
\begin{exe}
	\ex\label{ex:3-33}
Ein Satz S\sub{\textit{i}} hat den \textit{möglichen Fokus} Fk\sub{\textit{j}} (S\sub{\textit{i}}), wenn Fk\sub{\textit{j}} (S\sub{\textit{i}}) bei einer
Äußerung von S\sub{\textit{i}} \isi{Fokus} sein kann.
\end{exe}
Mögliche Foki notiere ich, wie in (\ref{ex:3-33}), künftig mit einem Index an
"`Fk"'. Wo kein Mißverständnis nahe liegt, spreche ich oft einfach von
\textsq{Fokus} statt von \textsq{möglichem Fokus}; ebenso unterdrücke
ich gelegentlich die Angabe "`(S\sub{\textit{i}})".

Für formale Zwecke kann man den RK als eine Menge von Sätzen und/""oder Propositionen repräsentieren. Wir können dann RK weitgehend mit dem identifizieren, was Kempson \textsq{Pragmatic Universe of Discourse} nennt:
\begin{exe}
\settowidth\jamwidth{(Altmann 1977: 100)}
\ex{\label{ex:3-34}
"`More formally we can say that for every conversational exchange in
which S is the speaker and H the hearer, the set of propositions which
constitute the knowledge which two such speakers will believe they
share must meet the following four conditions:
\begin{enumerate}
\item[(1)] S believes P\sub{\textit{i}}
\item[(2)] S believes H knows P\sub{\textit{i}}
\item[(3)] S believes H knows S believes P\sub{\textit{i}}
\item[(4)] S believes H knows S believes H knows P\sub{\textit{i}}
\end{enumerate}
I shall call this set of propositions the Pragmatic Universe of
Discourse."'} \jambox{\citep[167]{Kempson75}}
\end{exe}
\largerpage
Allerdings muß die Repräsentation von RK auch natursprachliche Sätze
enthalten können (Propositionen formuliert man gewöhnlich in einer
formalen Sprache), da Sprecher und Hörer u.\,U.\ wissen, daß und wie
ein bestimmter Satz geäußert worden ist (und sich nicht nur an seinen
Inhalt erinnern); außerdem können in RK Sätze vorkommen, die außer
natursprachlichen auch formale Elemente enthalten. So kann es sein,
daß der Sprecher etwas gesagt hat, das der Hörer wie (\ref{ex:3-35}) gespeichert
  hat, wobei "`\textit{W}"' eine Variable für phonetisches Material ist; (\ref{ex:3-35})
  repräsentiert dann das, was Sprecher und Hörer über die fragliche
  Äußerung wissen. Solche Fälle führen dann oft zu Rückfragen etwa
  wie (\ref{ex:3-36}):
\eal \label{ex:3-35}
\ex
\label{ex:3-35a}
Karl hat \textit{W} geschlagen
\ex
\label{ex:3-35b}
Karl ist etwas wichtiges \textit{W} gefallen
\zl
\eal \label{ex:3-36}
\ex
\label{ex:3-36a}
was hast du gesagt? Karl hat \textit{wen} geschlagen?
\ex
\label{ex:3-36b}
hast du gesagt "`\textit{ein}gefallen"' oder "`\textit{auf}\hspace{-0.15em}gefallen"'?
\zl
Wenn wir eine solche Repräsentation von RK voraussetzen, können wir \textsq{Fokus} auch, formal bestimmen:\footnote{\label{fn:3-10}%
	Eine inhaltlich mit (\ref{ex:3-37}) weitgehend übereinstimmende Bestimmung gibt \citet[192f]{Kempson75} anlässlich eines Falles von \textsq{Kontrastbetonung}. Eng verwandt ist auch die durch einige sorgfältige Fallanalysen gestützte Explikation von \citet{Pasch78a}.

  Die Definition (\ref{ex:3-37b}) macht explizit, daß es bei der Bestimmung des
  (semantischen) \isi{Fokus} von S\sub{\textit{i}} um das Verhältnis zwischen RK und der
  logischen Charakterisierung von S\sub{\textit{i}} geht. Im Ergebnis erhalten wir
  ein Paar (S\sub{\textit{i}}, Fk (S\sub{\textit{i}})), das die logischen Eigenschaften von S\sub{\textit{i}}
  zusammen mit den kontextuellen Verwendungsmöglichkeiten von S\sub{\textit{i}}
  bestimmt. 

  Denselben Grundgedanken hat Stechow \citep{Stechow80a,Stechow80b} in völlig anderer Weise verwendet: An Stelle des Paars (S\sub{\textit{i}}, Fk (S\sub{\textit{i}})) gebraucht er den Begriff \textsq{Fokusinformation} und definiert ihn als die materiale \isi{Implikation} zwischen der \textsq{Topikinformation} und dem \textsq{Inhalt} von S\sub{\textit{i}}; mit unwesentlicher Vergröberung können wir dafür die (materiale) \isi{Implikation} \textsq{RK $\supset$ S\sub{\textit{i}}} einsetzen. Diese Deutung der \textsq{Funktion} des \isi{Fokus} ist nicht nur intuitiv überraschend, sondern führt auch zu unüberwindlichen Schwierigkeiten, besonders bei Sätzen wie (\ref{ex:3-43b}) in unserem Text sowie (i), (iii) und (iv) von Fn.\,\ref{fn:3-13}. (Deren RK ist trivial wahr, ihre \textsq{Fokusinformation} wäre deshalb äquivalent mit S\sub{\textit{i}}~-- ein offensichtlich unerwünschtes Ergebnis. Die in \citet{Stechow80b} erörterten Lösungsvorschläge halte ich für unzureichend.) Soweit ich
  sehe, wird alles, was Stechows Fokustheorie leisten soll, auf
  einleuchtendere und einfachere Weise von unserer Theorie geleistet,
  ohne daß Probleme der angedeuteten Art dabei auftreten können.%
}
\begin{exe}
\ex
\label{ex:3-37}
\textit{Formale Bestimmung von \textsq{Fokus}:} \\
Bei einer Äußerung von S\sub{\textit{i}} ist jener Teil von S\sub{\textit{i}} der Fk (S\sub{\textit{i}}), der entweder
\begin{xlist}
\ex
\label{ex:3-37a}
einer \textit{W}"=Variablen in einem Satz S\supsub{W}{\textit{i}} \ in RK oder
\ex
\label{ex:3-37b}
einem Satzoperator oder einem \isi{Konjunkt} in der normalisierten logischen
Charakterisierung von S\sub{\textit{i}}
\end{xlist}
entspricht, das nicht in RK enthalten ist.
\end{exe}
Dabei ist "`S\supsub{W}{\textit{i}} "' ein Satz, der mit S\sub{\textit{i}} identisch ist bis
auf die Tatsache, daß er eine oder mehrere \textit{W}"=Variablen enthält, wo
S\sub{\textit{i}} phonetisches Material hat. Was ich unter einer
\textsq{normalisierten logischen Charakterisierung} verstehe, möchte
ich an einem Beispiel verdeutlichen. Nehmen wir an, daß die auf
reguläre Weise gewonnene logische Charakterisierung der Sätze in (\ref{ex:3-31})
(unter Absehung von \isi{Tempus} usw.) die Form (\ref{ex:3-38a}) hat. Dann hat die
\textsq{normalisierte logische Charakterisierung} die Form (\ref{ex:3-38b}):
\eal \label{ex:3-38}
\ex
\label{ex:3-38a}
SCHENK (KARL, KIND, BUCH)
\ex
\label{ex:3-38b}
∃r ∃x ∃y ∃z (r (x, y, z) \\
\hspace{6em} \& r (x, y, z) = SCHENK (x, y, z) \\
\hspace{6em} \& x = KARL \\
\hspace{6em} \& y = KIND \\
\hspace{6em} \& z = BUCH)
\zl
Jede \textit{n}"=stellige Proposition wird also in offensichtlicher Weise in
eine Proposition mit \textit{n + 2} Konjunkten umgeformt. Verbal kann man
(\ref{ex:3-38b}) etwa so umschreiben: Es gibt eine Relation \textit{r} (1.\,\isi{Konjunkt}),
die eine Relation des Schenkens (2.\,\isi{Konjunkt}) ist; daran sind Karl
(3.\,\isi{Konjunkt}), das Kind (4.\,\isi{Konjunkt}) und das Buch (5.\,\isi{Konjunkt}) in je
spezifischer Weise beteiligt.

Einen \isi{Fokus}, der nach (\ref{ex:3-37a}) bestimmt ist, will ich \textit{W}"=\isi{Fokus} nennen;
einen, der nach (\ref{ex:3-37b}) bestimmt ist, nenne ich \textsq{semantischen
  Fokus}. Wir nehmen an, daß alle Propositionen in RK die
normalisierte Form haben. Für (\ref{ex:3-31a}) gilt dann, daß der Satz unter
Wahrung konversationeller Maximen nur dann geäußert werden kann, wenn
es (a) in RK den Satz "`\textit{W} hat dem Kind das Buch geschenkt"' gibt, oder
wenn (b) das erste, zweite, vierte und fünfte \isi{Konjunkt} von (\ref{ex:3-38b}) in
RK sind, nicht aber das dritte. Ähnlich kann (\ref{ex:3-31d}) mit semantischem
\isi{Fokus} nur dann geäußert werden, wenn das erste und die drei letzten
Konjunkte von (\ref{ex:3-38b}), nicht aber das zweite, in RK sind.\footnote{\label{fn:3-11}%
  Das in den Grundzügen wohl ähnlich gemeinte Repräsentationssystem in \citet[542f]{Bartsch76}
  enthält keine Möglichkeit, die Zugehörigkeit des Verbs zum \isi{Fokus} formal zu kennzeichnen.%
}
(Das gilt nach unseren bisherigen Annahmen über die möglichen Foki der Sätze in (\ref{ex:3-31}), die
wir in \ref{subsubsec:3-1-3-2} z.\,T.\ korrigieren werden.)

Daß ein Satz mit einem bestimmten \isi{Fokus} nur geäußert werden kann, wenn
der relevante Kontext bestimmte Eigenschaften hat, heißt: Nur in einem
solchen \isi{Kontexttyp} kann der Satz unter Wahrung konversationeller
Maximen (vgl.\ \mbox{Grice} \citeyear{Grice75}) geäußert werden. Es heißt natürlich nicht, daß Äußerungen, die die Bestimmungen von (\ref{ex:3-32}) und (\ref{ex:3-37}) verletzen, nicht vorkommen könnten; tatsächlich sind Verletzungen dieser Bedingungen in alltäglicher Kommunikation nicht selten. Sie
lösen jedoch Kommunikationsprobleme aus: Der Hörer muß entweder auf relativ zuverlässige konversationelle Implikaturen zurückgreifen, oder er ist auf sehr unsichere Spekulationen über die Intentionen des Sprechers angewiesen; u.\,U.\ bricht die Verständigung zusammen. Für
weitere Diskussion vgl.\ \ref{subsubsec:3-1-4-4}.\footnote{\label{fn:3-12}%
	Siehe auch
  \citet{Clark77}, wo anhand verschiedener sprachlicher Phänomene die Fruchtbarkeit des
  Fokusbegriffs von (\ref{ex:3-32}) und (\ref{ex:3-37}) in Verbindung mit der Annahme
  konversationeller Maximen demonstriert wird.

  In diesem Zusammenhang finden auch Strategien der Wiederholung eine
  Erklärung; einige Typen werden in \citet{Lehman77} besprochen.

  Die Theorie der konversationellen Maximen zieht ihre Attraktivität und
  explanatorische Potenz wesentlich aus der Tatsache, daß sie eine
  allgemeine (nicht spezifisch sprachliche und nicht kulturspezifische)
  Theorie kooperativen Verhaltens ist. \citet{Keenan77} versucht
  diese Theorie zu erschüttern, indem sie auf systematische Verletzungen
  der konversationellen Maximen in der Kultur von Madagaskar
  hinweist. Tatsächlich bestätigen ihre Beobachtungen jedoch die
  Theorie: Die Maximen werden dort~-- wie überall sonst~-- gerade dann
  verletzt, wenn unkooperatives Verhalten zum Prinzip gemacht wird;
  wo dies nicht der Fall ist (nach Keenan besonders innerhalb der
  Familie und der nächsten Verwandtschaft), werden die Maximen beachtet.%
}

Natürlich ist eine \textsq{logische Charakterisierung} wie
(\ref{ex:3-38a}) und damit auch ihre normalisierte Form
(\ref{ex:3-38b}) nur in sehr einfachen Fällen zureichend. Wir wollen
anhand von (\ref{ex:3-39}) überlegen, welche in (\ref{ex:3-38}a,b) nicht berücksichtigten Faktoren in einer expliziteren
Darstellung noch berücksichtigt werden müßten.
\ea \label{ex:3-39}
Karl \textit{ist} krank
\z
Man kann \textit{ist} in (\ref{ex:3-39}) als \textit{W}"=\isi{Fokus} interpretieren, etwa als
Antwort auf (\ref{ex:3-40a}), wobei in RK (\ref{ex:3-40b}) enthalten ist. Es kann aber
auch ein semantischer \isi{Fokus} vorliegen, der etwa im Gegensatz zu (\ref{ex:3-41a}) steht, wobei in RK ist,
\eal \label{ex:3-40}
\ex
\label{ex:3-40a}
Karl \textit{spielt} krank?
\ex
\label{ex:3-40b}
Karl \textit{W} krank
\zl
\eal \label{ex:3-41}
\ex
\label{ex:3-41a}
Karl \textit{wird} krank
\ex
\label{ex:3-41b}
Karl \emph{war} krank
\zl
daß eine Relation zwischen Karl und Kranksein besteht. Ebenso kann ein
impliziter Gegensatz zu (\ref{ex:3-41b}) vorliegen; in diesem Fall ist in RK,
daß Karl zu einem bestimmten Zeitpunkt \textit{t} krank (gewesen) ist, und
die in der \isi{Flexionsform} von \textit{sei-} enthaltene Bestimmung von
\textit{t} ist das, was nicht in RK ist. Um dies darzustellen, müssen wir in
der normalisierten logischen Charakterisierung eine Variable für \textit{t}
ansetzen.

Auf eine vierte Verwendungsweise für (\ref{ex:3-39}), die weniger leicht zu
analysieren ist, hat \citet[309f]{Fuchs76} aufmerksam gemacht:
wenn der Satz nämlich als nachdrückliche Affirmation, etwa auf die
Frage (\ref{ex:3-42}), gebraucht wird. Essentiell das gleiche scheint
\ea
\label{ex:3-42}
ist Karl \textit{krank}?
\z
\eal \label{ex:3-43}
\ex
\label{ex:3-43a}
ich glaube nicht, daß Karl den Hund getreten hat
\ex
\label{ex:3-43b}
doch, Karl \textit{hat} den Hund getreten
\zl
%\addlines
\largerpage
in (\ref{ex:3-43b}) als Entgegnung auf (\ref{ex:3-43a}) vorzuliegen; es scheint nicht
möglich, \textit{hat} in (\ref{ex:3-43b}) als \textit{W}"=\isi{Fokus} oder als semantischen \isi{Fokus} wie
in (\ref{ex:3-41}a,b) zu verstehen; jedenfalls nicht im Kontext (\ref{ex:3-43a}). Besonders interessant ist dieses Phänomen mit \isi{Negation} (\ref{ex:3-44a}) und
Fragen (\ref{ex:3-44}b,c). (Dabei kommt als komplizierendes Moment hinzu,
\eal \label{ex:3-44}
\ex
\label{ex:3-44a}
aber Karl \textit{hat} den Hund doch gar nicht getreten
\ex
\label{ex:3-44b}
\textit{hat} Karl den Hund denn getreten?
\ex
\label{ex:3-44c}
wer \textit{hat} den Hund denn eigentlich getreten?
\zl
daß Frageelemente (bei Satz- wie bei Wortfragen),\footnote{\label{fn:3-12a}%
  Mit "`Frageelement"' meine ich ein abstraktes Element, das den Satz als
  Frage kennzeichnet; bei Ergänzungsfragen ist es Teil des
  Interrogativpronomens. Daß solche Frageelemente generell zum \isi{Fokus}
  gehören, folgt aus (\ref{ex:3-37}), wenn wir folgende Annahmen machen:

  Aussagesätze werden durch einen \isi{Operator} für die Wahrheitsbehauptung \emph{T}
  (entsprechend Freges Urteilsstrich) gekennzeichnet; direkte Fragesätze
  erhalten statt dessen einen \isi{Operator} "`?"'. Satz (\ref{ex:3-fn12aia}) wäre dann etwa
  als (\ref{ex:3-fn12aib}) zu repräsentieren, Satz (\ref{ex:3-fn12aiia}) als (\ref{ex:3-fn12aiib}):
  \eal
  \label{ex:3-fn12ai}
  \ex
  \label{ex:3-fn12aia}
  schlägt \textit{Karl} den Hund?
  \ex
  \label{ex:3-fn12aib}
  ? (∃r ∃x ∃y (r (x, y) \& r (x, y) = SCHLAG (x, y) \& x = KARL \& y = HUND))
  \zlmid
  \eal
  \label{ex:3-fn12aii}
  \ex
  \label{ex:3-fn12aiia}
  wen \textit{schlägt} Karl?
  \ex
  \label{ex:3-fn12aiib}
  ? (∃r ∃x ∃y (r (x, y) \& r (x, y) = SCHLAG (x, y) \& x = KARL \& y = z))
  \zl
  (Dabei ist \textit{wen} in (\ref{ex:3-fn12aiib}) durch "`?"' und die freie Variable \textit{z} repräsentiert.)

  Die Verwendungsbedingungen scheinen so zu sein, daß für (\ref{ex:3-fn12aia}) ein
  Kontext wie (\ref{ex:3-fn12aiiia}) gegeben sein muß, also (\ref{ex:3-fn12aiiib}) in RK ist; für
  (\ref{ex:3-fn12aiia}) ein Kontext wie (\ref{ex:3-fn12aiva}), also (\ref{ex:3-fn12aivb}) in RK:
  \eal
  \label{ex:3-fn12aiii}
  \ex
  \label{ex:3-fn12aiiia}
  jemand schlägt (vielleicht) den Hund
  \ex
  \label{ex:3-fn12aiiib}
  ∃r ∃x ∃y (r (x, y) \& r (x, y) = SCHLAG (x, y) \& y = HUND)
  \zlmid
  \eal
  \label{ex:3-fn12aiv}
  \ex
  \label{ex:3-fn12aiva}
  Karl tut etwas mit jemandem
  \ex
  \label{ex:3-fn12aivb}
  T (∃r ∃x ∃y (r (x, y) \& x = KARL))
  \zl
  Nach (\ref{ex:3-37}) ist dann in (\ref{ex:3-fn12aia}) \textit{Karl} Teil des \isi{Fokus}, und da der \isi{Operator}
  "`?"' von (\ref{ex:3-fn12aib}) nicht in RK (\dash nicht in (\ref{ex:3-fn12aiiib})) enthalten ist, ist
  auch der Teil des Satzes, dem der \isi{Operator} "`?"' zugeordnet ist, dem
  \isi{Fokus} zuzurechnen. Entsprechend ist bei (\ref{ex:3-fn12aiia}) nach (\ref{ex:3-37}) \textit{schlägt}
  Teil des \isi{Fokus}, und da weder der \isi{Operator} "`?"' noch die freie Variable
  \textit{z} von (\ref{ex:3-fn12aiib}) in (\ref{ex:3-fn12aivb}) enthalten ist, ist die ihnen zugeordnete
  \isi{Konstituente} \textit{wen} ebenfalls Teil des \isi{Fokus}.%
}
\isi{Adverbiale} wie \textit{eigentlich}, \textit{vermutlich} und \textsq{Gradpartikeln} wie \textit{sogar}, \textit{nur}, \textit{auch}
generell und die \isi{Negation} häufig zum \isi{Fokus} gehören; zu \isi{Negation} und
Gradpartikeln vgl.\ die~-- trotz aller Mängel, cf.\ \citet[266-73]{Hust76}~-- relevante Diskussion in
\citet[Kapitel 6 und 8.6]{Jackendoff72}; zu \isi{Satzadverbialen}
vgl.\ \citet{Verhagen79}.) Im \isi{Fokus} steht hier offenbar immer die
Wahrheitsbehauptung, wie ich es in den Umschreibungen in (\ref{ex:3-45})
anzudeuten versuche:
\eal \label{ex:3-45}
\ex
\label{ex:3-45a}
es trifft zu, daß Karl krank ist (zu (\ref{ex:3-39}))
\ex
\label{ex:3-45b}
doch, es ist wahr, daß Karl den Hund getreten hat (zu (\ref{ex:3-43b}))
\ex
\label{ex:3-45c}
aber wahr ist doch, daß Karl den Hund gar nicht getreten hat (zu (\ref{ex:3-44a}))
\ex
\label{ex:3-45d}
ist es denn wahr, daß Karl den Hund getreten hat? (zu (\ref{ex:3-44b}))
\ex
\label{ex:3-45e}
für welche Person x gilt, daß es wahr ist, daß x den Hund getreten hat? (zu (\ref{ex:3-44c}))
\zl
Um dies in der normalisierten logischen Charakterisierung zu erfassen,
muß das finite \isi{Verb} in diesen Konstruktionen einem \isi{Operator} für die
Wahrheitsbehauptung \textit{T} bzw.\ einem Frageoperator "`?"' (vgl.\ Fn.\,\ref{fn:3-12a})
zugeordnet werden.\footnote{\label{fn:3-12b}%
	Satz (\ref{ex:3-39}) hat die logische Repräsentation (39i) und benötigt in der	
  relevanten Interpretation einen Kontext wie (39ii\ref{ex:3-fn12b39iia}, b oder c); der
  relevante Teil des RK ist also wie (39iii) zu repräsentieren:   
  
  \gblabelsep{1em}  
  \begin{exe}
  \exi{(39i)}{\exalph{39i}\label{ex:3-fn12b39i}
  ~~T (∃r ∃x (r (x) \& r (x) = KRANK (x) \& x = KARL))}  %manueller Ausgleich mit ~
  \exi{(39ii)}%{\exalph{39ii}\label{ex:3-fn12b39ii}}
  \begin{xlist}
  \ex\label{ex:3-fn12b39iia}%
  vielleicht ist Karl \textit{krank}
  \exi{~~~b.} \label{ex:3-fn12b39iib}                   %manueller Ausgleich mit ~
  ist Karl \textit{krank}?
  \exi{~~~c.} \label{ex:3-fn12b39iic}                   %manueller Ausgleich mit ~
  Heinz behauptet, daß Karl \textit{krank} ist
  \end{xlist}
  \exi{(39iii)}\exalph{39iii}\label{ex:3-fn12b39iii}
  ∃r ∃x (r (x) \& r (x) = KRANK (x) \& x = KARL)
  \end{exe}
  Wenn das finite \isi{Verb} in Konstruktionen dieser Art (nicht-eingebettete
  Sätze mit \isi{Anfangsstellung} des finiten Verbs) dem \isi{Operator} \emph{T} (bzw.\ "`?"')
  zugeordnet ist, folgt aus (\ref{ex:3-37}), daß es zum \isi{Fokus} gehört (da dieser
  \isi{Operator} nicht in (\ref{ex:3-fn12b39iii}) enthalten ist).

  In Fällen wie (\ref{ex:3-39}), (\ref{ex:3-43b}), (\ref{ex:3-44}) bildet die betonte \isi{Konstituente} (das
  finite \isi{Verb}) also eigentlich nicht als ganze den \isi{Fokus}, sondern nur
  ein Teil von ihr (das mit der finiten Form verbundene flexivische
  Element). Das entspricht nicht ganz der Hypothese (\ref{ex:3-66}), ist aber
  insofern leicht zu verstehen, als dieser relevante Teil aus
  phonologischen Gründen gegenüber anderen Teilen derselben \isi{Konstituente}
  \isi{intonatorisch} im Allgemeinen nicht hervorgehoben werden kann.

  Anders ist es bei gewissen Typen von Exklamativen. Hans Altmann hat
  mich auf Sätze wie (i) aufmerksam gemacht, die man wie (ii)
  \ea%exehack
  \label{ex:3-fn12bi}
  hat \textit{der} aber Muskeln!
  \ex 
  \label{ex:3-fn12bii}
  es ist \textit{erstaunlich}, was der für Muskeln hat
  \ex
  \label{ex:3-fn12biii}
  es ist erstaunlich, was \textit{der} für Muskeln hat
  \z
  (aber kaum wie (iii)) gebrauchen kann. Hier scheint das betonte \textit{der}
  entweder gar nicht zum \isi{Fokus} zu gehören oder jedenfalls keinen
  minimalen \isi{Fokus} bilden zu können. Dies ist ein offenes
  Beschreibungsproblem, und mehr noch ein Erklärungsproblem.%
}

Daß diese Analyse intuitiv richtig ist, zeigt sich deutlich an einem Beispiel von Fuchs:
\begin{quotation}
  "`(Fritz hat seine Mutter gebeten, zum Mittagessen Ravioli zu kochen,
  sie hat es halb versprochen. Kurz darauf kommt er in die Küche,
  sieht die Mutter am Herd und fragt:) Gelt du \textit{kochst} Ravioli?"'
  \citep[309 (25)]{Fuchs76}
\end{quotation}
Aufgrund des gegebenen Kontexts kann nicht die Frage sein, was die
Mutter tut oder was sie mit Ravioli tut. In RK ist, daß es möglich
ist, daß die Mutter Ravioli kocht; gefragt ist im Beispiel, ob es wahr
ist, daß sie das tut. Durch die \isi{Betonung} des Finitums ist hier wieder
die Wahrheitsbehauptung als das nicht in RK enthaltene Element
bestimmt.

Bestandteil des \textsq{Vorwissens}, das Sprecher und Hörer gemeinsam
ist, ist in solchen Fällen also nur der \textsq{propositionale Gehalt}
des Satzes, also das, was Frege in seinen frühen Schriften
\textsq{beurteilbaren Inhalt} \citep[2--5]{Frege1879} und später
\textsq{Gedanken} (\zb \citealt[32]{Frege1892}) genannt hat; ohne
die \textsq{Anerkennung der Wahrheit} des Gedankens (die das
\textsq{Urteil} ausmacht; cf.\ \zb Frege (\citeyear{Frege1879}: 1f.;
\citeyear{Frege1892}: 35)).

Die P\sub{\textit{i}} im Zitat (\ref{ex:3-34}) von Kempson sind daher im Allgemeinen -- vermutlich gegen
Kempsons Absicht -- als Propositionen mit unbekanntem Wahrheitswert anzusehen. Daß nur der propositionale Gehalt eines Satzes in RK ist, ist dabei nicht etwa eine pathologische Eigenschaft von Spezialfällen wie (\ref{ex:3-39}), (\ref{ex:3-43b}) usw.\ Auch \zb in (\ref{ex:3-46b}) als Erwiderung auf (\ref{ex:3-46a}) ist der \textit{ob}"=Satz offensichtlich in RK, \dash \isi{Topik} von (\ref{ex:3-46b}),
\eal \label{ex:3-46}
\ex
\label{ex:3-46a}
A: Karl ist sehr nett; ob Frau Müller ihn wohl liebt?
\ex
\label{ex:3-46b}
B: es ist mir ziemlich egal, ob sie ihn liebt
\ex
\label{ex:3-46c}
A: Karl behauptet, daß Hans eine geniale Entdeckung gemacht hat
\ex
\label{ex:3-46d}
B: daß Hans eine geniale Entdeckung gemacht hat, kann ich mir\\
\hspace{1em}beim besten Willen nicht vorstellen
\zl
ohne daß sein Wahrheitswert aus RK bekannt wäre; ebenso bei dem
\textit{daß}"=Satz von (\ref{ex:3-46d}) als Antwort auf (\ref{ex:3-46c}). Entgegen weit verbreiteter
(\zb auch in \citet{Clark77} vertretener) Ansicht kann man also RK keinesfalls mit der Menge von
Propositionen identifizieren, die Sprecher und Hörer als wahr akzeptieren.\footnote{\label{fn:3-13}%
  Dies ist mehrfach (\zb in \citet[246]{Jackendoff72}, \citet[98]{Rochemont78}) für Fälle wie (i)
  beobachtet worden. Hier ist \textit{niemand} = Fk (i), und in RK ist ein Äquivalent von (ii),
  \ea%exehack
  \label{ex:3-fn13i}
  \textit{niemand} hat gelacht
  \ex
  \label{ex:3-fn13ii}
  jemand hat gelacht
  \z
  aber offensichtlich wird (ii) von einem Sprecher von (i) nicht als
  wahr akzeptiert.

  Entsprechendes gilt für Beispiele wie (iii) und (iv):
  \ea%exehack
  \label{ex:3-fn13iii}
  \textit{vielleicht} hat Karl gelacht
  \ex
  \label{ex:3-fn13iv}
  Karl hat \textit{nicht} gelacht
  \zlast %
}

Wir haben eine inhaltliche und eine formale Bestimmung für
\textsq{Fokus} gegeben. Wie kann man~-- in Fällen, die intuitiv nicht
völlig klar sind~-- den \isi{Fokus} praktisch bestimmen? Automatische Entdeckungs"= oder Entscheidungsprozeduren gibt es da nicht. Grundsätzlich muß man einen Kontext konstruieren oder aufsuchen, der soweit spezifiziert ist, daß man mit einiger intuitiven Sicherheit sagen kann, ob der fragliche Satz in diesem Kontext den hypothetisch angenommenen \isi{Fokus} haben kann. In vielen Fällen reicht es aus, eine \textsq{natürliche} Erwiderung zu konstruieren, deren Form es klar macht, welche Teile der Sätze einander gegenüber gestellt werden (und damit den \isi{Fokus} bilden). Zu betonen ist, daß der in (\ref{ex:3-23}) angesprochene \textsq{Fragesatztest} kein \textsq{Test} ist. Wenn die Frage (\ref{ex:3-47a}) als Nachweis dafür dienen soll, daß \textit{den Jungen} ein Fk\sub{\textit{i}} (\ref{ex:3-47b}), aber nicht ein Fk\sub{\textit{i}} (\ref{ex:3-47c}) ist,
\begin{exe}
\ex
\label{ex:3-47}
\begin{xlist}
\ex
\label{ex:3-47a}
wen hat der Mann gestreichelt?
\ex
\label{ex:3-47b}
der Mann hat den \textit{Jungen} gestreichelt
\ex
\label{ex:3-47c}
der \textit{Mann} hat den Jungen gestreichelt
\ex
\label{ex:3-47d}
er hat den \textit{Jungen} gestreichelt
\ex
\label{ex:3-47e}
den \textit{Jungen}
\end{xlist}
\end{exe}
müßte (\ref{ex:3-47b}) zumindest als
\textsq{natürliche} Antwort auf (\ref{ex:3-47a}) gelten können. Das ist im Allgemeinen jedoch nicht der Fall; typischerweise wird in \textsq{natürlichen} Antworten pronominalisiert (\ref{ex:3-47d}) oder
überhaupt nur die erfragte \isi{Konstituente} angegeben (\ref{ex:3-47e}), so daß (\ref{ex:3-47a}) als \textsq{Test} für (\ref{ex:3-47b}) entfällt. Vgl.\ auch Fn.\,\ref{fn:3-7}.

Trotzdem können solche Fragen in einfachen Fällen die ihnen zugedachte
Aufgabe erfüllen; nicht als \textsq{Test} für einen vorgegebenen
Satz, sondern dadurch, daß der verständige Leser aus der Frage
entnimmt, was Teil des RK sein soll und was als \textsq{neue Information} gelten kann.

\subsubsection{Verschiedene Fokusmöglichkeiten}
\label{subsubsec:3-1-3-2}

Kehren wir noch einmal zu den Beispielen in (\ref{ex:3-31}) zurück; ich wiederhole sie hier einzeln mit neuer Numerierung. Wenn wir (\ref{ex:3-48}) (= (\ref{ex:3-31c})) näher betrachten, zeigt sich,  daß der Satz nicht nur in Kontexten geäußert
werden kann, in denen, wie die Frage (\ref{ex:3-49a}) andeutet, allein \textit{das Buch}
\isi{Fokus} ist. Vielmehr hat (\ref{ex:3-48}) alle in (\ref{ex:3-50}) angegebenen möglichen Foki,
entsprechend den Fragen (\ref{ex:3-49}a--e);\footnote{\label{fn:3-14}%
  Griesbach behauptet, daß bei minimalem \isi{Fokus} wie in (\ref{ex:3-50a})~-- im
  Gegensatz zu einem maximalen \isi{Fokus} wie (\ref{ex:3-50e})~-- das \isi{Akkusativobjekt}
  "`beim Sprechen [\ldots] hervorgehoben"' werde. Dafür sehe ich keine
  empirische Grundlage. Möglicherweise gibt es spezielle
  Intonationsmittel, um in solchen Fällen den minimalen \isi{Fokus} als
  solchen auszuzeichnen. Sie sind jedoch keinesfalls obligatorisch, und
  es gilt durchaus nicht nur für die "`geschrieben[e] Aussage"', daß "`nur
  der Zusammenhang Aufschluß geben [kann], wie dieser Satz gemeint ist"'
  (\citealt{Griesbach1961a}~(IV): 85).%
}
hingegen
\begin{exe}
\ex
\label{ex:3-48}
Karl hat dem Kind das \textit{Buch} geschenkt
\ex
\label{ex:3-49}
\begin{xlist}
\ex
\label{ex:3-49a}
was hat Karl dem Kind geschenkt?
\ex
\label{ex:3-49b}
was hat Karl hinsichtlich des Kindes getan?
\ex
\label{ex:3-49c}
was hat Karl getan?
\ex
\label{ex:3-49d}
was hat das Kind erlebt?
\ex
\label{ex:3-49e}
was ist geschehen?
\ex
\label{ex:3-49f}
wer hat dem Kind das Buch geschenkt?
\ex
\label{ex:3-49g}
wem hat Karl das Buch geschenkt?
\ex
\label{ex:3-49h}
was hat Karl hinsichtlich des Kindes mit dem Buch getan?
\ex
\label{ex:3-49i}
was ist hinsichtlich des Kindes mit dem Buch geschehen?
\ex
\label{ex:3-49j}
was hat Karl mit dem Buch gemacht?
\ex
\label{ex:3-49k}
wem hat Karl was geschenkt?
\end{xlist}
\ex
\label{ex:3-50}
\begin{xlist}
\ex
\label{ex:3-50a}
Fk\sub{1} (\ref{ex:3-48}) = das Buch
\ex
\label{ex:3-50b}
Fk\sub{2} (\ref{ex:3-48}) = das Buch + geschenkt
\ex
\label{ex:3-50c}
Fk\sub{3} (\ref{ex:3-48}) = dem Kind + das Buch + geschenkt
\ex
\label{ex:3-50d}
Fk\sub{4} (\ref{ex:3-48}) = Karl + das Buch + geschenkt
\ex
\label{ex:3-50e}
Fk\sub{5} (\ref{ex:3-48}) = MK (\ref{ex:3-48}) = Karl + dem Kind + das Buch + geschenkt
\end{xlist}
\end{exe}
paßt (\ref{ex:3-48}) nicht in die durch (\ref{ex:3-49}f--k) angedeuteten
Kontexte. (Insbesondere paßt (\ref{ex:3-48}) nicht, wie \citet[523]{Bartsch76}
behauptet, zu (\ref{ex:3-49k})).

Dabei ist der vierte \isi{Fokus} (\ref{ex:3-50d}) besonders wichtig, weil er zeigt,
daß ein \isi{Fokus} nicht immer eine \isi{Konstituente} des Satzes ist, wie
\zb in \citet{Chomsky76a} angenommen: Während die übrigen Foki in
einigen Theorien über Konstituentenstrukturen deutscher\il{Deutsch} Sätze als
Konstituenten analysiert werden können, ist dies bei (\ref{ex:3-50d})
ausgeschlossen.\footnote{\label{fn:3-15}%
	Für Chomskys Zusammenhänge in \citet{Chomsky76a} ist das
  besonders wichtig, weil sich seine Argumentation, der \isi{Fokus} müsse an der Oberflächenstruktur
  bestimmt werden, im wesentlichen auf die Annahme stützt, daß der \isi{Fokus} eine \isi{Konstituente} ist.%
}
Um sicherzustellen, daß dies wirklich ein möglicher \isi{Fokus} von (\ref{ex:3-48}) ist, sei noch eine Kontrastierung wie (\ref{ex:3-51b}) vs.\ (\ref{ex:3-51c}) angegeben, wo die nicht"=identischen Teile der Sätze offensichtlich nicht aus RK bekannt sind:
\begin{exe}
\ex
\label{ex:3-51}
\begin{xlist}
\ex
\label{ex:3-51a}
warum freut sich das Kind so?
\ex
\label{ex:3-51b}
hat Karl dem Kind (schon) das \textit{Buch} geschenkt?
\ex
\label{ex:3-51c}
oder hat das Kind heute \textit{schulfrei}?
\end{xlist}
\end{exe}
Im Fall von (\ref{ex:3-50e}) haben wir als \isi{Fokus} die Menge aller Konstituenten von (\ref{ex:3-48}), notiert als "`MK (\ref{ex:3-48})"'.

Wenn wir (\ref{ex:3-52}) (= (\ref{ex:3-31a})) betrachten, finden wir, daß unter allen in
(\ref{ex:3-49}) angedeuteten Kontexten einzig (\ref{ex:3-49f}) passend ist, so daß Fk\sub{1} (\ref{ex:3-52})
= \textit{Karl} der einzig mögliche \isi{Fokus} ist. Ähnlich bei (\ref{ex:3-53}) (= (\ref{ex:3-31b}));
unter allen Fragen in (\ref{ex:3-49}) paßt nur (\ref{ex:3-49g}), so daß Fk\sub{1} (\ref{ex:3-53}) = \textit{dem Kind}
der einzig mögliche \isi{Fokus} ist:
\begin{exe}
\ex \label{ex:3-52}
\textit{Karl} hat dem Kind das Buch geschenkt
\ex
\label{ex:3-53}
Karl hat dem \textit{Kind} das Buch geschenkt
\end{exe}
Bei (\ref{ex:3-54}) (= (\ref{ex:3-31d})) haben wir (\ref{ex:3-49h}) als passende Frage, also \isi{Fokus}
wie in (\ref{ex:3-55a}) angegeben, aber auch (\ref{ex:3-49i}) ist ein passender Kontext,
wie man an der Kontrastierung in (\ref{ex:3-56}) überprüfen kann; daher (\ref{ex:3-55b}):
\begin{exe}
\ex
\label{ex:3-54}
Karl hat dem Kind das Buch \emph{geschenkt}
\ex
\label{ex:3-55}
\begin{xlist}
\ex
\label{ex:3-55a}
Fk\sub{1} (\ref{ex:3-54}) = geschenkt
\ex
\label{ex:3-55b}
Fk\sub{2} (\ref{ex:3-54}) = Karl + geschenkt
\end{xlist}
\ex
\label{ex:3-56}
es ist nicht so, daß Karl dem Kind das Buch \textit{geschenkt} hat;
vielmehr hat das Kind das Buch \textit{gefunden}
\end{exe}
Zum Abschluß dieser Übersicht wollen wir noch (\ref{ex:3-57}) betrachten; ein
Beispiel, das bisher nicht vorgekommen ist. Hier passen außer (\ref{ex:3-49k})
auch (\ref{ex:3-49c}) und (\ref{ex:3-49e}), daher (\ref{ex:3-58}):
\begin{exe}
\ex
\label{ex:3-57}
Karl hat dem \textit{Kind} das Buch geschenkt
\ex
\label{ex:3-58}
\begin{xlist}
\ex
\label{ex:3-58a}
Fk\sub{1} (\ref{ex:3-57}) = dem Kind + das Buch
\ex
\label{ex:3-58b}
Fk\sub{2} (\ref{ex:3-57}) = dem Kind + das Buch + geschenkt (= Fk\sub{3} (\ref{ex:3-48}))
\ex
\label{ex:3-58c}
Fk\sub{3} (\ref{ex:3-57}) = MK (\ref{ex:3-57}) = Karl + dem Kind + das Buch + geschenkt\\
\hphantom{Fk\sub{3} (\ref{ex:3-57}) =MK (\ref{ex:3-57}) =} (= Fk\sub{5} (\ref{ex:3-48}))
\end{xlist}
\end{exe}
Es zeigt sich, daß die meisten Fragen in (\ref{ex:3-49}) nur zu je einem Beispiel passen, einige aber~-- \zb (\ref{ex:3-49}c,e)~-- zu mehreren (nämlich zu
(\ref{ex:3-48}) und (\ref{ex:3-57})). (\ref{ex:3-49j}) paßt zu keinem bisher diskutierten Beispiel;
wir kommen bei der Besprechung von (\ref{ex:3-98}) (= (\ref{ex:3-4}) in (\ref{ex:3-4})) darauf zurück.

\subsubsection{Explikation von "`stilistisch normale Betonung"'}
\label{subsubsec:3-1-3-3}

Wir haben gesehen, daß verschiedene Sätze, die außer in ihrer \isi{Betonung}
identisch sind, verschiedene mögliche Foki haben. Manche haben
mehrere. Dabei legt zum einen rein definitorisch~-- cf.\ (\ref{ex:3-32}) und
(\ref{ex:3-37})~-- jeder \isi{Fokus} eines Satzes fest, in welchen Kontexttypen der
Satz vorkommen kann; wenn zwei Sätze S\sub{\textit{i}}, S\sub{\textit{j}} logisch identisch sind,
aber S\sub{\textit{i}} \textit{n} und S\sub{\textit{j}} \textit{m} mögliche Foki hat, für \textit{n} $>$ \textit{m}, dann kann
S\sub{\textit{i}} in mehr verschiedenen Kontexttypen vorkommen als S\sub{\textit{j}}. Zum
anderen haben wir gesehen, daß Art und Zahl der möglichen Foki eines
Satzes wesentlich von seiner \isi{Betonung} beeinflußt werden, und zwar in
der Weise, daß die betonten Konstituenten immer Teil des \isi{Fokus} sind
und unbetonte dazu kommen können. Diesen durch mögliche Foki
vermittelten Zusammenhang zwischen möglichen Kontexttypen und \isi{Betonung}
will ich zur Explikation von \textsq{Normalbetonung}
nutzen.\footnote{\label{fn:3-16}%
	Ich beachte durchweg nur die \isi{Betonung} (den Druckakzent), nicht das
  Intonationsmuster (den Tonhöhenverlauf). In dem hier interessierenden
  Phänomenbereich ist allein die \isi{Betonung} topologisch relevant;
  verschiedene Intonationen beeinflussen die Wortstellungsmöglichkeiten
  im Allgemeinen nicht. Auch unterscheide ich nicht zwischen
  \textsq{emphatischer}, \textsq{kontrastiver} usw.\ \isi{Betonung}: Jeder
  gewählte \isi{Fokus} kann für den Zweck einer \textsq{normalen Mitteilung},
  einer emphatischen (\zb empörten) \isi{Hervorhebung}, einer
  Kontrastierung gebraucht werden. (Allerdings scheint es gewisse
  Intonationen zu geben, die nur für kontrastierende Verwendung geeignet
  sind.)

  Es ist bemerkenswert, daß Contreras aus völlig unabhängigen Gründen
  ebenso verfährt: "`\ldots{} even though the intonational contours of a
  Mexican and those of an Argentinian may differ widely, the location of
  the main sentential stress, which may be signaled by different
  phonetic devices, seems to be conditioned by the same factors of
  rhematic structure"' \citep[105]{Contreras76}.%
}

Dabei gehe ich von der Annahme aus, daß die Regeln der Satzgrammatik
einen Satz S\sub{\textit{i}} dadurch beschreiben, daß sie ihm (mindestens) (a)
eine phonologische, (b) eine morphologische, (c) eine syntaktische,
  (d) eine logische und (e) eine pragmatische Charakterisierung
  zuweisen. Für die Zwecke dieses Aufsatzes identifiziere ich die
  pragmatische Charakterisierung (PC) mit der Angabe der Menge der
  möglichen Foki von S\sub{\textit{i}} (notiert als "`MF (S\sub{\textit{i}})"'). Die PC ist nicht
  Teil der logischen Charakterisierung (LC).

In der einschlägigen Literatur gibt es ähnliche Vorstellungen, \zb bei \citet{Bartsch76} und bei \citet{Rochemont78}; zahlreiche Autoren weichen jedoch davon ab. Viele möchten die Betonungsverhältnisse im Satz aufgrund syntaktischer \mbox{und}/""oder pragmatischer Regularitäten voraussagen. Wenn man dies versucht, ist nicht leicht zu sehen, wie es ohne Bezug auf den Kontext geschehen kann. Ich nehme demgegenüber an, daß die \isi{Betonung} im Satz frei ist und daß die Effekte verschiedener Betonungen durch Regeln für die PC erfaßt werden.~-- Viele Autoren verbauen sich die Möglichkeit, die PC innerhalb der Satzgrammatik zu behandeln, indem sie nur aktuelle Foki von Äußerungen betrachten. Sobald man, wie wir es tun, die möglichen Foki von Sätzen betrachtet, entfällt dieses Problem; in dem Maße, wie eine adäquate Explikation von "`\isi{Normalbetonung}"' und "`normaler \isi{Wortstellung}"' speziell von der Betrachtung \textit{möglicher} Foki abhängig ist, ist dieses Verfahren auch das einzig adäquate.~-- Aus ganz anderen Gründen wendet \citet{Verhagen79} sich dagegen, die PC innerhalb der Satzgrammatik zu behandeln. Ich halte seine Gründe nicht für stichhaltig, kann aus Raumgründen jedoch nicht darauf eingehen.

\largerpage
  Für die unmittelbaren Zwecke dieses Aufsatzes ist es nicht
  entscheidend, auf der \textit{formalen} Trennung von LC und PC~-- die von
  vielen bestritten wird~-- zu bestehen. Jedoch hat die Unterscheidung
  von Logik und Pragmatik, gerade in dem hier betrachteten Teilgebiet,
  so fundamentale Bedeutung für die allgemeine Sprachtheorie und die
  Organisation der Grammatik, daß ich auf die \textit{inhaltlichen}
  Unterschiede zwischen LC und PC etwas eingehen will. Dies ist umso
  wichtiger, als die empirischen und methodologischen Argumente für
  oder gegen die Trennung von LC und PC in der Literatur nicht ganz
  deutlich zu sein scheinen.

  Aus der Definition von \isi{Fokus} und \isi{Topik} in (\ref{ex:3-32}) und (\ref{ex:3-37}) folgen
  einige wesentliche Eigenschaften dieser Unterscheidung. Das
  wichtigste ist: Die Unterscheidung ist an Kontextbedingungen,
  \dash an \textsq{Vorwissen} von Sprecher und Hörer geknüpft. Dies
  unterscheidet sie grundsätzlich von allem, was man als genuin
  logische Eigenschaften von Sätzen (Wahrheitsbedingungen)
  betrachtet. Vollkommen deutlich ist das bei den Beispielen in (\ref{ex:3-31}):
  Sie alle haben die LC (\ref{ex:3-38}), \dash dieselben (durch Syntax und
  morphologisches Material festgelegten) Wahrheitsbedingungen; aber
  ihre Gebrauchsbedingungen (ihre Möglichkeiten, ohne Verletzung
  konversationeller Maximen geäußert zu werden) sind völlig
  verschieden. Es hieße, eine wesentliche Unterscheidung zu
  verwischen, wenn man Gebrauchsbedingungen solcher Art in gleicher
  Weise wie Wahrheitsbedingungen darstellen wollte.

  In (\ref{ex:3-37}) haben wir 2 Fokustypen unterschieden: \textit{W}"=\isi{Fokus} und
  semantischen \isi{Fokus}: Es liegt auf der Hand, daß ein \textit{W}"=\isi{Fokus} nicht
  als Teil der logischen Charakterisierung eines Satzes dargestellt
  werden kann, da es bei ihm wesentlich um die phonetische Form eines
  Satzbestandteils geht. Ansonsten liegt aber das gleiche Phänomen wie
  beim semantischen \isi{Fokus} vor: kontextrelative Gebrauchsbedingungen
  und Determiniertheit durch die \isi{Betonung}. Wenn ein \textit{W}"=\isi{Fokus} durch
  eine von der LC verschiedene pragmatische Komponente charakterisiert
  wird, sollte auch ein semantischer \isi{Fokus} von dieser Komponente
  charakterisiert werden.

\addlines[2]
  Aufgrund der Definition von \textsq{Fokus} sind bei verschieden
  betonten, aber syntaktisch und \isi{morphologisch} identischen Sätzen
  im Allgemeinen keine logischen Unterschiede zu erwarten. Allerdings wird
  durch die Operation der Fokusregeln bei einem semantischen \isi{Fokus} die
  LC in gewisser Weise modifiziert: indem eins oder mehrere ihrer
  Konjunkte indirekt dem \isi{Fokus} zugeordnet werden und alle anderen dem
  \isi{Topik}. Es ist daher nicht grundsätzlich ausgeschlossen, daß mit
  verschiedenen Foki auch verschiedene Wahrheitsbedingungen verknüpft
  sind. Insofern sagt die bloße Beobachtung, daß zwei \isi{morphologisch}
  und syntaktisch gleiche Sätze je nach \isi{Betonung} verschiedene logische
  Eigenschaften haben (oder zu haben scheinen), als solche wenig
  aus. Solche Unterschiede sind jedoch nur zu erwarten, wenn sie auf
  allgemeine Regeln zur Fokusbestimmung zurückgeführt werden können,
  insbesondere also wenn sie daraus folgen, daß ein Teil des Satzes
  aus RK bekannt ist. Es sind keinerlei Unterschiede zu erwarten, die
  den Unterschieden zwischen verschiedenen Morphemen entsprechen oder
  der Unterscheidung verschiedener syntaktischer Relationen zu einem
  \isi{Verb}; zu rechnen ist in gewissen Fällen \zb mit Effekten, die man
  als Skopusunterschiede von Quantoren darstellen kann; vgl.\ etwa
  \citet[45]{Rochemont78}. In dem Maße, wie es gelingt, vorhandene oder
  angebliche logische Unterschiede zwischen solchen Sätzen auf
  Unterschiede in den möglichen Kontexttypen zurückzuführen, finden
  diese logischen Unterschiede also eine \textit{Erklärung}~-- was nicht der
  Fall wäre, wenn man sie als logische Eigenschaften wie andere in der
  LC darstellen würde.

Wichtig ist das u.\,a.\ bei Fällen wie (\ref{ex:3-59}a,b). Man kann diese Sätze verwenden, um Karl Annahmen zuzusprechen (bzw.\ abzusprechen), die in
einem intuitiven Sinne verschieden sind; insofern mag man sie als logisch verschieden betrachten. Bei (\ref{ex:3-59a}) glaubt Karl
\begin{exe}
\ex
\label{ex:3-59}
\begin{xlist}
\ex
\label{ex:3-59a}
Karl glaubt nicht, daß man \textit{Heinz} entlassen hat
\ex
\label{ex:3-59b}
Karl glaubt nicht, daß man Heinz ent\textit{lassen} hat
\end{xlist}
\end{exe}
möglicherweise, daß jemand entlassen wurde, aber nicht, daß dies Heinz
war; bei (\ref{ex:3-59b}) glaubt Karl eventuell, daß mit Heinz etwas geschehen ist,
aber nicht, daß er entlassen wurde. (Über die allgemeinen Probleme bei
opaken Kontexten vgl.\ \citealt{Hoehle79c}). Diese Unterschiede folgen jedoch aus der von uns vorgesehenen pragmatischen Charakterisierung: Da im Allgemeinen die \isi{Negation} zum \isi{Fokus} gehört, ist Fk (\ref{ex:3-59a}) = \textit{nicht} + \textit{Heinz} und Fk (\ref{ex:3-59b}) = \textit{nicht} + \textit{entlassen}; der Rest des Satzes ist jeweils \isi{Topik}. Das bedeutet, daß bei Äußerung von (\ref{ex:3-59a}) ein Äquivalent von (\ref{ex:3-60a}) in RK sein muß und bei Äußerung von (\ref{ex:3-59b}) ein Äquivalent von (\ref{ex:3-60b}); die Verschiedenartigkeit der Annahmen, die Karl durch
\begin{exe}
\ex
\label{ex:3-60}
\begin{xlist}
\ex
\label{ex:3-60a}
Karl glaubt, daß man jemand entlassen hat
\ex
\label{ex:3-60b}
Karl glaubt, daß man mit Heinz etwas gemacht hat
\end{xlist}
\end{exe}
(\ref{ex:3-59}a,b) zugesprochen werden, folgt daraus unmittelbar.

Dieses Verhältnis zwischen Logik und Pragmatik wird häufig nicht
beachtet. So möchten \zb Sgall et al.\ (\citealt{Sgall73}; \citealt[16ff]{Sgall77}) zeigen, daß Fokusunterscheidungen als Teil der Wahrheitsbedingungen behandelt werden müssen. Ihre eigene Diskussion zeigt aber, daß bei den meisten ihrer Beispiele die verschiedenen kommunikativen Effekte der Sätze aus rein pragmatischen Bedingungen folgen und gerade nicht als verschiedene
Wahrheitsbedingungen dargestellt werden dürfen.

Bo{\"e}r behauptet (in einer Arbeit, in der er ansonsten versucht,
behauptete logische Unterschiede zwischen verschieden betonten Sätzen
auf pragmatische Unterschiede zurückzuführen), daß zwischen Sätzen
wie (\ref{ex:3-61}a,b) (i) eine echte (nicht"=pragmatische) Bedeutungsdifferenz
bestehe, der (ii) ein syntaktischer Unterschied in ihren Tiefenstrukturen entspricht, der sich (iii) infolge automatischer (von logischen und pragmatischen Faktoren unabhängiger) Regeln in
verschiedenen Betonungen niederschlägt \citep[276]{Boer79}:
\begin{exe}
\ex
\label{ex:3-61}
\begin{xlist}
\ex
\label{ex:3-61a}
John \textit{shot} the horse by mistake
\ex
\label{ex:3-61b}
John shot the \textit{horse} by mistake
\end{xlist}
\end{exe}
Demgegenüber scheint mir, daß es nicht offensichtlich ist, daß (i) (\ref{ex:3-61}a,b) verschiedene Wahrheitsbedingungen haben, noch (ii) daß sie
verschiedene Tiefenstrukturen haben. Es spricht wohl nichts dafür, daß
sie verschiedene Oberflächenstrukturen haben; und (iii) für die
Annahme, daß die Betonungsunterschiede von (\ref{ex:3-61}a,b) ein automatischer Reflex verschiedener Tiefenstrukturen seien, spricht erst recht
nichts. Dagegen ist es offensichtlich, daß Fk (\ref{ex:3-61a}) = \textit{shot} und Fk
(\ref{ex:3-61b}) = \textit{the horse}; demzufolge muß bei einer Äußerung von (\ref{ex:3-61a}) ein Äquivalent von (\ref{ex:3-62a}) und bei Äußerung von (\ref{ex:3-61b}) ein Äquivalent von (\ref{ex:3-62b}) in RK
\begin{exe}
\ex
\label{ex:3-62}
\begin{xlist}
\ex
\label{ex:3-62a}
John hat versehentlich etwas mit dem Pferd gemacht
\ex
\label{ex:3-62b}
John hat versehentlich ein Lebewesen erschossen
\end{xlist}
\end{exe}
sein. Die "`strongly felt meaning-difference"' \citep[275]{Boer79}
zwischen (\ref{ex:3-61}a,b) ist danach zu erwarten. (Andere von Bo{\"e}r diskutierte Beispiele sind z.\,T.\ wesentlich komplizierter. Seine pragmatischen Erklärungsversuche scheinen mir im Prinzip richtig zu sein, wenn sie auch mittels der pragmatischen Fokustheorie präzisiert und vereinfacht werden müßten.)

\addlines[2]
Es ist bekannt, daß je nach \isi{Betonung} die Bezugsmöglichkeiten
anaphorischer Elemente verschieden sein können; vgl.\ \zb \citet{Akmajian70}, \citet{AkmaJack70}. Mindestens für die typischen Fälle von pronominalen Anaphern von NPs gibt es aber ohnehin
gute Gründe, ihre Interpretation nicht innerhalb der LC zu kennzeichnen; intendierte \isi{Koreferenz} gehört daher (im Gegensatz zu grammatisch notwendiger \isi{Koreferenz} wie bei Reflexiv- und
Reziprokpronomen) im Allgemeinen nicht zu den grammatisch relevanten logischen
Eigenschaften von Sätzen.\footnote{\label{fn:3-17}%
	Dies gilt insbesondere auch für Fälle wie (i) und (ii), wo nach Meinung vieler Logiker bei intendierter \isi{Koreferenz} (a) die \isi{Koreferenz}
  explizit dargestellt werden muß und wo dies (b) nur durch eine
  gebundene Variable geschehen kann; vgl.\ \zb \citet{Evans80}.
  \eal
  \label{ex:3-i1}
  \ex
  \label{ex:3-ia}
  jeder Engländer liebt seine Mutter
  \ex
  \label{ex:3-ib}
  wer liebt seine Mutter?
  \ex
  \label{ex:3-ic}
  keiner liebt seine Mutter
  \zlmid
  \eal
  \label{ex:3-ii1}
  \ex
  \label{ex:3-iia}
  jeder ist glücklich, wenn er singt
  \ex
  \label{ex:3-iib}
  wer hat behauptet, daß er krank ist?
  \ex
  \label{ex:3-iic}
  keiner von uns glaubt, daß er im Recht ist
  \zl
  Die Annahme (a) beruht m.\,E.\ auf einer Verkennung der Aufgaben einer
  sprachwissenschaftlichen Semantik, (b) auf inkonsequenter Anwendung
  der Quantifikationstheorie. Vgl.\ zu diesem Komplex \citeauthor{Hoehle79c} (in Vorb.).%
  } 
Etwas anders ist es dort, wo
\isi{Koreferenz} zwischen einer \isi{NP} und einem \isi{Pronomen} ausgeschlossen
ist. Hier gibt es teilweise grammatische Bedingungen. Chomsky macht in
\citet[340--345]{Chomsky76b} geltend, daß solche Bedingungen in wenigstens einem Fall
auch logischer Natur sind: In (\ref{ex:3-63a}) ist \isi{Koreferenz} zwischen \textit{he} und
\textit{John} möglich, nicht aber in (\ref{ex:3-63b}) zwischen \textit{he} und \textit{someone}. Chomsky
nimmt an, daß \textit{he} in dieser syntaktischen Konfiguration nicht koreferent
mit einer logischen Variablen sein kann. Da auch in (\ref{ex:3-63c}), wo \textit{John}
\isi{Fokus} ist, \isi{Koreferenz}
\eal
\label{ex:3-63}
\ex
\label{ex:3-63a}
the woman he loved \textit{betrayed} John
\ex
\label{ex:3-63b}
the woman he loved \textit{betrayed} someone
\ex
\label{ex:3-63c}
the woman he loved betrayed \textit{John}
\zl
zwischen \textit{he} und \textit{John} ausgeschlossen sein soll,
schließt er, daß der \isi{Fokus} in der LC als logische Variable zu
kennzeichnen sei. Hier ist jedoch eine rein pragmatische Analyse der
Daten mindestens ebenso gut möglich: In Kontexten wie diesen muß die
Referenz von \textit{he} durch den RK gegeben sein.\footnote{\label{fn:3-18}%
	In (\ref{ex:3-63}) haben wir \textsq{Rückwärts"=Pronominalisierung} in einen
  eingebetteten Satz hinein. Offenbar gilt die gleiche Beschränkung für
  \textsq{Vorwärts"=Pronominalisierung} aus einem eingebetteten Satz
  heraus: In (i) kann \textit{ihn} eine \isi{Anapher} des Subjekts des Relativsatzes
  sein, in (ii) nicht (bzw.\ wie bei (\ref{ex:3-63}) nur dann, wenn \isi{Pronomen} und \textsq{Antezedens-NP} in RK sind).
  \ea
  \label{ex:3-fn18i}
  die Frau, die Karl \textit{geliebt} hat, hat ihn \textit{betrogen}
  \zmid 
  \eal
  \label{ex:3-fn18ii}
  \ex
  \label{ex:3-fn18iia}
  die Frau, die jemand/""jeder/""niemand \textit{geliebt} hat, hat ihn \textit{betrogen}
  \ex
  \label{ex:3-fn18iib}
  die Frau, die \textit{Karl} geliebt hat, hat ihn \textit{betrogen}
  \zllast%
}
Diese Bedingung ist in (\ref{ex:3-63b}) nur gegeben, wenn sich \textit{he} nicht auf
\textit{someone} bezieht, da dies~-- jedenfalls in der naheliegendsten
Interpretation~-- zum \isi{Fokus} gehört: Fk (\ref{ex:3-63b}) = \textit{betrayed} +
\textit{someone}. In (\ref{ex:3-63a}) ist die Bedingung erfüllt, da Fk (\ref{ex:3-63a}) =
\textit{betrayed} and demzufolge eine Relation zwischen \textsq{the
  woman he loves} und John in RK sein muß. Sie ist in (\ref{ex:3-63c}) nicht erfüllt, wenn der Satz ohne Kontext gegeben ist: Da Fk (\ref{ex:3-63c}) =
\textit{John}, kann unter dieser Voraussetzung die Referenz von
\textit{he} nicht in RK durch \textit{John} festgelegt sein. In einem Kontext
dagegen, wie er durch (\ref{ex:3-64}) (= \citealt[101 (25)]{Rochemont78}) gegeben
ist, ist die Referenz von \textit{he} = \textit{John} bereits in RK
gegeben, so daß in diesem Fall \isi{Koreferenz} zwischen \textit{he} und
\textit{John} in (\ref{ex:3-64b}) möglich ist, obwohl \textit{John} den \isi{Fokus} bildet:
\eal
\label{ex:3-64}
\ex
\label{ex:3-64a}
A: \hspace{1ex} Sally and the woman John loves are leaving the country today\\
B: \hspace{1ex} I thought that the woman he loves had \textit{betrayed} Sally
\ex
\label{ex:3-64b}
A: \hspace{1ex} No~-- the woman he loves betrayed \textit{John}; Sally and she are
the best of friends\\
\zl
Wir werden anhand von Beispielen wie (\ref{ex:3-65}) in \ref{subsubsec:3-1-4-4} sehen, welchen
erklärenden Gehalt die rein pragmatische Fokustheorie (\ref{ex:3-37}) in
Verbindung mit der Theorie konversationeller Implikaturen hat.
\ea
\label{ex:3-65}
John called Bill a Republican, and then \textit{he} insulted \textit{him}
\z
Für einige einfache Fälle formuliere ich im folgenden Hypothesen zur
Generierung der PC. Zweifellos erscheinen diese übermäßig
\textsq{oberflächlich}; man würde sich Aussagen wünschen, von denen
diese Hypothesen als Theoreme ableitbar wären. Angesichts der
Tatsache, daß es in diesem Gebiet so gut wie gar keine sicheren
Forschungsergebnisse, aber einen Wust an terminologischen Unklarheiten
und loser Spekulation über die Fakten gibt, scheint mir das hier
gewählte Verfahren vorteilhaft.
\ea
\label{ex:3-66}
\textit{Hypothese:}\\
Dann und nur dann wenn ein Satz S\sub{\textit{i}} \textit{n} betonte Elemente BK enthält,
\emph{n}~$\geq$~1, bilden diese BK einen möglichen \isi{Fokus}. Dies ist der
\textit{minimale}\\
Fk\sub{1} (S\sub{\textit{i}}).
\z
Diese Hypothese legt sich aufgrund der Beispiele des vorigen
Abschnitts nahe; insbesondere müssen in (\ref{ex:3-57}) beide betonten
Konstituenten im minimalen \isi{Fokus} sein. Einen minimalen \isi{Fokus} gibt es
(fast) immer als \textit{W}"=\isi{Fokus}; als semantischer \isi{Fokus} ist er naturgemäß
nur möglich, wenn die betonte \isi{Konstituente} selbst bedeutungstragend
ist, also nicht \zb bei Teilen von idiomatischen Ausdrücken wie in
(\ref{ex:3-67}):
\begin{exe}
\ex
\label{ex:3-67}
\begin{xlist}
\ex
\label{ex:3-67a}
das ist ihm \textit{ein}gefallen?~--  nein, das ist ihm \textit{auf}gefallen
\ex
\label{ex:3-67b}
hat sie ihr den \textit{Hof} gemacht?~--  nein, sie hat ihr den \textit{Garaus} gemacht
\end{xlist}
\end{exe}
\citet[192f]{Kempson75} verknüpft~-- allerdings in Hinsicht auf
sehr spezielle Beispiele~-- \textsq{betonte Konstituente} und
\textsq{möglichen Kontexttyp} sogar unmittelbar
miteinander. Allerdings ist (\ref{ex:3-66}) weniger trivial, als es scheinen
könnte: Weder \citet{Bartsch76} noch \citet{Smyth79} haben die
zugrundeliegende Regularität erkannt.\footnote{\label{fn:3-19}%
  Die Hypothese (\ref{ex:3-66}) gilt in dieser Formulierung für das Deutsche\il{Deutsch}, Englische\il{Englisch}, Spanische\il{Spanisch}
  und vermutlich für viele andere Sprachen; man ist geneigt, sie für ein Universale zu halten. Nach
  \citet{Schauber78} ist dies jedoch nicht der Fall: Im Navajo wird der \isi{Fokus} durch die Position
  verschiedener Affixe festgelegt, offenbar ohne daß dabei die \isi{Betonung} eine Rolle spielt. (Ähnlich
  scheint es im Ojibwa zu sein: "`Since there are \textit{no supersegmental devices}, nor special
  morphemes for marking contrastive NPs, Ojibwa is essentially forced to use its word order to
  mark the contrastiveness of NPs"' (\citet[317]{Tomlin79}; \isi{Hervorhebung} von mir)).

  Falls dies zutrifft, ist (\ref{ex:3-66}) allenfalls ohne die Klausel "`dann und nur dann"' ein Universale.%
}
\begin{exe}
\ex
\label{ex:3-68}
\textit{Hypothese:}\\
Wenn ein Satz S\sub{\textit{i}} einen möglichen \isi{Fokus} Fk\sub{\textit{n}} (S\sub{\textit{i}}) hat, Fk\sub{\textit{n}} (S\sub{\textit{i}}) $\neq$ Fk\sub{1} (S\sub{\textit{i}}), dann sind u.\,a.\ die BK in Fk\sub{\textit{n}} (S\sub{\textit{i}})
enthalten.
\end{exe}
Aus (\ref{ex:3-68}) und (\ref{ex:3-66}) folgt, daß die Fk\sub{\textit{n}} (S\sub{\textit{i}}) von (\ref{ex:3-68}) unbetonte,
aber nicht nur unbetonte Konstituenten enthalten. Die BK sind also
(echter oder unechter) Teil aller möglichen Foki. Gesamtsätze
(sentences) haben, wie es scheint, immer mindestens 1 voll betonte
\isi{Konstituente}. Für Teilsätze (clauses) gilt das jedoch nicht,
cf.\ (\ref{ex:3-69}). Folglich hat der \textit{daß}"=Satz in (\ref{ex:3-69}) keinen möglichen
\isi{Fokus}. Bei den Erörterungen, die folgen, habe ich nur einfache
\ea
\label{ex:3-69}
ich \textit{glaube} nicht daß Heinz den Hund gebissen hat
\z
Sätze im Auge. Ebenso sehe ich von den Verhältnissen innerhalb von
Nominal"= und Präpositionalphrasen ab.

Um die Tatsache terminologisch hervorzuheben, daß nicht"=minimale Foki
aus einem minimalen \isi{Fokus} plus zusätzlichen Konstituenten bestehen,
ähnlich wie sich im einfachen Fall die LC einer \isi{Konstituente} aufgrund
von Projektionsregeln kompositionell aus den LC ihrer Subkonstituenten
ergibt (zu Einschränkungen des strikten Kompositionalitätsprinzips
vgl.\ \citealt{Hoehle79c}), führen wir den Begriff
\textsq{Fokusprojektion} ein:
\begin{exe}
	\ex \textit{Definition:}\label{ex:3-70}
\begin{xlist}
\ex
\label{ex:3-70a}
Im Fall von (\ref{ex:3-68}) liegt eine \textit{Fokusprojektion} (von Fk\sub{1}
(S\sub{\textit{i}}) zu Fk\sub{\textit{n}} (S\sub{\textit{i}})) vor.  
\ex
\label{ex:3-70b}
Bedingungen, unter denen eine \isi{Fokusprojektion} vorliegt, sind
\textit{fokusprojektiv}. 
\ex
\label{ex:3-70c}
Unter allen möglichen Foki eines Satzes S\sub{\textit{i}} sind die Foki
\textit{maximal} (= Fk\sub{m} (S\sub{\textit{i}})), die die meisten Konstituenten
enthalten.  
\ex
\label{ex:3-70d}
Wenn in S\sub{\textit{i}} Fk\sub{m} (S\sub{\textit{i}}) = Fk\sub{1} (S\sub{\textit{i}}), liegen
nicht"=fokusprojektive Bedingungen vor.
\end{xlist}
\end{exe}
Das fundamentale Problem bei allen Untersuchungen zu Fokusphänomenen
ist es, Regeln zur \isi{Fokusprojektion} zu finden. Ein besonders
interessantes Teilproblem können wir etwas schärfer formulieren, wenn
wir, in Anlehnung an \citet[307f]{Fuchs76}, einen Begriff
\textsq{Fokusexponent} einführen:
\begin{exe}
	\ex\label{ex:3-71}
\textit{Definition:} \\
Innerhalb einer komplexen \isi{Konstituente} K\sub{\textit{i}} ist die \isi{Konstituente} K\sub{\textit{j}}
der \textit{Fokusexponent} von K\sub{\textit{i}}, für die gilt: Wenn K\sub{\textit{j}} ein möglicher \isi{Fokus}
von K\sub{\textit{i}} ist, ist auch K\sub{\textit{i}} ein möglicher \isi{Fokus}; \dash es gibt eine
\isi{Fokusprojektion} von Fk\sub{\textit{k}} (K\sub{\textit{i}}) = K\sub{\textit{j}} zu Fk\sub{m} (K\sub{\textit{i}}) = MK (K\sub{\textit{i}}).
\end{exe}
Das Problem ist dann: Kann man den Fokusexponenten von
Konstituententypen generell bestimmen? In Präpositional- und
Nominalphrasen scheinen die Verhältnisse in der Tat einfach zu sein:
Dort ist anscheinend immer das letzte \isi{Substantiv} bzw.~-- wo vorhanden,
der letzte Satz~-- der \isi{Konstituente} ihr \isi{Fokusexponent}. Für Sätze ist
die Frage weitaus schwieriger; wie unsere Beispiele zeigen werden,
sind wir von einer allgemeinen Lösung weit entfernt. Im übrigen würden
durch Regeln für den Fokusexponenten \zb die Fokusmöglichkeiten von
(\ref{ex:3-54}) sicherlich noch nicht erfaßt.

Soweit die Rede von \textsq{communicative dynamism} und
\textsq{communicative importance} in \citet{Sgall73}, \citet{Sgall77, Sgall78}
überhaupt verständlich ist, kann man sie vielleicht als Versuch verstehen, eine
Basis für die Formulierung von Regeln zur \isi{Fokusprojektion} zu legen; soweit das
Vorhaben erfolgreich ist, könnte man darin vielleicht eine Erklärung
(nicht aber ein Indiz) dafür sehen, daß~-- nach Ansicht vieler Autoren~--
gewisse Konstituenten \textsq{näher zum \isi{Verb} gehören} als andere. Die
Erfolgsaussichten dieses speziellen Ansatzes scheinen mir jedoch
äußerst gering; cf.\ auch \citet[348ff]{Dahl75}.

Einen interessanten Versuch, die sehr komplexen Zusammenhänge zwischen
\isi{Betonung}, Fokusmöglichkeiten, \isi{Wortstellung} und semantischer Struktur
der Sätze im Spanischen\il{Spanisch} explizit zu erfassen, hat \citet{Contreras76} unternommen. Auch hier gibt es viele Bedenken im einzelnen, und die Ergebnisse der Arbeit sind allenfalls zum Teil auf das Deutsche\il{Deutsch} anwendbar; am ehesten zu überprüfen wäre die
\textsq{Rheme Selection Hierarchy}, die festlegt, welche Teile des
Satzes unter gegebenen Umständen zum \isi{Fokus} gehören können (und
insofern wahrscheinlich, in meinem Verständnis von \citet{Sgall73} und \citet{Sgall77, Sgall78}, einer \textsq{scale of communicative dynamism} entspricht). Aber wegen einer Fülle differenzierter Beobachtungen und der Explizitheit, Kontrollierbarkeit und Konsistenz der Darstellung,
die im Gegensatz zu fast der gesamten Literatur über solche Themen
steht, ist dies (neben \citealt{Jackendoff72}) die Arbeit, an der sich
künftige Untersuchungen zu dem Thema zu orientieren haben. (Dabei ist
das an der \textsq{generativen Semantik} orientierte Regelwerk von
\citet{Contreras76} leicht in eins zu übersetzen, das Oberflächenstrukturen \textsq{interpretiert}. Ob diese verschiedenen Regelformulierungen rein notationelle Varianten sind oder
verschiedene empirische Implikationen haben, ist ein Thema für
künftige Untersuchungen.)

Aus (\ref{ex:3-66}) und (\ref{ex:3-68}) folgt (\ref{ex:3-72}):
\begin{exe}
\ex \label{ex:3-72}
Wenn zwei Sätze S\sub{\textit{i}}, S\sub{\textit{j}} sich nur dadurch unterscheiden, daß sie
verschiedene Betonungen haben, sind die Mengen ihrer möglichen Foki
verschieden (MF (S\sub{\textit{i}}) $\neq$ MF (S\sub{\textit{j}})).
\end{exe}
Auch wenn sowohl in S\sub{\textit{i}} als auch in S\sub{\textit{j}} fokusprojektive
Bedingungen vorliegen, so daß unter Umständen für alle nicht"=minimalen Foki
Fk\sub{\textit{n}} (S\sub{\textit{i}}) = Fk\sub{\textit{n}} (S\sub{\textit{j}}) gilt, ist doch in jedem Fall Fk\sub{1}
(S\sub{\textit{i}}) $\neq$ Fk\sub{1} (S\sub{\textit{j}}). (In dem Fall daß Fk\sub{1}(S\sub{\textit{i}}) = Fk\sub{\textit{n}}(S\sub{\textit{j}}),
ist immer noch Fk\sub{1} (S\sub{\textit{j}}) nicht in MF (S\sub{\textit{i}}).)

Die je nach \isi{Betonung} mögliche Menge der Foki wird weiter dadurch
eingeschränkt, daß empirisch die \isi{Betonung} verschiedener Konstituenten
offenbar niemals in gleicher Weise \isi{fokusprojektiv} ist:
\begin{exe}
	\ex \label{ex:3-73}
\textit{Hypothese:} \\
Wenn zwei Sätze S\sub{\textit{i}}, S\sub{\textit{j}} sich nur durch ihre \isi{Betonung}
unterscheiden und es in S\sub{\textit{i}} eine \isi{Fokusprojektion} von Fk\sub{\textit{i}} (S\sub{\textit{i}})
= K\sub{\textit{j}} zu Fk\sub{\textit{n}} (S\sub{\textit{i}}) = K\sub{\textit{j}} + K\sub{\textit{k}} gibt, dann gibt es in S\sub{\textit{j}}
keine \isi{Fokusprojektion} von Fk\sub{\textit{j}} (S\sub{\textit{j}}) = K\sub{\textit{k}} zu Fk\sub{\textit{n}} (S\sub{\textit{j}}) = K\sub{\textit{j}}
+ K\sub{\textit{k}} (für Fk\sub{\textit{i}} (S\sub{\textit{i}}) $\cap$ Fk\sub{\textit{j}}(S\sub{\textit{j}}) = $\emptyset$).
\end{exe}
Die Hypothese ist so formuliert, daß sie die \isi{Projektion} von
verschiedenen Foki (K\sub{\textit{j}} bzw.\ K\sub{\textit{k}}) zu aus diesen verschiedenen Foki
kombinierten gleichen Foki (K\sub{\textit{j}} + K\sub{\textit{k}}) verbietet; betrachten wir
als Beispiel dazu noch einmal (\ref{ex:3-48}) vs.\ (\ref{ex:3-54}):
\begin{exe}
\exi{(48)}{\exalph{48}\label{ex:3-48-2}
Karl hat dem Kind das \textit{Buch} geschenkt}
\exi{(54)}{\exalph{54}\label{ex:3-54-2}
Karl hat dem Kind das Buch \textit{geschenkt}}
\end{exe}
Wir haben früher gesehen, daß es in (\ref{ex:3-48}) eine \isi{Fokusprojektion} von
Fk\sub{1} (\ref{ex:3-48}) = \textit{Buch} zu Fk\sub{2} (\ref{ex:3-48}) = \textit{Buch} +
\textit{geschenkt} gibt. In (\ref{ex:3-54}) ist Fk\sub{1} (\ref{ex:3-54}) = \textit{geschenkt};
(\ref{ex:3-73}) sagt korrekt voraus, daß es keinen Fk\sub{\textit{n}} (\ref{ex:3-54}) = \textit{Buch} +
\textit{geschenkt} gibt (obwohl die \isi{Betonung} in (\ref{ex:3-54}) \isi{fokusprojektiv}
ist: Es gibt den Fk\sub{2} (\ref{ex:3-54}) = \textit{Karl} +
\textit{geschenkt}). (\ref{ex:3-73}) verbietet jedoch nicht den Fall, daß es eine
\isi{Projektion} von partiell gleichen Foki (etwa Fk\sub{\textit{l}} (S\sub{\textit{i}}) = K\sub{\textit{i}} +
K\sub{\textit{j}} und Fk\sub{\textit{m}} (S\sub{\textit{j}}) = K\sub{\textit{j}} + K\sub{\textit{k}}) zu einem gleichen \isi{Fokus}
K\sub{\textit{i}} + K\sub{\textit{j}} + K\sub{\textit{k}} gibt; einen solchen Fall haben wir in (\ref{ex:3-58b}). Auch die \isi{Projektion} von gleichen Foki zu gleichen Foki ist
natürlich erlaubt; vgl.\ (\ref{ex:3-58c}). Vgl.\ dazu (\ref{ex:3-105}).

An dieser Stelle muß auf eine Mehrdeutigkeit des Ausdrucks
"`verschiedene/""gleiche \isi{Betonung}"' aufmerksam gemacht werden. Man kann
sagen, daß zwei Sätze die gleiche \isi{Betonung} haben, wenn sie die gleiche
Abfolge von betonten und unbetonten Konstituenten haben. Oder man kann
von gleicher \isi{Betonung} sprechen, wenn die gleichen Satzteile betont
sind. Im ersten Fall spreche ich von gleichem
\textsq{Betonungsmuster}, im zweiten von gleicher
\textsq{Konstituentenbetonung}. So haben wir in (\ref{ex:3-74}a,b) das gleiche
\isi{Betonungsmuster}, in (\ref{ex:3-74}b,c) dagegen die gleiche
Konstituentenbetonung:
\eal
\label{ex:3-74}
\ex
\label{ex:3-74a}
Karl hat den \textit{Hund} gestreichelt
\ex
\label{ex:3-74b}
den Hund hat \textit{Karl} gestreichelt
\ex
\label{ex:3-74c}
\textit{Karl} hat den Hund gestreichelt
\zl
Für (\ref{ex:3-72}) und (\ref{ex:3-73}) ist diese Unterscheidung ohne Belang, aber man muß
beachten, daß in topologischen Untersuchungen gewöhnlich das
\isi{Betonungsmuster}, im Gegensatz zur Konstituentenbetonung, konstant
gehalten wird; andernfalls wären \zb Kirkwoods Äußerungen in (4)
überhaupt nicht verständlich. Beim Vergleich von aktiven und passiven
Konstruktionen gilt etwas ähnliches: Man findet häufig Behauptungen
des Inhalts, daß etwa (\ref{ex:3-75}a,b) sich hinsichtlich der
\isi{Topik}"=\isi{Fokus}"=Gliederung unterscheiden. Die Beurteilung passiver Sätze
ist oft recht schwierig, aber hier ist diese Behauptung richtig, wenn
das \isi{Betonungsmuster} konstant bleibt, so daß in (\ref{ex:3-75}a,b) jeweils die
dritte \isi{Konstituente} betont ist, wie in (\ref{ex:3-75}c,d). Wenn jedoch die Konstituentenbetonung konstant gehalten wird, also auf dem logischen Objekt (\ref{ex:3-75}c,e)) oder auf dem \isi{Verb} (\ref{ex:3-75}d,f), ergeben sich keine
Unterschiede in den Fokusmöglichkeiten, \dash in (\ref{ex:3-75}c,e) kann der
ganze Satz den \isi{Fokus} bilden, in (\ref{ex:3-75}d,f) nur das \isi{Verb}.
\begin{exe}
\ex
\label{ex:3-75}
\begin{xlist}
\ex
\label{ex:3-75a}
jemand hat den Hund getreten
\ex
\label{ex:3-75b}
der Hund ist getreten worden
\ex
\label{ex:3-75c}
jemand hat den \textit{Hund} getreten
\ex
\label{ex:3-75d}
der Hund ist \textit{getreten} worden
\ex
\label{ex:3-75e}
der \textit{Hund} ist getreten worden
\ex
\label{ex:3-75f}
jemand hat den Hund \textit{getreten}
\end{xlist}
\end{exe}
Um eine kurze Ausdrucksweise zur Hand zu haben, führe ich folgende Notationen ein:
%cmldcomm: fix

\begin{exe}
\ex 
\begin{xlist}
	\ex \label{ex:3-76a}
\begin{itemize}
	\item[EB\sub{\textit{i}}:] eine Menge von Sätzen S\sub{\textit{i}\sub{1}}, S\sub{\textit{i}\sub{2}}, \ldots,
	S\sub{\textit{i\sub{n}}}, die sich (abgesehen von möglichen Foki) nur hinsichtlich
	der Konstituentenbetonung unterscheiden
\end{itemize}
\ex \label{ex:3-76b}
\begin{itemize}
	\item[ES\sub{\textit{i}}:] eine Menge von Sätzen S\sub{\textit{i}\sub{1}}, S\sub{\textit{i}\sub{2}}, \ldots,
S\sub{\textit{i\sub{n}}}, die sich (bei gleicher Konstituentenbetonung und
abgesehen von möglichen Foki) nur hinsichtlich der linearen Abfolge
der Konstituenten unterscheiden
\end{itemize}
\ex \label{ex:3-76c}
\begin{itemize}
\setlength{\itemindent}{1ex}
\item[EBS\sub{\textit{i}}:] eine Menge von Sätzen S\sub{\textit{i}\sub{1}}, S\sub{\textit{i}\sub{2}}, \ldots,
S\sub{\textit{i\sub{n}}}, die sich (abgesehen von möglichen Foki) nur hinsichtlich
der linearen Abfolge der Konstituenten und/""oder der
Konstituentenbetonung unterscheiden
\end{itemize}
\end{xlist}
\end{exe}
\addlines
Diese Mengen sollen vollständig sein: Wenn zwei Sätze S\sub{\textit{i}}, S\sub{\textit{j}}
sich nur hinsichtlich der Konstituentenbetonung unterscheiden und
S\sub{\textit{j}} in EB\sub{\textit{i}} ist, dann ist auch S\sub{\textit{i}} in EB\sub{\textit{i}}; entsprechend für
ES\sub{\textit{i}} und EBS\sub{\textit{i}}. Für alle \textit{i} und \textit{j}: Wenn \textit{i} = \textit{j}, dann EB\sub{\textit{i}}
$\subseteq$ EBS\sub{\textit{j}} und ES\sub{\textit{i}} $\subseteq$ EBS\sub{\textit{j}}.

Aus (\ref{ex:3-37}) und (\ref{ex:3-72}) folgt (\ref{ex:3-77}):
\begin{exe}
\ex
\label{ex:3-77}
Wenn zwei Sätze S\sub{\textit{i}}, S\sub{\textit{j}} in EB\sub{\textit{i}} sind, sind die Mengen von
Kontexttypen, in denen sie vorkommen können, verschieden.
\end{exe}
Dabei ist ein \textsq{Kontexttyp} die Menge MRK\sub{\textit{i}} von relevanten
Kontexten, die, für einen gegebenen Satz S\sub{\textit{i}} bei einem gegebenen Fk\sub{\textit{j}}
(S\sub{\textit{i}}), gemäß (\ref{ex:3-37a}) einen Satz S\supsub{W}{\textit{i}} \ oder gemäß (\ref{ex:3-37b}) die
Konjunkte der normalisierten LC von S\sub{\textit{i}} enthalten, die dem Tk\sub{\textit{j}} (S\sub{\textit{i}})
entsprechen. (Vgl.\ (\ref{ex:3-38}). N.B.: Wenn dort \zb das 3.\ oder
4.\,\isi{Konjunkt} einem Teil des Topiks entspricht, ist immer auch das
1.\,\isi{Konjunkt} in RK; der Ausdruck des 3.\ (bzw.\ 4.) Konjunkts wäre sonst
  sinnlos.)

  Während jede solche Menge MRK\sub{\textit{i}} natürlich infinit ist, ist die
  Anzahl der MRK\sub{\textit{i}} pro Satz S\sub{\textit{i}} gleich der Anzahl der möglichen
  Topiks von S\sub{\textit{i}}, wenn wir im Fall eines topiklosen Satzes ein
  \textsq{leeres} \isi{Topik} mitzählen.
\begin{exe}
\ex
\label{ex:3-78}
\textit{Definition:} \\
Unter den Sätzen in EB\sub{\textit{i}} sind die S\sub{\textit{j}} \textit{kontextuell relativ unmarkiert} hinsichtlich der \isi{Betonung}, die in der größten Zahl von Kontexttypen
vorkommen können; alle anderen S\sub{\textit{k}} in EB\sub{\textit{i}} sind hinsichtlich der \isi{Betonung} kontextuell markiert.
\end{exe}
Der Sinn dieser Definition dürfte klar sein. Wenn es je nach \isi{Betonung}
verschieden viele mögliche Foki gibt, dann ist der Satz, der die
meisten möglichen Foki hat, im Vergleich zu den anders betonten Sätzen
kontextuell am wenigsten eingeschränkt. Ein Satz, der nur aus einem
Wort besteht, etwa \textit{geh!}, ist trivial \isi{unmarkiert}.

Wir können jetzt den in (\ref{ex:3-30}) gebrauchten Begriff \textsq{stilistisch
  normale Betonung} explizieren:
\begin{exe}
\ex
\label{ex:3-79}
Die \isi{Betonung} eines Satzes S\sub{\textit{i}} ist stilistisch normal, wenn S\sub{\textit{i}}
hinsichtlich der \isi{Betonung} kontextuell relativ \isi{unmarkiert} ist; sie ist
stilistisch nicht-normal, wenn S\sub{\textit{i}} hinsichtlich der \isi{Betonung}
kontextuell markiert ist.
\end{exe}
Unter den Beispielen, die wir in \ref{subsubsec:3-1-3-2} betrachtet haben, ist allein
 (\ref{ex:3-48}) hinsichtlich der \isi{Betonung} kontextuell relativ \isi{unmarkiert}, da er
die meisten möglichen Foki hat und daher in den meisten Kontexttypen
vorkommen kann. Eben dieser Satz weist auch eine \isi{Betonung} auf, die, im
Unterschied zu alternativen Betonungsmustern, allgemein als
stilistisch normal empfunden wird.

\subsection{Fruchtbarkeit der Explikation}
\label{subsec:3-1-4}

Ich möchte auf einige Eigenschaften dieser Explikation von
"`\isi{Normalbetonung}"' eingehen.

\subsubsection{Adäquatheit}
\label{subsubsec:3-1-4-1}

Wenn man die intuitiv wirklich klaren Fälle von \textsq{normaler} und
\textsq{nicht-normaler} \isi{Betonung} untersucht, zeigt sich, daß (\ref{ex:3-79}) die
Unterscheidung in genauer Übereinstimmung mit der Intuition
vornimmt. Der Zusammenhang zwischen \textsq{Normalbetonung} und
relativ großem Umfang des \isi{Fokus} ist in der Literatur mehrfach
beobachtet worden. Die~-- terminologisch komplizierten und teilweise
recht unklaren~-- Ausführungen von Halliday verstehe ich in genau
diesem Sinne, so z.\,B.:
\begin{quotation}
  "`A specific question is derivable from any information unit except
  one with unmarked focus; one with unmarked focus does not imply any
  preceding information [\ldots] Where the focμs is unmarked, [\ldots]
  its domain may be the whole of the information unit. An item with
  unmarked focus may thus be represented as being ambiguous, as having
  the structure either given~-- new or simply new."'
  \citep[208]{Halliday67}
\end{quotation}
Ähnlich Chomsky, z.\,B.:
\begin{quotation}
  "`This [\ldots] suggests that when expressive or contrastive stress
  shifts \isi{intonation} center, the same principle applies as in normal
  cases for determining focus and presupposition, but with the
  additional proviso that naturalness declines far more sharply as
  larger and larger phrases containing the \isi{intonation} center are
  considered as a possible focus."' \citep[98f]{Chomsky76a}
\end{quotation}
Wenn ich die~-- in wesentlichen Teilen höchst unklaren~-- Ausführungen
von \citet{Sgall73} richtig verstehe, machen auch sie diese
Beobachtung: Bei \textsq{normaler} \isi{Betonung} kann der \isi{Fokus} umfangreich
sein, während er bei \textsq{nicht"=normaler} \isi{Betonung} eingeschränkt
ist.

Dabei kann es bei allen Autoren nicht um den Umfang des \isi{Fokus} als
solchen gehen. In (\ref{ex:3-80}) \zb ist der ganze Satz (minimaler) \isi{Fokus},
\begin{exe}
\ex
\label{ex:3-80}
\textit{Karl} hat den \textit{Hund} \textit{getreten}
\end{exe}
aber keiner der Zitierten betrachtet diese \isi{Betonung} als normal. Das
entscheidende Faktum ist offensichtlich, daß bei \isi{Normalbetonung} der
\isi{Fokus} wesentlich mehr umfaßt bzw.\ umfassen kann als eine einzige
betonte \isi{Konstituente}, wenn also \isi{Fokusprojektion}~-- bei \citet[301ff]{Fuchs76}:
\textsq{Integration}~-- vorliegt (soweit überhaupt eine theoretische Möglichkeit dafür gegeben
ist).\footnote{\label{fn:3-20}%
  Unter \textsq{emphatisch} oder \textsq{kontrastiv} betonten Sätzen
  verstehen die meisten Autoren offenbar solche, die gemäß (\ref{ex:3-79})
  hinsichtlich der \isi{Betonung} kontextuell markiert sind; insbesondere jene
  kontextuell markierten Sätze, die nur einen einzigen möglichen \isi{Fokus}
  haben (wenn der Satz mehr als 1 \isi{Konstituente} hat). Nach dieser
  Explikation wäre (\ref{ex:3-80}) als \isi{emphatisch} und/""oder \isi{kontrastiv} zu
  bezeichnen.

  In \citet[§2.3]{Fuchs80} sagt Fuchs allerdings, Fälle wie (i)
  seien im Verhältnis zu (ii) keineswegs \textsq{neutral} oder
  \textsq{unmarkiert}; vielmehr sei der Typ (i) "`more restricted, more
  specific"':
  \ea
  \label{ex:3-i2}
  der \textit{Junge} kommt
  \ex
  \label{ex:3-ii2}
  der \textit{Junge kommt}
  \z
  Sie gibt zwei Typen von Argumenten dafür: (a) "`Taking placement in
  discourse into account, [(i)] has far more limited possibilities of
  occurrence than [(ii)], and a much more specific mean\-ing (or discourse
  function)."' Soweit "`placement in discourse"', "`mean\-ing"' und "`discourse
  function"' mir verständlich erscheinen, ist diese Behauptung
  schlechterdings falsch: (i) hat 2 mögliche Foki (vgl.\ (\ref{ex:3-83})), (ii) nur
  1, so daß (i) in mehr Kontexttypen vorkommen kann als (ii). (b) Der
  Typ (i) sei hinsichtlich seines syntaktischen Aufbaus und seiner
  prosodischen Möglichkeiten stärker beschränkt als der Typ
  (ii). Hinsichtlich des \textsq{syntaktischen Aufbaus} ist das
  irreführend. Zwar kommt \isi{Fokusprojektion} vom \isi{Subjekt} zum Satz nur bei
  einigen Prädikaten vor; aber genauso gut könnte man sagen, daß nur bei
  einigen Prädikaten \isi{Fokusprojektion} vom \isi{Prädikat} zum Satz
  vorkommt. Außerdem läßt Fuchs hier die allgemeine Regel (\ref{ex:3-115}) außer
  Acht. Hinsichtlich der prosodischen Eigenschaften mag die Behauptung
  stimmen. Dies hat jedoch mit \textit{intuitiver} (stilistischer)
  Unmarkiertheit, um die es hier geht, nichts zu tun; es ist allenfalls
  für einen Begriff der \textit{strukturellen} (Un-)Markiertheit relevant. (Nicht
  ohne Grund bezeichnen fast alle von Fuchs zitierten Autoren (i) als
  \textsq{normal}, ohne (ii) überhaupt zu berücksichtigen.) Wir
  besprechen in Abschnitt~\ref{sec:3-3}, daß derartige strukturelle Markiertheitsbegriffe
  streng von Begriffen der stilistischen Markiertheit unterschieden
  werden müssen und, im Gegensatz zu diesen, von sehr zweifelhafter
  Relevanz sind.%
}

Die gegebene Explikation ist noch in einer anderen Hinsicht adäquat:
insofern sie auf die Zahl \textit{möglicher} Foki von \textit{Sätzen}
abgestellt ist und nicht auf Äußerungen. Altmann bezieht in dem Zitat
unter (\ref{ex:3-28}) die \textsq{Normalität} einer \isi{Betonung} auf den Gebrauch von
Sätzen, \dash auf Äußerungen. Eine Äußerung hat immer genau 1
intendierten \isi{Fokus}, und man kann nur prüfen, ob die Äußerung
konversationellen Maximen entsprechend im gegebenen Kontext
\textsq{passend} ist oder nicht. Da die \textsq{Normalität} einer
\isi{Betonung} aber ersichtlich mit der Art und Zahl möglicher Foki
zusammenhängt, kann sie primär nur für Sätze, nicht für Äußerungen
definiert werden. Anders gesagt: Wer \textsq{Normalbetonung} primär
auf Äußerungen und nicht auf Sätze bezieht (also mit
\textsq{kontextuell angemessen} identifiziert), muß zu dem paradoxen
Schluß kommen, daß die "`Annahme eines Normalakzents für jeden
satzwertigen Ausdruck [\ldots] theoretisch nicht haltbar ist"', obwohl
sie sich "`auf tiefwurzelnde Intuitionen stützen kann"', vgl.\ (\ref{ex:3-28})~--
aber ein solches Schlußverfahren entzieht der empirischen
Sprachforschung den Boden. (Aber man kann \textsq{Normalbetonung}
natürlich auch für Äußerungen definieren: Eine Äußerung von S\sub{\textit{i}} hat
stilistisch normale Betonung\is{Normalbetonung}, wenn S\sub{\textit{i}} stilistisch normale
\isi{Betonung} hat.)

\subsubsection{Nützlichkeit}
\label{subsubsec:3-1-4-2}

Die in (\ref{ex:3-79}) gegebene Explikation von "`\isi{Normalbetonung}"' ersetzt das
Explikandum durch ein Explikans, das (a) präziser bestimmt und (b)
empirisch besser kontrollierbar ist. (Die in \ref{subsubsec:3-1-4-1} zitierten Autoren
setzen wohlgemerkt \textsq{Normalbetonung} immer als undefinierten
Begriff voraus und teilen empirische Beobachtungen über Fälle mit
\textsq{Normalbetonung} mit. In (\ref{ex:3-79}) ist es umgekehrt: Empirisch
kontrollierbare Phänomene werden zur Explikation von "`\isi{Normalbetonung}"'
verwendet.)

Diese Explikation ermöglicht es, eine Reihe von Zweifelsfällen zu
klären. Untersuchen wir einige Beispiele aus \ref{subsec:3-1-1}, hier mit neuer
Numerierung wiederholt.

Zunächst zu Beispielen, die für Bartschs Aussagen über
\textsq{Normalbetonung} von Belang sind:
\begin{exe}
\ex
\label{ex:3-81}
\begin{xlist}
\ex
\label{ex:3-81a}
es heißt, daß der \textit{Junge} getanzt hat (= (\ref{ex:3-1c}))
\ex
\label{ex:3-81b}
~\hphantom{es heißt} daß der Junge \textit{getanzt} hat (= (\ref{ex:3-2c}))
\end{xlist}
\end{exe}
Hier haben wir die minimalen Foki Fk\sub{1} (\ref{ex:3-81a}) = \textit{der Junge}
und Fk\sub{1} (\ref{ex:3-81b}) = \textit{getanzt}. Gibt es eine \isi{Fokusprojektion}?
Betrachten wir die Diskurse in (\ref{ex:3-82}):
\begin{exe}
\ex
\label{ex:3-82}
\begin{xlist}
\ex
\label{ex:3-82a}
A: was ist geschehen, als die Nachricht eintraf?
\ex
\label{ex:3-82b}
B: (\ref{ex:3-81a})
\ex
\label{ex:3-82c}
B: (\ref{ex:3-81b})
\end{xlist}
\end{exe}
Auf die Frage (\ref{ex:3-82a}) kann mit (\ref{ex:3-81b}) geantwortet werden, wenn mehrere Leute,
darunter der Junge, die Nachricht gehört haben und nicht zu erwarten
war, daß gerade der Junge etwas Besonderes tun würde. Mit (\ref{ex:3-81a}) kann
offenbar nur geantwortet werden, wenn zu erwarten oder bekannt war,
daß jemand tanzt. In (\ref{ex:3-81a}) ist \textit{getanzt} also notwendig \isi{Topik},
in (\ref{ex:3-81b}) dagegen haben wir Fk\sub{2} (\ref{ex:3-81b}) \textit{der Junge} +
\textit{getanzt}. Demnach hat (\ref{ex:3-81b}) mehr mögliche Foki als (\ref{ex:3-81a}) und
ist im Unterschied zu (\ref{ex:3-81a}) nach (\ref{ex:3-79}) normalbetont\is{Normalbetonung}, entgegen der
Aussage von Bartsch.

Allerdings ist es seit \citet{Hatcher56b, Hatcher56a} wohlbekannt, daß
andere einstellige Prädikate sich anders verhalten,
vgl.\ \zb \citet[81 Fn.\,8]{Heidolph70}, \citet[41ff]{Schmerling76},
\citet{Fuchs76}, \citet{AllCrutt79}. Auf diese treffen Bartschs Aussagen
zu, vgl.:
\eal
\label{ex:3-83}
\ex
\label{ex:3-83a}
es heißt, daß der \textit{Junge} kommt (= (\ref{ex:3-1d}))
\ex
\label{ex:3-83b}
es heißt, daß der Junge \textit{kommt} (= (\ref{ex:3-2d}))
\zl
In (\ref{ex:3-83a}), aber nicht in (\ref{ex:3-83b}), ist eine \isi{Fokusprojektion} zu \textit{der Junge} +
\textit{kommt} möglich, so daß in diesem Fall die \isi{Betonung} des Subjekts
stilistisch normal ist.

\addlines
Weniger bekannt ist, daß es ähnliche Unterschiede auch bei
zweistelligen Verben gibt. So ist in (\ref{ex:3-84}) die \isi{Betonung} des Objekts
\eal
\label{ex:3-84}
\ex
\label{ex:3-84a}
es heißt, daß der Junge dem \textit{Pfarrer} begegnet ist (= (\ref{ex:3-1b}))
\ex
\label{ex:3-84b}
es heißt, daß der Junge dem Pfarrer \textit{begegnet} ist (= (\ref{ex:3-2b}))
\zl
stilistisch normal; vgl.\ die MF (\ref{ex:3-84a}) in (\ref{ex:3-85}) mit der MF (\ref{ex:3-84b}) in (\ref{ex:3-86}):
\begin{exe}
\ex
\label{ex:3-85}
\begin{xlist}
\ex
\label{ex:3-85a}
Fk\sub{1} (\ref{ex:3-84a}) = dem Pfarrer
\ex
\label{ex:3-85b}
Fk\sub{2} (\ref{ex:3-84a}) = dem Pfarrer + begegnet
\ex
\label{ex:3-85c}
Fk\sub{3} (\ref{ex:3-84a}) = MK (\ref{ex:3-84a}) = der Junge + dem Pfarrer + begegnet
\end{xlist}
\ex
\label{ex:3-86}
Fk\sub{1} (\ref{ex:3-84b}) = Fk\sub{m} (\ref{ex:3-84b}) = begegnet
\end{exe}
Anders als Bartsch es postuliert verhält sich (\ref{ex:3-87}). Wie die MF (\ref{ex:3-87a}) in (\ref{ex:3-88}) und die MF (\ref{ex:3-87b}) in (\ref{ex:3-89}) ausweisen, scheint hier die \isi{Betonung}
\begin{exe}
\ex
\label{ex:3-87}
\begin{xlist}
\ex
\label{ex:3-87a}
es heißt, daß die Theorie den \textit{Fach}leuten gefallen hat (= (\ref{ex:3-1a}))
\ex
\label{ex:3-87b}
es heißt, daß die Theorie den Fachleuten \textit{gefallen} hat (= (\ref{ex:3-2a}))
\end{xlist}
\end{exe}
des Verbs stilistisch normal, die des Objekts nicht-normal zu sein:
\begin{exe}
\ex
\label{ex:3-88}
Fk\sub{1} (\ref{ex:3-87a}) = Fk\sub{m} (\ref{ex:3-87a}) = den Fachleuten
\ex
\label{ex:3-89}
\begin{xlist}
\ex
\label{ex:3-89a}
Fk\sub{1} (\ref{ex:3-87b}) = gefallen
\ex
\label{ex:3-89b}
Fk\sub{2} (\ref{ex:3-87b}) = den Fachleuten + gefallen
\ex
\label{ex:3-89c}
Fk\sub{3} (\ref{ex:3-87b}) = MK (\ref{ex:3-87b}) = die Theorie + den Fachleuten + gefallen
\end{xlist}
\end{exe}
Es scheint, daß die Projektionsmöglichkeiten nicht nur vom
Prädikatstyp, sondern auch vom internen Aufbau des Objekts beeinflußt
werden. So erlauben (\ref{ex:3-90a}) wie (\ref{ex:3-90b}) eine \isi{Fokusprojektion};
\begin{exe}
\ex
\label{ex:3-90}
\begin{xlist}
\ex
\label{ex:3-90a}
er hat sich (sogar) ein \textit{Buch} gekauft
\ex
\label{ex:3-90b}
er hat (sogar) ein \textit{Buch} verbrannt
\end{xlist}
\end{exe}
dies scheint auch bei (\ref{ex:3-91a}) der Fall zu sein, aber nicht bei (\ref{ex:3-91b}):
\begin{exe}
\ex
\label{ex:3-91}
\begin{xlist}
\ex
\label{ex:3-91a}
er hat sich (sogar) ein \textit{Buch} über Mozart gekauft
\ex
\label{ex:3-91b}
er hat (sogar) ein \textit{Buch} über Mozart verbrannt
\end{xlist}
\end{exe}
(vgl.\ \citet{Erteschick79b}, wo ähnliche Unterschiede für das Englische\il{Englisch}
diskutiert werden). 

Gehen wir zu Kiparskys Beispiel (\ref{ex:3-92}) über. Der
Satz paßt zu den
\begin{exe}
\ex
\label{ex:3-92}
der Arzt wird den Patienten unter\textit{suchen} (= (\ref{ex:3-3}))
\end{exe}
Fragen (\ref{ex:3-93}), also Fk\sub{1} (\ref{ex:3-92}) = \textit{untersuchen} und Fk\sub{2} (\ref{ex:3-92}) =
\textit{der Arzt} + \textit{untersuchen}. Vergleichen wir damit (\ref{ex:3-94});
dieser Satz paßt zu den 
\begin{exe}
\ex
\label{ex:3-93}
\begin{xlist}
\ex
\label{ex:3-93a}
was wird der Arzt mit dem Patienten tun?
\ex
\label{ex:3-93b}
was wird mit dem Patienten geschehen?
\end{xlist}
\ex
\label{ex:3-94}
der Arzt wird den Pa\textit{tienten} untersuchen
\ex
\label{ex:3-95}
\begin{xlist}
\ex
\label{ex:3-95a}
wen wird der Arzt untersuchen?
\ex
\label{ex:3-95b}
was wird der Arzt tun?
\ex
\label{ex:3-95c}
was wird geschehen?
\end{xlist}
\end{exe}
Fragen (\ref{ex:3-95}), also Fk\sub{1} (\ref{ex:3-94}) = \textit{Patienten}, Fk\sub{2} (\ref{ex:3-94}) =
\textit{Patienten} + \textit{untersuchen}, Fk\sub{3} (\ref{ex:3-94}) = MK (\ref{ex:3-94}). Da
(\ref{ex:3-94}) mehr mögliche Foki hat als (\ref{ex:3-92}), ist (\ref{ex:3-92}) nach (\ref{ex:3-79}), entgegen
Kiparskys Meinung, aber in Übereinstimmung mit dem intuitiven Urteil
vieler Sprecher, nicht"=normal betont.

Betrachten wir Kirkwoods Beispiele. Man überzeugt sich leicht, daß für
(\ref{ex:3-96}) das gleiche gilt wie für (\ref{ex:3-48}), \dash wir haben die MF (\ref{ex:3-96}) in
(\ref{ex:3-97}). Unter allen möglichen Betonungsmustern für diese \isi{Wortfolge} hat
(\ref{ex:3-96}) die meisten möglichen Foki; nach (\ref{ex:3-79}) ist der Satz damit normal
betont.
\begin{exe}
\ex
\label{ex:3-96}
er gab dem Kind das \textit{Buch} (= (\ref{ex:3-3}) in (\ref{ex:3-4}))
\ex
\label{ex:3-97}
\begin{xlist}
\ex
\label{ex:3-97a}
Fk\sub{1} (\ref{ex:3-96}) = das Buch
\ex
\label{ex:3-97b}
Fk\sub{2} (\ref{ex:3-96}) = gab + das Buch
\ex
\label{ex:3-97c}
Fk\sub{3} (\ref{ex:3-96}) = gab + dem Kind + das Buch
\ex
\label{ex:3-97d}
Fk\sub{4} (\ref{ex:3-96}) = er + gab + das Buch
\ex
\label{ex:3-97e}
Fk\sub{5} (\ref{ex:3-96}) = MK (\ref{ex:3-96}) = er + gab + dem Kind + das Buch
\end{xlist}
\end{exe}
Wie ist es bei (\ref{ex:3-98})? Der Satz paßt zu den Fragen in (\ref{ex:3-99}), hat also die
MF (\ref{ex:3-98}) in (\ref{ex:3-100}), die z.\,T.\ mit den Foki von (\ref{ex:3-96}) identisch sind.
\begin{exe}
\ex
\label{ex:3-98}
er gab das Buch dem \textit{Kind} (= (\ref{ex:3-4}) in (\ref{ex:3-4}))
\ex
\label{ex:3-99}
\begin{xlist}
\ex
\label{ex:3-99a}
wem gab er das Buch?
\ex
\label{ex:3-99b}
was hat er mit dem Buch gemacht? (cf.\ (\ref{ex:3-49j}))
\ex
\label{ex:3-99c}
was hat er getan?
\ex
\label{ex:3-99d}
was ist mit dem Buch geschehen?
\ex
\label{ex:3-99e}
was ist geschehen?
\end{xlist}
\ex
\label{ex:3-100}
\begin{xlist}
\ex
\label{ex:3-100a}
Fk\sub{1} (\ref{ex:3-98}) = dem Kind
\ex
\label{ex:3-100b}
Fk\sub{2} (\ref{ex:3-98}) = gab + dem Kind
\ex
\label{ex:3-100c}
Fk\sub{3} (\ref{ex:3-98}) = gab + das Buch + dem Kind (= Fk\sub{3} (\ref{ex:3-96}))
\ex
\label{ex:3-100d}
Fk\sub{4} (\ref{ex:3-98}) = er + gab + dem Kind
\ex
\label{ex:3-100e}
Fk\sub{5} (\ref{ex:3-98}) = MK (\ref{ex:3-98}) = er + gab + das Buch + dem Kind (= Fk\sub{5} (\ref{ex:3-96}))
\end{xlist}
\end{exe}
Unter allen möglichen Betonungsmustern für diese \isi{Wortfolge} hat (\ref{ex:3-98})
die meisten möglichen Foki; unter der Explikation (\ref{ex:3-79}) ist (\ref{ex:3-98}) daher
in genau derselben Weise normal betont wie (\ref{ex:3-96}); für Kirkwoods und
Griesbachs Annahme, in Fällen wie (\ref{ex:3-98}) liege nicht"=normale Betonung\is{Normalbetonung}
vor, sehe ich keine Grundlage.

Bartsch sagt zwar, daß in (\ref{ex:3-96}) wie in (\ref{ex:3-98}) in gleicher Weise
\textsq{standard \isi{intonation} pattern} vorliege, leugnet aber
\citep[523]{Bartsch76}, daß in (\ref{ex:3-98}) das \isi{Akkusativobjekt} Teil eines \isi{Fokus}
(wie in (\ref{ex:3-100c})) sein kann. (Die Möglichkeit, daß das \isi{Subjekt} bei mehrstelligen Verben im \isi{Fokus} ist wie in (\ref{ex:3-50e}) oder (\ref{ex:3-97}d,e) und
(\ref{ex:3-100}d,e), läßt sie ausdrücklich nicht zu,
vgl.\ \citet[522]{Bartsch76}. Für 1"=stellige Verben sind die
Implikationen ihrer Fokustheorie nicht klar.) Natürlich muß man bei
Fragen der Fokuseruierung wie in allen anderen Fällen, in denen die
Intuitionen von Sprechern eine Rolle spielen, im Prinzip mit gewissen
idiolektalen Unterschieden rechnen. In diesem speziellen Fall bin ich
aber davon überzeugt, daß die drei genannten Autoren sich nicht von
empirischen Beobachtungen, sondern von gewissen Traditionen des
Schulunterrichts haben leiten lassen.\footnote{\label{fn:3-21}%
    Möglicherweise liegt auch eine Übergeneralisierung vor. Denn während
  es scheint, daß die Fokusmöglichkeiten bei \textit{geb-} und
  \textit{schenk-} so sind, wie ich in (\ref{ex:3-97}) und (\ref{ex:3-100}) angegeben habe,
  verhalten sich andere 3"=stellige Verben offenbar anders. Bei \textit{zeig-}
  \zb scheint mir nur in (i), nicht aber in (ii), ein \isi{Fokus} möglich,
  der~-- parallel zu (\ref{ex:3-100}c,e)~-- die beiden Objekte und das \isi{Verb}
  enthält:
  \ea
  \label{ex:3-i3}
  Karl hat den Kindern den \textit{Garten} gezeigt
  \ex
  \label{ex:3-ii3}
  Karl hat den Garten den \textit{Kindern} gezeigt
  \zlast%
}

Für (\ref{ex:3-101}a,b) gilt dasselbe wie für (\ref{ex:3-96}), (\ref{ex:3-98}); wir können Lenerz
zustimmen, daß \isi{Normalbetonung} vorliegt. Gilt das auch für (\ref{ex:3-101c})? In
(\ref{ex:3-102}) formuliere ich einige Fragen, um zu prüfen, ob (\ref{ex:3-101c}) zu ihnen
passen könnte. Es scheint mir~-- besonders im Vergleich zu (\ref{ex:3-101a}) --
ganz deutlich, daß (\ref{ex:3-101c}) zu keiner der Fragen (\ref{ex:3-102}b--e)
\begin{exe}
\ex
\label{ex:3-101}
\begin{xlist}
\ex
\label{ex:3-101a}
ich habe das Geld dem \textit{Kassierer} gegeben (= (2b) in (\ref{ex:3-5}))
\ex
\label{ex:3-101b}
ich habe dem Kassierer das \textit{Geld} gegeben (= (3a) in (\ref{ex:3-5}))
\ex
\label{ex:3-101c}
ich habe dem \textit{Kassierer} das Geld gegeben (= (2a) in (\ref{ex:3-5}))
\end{xlist}
\ex
\label{ex:3-102}
\begin{xlist}
\ex
\label{ex:3-102a}
wem hast du das Geld gegeben?
\ex
\label{ex:3-102b}
was hast du mit dem Geld gemacht?
\ex
\label{ex:3-102c}
was ist mit dem Geld geschehen?
\ex
\label{ex:3-102d}
was hast du dann getan?
\ex
\label{ex:3-102e}
was ist dann geschehen?
\end{xlist}
\end{exe}
paßt, also nur den minimalen \isi{Fokus} \textit{dem Kassierer} hat. Unter der
Explikation (\ref{ex:3-79}) hat (\ref{ex:3-101c}) also (da (\ref{ex:3-101b}) bei gleicher
\isi{Wortfolge} mehr Foki hat) eine entschieden nicht-normale Betonung\is{Normalbetonung},
entgegen der Ansicht von Lenerz, aber in Übereinstimmung mit der
Intuition vieler Sprecher.

\subsubsection{Heuristische Fruchtbarkeit}
\label{subsubsec:3-1-4-3}

Die Explikation (\ref{ex:3-79}) ist infolge des Systems von Hypothesen und
Definitionen, auf dem sie aufbaut, direkt mit einer Fülle von
empirisch interessanten Fragen verknüpft, die so entweder zuvor nicht
gestellt worden sind oder nicht in kontrollierbarer Weise einer
Antwort nähergebracht werden konnten; in vielen Fällen legt sie die
Antworten unmittelbar nahe; z.\,B.:
\begin{itemize}
\item Gibt es Fälle, in denen verschiedene \isi{Betonungsmuster} bei
  gleicher \isi{Wortfolge} gleiche Fokusprojektionen auslösen? Vgl.\ (\ref{ex:3-73}).
\item Gibt es Fälle, wo es in einer EB\sub{\textit{i}} mehrere kontextuell relativ
  unmarkierte Sätze gibt?
\item Gibt es (außer bei \isi{Negation}, Gradpartikeln u.\,ä.) Fälle, in
  denen nicht der ganze Satz \isi{Fokus} sein kann?~-- Vgl.\ Abschnitt~\ref{sec:3-2}.
\item Unter welchen Bedingungen gibt es einen Fokusexponenten
  (cf.\ (\ref{ex:3-71})) des Satzes?
\item Von welchen Faktoren hängt es ab, was \isi{Fokusexponent} des Satzes ist?
\item Wie ist das Verhältnis zwischen \isi{Topik}/""\isi{Fokus} und \textsq{alt
    (bekannt, gegeben)} und \textsq{neu (unbekannt)}?
\end{itemize}
Der letztgenannten Frage wenden wir uns weiter unten zu; vorher wollen
wir einige Spezialfälle der \isi{Fokusprojektion} betrachten.

Zur partiellen Erklärung der Unterschiede zwischen (\ref{ex:3-48}) und (\ref{ex:3-57}) legen
sich die folgenden Generalisierungen nahe.
\begin{exe}
\ex
\label{ex:3-103}
\textit{Hypothese:}\\
Bei gewissen dreistelligen Verben und der \isi{Wortfolge}\\
\hphantom{Bei gewiss} \isi{Subjekt} $>$ Objekt $>$ Objekt\\
ist, wenn die Objekte definite nicht"=pronominale Nominalphrasen sind,
das letzte Objekt \isi{Fokusexponent} von S.
\end{exe}
(Dabei drückt "`A $>$ B"' aus, daß B auf A linear folgt. Die Relation
ist asymmetrisch und transitiv.)

In dieser Formulierung ist (\ref{ex:3-103}) für (\ref{ex:3-48}) und (\ref{ex:3-57}) ebenso anwendbar
wie für (\ref{ex:3-104}a,b):
\begin{exe}
\exi{(48)}{\exalph{48}
Karl hat dem Kind das \textit{Buch} geschenkt}
\exi{(57)}{\exalph{57}\label{ex:3-57-2}
Karl hat dem \textit{Kind} das \textit{Buch} geschenkt}
\ex
\label{ex:3-104}
\begin{xlist}
\ex
\label{ex:3-104a}
Karl hat das Buch dem \textit{Kind} geschenkt
\ex
\label{ex:3-104b}
Karl hat das \textit{Buch} dem \textit{Kind} geschenkt
\end{xlist}
\ex
\label{ex:3-105}
\textit{Hypothese:}
\begin{xlist}
\ex
\label{ex:3-105a}
\emph{Jede} \isi{Betonung} eines Fokusexponenten von K\sub{\textit{i}} führt dazu, daß es einen Fk\sub{m} (K\sub{\textit{i}}) = MK (K\sub{\textit{i}}) gibt.
\ex
\label{ex:3-105b}
In jedem Satz S\sub{\textit{i}} ist in jedem nicht-minimalen \isi{Fokus} Fk\sub{\textit{n}} (S\sub{\textit{i}}) das \isi{Verb} enthalten.
\end{xlist}
\end{exe}
Wenn in einem Satz S\sub{\textit{i}} nur der \isi{Fokusexponent} von S betont ist (und
dieser in S\sub{\textit{i}} nicht das \isi{Verb} ist), folgt aus (\ref{ex:3-105b}), daß
Fk\sub{2} (S\sub{\textit{i}}) = Fk\sub{1} (S\sub{\textit{i}}) + \isi{Verb} und (gegebenenfalls) Fk\sub{3}
(S\sub{\textit{i}}) = Fk\sub{2} (S\sub{\textit{i}}) + K\sub{\textit{j}}. Aus (\ref{ex:3-105a}) folgt, daß diese
Regel für (\ref{ex:3-48}) und (\ref{ex:3-104a}) in derselben Weise
anzuwenden ist wie für (\ref{ex:3-57}) und (\ref{ex:3-104b}). Aus
(\ref{ex:3-103}), (\ref{ex:3-105}) und (\ref{ex:3-66}) folgt dann, daß
(\ref{ex:3-48}) mindestens die Foki (\ref{ex:3-50}a,b,e) hat und daß (\ref{ex:3-57}) mindestens die Foki
(\ref{ex:3-58}a,b,c) hat. Aus
(\ref{ex:3-105}) und
\begin{exe}
\exi{(50)}{\exalph{50}\label{ex:3-50-2}
\begin{xlist}
\ex
\label{ex:3-50a-2}
Fk\sub{1} (\ref{ex:3-48}) = das Buch
\ex
\label{ex:3-50b-2}
Fk\sub{2} (\ref{ex:3-48}) = das Buch + geschenkt
\ex
\label{ex:3-50c-2}
Fk\sub{3} (\ref{ex:3-48}) = dem Kind + das Buch + geschenkt
\ex
\label{ex:3-50d-2}
Fk\sub{4} (\ref{ex:3-48}) = Karl + das Buch + geschenkt
\ex
\label{ex:3-50e-2}
Fk\sub{5} (\ref{ex:3-48}) = MK (\ref{ex:3-48}) = Karl + dem Kind + das Buch + geschenkt
\end{xlist}}

\exi{(58)}{\exalph{58}\label{ex:3-58-2}
\begin{xlist}
\ex
\label{ex:3-7-58a-2}
Fk\sub{1} (\ref{ex:3-57}) = dem Kind + das Buch
\ex
\label{ex:3-7-58b-2}
Fk\sub{2} (\ref{ex:3-57}) = dem Kind + das Buch + geschenkt (= Fk\sub{3} (\ref{ex:3-48}))
\ex
\label{ex:3-7-58c-2}
Fk\sub{3} (\ref{ex:3-57}) = MK (\ref{ex:3-57}) = Karl + dem Kind + das Buch + geschenkt\\
\hphantom{Fk\sub{3} (\ref{ex:3-57}) = MK (\ref{ex:3-57}) =} (= Fk\sub{5} (\ref{ex:3-48}))
\end{xlist}}
\end{exe}
(\ref{ex:3-66}) folgt, daß (\ref{ex:3-57}) höchstens die Foki von (\ref{ex:3-58}) haben kann,
insbesondere nicht etwa Fk\sub{\textit{n}} (\ref{ex:3-57}) = \textit{Karl} + \textit{dem
  Kind} + \textit{das Buch}; ebenso sind für (\ref{ex:3-48}) die denkbaren Foki \textit{dem Kind} + \textit{das Buch} (= Fk\sub{1} (\ref{ex:3-57})), \textit{Karl} +
\textit{das Buch} und \textit{Karl} + \textit{dem Kind} + \textit{das
  Buch} ausgeschlossen. Wir haben keine Hypothese formuliert (und
wollen dies hier auch nicht tun), aus der folgt, daß (\ref{ex:3-48}) die Foki
(\ref{ex:3-50}c,d) hat; aber alle anderen Eigenschaften von (\ref{ex:3-48}) und (\ref{ex:3-57})
folgen aus unseren Annahmen. Sinngemäß gilt das gleiche für (\ref{ex:3-104}a,b).

Zur Illustration zwei weitere Vergleiche. Zu (\ref{ex:3-106}) passen die Fragen
(\ref{ex:3-107}); die entsprechenden Foki sind in (\ref{ex:3-108}) angegeben.
\begin{exe}
\ex
\label{ex:3-106}
\textit{Karl} hat dem Kind das \textit{Buch} geschenkt
\ex
\label{ex:3-107}
\begin{xlist}
\ex
\label{ex:3-107a}
wer hat dem Kind was geschenkt?
\ex
\label{ex:3-107b}
wer hat hinsichtlich des Kindes was getan? was hat das Kind erlebt?
\ex
\label{ex:3-107c}
was ist geschehen?
\end{xlist}
\ex
\label{ex:3-108}
\begin{xlist}
\ex
\label{ex:3-108a}
Fk\sub{1} (\ref{ex:3-106}) = Karl + das Buch
\ex
\label{ex:3-108b}
Fk\sub{2} (\ref{ex:3-106}) = Karl + das Buch + geschenkt (= Fk\sub{4} (\ref{ex:3-48}))
\ex
\label{ex:3-108c}
Fk\sub{3} (\ref{ex:3-106}) = MK (106) = Karl + dem Kind + das Buch + geschenkt \\
\hphantom{Fk\sub{3} (\ref{ex:3-106}) = MK (106) =} (= Fk\sub{5} (\ref{ex:3-48}))
\end{xlist}
\end{exe}
Aus (\ref{ex:3-103}), (\ref{ex:3-105}) und (\ref{ex:3-66}) folgt, daß (\ref{ex:3-106}) genau diese
Foki hat.\footnote{\label{fn:3-22}%
  Die Hypothese (\ref{ex:3-105a}) gilt offenbar nicht innerhalb von Nominalphrasen (zu komplexen NPs
  vgl.\ allgemein \citet{Abraham77} und \citet{Fuchs76}). In der \isi{NP} \textit{den blonden Mann mit dem
    großen Auto} ist das letzte \isi{Substantiv} \textit{Auto} zunächst \isi{Fokusexponent} der PP \textit{mit
    dem großen Auto} und indirekt der ganzen \isi{NP}. Bei einer \isi{Betonung} wie in (i) gibt es jedoch keine
  \isi{Fokusprojektion} zu MK (\isi{NP}), sondern nur den minimalen \isi{Fokus} Fk\sub{1} (i) = \textit{blonden} +
  \textit{Auto}: 
  \ea
  \label{ex:3-fn21i}
  ich suche den \textit{blonden} Mann mit dem großen \textit{Auto}.
  \zlast%
}

Zu (\ref{ex:3-109}) passen die Fragen in (\ref{ex:3-110}), daher die Foki von (\ref{ex:3-111}). Aus den
Hypothesen folgt, daß (\ref{ex:3-109}) mindestens die Foki (\ref{ex:3-111}a,d) hat; wie bei (\ref{ex:3-50}c,d) wird auch hier nicht vorausgesagt, daß die
Foki (\ref{ex:3-111}b,c) existieren.
\begin{exe}
\ex
\label{ex:3-109}
Karl hat dem Kind das \textit{Buch geschenkt}
\ex
\label{ex:3-110}
\begin{xlist}
\ex
\label{ex:3-110a}
was hat Karl hinsichtlich des Kindes getan?
\ex
\label{ex:3-110b}
was hat Karl getan?
\ex
\label{ex:3-110c}
was hat das Kind erlebt?
\ex
\label{ex:3-110d}
was ist geschehen?
\end{xlist}
\ex
\label{ex:3-111}
\begin{xlist}
\ex
\label{ex:3-111a}
Fk\sub{1} (\ref{ex:3-109}) = das Buch + geschenkt (= Fk\sub{2} (\ref{ex:3-48}))
\ex
\label{ex:3-111b}
Fk\sub{2} (\ref{ex:3-109}) = dem Kind + das Buch + geschenkt (= Fk\sub{3} (\ref{ex:3-48}))
\ex
\label{ex:3-111c}
Fk\sub{3} (\ref{ex:3-109}) = Karl + das Buch + geschenkt (= Fk\sub{4} (\ref{ex:3-48}))
\ex
\label{ex:3-111d}
Fk\sub{4} (\ref{ex:3-109}) = MK (\ref{ex:3-109}) = Karl + dem Kind + das Buch + geschenkt\\
\hphantom{Fk\sub{4} (\ref{ex:3-109}) = MK (\ref{ex:3-109}) =} (= Fk\sub{5} (\ref{ex:3-48}))
\end{xlist}
\end{exe}
%\addlines[2.5]
Gehen wir noch kurz auf den Unterschied zwischen (\ref{ex:3-112}a,b) ein. Unter
der Voraussetzung, daß die \isi{Betonung} in (\ref{ex:3-112a}) \isi{fokusprojektiv} ist,
existieren nach (\ref{ex:3-66}) und (\ref{ex:3-105}) die Foki von (\ref{ex:3-113}). Unter derselben
Voraussetzung
\begin{exe}
\ex
\label{ex:3-112}
\begin{xlist}
\ex
\label{ex:3-112a}
Karl hat den \textit{Hund} geschlagen
\ex
\label{ex:3-112b}
Karl hat den Hund \textit{geschlagen}
\end{xlist}
\ex
\pagebreak
\label{ex:3-113}
\begin{xlist}
\ex
\label{ex:3-113a}
Fk\sub{1} (\ref{ex:3-112a}) = den Hund
\ex
\label{ex:3-113b}
Fk\sub{2} (\ref{ex:3-112a}) = den Hund + geschlagen
\ex
\label{ex:3-113c}
Fk\sub{3} (\ref{ex:3-112a}) = MK (\ref{ex:3-112a}) = Karl + den Hund + geschlagen
\end{xlist}
\ex
\label{ex:3-114}
\begin{xlist}
\ex
\label{ex:3-114a}
Fk\sub{1} (\ref{ex:3-112b}) = geschlagen
\ex
\label{ex:3-114b}
Fk\sub{2} (\ref{ex:3-112b}) = Karl + geschlagen
\end{xlist}
\end{exe}
(daß es in (\ref{ex:3-112a}) eine \isi{Fokusprojektion} gibt) ist nach (\ref{ex:3-73}) für (\ref{ex:3-112b}) ein \isi{Fokus} Fk\sub{\textit{n}} (\ref{ex:3-112b}) = \textit{den Hund} + \textit{geschlagen}
ausgeschlossen, so daß nur die Foki (\ref{ex:3-114}) theoretisch möglich
sind. Diese Voraussagen der Hypothesen stimmen mit den empirischen
Fakten genau überein.\\

\largerpage
\noindent Wir haben einige Fragen im Zusammenhang mit Fokusexponenten
betrachtet. Kommen wir nun zu der Frage, was die Unterscheidung von
\isi{Topik} und \isi{Fokus} mit der Unterscheidung von \textsq{alt} und
\textsq{neu} zu tun hat. Sie ist dadurch motiviert, daß man in vielen
Ausführungen zur Topik-Fokus-Unterscheidung Bestimmungen der Art
findet, daß der \isi{Fokus} \textsq{neue Information} enthalte, während im
\isi{Topik} \textsq{gegebene (alte) Information} enthalten sei. Wenn man das
\zb auf den minimalen \isi{Fokus} von (\ref{ex:3-48}) \textit{das Buch} anwendet, erhebt sich
die Frage, in welchem Sinne eine \isi{Nominalphrase} überhaupt
\textsq{Information} enthalten soll. Am ehesten wird man darunter
verstehen, daß durch die \isi{NP} ein bis dahin unbekannter Referent in den
Diskurs eingeführt wird. Das kann mit definiten NPs wie \textit{das Buch}
im Allgemeinen jedoch nicht geschehen; deren Verwendung setzt im Allgemeinen voraus,
daß ihr Referent aus RK bekannt ist.\footnote{\label{fn:3-23}%
    Dies gilt für NPs wie \textit{das Buch}, aber nicht für definite NPs
  schlechthin. Mit Ausdrücken wie \textit{mein Sohn, mein Nachbar, der Chef
  meiner Frau} kann ohne weiteres ein neuer Referent in den Diskurs
  eingeführt werden. Starke Beschränkungen für diese Verwendung
  definiter NPs gibt es bei Eigennamen. Zu einigen Aspekten solcher
  Fälle vgl.\ \citet{Prince78, Prince79}.~-- Definite \isi{Personalpronomen} können überhaupt nur gebraucht werden, wenn ihr Referent aus RK bekannt ist, \dash wenn er aufgrund des weiteren
  Kontexts identifizierbar ist wie in (\ref{ex:3-63a}), (\ref{ex:3-64b}) oder aber auf eine
  \textsq{koreferente} \isi{NP} in einem anderen Teil-Satz (clause) des gleichen Satzgefüges folgt wie \zb in (ii) von Fn.\,\ref{fn:3-17}. (Das sind notwendige Bedingungen, nicht hinreichende.) 

  Während es für definite
  NPs in dieser Hinsicht gewisse Gebrauchsbeschränkungen gibt, sind präsupponierte Sätze völlig frei. Sie können nicht nur dann im \isi{Fokus} stehen, wenn ihre Funktion im Satzgefüge nicht aus RK bekannt ist, sondern auch zur Vermittlung neuer Information dienen. So kann (i)
  \ea
  \label{ex:3-23i}
  Karl wußte, daß das Leben in Norwegen teuer ist
  \z
  verwendet werden, um dem Hörer/""Leser mitzuteilen, daß das Leben in
  Norwegen teuer ist; entsprechend \zb bei (\ref{ex:3-23iia}), etwa als Antwort
  auf (\ref{ex:3-23iib}), oder in (\ref{ex:3-23iiib}) im Kontext (\ref{ex:3-23iiia}):
  \eal
  \label{ex:3-23ii}
  \ex
  \label{ex:3-23iia}
  Karl freut sich, daß sein Freund zu Besuch kommt
  \ex
  \label{ex:3-23iib}
  warum ist Karl so vergnügt?
  \zlmid
  \eal
  \label{ex:3-23iii}
  \ex
  \label{ex:3-23iiia}
  Karl war erkrankt
  \ex
  \label{ex:3-23iiib}
  da es Maria bedrückte, daß sie Karl nicht helfen konnte, fühlte sie
  sich unglücklich
  \zl
  Wir wissen aus der Diskussion im Zusammenhang mit (\ref{ex:3-43b}), daß die
  Elemente von RK nicht notwendig von Sprecher und Hörer als wahr
  akzeptiert werden, und wir sehen hier, daß Propositionen, die der
  Sprecher notwendig als wahr akzeptiert, nicht Elemente von RK sein
  müssen. \textsq{Relevanter Kontext} und \textsq{logische Präsupposition} sind also nicht nur definitorisch, sondern auch empirisch völlig verschiedene Gegenstände. Vgl.\ die ausführliche
  Diskussion in \citet[3.3.3]{Reis77}.

  (Unter manchen sprachwissenschaftlichen Präsuppositionstheorien, die
  die Existenzpräsupposition von Eigennamen und die Wahrheitspräsupposition von faktiven Nebensätzen als essentiell gleiches Phänomen behandeln, ist dieser Unterschied überraschend. Ganz anders in Freges Theorie der Nebensätze: Der Gedanke, daß ein
  gegebener Eigenname etwas bezeichnet, ist nicht Teil des Gedankens,
  den ein Satz ausdrückt oder bezeichnet, der den Eigennamen enthält,
  sondern er ist \emph{vorausgesetzt} \citep[40]{Frege1892}. Faktive Nebensätze
  haben nach Frege dagegen im Allgemeinen eine doppelte Funktion: Zum einen
  bezeichnen sie den Gedanken, den der Nebensatz sonst (\dash als
  selbständiger Aussagesatz) ausdrückt; zum anderen \emph{drücken} sie diesen
  Gedanken auch selber \emph{aus} \citep[47f]{Frege1892}, ihre Wahrheit wird
  also nicht vorausgesetzt, sondern behauptet. Vgl.\ zu interessanten
  Aspekten und Problemen dieser Auf"|fassung \citeauthor{Hoehle79c} (in Vorb.)).%
}
In diesem Sinne kann daher \textit{das
Buch} nicht \textsq{neue Information} sein. (Diese Kritik ist häufig
erhoben worden, cf.\ \zb \citet{Weiss75}). \textsq{Neu}, \dash nicht aus RK bekannt, ist vielmehr, wie wir in (\ref{ex:3-32}) ausgeführt haben, die Information, daß bei dem Vorgang des Schenkens, bei dem Karl als Agens und das Kind als Empfänger beteiligt waren, \textit{das Buch}
als Objekt betroffen war, \dash die Funktion, die \textit{das Buch} in (\ref{ex:3-48})
hat.\footnote{\label{fn:3-24}%
	Dies wird in einem großen Teil der Literatur übersehen, aber durchaus
  nicht überall: "`Note that in the representation in (\ref{ex:3-7}) the focus
  component of the semantic reading is given as a semantic \textit{relation}, not
  a single term. This reflects the fact that the focus constituent of a
  sentence represents novel information not because the constituent
  itself is necessarily novel, but rather because the semantic
  relation which the constituent enters into is novel with respect to a
  given universe of discourse"' \citep[218]{Akmajian70}.%
}

Definite Nominalphrasen können also im Allgemeinen ohne weiteres im \isi{Fokus}
sein; es gibt gar keinen Grund, warum dies anders sein sollte. Für
\isi{Personalpronomen} gilt empirisch jedoch eine eigenartige Regularität,
die man wie in (\ref{ex:3-115}) formulieren kann:
\begin{exe}
\ex
\label{ex:3-115}
\textit{Hypothese:}\\
Wenn zwei Sätze S\sub{\textit{i}}, S\sub{\textit{j}}
\begin{xlist}
\ex
\label{ex:3-115a}
sich außer durch die \isi{Betonung} nur dadurch unterscheiden, daß anstelle
einer \isi{NP}\sub{\textit{i}} in S\sub{\textit{i}}, die nicht"=pronominal ist, in S\sub{\textit{j}} eine \isi{NP}\sub{\textit{j}} steht,
die aus einem \isi{Personalpronomen} besteht, und wenn es
\ex
\label{ex:3-115b}
in S\sub{\textit{i}} eine \isi{Fokusprojektion} von Fk\sub{1} (S\sub{\textit{i}}) = \isi{NP}\sub{\textit{i}} zu Fk\sub{\textit{n}}
(S\sub{\textit{i}}) gibt, dann gibt es \ex
\label{ex:3-115c}
bei Betonung des Verbs in S\sub{\textit{j}} eine \isi{Fokusprojektion} von Fk\sub{1} (S\sub{\textit{j}})
= \isi{Verb} zu Fk\sub{2} (S\sub{\textit{j}}) = \isi{Verb} + \isi{NP}\sub{\textit{j}}.
\end{xlist}
\end{exe}
Vergleichen wir zur Illustration (\ref{ex:3-116}a,b). Die beiden Sätze
unterscheiden sich gemäß (\ref{ex:3-115a}) außer durch ihre \isi{Betonung} dadurch,
daß an Stelle der substantivischen \isi{NP} \textit{den Hund} von (\ref{ex:3-116a}) das
\isi{Personalpronomen} \textit{mich} in (\ref{ex:3-116b}) steht. Entsprechend (\ref{ex:3-115b}) gibt es
in (\ref{ex:3-116a}) eine \isi{Fokusprojektion} von Fk\sub{1} (\ref{ex:3-116a}) = \textit{den Hund} zu Fk\sub{2}
(\ref{ex:3-116a}) = \textit{den Hund} + \textit{getreten}.
\begin{exe}
\ex
\label{ex:3-116}
\begin{xlist}
\ex
\label{ex:3-116a}
Karl hat den \textit{Hund} getreten
\ex
\label{ex:3-116b}
Karl hat mich \textit{getreten}
\ex
\label{ex:3-116c}
Karl hat den Hund \textit{getreten}
\ex
\label{ex:3-116d}
Karl hat \textit{mich} getreten
\end{xlist}
\end{exe}
(Nach (\ref{ex:3-105b}) muß ein Fk\sub{\textit{n}} in (\ref{ex:3-115b}) das \isi{Verb}
enthalten). Entsprechend (\ref{ex:3-115c}) gibt es in (\ref{ex:3-116b}) eine
\isi{Fokusprojektion} von Fk\sub{1} (\ref{ex:3-116b}) = \textit{getreten} zu Fk\sub{2} (\ref{ex:3-116b})
= \textit{mich} + \textit{getreten}. Wichtig ist dabei, daß es bei den umgekehrten Betonungsverhältnissen in (\ref{ex:3-116}c,d) nach (\ref{ex:3-73}) in (\ref{ex:3-116c}) keinen \isi{Fokus} \textit{den Hund} + \textit{getreten} und in (\ref{ex:3-116d}) keinen \isi{Fokus} \textit{mich} + \textit{getreten} gibt. Solange nicht
spezielle Zusatzhypothesen aufgestellt werden (für die es keine empirische Motivation gibt), folgt daraus, daß \isi{Personalpronomen}
generell nicht Ausgangspunkt einer \isi{Fokusprojektion} sein können: Sie sind entweder unbetont; dann können sie nach (\ref{ex:3-66}) keinen minimalen
\isi{Fokus} bilden. Oder sie sind betont; dann geht aber eine denkbare \isi{Fokusprojektion} nach (\ref{ex:3-115c}) vom \isi{Verb} und (nach (\ref{ex:3-73})) nicht von
ihnen aus. Natürlich heißt das nicht, daß \isi{Personalpronomen} nicht im \isi{Fokus} sein könnten: In (\ref{ex:3-116b}) ist das \isi{Pronomen} Teil von Fk\sub{2}
(\ref{ex:3-116b}), in (\ref{ex:3-116d}) bildet es den einzigen möglichen \isi{Fokus}.

Dieselben Überlegungen sind auf (\ref{ex:3-117}), (\ref{ex:3-118}) anzuwenden: In (\ref{ex:3-117a}) mit einer substantivischen \isi{NP} führt die \isi{Betonung} der \isi{NP} zu einer
\isi{Fokusprojektion}, in (\ref{ex:3-117b}) mit einem \isi{Personalpronomen} die \isi{Betonung}
des Verbs.
\begin{exe}
\ex
\label{ex:3-117}
\begin{xlist}
\ex
\label{ex:3-117a}
es heißt, daß dein \textit{Vater} kommt
\ex
\label{ex:3-117b}
es heißt, daß er \textit{kommt}
\end{xlist}
\ex
\label{ex:3-118}
\begin{xlist}
\ex
\label{ex:3-118a}
es heißt, daß dein Vater \textit{kommt}
\ex
\label{ex:3-118b}
es heißt, daß \textit{er} kommt
\end{xlist}
\end{exe}
Umgekehrt ist bei der Betonung des Verbs in (\ref{ex:3-118a}) bzw.\ bei \isi{Betonung}
des Personalpronomens in (\ref{ex:3-118b}) keine \isi{Fokusprojektion} möglich.

Es ist häufig (wenn auch undeutlich) bemerkt worden, daß es mit definiten \isi{Personalpronomen} eine besondere Bewandtnis hat. (Selbst wenn
es ein allgemeines Betonungsverbot für \isi{Personalpronomen} gäbe~-- was es nicht gibt, cf.\ (\ref{ex:3-116d}), (\ref{ex:3-118b})~--, würde (\ref{ex:3-115c}) daraus nicht
folgen; man beachte, daß (\ref{ex:3-116c}) zwar keinen \isi{Fokus} \textit{den Hund} + \textit{getreten}, wohl aber Fk\sub{2} (\ref{ex:3-116c}) = \textit{Karl} + \textit{getreten} hat). Die Besonderheiten werden oft damit in Zusammenhang gebracht, daß \isi{Personalpronomen} per Definition \textsq{bekannte Information} darstellten oder allgemein \textsq{anaphorisch} seien. \textsq{Bekannte Information} stellen definite \isi{Personalpronomen} aber nur in demselben Sinne dar wie viele definite Nominalphrasen ganz allgemein; für diese gilt (\ref{ex:3-115c}) jedoch nicht. \textsq{Anaphorisch} sind \isi{Personalpronomen} nicht ohne weiteres: Die \isi{Personalpronomen} der ersten und zweiten Person sind per Definition nie \isi{anaphorisch} (verhalten sich aber gemäß (\ref{ex:3-115c})), und die der dritten Person sind es nicht unbedingt: Man kann, unter geeigneten Umständen, ohne weiteres ein \isi{Pronomen} der 3.\ Person verwenden, ohne daß irgendein sprachlicher Kontext gegeben ist. Obendrein verhalten sich echt anaphorische NPs offenbar nicht allgemein wie \isi{Personalpronomen}. Im übrigen gilt (\ref{ex:3-115c}) auch für indefinite \isi{Pronomen} wie \textit{jemand}, die in keinem Sinne des Wortes \textsq{bekannte Information} darstellen oder \isi{anaphorisch} sind.

Um zu prüfen, wie sich echt anaphorische Nominalphrasen verhalten,
betrachten wir noch einmal das Beispiel (\ref{ex:3-9-2}). Wir haben in \ref{subsec:3-1-2} festgestellt, daß man, soweit keine zusätzliche Information
gegeben ist, nur (\ref{ex:3-9b-2}) und nicht 
\begin{exe}
\exi{(9)}\exalph{9}\label{ex:3-9-2}
\begin{xlist}
\ex \exalph{9a}
\label{ex:3-9a-2}
er will seinem Freund das \textit{Auto} schenken
\ex \exalph{9b}
\label{ex:3-9b-2}
er will das Auto seinem \textit{Freund} schenken
\ex \exalph{9c}
\label{ex:3-9c-2}
Karl hat gestern einen \textit{Porsche} gekauft
\end{xlist}
\end{exe}
%\addlines[2]
(\ref{ex:3-9a-2}) als möglichen Nachfolgersatz zu (\ref{ex:3-9c-2}) ansehen wird. Woran liegt das? Offenbar versteht man (mangels zusätzlicher Information) \textit{das Auto}
als \isi{Anapher} zu \textit{Porsche}. Ist es also so, daß sich Anaphern ähnlich wie \isi{Pronomen} verhalten, indem sie bei \isi{Betonung} keine \isi{Fokusprojektion} zu einem \isi{Fokus} auslösen, der das \isi{Verb} umfaßt (denn das \isi{Verb} muß im Nachfolgersatz zu (\ref{ex:3-9c-2}) offenbar im \isi{Fokus} sein)? Betrachten wir zum Vergleich die Satzfolge in (\ref{ex:3-119}). In diesem Kontext scheint mir die \isi{Betonung} von \emph{Auto}
\eal
\label{ex:3-119}
\ex
\label{ex:3-119a}
Karl ist ein verrückter Kerl
\ex
\label{ex:3-119b}
erst \textit{neulich} hat er sich einen \textit{Porsche} gekauft, und
weißt du, was er \textit{jetzt} vorhat?  
\ex
\label{ex:3-119c}
er will seinem Freund das \textit{Auto} schenken und sich per Schiff auf eine
\textit{Welt}reise begeben
\zl
in (\ref{ex:3-119c}) völlig einwandfrei zu sein, obwohl es nach wie vor eine
\isi{Anapher} von \textit{Porsche} in (\ref{ex:3-119b}) ist. Ähnlich in (\ref{ex:3-120}). Aus dem
Kontext ist klar, daß \textit{das Auto} + \textit{verkaufen} als \isi{Fokus}
von (\ref{ex:3-120d})
\eal
\label{ex:3-120}
\ex
\label{ex:3-120a}
Karl geht es prächtig
\ex
\label{ex:3-120b}
er hat ein fabelhaftes \textit{Ein}kommen, besitzt einen neuen \textit{Porsche}, ein
\textit{Haus}, und obendrein eine Apparte\textit{ment}wohnung
\ex
\label{ex:3-120c}
und weißt du, was der komische Kerl \textit{vor}hat?
\ex
\label{ex:3-120d}
er will das \textit{Auto} verkaufen
\zl
intendiert ist, und dies, obwohl \textit{Auto} \isi{Anapher} von
\textit{Porsche} in (\ref{ex:3-120b}) und betont ist.

Wenn diese Beobachtungen zutreffen, bietet sich eine Erklärung für die Verhältnisse in (\ref{ex:3-9}) mit Hilfe Gricescher Konversationsmaximen an. In
(\ref{ex:3-119c}) wie in (\ref{ex:3-120d}) ist \textit{Auto} aufgrund des RK notwendig Teil des \isi{Fokus} und kann deshalb in dieser Konfiguration betont
auftreten. In (\ref{ex:3-9}) dagegen würde es gegen die Maxime der Quantität verstoßen, wenn \textit{Auto} zum \isi{Fokus} gehören würde: In der Abfolge
(\ref{ex:3-9c}) vor (\ref{ex:3-9b}) wie (\ref{ex:3-9c}) vor (\ref{ex:3-9a}) ist es aus RK bekannt, daß
zwischen Karl und dem Auto eine gewisse Relation besteht. Als neue Information kommt nur in Frage, daß diese Relation ein Vorgang des
Schenkens mit Karls Freund als Empfänger ist; daher darf, bei Wahrung konversationeller Maximen, nur \textit{seinem Freund} + \textit{schenken} zum \isi{Fokus}
gehören. Dies ist in (\ref{ex:3-9a}) nicht möglich, wohl aber in (\ref{ex:3-9b}).

Wenn diese Deutung richtig ist, können wir feststellen: Für die
Fokusmöglichkeiten von \isi{Personalpronomen} gilt eine besondere Regel
(\ref{ex:3-115}), die von der Regel für nicht"=pronominale NPs abweicht. Für
definite NPs gibt es im Allgemeinen keine besonderen Beschränkungen
hinsichtlich der Verteilung auf \isi{Topik} und \isi{Fokus}; die Verteilung von
anaphorischen (definiten) NPs folgt allgemeinen konversationellen
Maximen. Wie steht es mit indefiniten NPs?

Im Gegensatz zu definiten NPs dienen indefinite NPs nach gängiger
Vorstellung typischerweise dazu, bis dahin unbekannte Referenten in
den Diskurs einzuführen; auf einen so eingeführten Referenten kann
man sich im nachfolgenden Diskurs nicht erneut mit einer indefiniten
\isi{NP} beziehen. Man könnte meinen, daß aus der Bestimmung von
\textsq{Fokus} in (\ref{ex:3-32}) und (\ref{ex:3-37}) folgt, daß NPs mit dieser Funktion nur
Teil des \isi{Fokus}, nicht Teil des Topiks sein können. Dies ist jedoch
nicht so. (Daß indefinite NPs nicht notwendig im \isi{Fokus} sind, hat
m.\,W.\ erstmals \citet[82]{Heidolph70} betont.) Dazu zwei Beispiele.
\begin{exe}
\ex
\label{ex:3-121}
\begin{xlist}
\ex
\label{ex:3-121a}
A: hast du schon gehört?
\ex
\label{ex:3-121b}
~\phantom{A} Karl soll ein \textit{Kind} erschlagen haben
\ex
\label{ex:3-121c}
B: \textit{gehört} habe ich das \textit{auch}, aber es \textit{stimmt} nicht
\ex
\label{ex:3-121d}
~\phantom{B} Karl hat ein Kind über\textit{fahren}, und zwar \textit{ohne} eigene \textit{Schuld}
\end{xlist}
\end{exe}
Bei Äußerung von (\ref{ex:3-121d}) ist aus dem RK bekannt, daß es einen Vorgang
gegeben hat, an dem Karl und ein Kind beteiligt waren, \textit{ein Kind} ist
also Teil des Topiks. Bei Äußerung von (\ref{ex:3-122c}) ist aus RK
\eal
\label{ex:3-122}
\ex
\label{ex:3-122a}
A: wie ich höre, hat Karl eine \textit{Amerikanerin} geheiratet
\ex
\label{ex:3-122b}
B: das ist eine Verwechslung
\ex
\label{ex:3-122c}
~\phantom{B} \textit{Heinz} hat eine Amerikanerin geheiratet
\zl
bekannt, daß jemand eine Amerikanerin geheiratet hat; \textit{eine Amerikanerin} kann deshalb in (\ref{ex:3-122c}) nicht im \isi{Fokus} sein. Gleichwohl kann man den Dialog (\ref{ex:3-121}a--d) durch (\ref{ex:3-121e}) fortsetzen, wo die definite \isi{NP} \textit{das Kind} darauf hinweist, daß durch \textit{ein Kind} in (\ref{ex:3-121d}) ein Referent festgelegt worden ist; entsprechend kann (\ref{ex:3-122d}) auf (\ref{ex:3-122c}) folgen, wo das anaphorische \textit{sie} klar macht, daß durch \textit{eine Amerikanerin} in (\ref{ex:3-122c}) ein Referent eingeführt worden ist. In beiden Fällen akzeptiert der Sprecher B bei Äußerung von (\ref{ex:3-121d}) bzw.\ (\ref{ex:3-122c}) die in RK enthaltene Charakterisierung für \textsq{ein Kind}
\begin{exe}
\exi{(121)}{\exalph{121}\label{ex:3-121-2}
\begin{xlist}
\exi{e.}{\exalph{121e}\label{ex:3-121e}
B: dem Kind ist aber nicht viel passiert}
\end{xlist}}
\exi{(122)}{\exalph{122}\label{ex:3-122-2}
\begin{xlist}
\exi{d.}{\exalph{122d}\label{ex:3-122d}
B: sie soll sehr reich sein}
\end{xlist}}
\end{exe}
bzw.\ \textsq{eine Amerikanerin} nicht als ausreichend, um einen
bestimmten Referenten festzulegen; er betrachtet diese NPs als
\textsq{nicht-spezifisch}. Durch die Äußerung von (\ref{ex:3-121d}) bzw.\ (\ref{ex:3-122c}) wird jedoch, jedenfalls für die Zwecke von B, der Referent eindeutig bestimmt als das Kind, das Karl überfahren hat bzw.\ als die Amerikanerin, die Heinz geheiratet hat. Man sieht daraus, daß nicht
die indefinite \isi{NP} selbst einen Referenten einführt (dann könnte sie in (\ref{ex:3-121d}) und (\ref{ex:3-122c}) tatsächlich nach (\ref{ex:3-32}) und (\ref{ex:3-37}) nur zum \isi{Fokus}
gehören), sondern der ganze Satz dient~-- unter geeigneten Umständen~-- dazu, einen Referenten für die \isi{NP} zu etablieren. Daher ist es nicht überraschend, daß indefinite NPs im \isi{Topik} vorkommen und dennoch zur Festlegung eines neuen Referenten beitragen können.

\subsubsection{Explanatorische Fruchtbarkeit}
\label{subsubsec:3-1-4-4}

Die in (\ref{ex:3-79}) formulierte Explikation von \textsq{stilistisch normaler Betonung} ist nicht nur heuristisch fruchtbar, sondern sie hat Erklärungswert. Dies halte ich für das wichtigste an dem ganzen Ansatz: Sie macht verständlich, \textit{warum} ein gegebener Satz als stilistisch normal bzw.\ nicht"=normal betont empfunden wird, und in ihrem Licht sieht man, was an den in \ref{subsec:3-1-2} besprochenen Explikationsversuchen richtig ist und wo ihre Fehler liegen.

Alle Versuche, die auf \textsq{Kontextungebundenheit} von
\isi{Normalbetonung} abheben ((\ref{ex:3-7}), (\ref{ex:3-8}), (\ref{ex:3-10}), (\ref{ex:3-11})), sind insofern korrekt, als normal betonte Sätze in mehr Kontexttypen auftreten können als
entsprechende nicht"=normal betonte Sätze; das heißt nicht~-- und da
liegt der Fehler dieser Versuche~--, daß sie in beliebigen Kontexttypen auftreten könnten oder auch nur in allen Kontexttypen, in denen dieselbe \isi{Wortfolge} mit anderem \isi{Betonungsmuster} möglich ist. Im Gegenteil: Normalbetonte Sätze können in den meisten Fällen gerade nicht in denselben Kontexten wie die entsprechenden nicht"=normal betonten auftreten; in dem \isi{Kontexttyp} z.\,B., der für (\ref{ex:3-123a}) natürlich ist, ist (\ref{ex:3-123b}) trotz \isi{Normalbetonung} unmöglich:
\eal
\label{ex:3-123}
\ex
\label{ex:3-123a}
\textit{Karl} hat den Hund getreten
\ex
\label{ex:3-123b}
Karl hat den \textit{Hund} getreten
\zl
Auch die Explikation (\ref{ex:3-13}), die auf \textsq{Hervorhebung} bei
nicht"=normaler \isi{Betonung} abhebt, ist partiell korrekt: Bei normaler
\isi{Betonung} ist die betonte \isi{Konstituente} im Allgemeinen nicht der einzige
mögliche \isi{Fokus} des Satzes, so daß kein \isi{Fokus} besonders ausgezeichnet
ist; viele Fälle von nicht"=normaler \isi{Betonung} dagegen sind
nicht"=\isi{fokusprojektiv}, so daß nur die betonte(n) \isi{Konstituente}(n) im
\isi{Fokus} und insofern besonders \textsq{hervorgehoben} ist/""sind;
cf.\ besonders (\ref{ex:3-80}). In einem gegebenen Kontext jedoch kann auch bei
\isi{Normalbetonung} der \isi{Fokus} minimal und damit \textsq{hervorgehoben}
sein.

Im typischen Fall kann bei einem normalbetonten Satz S\sub{\textit{i}} der ganze
Satz im \isi{Fokus} sein; das \isi{Topik} ist dann leer, und kein Teil der LC von
S\sub{\textit{i}} ist in RK enthalten. Natürlicherweise sind solche Sätze für
Text- oder Diskursanfänge entsprechend (\ref{ex:3-10}), (\ref{ex:3-11}) besonders geeignet,
während innerhalb eines fortlaufenden Textes/""Diskurses gemäß
konversationellen Maximen ein Nachfolgersatz gewöhnlich auf den RK
Bezug nehmen muß; daher sind Äußerungen von normalbetonten Sätzen mit
maximalem \isi{Fokus} dort seltener zu finden.

Es ist zu erwarten, daß solche Sätze, wenn sie ohne jeden Kontext
präsentiert werden, eher als \textsq{normal} empfunden werden als
solche, die notwendig ein \isi{Topik} und damit implizit den Bezug auf einen
RK enthalten.\footnote{\label{fn:3-25}%
	Wir haben erwähnt, daß außer \isi{Topik}/""\isi{Fokus} u.\,a.\ auch die
  Identifizierbarkeit der Referenten von NPs für die Verwendbarkeit
  eines Satzes in gegebenem Kontext eine Rolle spielt,
  vgl.\ Fn.\,\ref{fn:3-23}. Daher ist \zb (i) von Fn.\,\ref{fn:3-2} nur unter besonders engen
  Voraussetzungen als \textsq{Textanfang} möglich (was auch immer man
  unter \textsq{Text} versteht).%
}
Auf diese Weise kommen Stockwells \textsq{citation
  patterns} zustande. Die Abwesenheit von \textsq{additional
  components or differential meaning} resultiert daraus, daß dem \isi{Fokus}
von S\sub{\textit{i}} kein \isi{Topik} gegenüber steht, wenn Fk (S\sub{\textit{i}}) = MK (S\sub{\textit{i}}).

Auch die Tatsache, daß Kiparsky und Bierwisch für
\textsq{Normalbetonung} genau 1 \isi{Hauptakzent} annehmen, so daß Fälle mit
mehr voll betonten Konstituenten in (\ref{ex:3-15}) als \textsq{kontrastiv}
ausgezeichnet werden, wird begreifbar: Im typischen Fall haben Sätze
mit \textit{n} betonten Konstituenten weniger Fokusmöglichkeiten als
vergleichbare Sätze mit \textit{m} betonten Konstituenten, wenn \textit{n} $>$ \textit{m};
cf.\ \zb (\ref{ex:3-48}) vs.\ (\ref{ex:3-57}), (\ref{ex:3-106}), (\ref{ex:3-109}).

Da die Explikation von \textsq{Normalbetonung} in eine pragmatische
Kontexttheorie \isi{eingebettet} ist, erlaubt sie die Klärung eines
bekannten Problems: Wieso wird in (\ref{ex:3-65-3}) \textit{insult} als \isi{Anapher} von
\textit{call s.o.\ a Republican} verstanden, und wieso kann dort
\textit{he} eine \isi{Anapher} von \textit{Bill} und \textit{him} eine
\isi{Anapher} von \textit{John} sein, aber nicht
umgekehrt?\footnote{\label{fn:3-26}%
  Für frühere Diskussionen zu diesen
  Fragen vgl.\ u.\,a.\ \citet{Lakoff71}, \citet[63-75]{Schmerling76},
  \citet[§8.6.1.]{Kempson75} und besonders \citet[§ 2.1.]{Prince79}.%
}
\begin{exe}
\exi{(65)}\exalph{65}\label{ex:3-65-3} John called Bill a Republican, and then \textit{he} insulted \textit{him} 
\end{exe}
Wir beobachten zunächst, daß im zweiten \isi{Konjunkt} von (\ref{ex:3-124a}) \textit{der Kunde} +
\textit{wurde} + \textit{unverschämt} ein möglicher und in diesem Kontext naheliegender
\isi{Fokus} ist. In (\ref{ex:3-124b}) dagegen kann nur \textit{der Kunde} \isi{Fokus} sein; \textit{wurde} +
\textit{unverschämt} ist \isi{Topik}. Damit muß aus dem RK
\eal
\label{ex:3-124}
\ex
\label{ex:3-124a}
der Lehrling hat den \textit{Meister} beschimpft, und dann wurde der Kunde \textit{unverschämt}
\ex
\label{ex:3-124b}
der Lehrling hat den \textit{Meister} beschimpft, und dann wurde der \textit{Kunde} unverschämt
\zl
bekannt sein, daß jemand unverschämt wurde. In diesem Kontext ist der
Zusammenhang klar: Den Meister zu beschimpfen kann als eine Form des
Unverschämtseins betrachtet werden. Daher wirkt (\ref{ex:3-125a}) redundant,
wenn mit \textit{er} der Lehrling gemeint ist. Dieser Eindruck verstärkt sich
\eal
\label{ex:3-125}
\ex
\label{ex:3-125a}
der Lehrling hat den \textit{Meister} beschimpft, und dann wurde er \textit{unverschämt}
\ex
\label{ex:3-125b}
der Lehrling hat den \textit{Meister} beschimpft, und dann wurde \textit{er} unverschämt
\zl
in (\ref{ex:3-125b}): Hier kann nur \textit{er} einziger \isi{Fokus} sein; daß jemand unverschämt war, muß aus RK bekannt sein. Aus dem ersten \isi{Konjunkt} ist
bekannt, daß der Lehrling unverschämt war (indem er den Meister beschimpfte); deshalb kann es keine neue Information sein, daß er, der
Lehrling, unverschämt wurde. Mit \textit{er} kann hier nur eine dritte Person gemeint sein (was ohne weiteren Kontext nicht nahe liegt) oder der
Meister. Dies ist die intuitiv nahe liegende Interpretation; sie folgt aus den Fokusverhältnissen.

Wir halten fest: Ein unbetontes \isi{Prädikat} wird, da es in diesen Fällen
\isi{Topik} ist, im Allgemeinen als \isi{Anapher} eines vorhergehenden Prädikats
interpretiert; ein betontes \isi{Personalpronomen} kann in solchen Fällen
natürlicherweise nicht eine \isi{Anapher} des vorhergehenden Subjekts
sein. Völlig deutlich wird dies in (\ref{ex:3-126b}). Wenn \textit{er} in (\ref{ex:3-126a}) eine
\eal
\label{ex:3-126}
\ex
\label{ex:3-126a}
der Lehrling hat den \textit{Meister} geschlagen, und dann hat er den \textit{Meister} geschlagen
\ex
\label{ex:3-126b}
der Lehrling hat den \textit{Meister} geschlagen, und dann hat \textit{er} den Meister geschlagen
\zl
\isi{Anapher} von \textit{Lehrling} sein soll, ist das zweite \isi{Konjunkt} eine
Wiederholung des ersten (und deshalb konversationell aberrant, weil
nach \textit{dann} keine Wiederholung erwartet wird). Wenn dasselbe für (\ref{ex:3-126b}) gelten sollte, wäre darüber hinaus die Tatsache, daß es der Lehrling
war, der den Meister geschlagen hat, als neue zusätzliche Information
ausgezeichnet, was absurd wäre: Hier muß \textit{er} eine dritte Person
bezeichnen.

Dabei kommt es natürlich nicht wesentlich darauf an, daß ein betontes
\textit{er} keine \isi{Anapher} des syntaktischen Subjekts im ersten \isi{Konjunkt} sein
kann; aufgrund der Definition von semantischem \isi{Fokus} geht es vielmehr
um die Relationen in der logischen Charakterisierung der Sätze. Daher
finden wir den oberflächlich umgekehrten Fall, wenn das zweite
\isi{Konjunkt} passiviert ist. Im zweiten \isi{Konjunkt} von (\ref{ex:3-126c}) ist
\textit{beschimpft} + \textit{worden} \isi{Topik}, \dash daß ein Äquivalent von "`jemand hat
jemand beschimpft"'
\begin{exe}
\exi{(126)}\exalph{126}\label{ex:3-126-2}
\begin{xlist}
\exi{c.}\exalph{126c}\label{ex:3-126c} der Lehrling hat den \textit{Meister} beschimpft, und dann ist \textit{er} beschimpft worden
\end{xlist}
\end{exe}
in RK ist. Wollte man \textit{er} auf \emph{Meister} beziehen, wäre als neue
Information gekennzeichnet, daß er, der Meister, es war, der
beschimpft wurde~-- aber das ist aus dem ersten \isi{Konjunkt} schon
bekannt. Daher kann hier mit \textit{er} nur der Lehrling oder eine
dritte Person gemeint sein.

Bei zweistelligen Verben sind außerdem die Objektpronomen zu beachten. In (\ref{ex:3-127a}) kann nur \textit{er} den einzigen \isi{Fokus} bilden; genau
parallel zu (\ref{ex:3-126b}) muß aus RK bekannt sein, daß jemand den Referenten von \textit{ihn} gepeitscht hat; \textit{ihn} ist also \isi{Anapher} von
\textit{Meister}, und \textit{er}
\eal
\label{ex:3-127}
\ex
\label{ex:3-127a}
der Lehrling hat den \textit{Meister} gepeitscht, und dann hat \textit{er} ihn gepeitscht
\ex
\label{ex:3-127b}
der Lehrling hat den \textit{Meister} gepeitscht, und dann hat er \textit{ihn} gepeitscht
\ex
\label{ex:3-127c}
der Lehrling hat den \textit{Meister} gepeitscht, und dann hat er den \textit{Chef} gepeitscht
\ex
\label{ex:3-127d}
der Lehrling hat den \textit{Meister} gepeitscht, und dann hat \textit{er} \textit{ihn} gepeitscht
\ex
\label{ex:3-127e}
der Lehrling hat den \textit{Meister} gepeitscht, und dann hat \textit{er} den \textit{Chef} gepeitscht
\ex
\label{ex:3-127f}
der Lehrling hat den \textit{Meister} gepeitscht, und dann hat der \textit{Kunde} \textit{ihn} gepeitscht
\zl
kann aus den oben besprochenen Gründen nicht \isi{Anapher} von
\textit{Lehrling} sein. In (\ref{ex:3-127b}) ist \textit{ihn} einziger möglicher
\isi{Fokus}, und aus RK muß bekannt sein, daß der Referent von \textit{er} jemand gepeitscht hat. Mit \textit{er} kann hier deshalb nur der Lehrling gemeint sein; \textit{ihn} muß eine dritte Person bezeichnen. (\ref{ex:3-127c}) ist genau entsprechend. Danach ist es klar, daß in (\ref{ex:3-127d}), wo \textit{er} + \textit{ihn} den \isi{Fokus} bildet, \textit{er} keine \isi{Anapher}
von \textit{Lehrling} und \textit{ihn} keine \isi{Anapher} von \textit{Meister} sein kann, denn sonst entspräche der \isi{Fokus} einem Teil des RK. Beide \isi{Pronomen} können hier Personen bezeichnen, die im ersten \isi{Konjunkt} nicht genannt sind; oder aber \textit{ihn} ist \isi{Anapher} von \textit{Lehrling} und/""oder \textit{er} ist \isi{Anapher} von \textit{Meister}. Dies sind genau die intuitiv möglichen Interpretationen von (\ref{ex:3-127d}). Entsprechend sind die
Bezugsmöglichkeiten der \isi{Pronomen} in (\ref{ex:3-127}e,f).

Genauso wie in (\ref{ex:3-127d}) sind die Antezedensmöglichkeiten der \isi{Pronomen}
in (\ref{ex:3-128a}), einer Parallele zu (\ref{ex:3-65}). Zu klären bleibt hier
\eal
\label{ex:3-128}
\ex
\label{ex:3-128a}
der Lehrling hat den \textit{Meister} gepeitscht, und dann hat \textit{er} \textit{ihn} liebkost
\ex
\label{ex:3-128b}
der Lehrling hat den \textit{Meister} gepeitscht, und dann hat der \textit{Kunde} den \textit{Chef} liebkost
\ex
\label{ex:3-128c}
der Lehrling hat den \textit{Meister} gepeitscht, und dann hat der \textit{Kunde} den Meister liebkost
\zl
nur die Interpretation des Verbs. Die Lage ist hier ähnlich wie in
(\ref{ex:3-124b}) und (\ref{ex:3-125b}): Das \isi{Verb} \textit{liebkost} muß aufgrund der
Betonungsverhältnisse \isi{Topik} sein, also muß im RK sein, daß jemand
jemanden liebkoste. Davon ist im ersten \isi{Konjunkt} jedoch nicht
ausdrücklich die Rede, und wir wollen annehmen, daß dies auch für den
weiteren Kontext gilt. Ganz entsprechend ist es in (\ref{ex:3-128b}). In (\ref{ex:3-128c}) ist es insofern anders, als das \isi{Topik} auch \textit{den Meister}
enthält; aus RK muß hier also bekannt sein, daß jemand den Meister
liebkoste. Aus dem ersten \isi{Konjunkt} ist bekannt, daß jemand den Meister
peitschte; um ein dem Kontext angemessenes Verständnis zu erlangen,
muß der Hörer deshalb annehmen, daß Peitschen eine Form der Liebkosung
ist. Zieht der Hörer diesen Schluß nicht~-- und es gibt nichts, was
ihn dazu zwingen könnte; je nach seinen eigenen Ansichten und seinen
Vermutungen über Gegebenheiten in der Welt mag er diesen Schluß als
vollkommen ausgeschlossen oder aber (\zb wenn er weiß, daß der
Meister Masochist ist) als naheliegend betrachten~--, dann muß er (\ref{ex:3-128c}) für inkohärent halten. Zieht er diesen Schluß jedoch und
interpretiert \textit{liebkost} als \isi{Anapher} von \textit{gepeitscht},
dann ist (\ref{ex:3-128c})~-- und in gleicher Weise (\ref{ex:3-128}a,b)~-- völlig
einwandfrei.

Die durch (\ref{ex:3-65}) bzw.\ (\ref{ex:3-128a}) präsentierten Probleme lösen sich also
ohne irgendwelche Zusatzannahmen als natürliche Konsequenz der in (\ref{ex:3-37})
formulierten Kontexttheorie und der darauf basierenden Theorie der
\isi{Betonung} (\ref{ex:3-66}).

Bei der Analyse von (\ref{ex:3-127}) und (\ref{ex:3-128}) haben wir einen erklärenden
Gebrauch von der Explikation von stilistisch normaler \isi{Betonung}
gemacht, indem wir sie auf Fälle von nicht"=normal betonten Sätzen
angewendet haben. Worin liegt nun der erklärende Charakter der
Explikation, wenn man sie auf normal betonte Sätze anwendet?

Ein Satz mit normaler \isi{Betonung} ist unter allen Sätzen mit gleicher
\isi{Wortfolge} kontextuell am wenigsten restringiert. Daher kann man ihn in
den relativ meisten verschiedenen Situationstypen verwenden, man kann
die meisten \textsq{verschiedenen Sprechhandlungen} damit
vollziehen. Mit einem normalbetonten \isi{Fragesatz} wie (\ref{ex:3-129a}) \zb kann
man, nach einer geläufigen Ausdrucksweise, \textsq{verschiedene}
Fragen stellen: ob es der Hund war, den Karl
\eal
\label{ex:3-129}
\ex
\label{ex:3-129a}
hat Karl den \textit{Hund} geschlagen?
\ex
\label{ex:3-129b}
hat \textit{Karl} den Hund geschlagen?
\zl
geschlagen hat; ob Karls Tätigkeit darin bestand, den Hund zu
schlagen; ob das Geschehen darin bestand, daß Karl den Hund geschlagen
hat. Mit dem nicht"=normal betonten \isi{Fragesatz} (\ref{ex:3-129b}) dagegen kann man
nur fragen, ob Karl es war, der den Hund geschlagen hat. Entsprechend
für andere Satzarten. Allerdings kann man den Bezug auf
\textsq{verschiedene mögliche Sprechhandlungen}, so verlockend es ist,
nicht anstelle des Bezugs auf mögliche Kontexttypen zur Grundlage der
Explikation machen, da Ausdrücke wie "`verschiedene
Sprechhandlungen/""Fragen/""Aussagen"' usw.\ in der hier gemeinten
Interpretation selbst nur unter Rekurs auf mögliche Kontexttypen
expliziert werden können. Die \textsq{verschiedenen} Fragen, die man
mit (\ref{ex:3-129a}) stellen kann, sind ja nicht etwa logisch verschieden.

Es scheint mir nun sehr natürlich zu sein, daß man ein Instrument, das (relativ) vielen verschiedenen Zwecken dienen kann, als \textsq{normal} bezeichnet, während man ein vergleichbares Instrument, das für relativ wenige Zwecke taugt, als \textsq{speziell} oder \textsq{nicht"=normal} bezeichnet. Insofern meine ich, daß unter der in \ref{subsubsec:3-1-3-3} entwickelten Explikation, die essentiell pragmatischer Natur ist, da sie auf mögliche Kontexttypen von Sätzen abstellt, erstmals inhaltlich verständlich wird, in welchem Sinne die \isi{Normalbetonung} \textsq{normal} ist; (\ref{ex:3-79}) sagt nicht nur, \textit{daß} normalbetonte Sätze gewisse Eigenschaften haben, sondern zeigt auch, \textit{warum} Sätze mit
solchen Eigenschaften als \textsq{normal} gelten.

\section{Stilistisch normale Wortstellung}
\label{sec:3-2}

\subsection{Einige Fakten}
\label{subsec:3-2-1}

Wir haben früher gesehen, daß Sätze wie (\ref{ex:3-130}a,b) die Foki (\ref{ex:3-131})
bzw.\ (\ref{ex:3-132}) haben können:
\eal
\label{ex:3-130}
\ex
\label{ex:3-130a}
Karl hat dem Kind das \textit{Buch} geschenkt
\ex
\label{ex:3-130b}
Karl hat das Buch dem \textit{Kind} geschenkt
\zl
\eal
\label{ex:3-131}
\ex
\label{ex:3-131a}
Fk\sub{1} (\ref{ex:3-130a}) = das Buch
\ex
\label{ex:3-131b}
Fk\sub{2} (\ref{ex:3-130a}) = das Buch + geschenkt
\ex
\label{ex:3-131c}
Fk\sub{3} (\ref{ex:3-130a}) = Karl + das Buch + geschenkt
\ex
\label{ex:3-131d}
Fk\sub{4} (\ref{ex:3-130a}) = dem Kind + das Buch + geschenkt
\ex
\label{ex:3-131e}
Fk\sub{5} (\ref{ex:3-130a}) = MK (\ref{ex:3-130a}) = Karl + dem Kind + das Buch + geschenkt
\zl
\eal
\label{ex:3-132}
\ex
\label{ex:3-132a}
Fk\sub{1} (\ref{ex:3-130b}) = dem Kind
\ex
\label{ex:3-132b}
Fk\sub{2} (\ref{ex:3-130b}) = dem Kind + geschenkt
\ex
\label{ex:3-132c}
Fk\sub{3} (\ref{ex:3-130b}) = Karl + dem Kind + geschenkt
\ex
\label{ex:3-132d}
Fk\sub{4} (\ref{ex:3-130b}) = das Buch + dem Kind + geschenkt
\ex
\label{ex:3-132e}
Fk\sub{5} (\ref{ex:3-130b}) = MK (\ref{ex:3-130b}) = Karl + das Buch + dem Kind + geschenkt
\zl
\addlines
Vergleichen wir damit die Sätze (\ref{ex:3-133}a,b) und prüfen wir, zu welchen
Fragen in (\ref{ex:3-134}) sie passen. (\ref{ex:3-133a}) paßt zu den Fragen (\ref{ex:3-134}a,c,e);
aber nicht zu (\ref{ex:3-134}b,d,f,g). Die Foki von (\ref{ex:3-133a}) sind daher die in
(\ref{ex:3-135}) angegebenen. Zu (\ref{ex:3-133b}) passen die
\begin{exe}
\ex
\label{ex:3-133}
\begin{xlist}
\ex
\label{ex:3-133a}
dem Kind hat Karl das \textit{Buch} geschenkt
\ex
\label{ex:3-133b}
das Buch hat Karl dem \textit{Kind} geschenkt
\end{xlist}
\ex
\label{ex:3-134}
\begin{xlist}
\ex
\label{ex:3-134a}
was hat Karl dem Kind geschenkt?
\ex
\label{ex:3-134b}
wem hat Karl das Buch geschenkt?
\ex
\label{ex:3-134c}
was hat Karl hinsichtlich des Kindes getan?
\ex
\label{ex:3-134d}
was hat Karl mit dem Buch gemacht?
\ex
\label{ex:3-134e}
was hat das Kind erlebt?
\ex
\label{ex:3-134f}
was ist mit dem Buch geschehen?
\ex
\label{ex:3-134g}
was ist geschehen?
\end{xlist}
\end{exe}
Fragen (\ref{ex:3-134}b,d,f); aber nicht (\ref{ex:3-134}a,c,e,g). Daher die in (\ref{ex:3-136})
genannten Foki.
\begin{exe}
\ex
\label{ex:3-135}
\begin{xlist}
\ex
\label{ex:3-135a}
Fk\sub{1} (\ref{ex:3-133a}) = das Buch (= Fk\sub{1} (\ref{ex:3-130a}))
\ex
\label{ex:3-135b}
Fk\sub{2} (\ref{ex:3-133a}) = das Buch + geschenkt (= Fk\sub{2} (\ref{ex:3-130a}))
\ex
\label{ex:3-135c}
Fk\sub{3} (\ref{ex:3-133a}) = Karl + das Buch+ geschenkt (= Fk\sub{3} (\ref{ex:3-130a}))
\end{xlist}
\ex
\label{ex:3-136}
\begin{xlist}
\ex
\label{ex:3-136a}
Fk\sub{1} (\ref{ex:3-133b}) = dem Kind (= Fk\sub{1} (\ref{ex:3-130b}))
\ex
\label{ex:3-136b}
Fk\sub{2} (\ref{ex:3-133b}) = dem Kind+ geschenkt (= Fk\sub{2} (\ref{ex:3-130b}))
\ex
\label{ex:3-136c}
Fk\sub{3} (\ref{ex:3-133b}) = Karl + dem Kind + geschenkt (= Fk\sub{3} (\ref{ex:3-130b}))
\end{xlist}
\end{exe}
Es fällt ins Auge, daß MF (\ref{ex:3-133a}) $\subset$ MF (\ref{ex:3-130a}) und MF (\ref{ex:3-133b}) $\subset$ MF (\ref{ex:3-130b}). 

Vergleichen wir auch (\ref{ex:3-137}a,b) mit (\ref{ex:3-138}a,b). (\ref{ex:3-137a}) hat die in (\ref{ex:3-139}) genannten Foki, (\ref{ex:3-137b}) die in (\ref{ex:3-140})
genannten. (\ref{ex:3-138a})
\begin{exe}
\ex
\label{ex:3-137}
\begin{xlist}
\ex
\label{ex:3-137a}
es heißt, daß Karl ihn dem \textit{Kind} geschenkt hat
\ex
\label{ex:3-137b}
es heißt, daß Karl ihm das \textit{Buch} geschenkt hat
\end{xlist}
\ex
\label{ex:3-138}
\begin{xlist}
\ex
\label{ex:3-138a}
es heißt, daß ihn Karl dem \textit{Kind} geschenkt hat
\ex
\label{ex:3-138b}
es heißt, daß ihm Karl das \textit{Buch} geschenkt hat
\end{xlist}
\ex
\label{ex:3-139}
\begin{xlist}
\ex
\label{ex:3-139a}
Fk\sub{1} (\ref{ex:3-137a}) = dem Kind
\ex
\label{ex:3-139b}
Fk\sub{2} (\ref{ex:3-137a}) = dem Kind + geschenkt
\ex
\label{ex:3-139c}
Fk\sub{3} (\ref{ex:3-137a}) = Karl + dem Kind + geschenkt
\ex
\label{ex:3-139d}
Fk\sub{4} (\ref{ex:3-137a}) = ihn + dem Kind + geschenkt
\ex
\label{ex:3-139e}
Fk\sub{5} (\ref{ex:3-137a}) = MK (\ref{ex:3-137a}) = Karl + ihn + dem Kind + geschenkt
\end{xlist}
\ex
\label{ex:3-140}
\begin{xlist}
\ex
\label{ex:3-140a}
Fk\sub{1} (\ref{ex:3-137b}) = das Buch
\ex
\label{ex:3-140b}
Fk\sub{2} (\ref{ex:3-137b}) = das Buch + geschenkt
\ex
\label{ex:3-140c}
Fk\sub{3} (\ref{ex:3-137b}) = Karl + das Buch + geschenkt
\ex
\label{ex:3-140d}
Fk\sub{4} (\ref{ex:3-137b}) = ihm + das Buch+ geschenkt
\ex
\label{ex:3-140e}
Fk\sub{5} (\ref{ex:3-137b}) = MK (\ref{ex:3-137b}) = Karl + ihm + das Buch + geschenkt
\end{xlist}
\ex
\label{ex:3-141}
\begin{xlist}
\ex
\label{ex:3-141a}
Fk\sub{1} (\ref{ex:3-138a}) = dem Kind (= Fk\sub{1} (\ref{ex:3-137a}))
\ex
\label{ex:3-141b}
Fk\sub{2} (\ref{ex:3-138a}) = dem Kind + geschenkt (= Fk\sub{2} (\ref{ex:3-137a}))
\ex
\label{ex:3-141c}
Fk\sub{3} (\ref{ex:3-138a}) = Karl + dem Kind + geschenkt (= Fk\sub{3} (\ref{ex:3-137a}))
\end{xlist}
\ex
\label{ex:3-142}
\begin{xlist}
\ex
\label{ex:3-142a}
Fk\sub{1} (\ref{ex:3-138b}) = das Buch (= Fk\sub{1} (\ref{ex:3-137b}))
\ex
\label{ex:3-142b}
Fk\sub{2} (\ref{ex:3-138b}) = das Buch + geschenkt (= Fk\sub{2} (\ref{ex:3-137b}))
\ex
\label{ex:3-142c}
Fk\sub{3} (\ref{ex:3-138b}) = Karl + das Buch + geschenkt (= Fk\sub{3} (\ref{ex:3-137b}))
\end{xlist}
\end{exe}
hat dagegen nur die Foki von (\ref{ex:3-141}) und (\ref{ex:3-138b}) hat die von
(\ref{ex:3-142}). Wieder ist MF (\ref{ex:3-137a}) $\neq$ MF (\ref{ex:3-138a}) und MF (\ref{ex:3-137b}) $\neq$ MF
(\ref{ex:3-138b}), und zwar derart, daß MF (\ref{ex:3-138a}) $\subset$ MF (\ref{ex:3-137a}) und MF (\ref{ex:3-138b}) $\subset$ MF
(\ref{ex:3-137b}).

Woran liegt es nun, daß die Mengen der Foki in (\ref{ex:3-133}) und (\ref{ex:3-138}) eingeschränkt sind, so daß diese Sätze relativ zu den verglichenen Sätzen in (\ref{ex:3-130}) und (\ref{ex:3-137}) kontextuell stärker restringiert sind? In \ref{subsubsec:3-1-3-2} haben wir ähnliche Einschränkungen bei der Menge der Foki in Abhängigkeit von der \isi{Betonung} beobachtet. An der \isi{Betonung} kann es jedoch nicht liegen, daß (\ref{ex:3-133}) und (\ref{ex:3-138}) in weniger Kontexttypen vorkommen als (\ref{ex:3-130}) und (\ref{ex:3-137}): Man überzeugt sich leicht davon, daß (\ref{ex:3-133}) und (\ref{ex:3-138}) unter einer anderen \isi{Betonung} nicht mehr Foki haben können, allenfalls weniger. Dementsprechend sind (\ref{ex:3-133}a,b) und (\ref{ex:3-138}a,b) nach (\ref{ex:3-78}) hinsichtlich der \isi{Betonung} kontextuell relativ \isi{unmarkiert}; sie sind nach (\ref{ex:3-79})~-- in Übereinstimmung mit der Intuition~-- stilistisch normal betont.

Zugleich scheint es intuitiv~-- bei (\ref{ex:3-133}) vielleicht deutlicher als
bei (\ref{ex:3-138})~-- , daß die \isi{Wortstellung} in diesen Sätzen stilistisch
weniger normal ist als in den Sätzen von (\ref{ex:3-130}) und (\ref{ex:3-137}), und
offensichtlich hängt damit die Einschränkung der Fokusmöglichkeiten
zusammen. Allerdings kann man \textsq{stilistisch normale Wortstellung} natürlich nicht unmittelbar als Definiens bei der Beschreibung von Fällen wie (\ref{ex:3-133}), (\ref{ex:3-138}) benutzen: Dieser Ausdruck bedarf vielmehr selbst der Explikation, und zwar mit Hilfe empirisch
kontrollierbarer Phänomene wie dem Vergleich zwischen (\ref{ex:3-130}) und (\ref{ex:3-133})
bzw.\ (\ref{ex:3-137}) und (\ref{ex:3-138}).

\subsection{Explikation von "`stilistisch normale Wortstellung"'}
\label{subsec:3-2-2}
\begin{exe}
\ex
\label{ex:3-143}
\textit{Hypothese:} \\
Wenn zwei Sätze S\sub{\textit{i}}, S\sub{\textit{j}} in ES\sub{\textit{i}} (vgl.\ (\ref{ex:3-76b})) sind, gilt:\\ MF (S\sub{\textit{i}}) $\subseteq$ MF (S\sub{\textit{j}}) oder MF (S\sub{\textit{j}}) $\subseteq$ MF (S\sub{\textit{i}}).
\end{exe}
Diese Hypothese stützt sich auf viele Beobachtungen; man beachte aber,
daß sie keineswegs trivial ist: Es wäre ohne weiteres denkbar, daß die
Regeln zur \isi{Fokusprojektion} je nach \isi{Wortstellung} verschiedene
nicht-minimale Foki zulassen, so daß MF (S\sub{\textit{i}}) $\cap$ MF (S\sub{\textit{j}})
$\neq$ MF (S\sub{\textit{j}}) und $\neq$ MF (S\sub{\textit{i}}) wäre. Das scheint nicht der
Fall zu sein.

\addlines[2]
In (\ref{ex:3-78}) haben wir, unter Bezug auf Sätze mit gegebener \isi{Wortstellung},
den Begriff \textsq{kontextuell relativ \isi{unmarkiert} (hinsichtlich der
\isi{Betonung})} definiert. Parallel dazu (\ref{ex:3-144}):
\begin{exe}
\ex\label{ex:3-144}
\textit{Definition:} \\
Unter allen Sätzen in ES\sub{\textit{i}} sind die S\textit{\sub{i\sub{j}}} hinsichtlich. der
\isi{Wortstellung} kontextuell relativ \isi{unmarkiert}, die in der größten Zahl
von Kontexttypen vorkommen können. Alle anderen Sätze S\textit{\sub{i\sub{k}}} in
ES\sub{\textit{i}} sind hinsichtlich der \isi{Wortstellung} kontextuell markiert.
\end{exe}
Aus (\ref{ex:3-143}) folgt, daß für die S\textit{\sub{i\sub{k}}} von (\ref{ex:3-144}) gilt: MF (S\textit{\sub{i\sub{k}}})
$\subset$ MF (S\textit{\sub{i\sub{j}}}).\vspace{-3pt}

Nach (\ref{ex:3-144}) sind (\ref{ex:3-133}a,b) und (\ref{ex:3-138}a,b) hinsichtlich der \isi{Wortstellung} kontextuell markiert, während (\ref{ex:3-130}a,b) und (\ref{ex:3-137}a,b) hinsichtlich der \isi{Wortstellung} kontextuell relativ \isi{unmarkiert} sind. Man
könnte versuchen, dies unmittelbar für die Explikation von \textsq{normaler Wortstellung} zu nutzen; aus intuitiven und systematischen Gründen sollte aus einer adäquaten Explikation aber auch folgen, daß \zb (\ref{ex:3-145a}) \textsq{normale} \isi{Wortstellung} aufweist und (\ref{ex:3-145b}) \textsq{nichtnormale}; diese Sätze sind aber nach (\ref{ex:3-144}) hinsichtlich der \isi{Wortstellung} beide kontextuell markiert, da MF (\ref{ex:3-145a}) = MF (\ref{ex:3-145b}) = \textit{dem Kind} und da es eine alternative \isi{Konstituentenfolge} mit einer größeren Menge von Foki gibt, nämlich (\ref{ex:3-130b}). Ebenso sind (\ref{ex:3-146}a,b) in gleicher Weise hinsichtlich der \isi{Wortstellung}
kontextuell
\eal
\label{ex:3-145}
\ex
\label{ex:3-145a}
Karl hat dem \textit{Kind} das Buch geschenkt
\ex
\label{ex:3-145b}
dem \textit{Kind} hat Karl das Buch geschenkt
\zl
\eal
\label{ex:3-146}
\ex
\label{ex:3-146a}
\textit{Karl} hat dem Kind das Buch geschenkt
\ex
\label{ex:3-146b}
das Buch hat \textit{Karl} dem Kind geschenkt
\zl
relativ \isi{unmarkiert}, da MF (\ref{ex:3-146a}) = MF (\ref{ex:3-146b}) und da bei gleicher Konstituentenbetonung bei keiner anderen \isi{Konstituentenfolge} mehr Foki als in (\ref{ex:3-146}a,b) möglich sind; trotzdem möchte man die \isi{Wortstellung} in (\ref{ex:3-146a}) als \textsq{normal} bezeichnen und die in (\ref{ex:3-146b}) nicht.

Eine Definition, die wie (\ref{ex:3-144}) auf Mengen von Sätzen mit einer
gegebenen Konstituentenbetonung rekurriert, ist also offenbar für die Explikation des Begriffs \textsq{stilistisch normale Wortstellung} nicht geeignet. Vielmehr scheint es so zu sein, daß Konstituentenfolgen wie in (\ref{ex:3-145a}), (\ref{ex:3-146a}) unter \textit{jeder} Konstituentenbetonung \textsq{normal} sind; \dash alle Sätze in der EB\sub{\textit{i}} (vgl.\ (\ref{ex:3-76a})), zu der (\ref{ex:3-145a}) gehört, weisen \textsq{normale} \isi{Wortstellung} auf. Entsprechend weisen alle Sätze in den EB\sub{\textit{i}}, zu denen (\ref{ex:3-145b}) bzw.\ (\ref{ex:3-146b}) gehören, \textsq{nicht-normale} \isi{Wortstellung} auf. Daß die \isi{Wortstellung} von (\ref{ex:3-145b}) und (\ref{ex:3-146b}) als nicht"=normal empfunden wird, hängt, wie wir bei der Besprechung von (\ref{ex:3-133}a,b) gesehen haben, offenbar damit zusammen, daß es bei diesen Konstituentenfolgen unter keiner wie immer gearteten Konstituentenbetonung eine \isi{Fokusprojektion} zur Menge der Konstituenten des Satzes gibt, während eine solche \isi{Fokusprojektion} bei der \isi{Konstituentenfolge} von (\ref{ex:3-145a}) unter einer geeigneten Konstituentenbetonung (nämlich wie in (\ref{ex:3-130a})) möglich ist. Da die Zahl der prinzipiell möglichen Foki also~-- unabhängig von der gewählten Konstituentenbetonung~-- bei Konstituentenfolgen wie in (\ref{ex:3-145b}) und (\ref{ex:3-146b}) gegenüber solchen wie in (\ref{ex:3-145a}) grundsätzlich
eingeschränkt ist, soll die gesuchte Explikation die \isi{Wortstellung} als
normal auszeichnen, die bei geeigneter \isi{Betonung} mit den wenigstens
kontextuellen Restriktionen verbunden ist.

Ich schlage daher die folgende Formulierung vor:
\begin{exe}
\ex\label{ex:3-147}
Unter allen Sätzen in EBS\sub{\textit{i}} (vgl.\ (\ref{ex:3-76c})) haben die Sätze S\textit{\sub{i\sub{j}}}
\textit{stilistisch normale Wortstellung}, für die gilt: In EB\sub{\textit{i}}, S\textit{\sub{i\sub{j}}} in EB\sub{\textit{i}},
ist ein Satz S\textit{\sub{i\sub{k}}}, der unter allen Sätzen in EBS\sub{\textit{i}} in den meisten
Kontexttypen vorkommen kann.
\end{exe}
Erläuterung: In wievielen Kontexttypen ein Satz vorkommen kann, hängt davon ab, wieviele Foki er hat. Ein Satz hat, bei gegebener \isi{Wortstellung} (\dash in EB\sub{\textit{i}}), normale Betonung\is{Normalbetonung}, wenn er unter allen Sätzen in EB\sub{\textit{i}} die meisten Foki hat. Um zu prüfen, ob ein Satz S\sub{\textit{i}} normale \isi{Wortstellung} hat, muß man also feststellen: (1) welche S\sub{\textit{j}} in der EB\sub{\textit{i}}, zu der S\sub{\textit{i}} gehört, normalbetont\is{Normalbetonung} sind, (2) ob eine andere \isi{Konstituentenfolge} unter \isi{Normalbetonung} mehr mögliche Foki als S\sub{\textit{j}} zuläßt. Ist dies nicht der Fall, hat S\sub{\textit{i}} nach (\ref{ex:3-147}) eine stilistisch normale \isi{Wortstellung}.

Man überzeugt sich leicht davon, daß diese Explikation für die diskutierten Beispiele das gewünschte Ergebnis liefert; vgl.\ die Diskussion von (\ref{ex:3-130a}) im Vergleich zu (\ref{ex:3-133}a,b) und (\ref{ex:3-137}) im Vergleich zu (\ref{ex:3-138}).

Besonders interessant sind solche EBS\sub{\textit{j}} für die (\ref{ex:3-148}) gilt:
\ea
\label{ex:3-148}
EBS\sub{\textit{i}} enthält mindestens einen Satz S\sub{m} derart, daß Fk\sub{1} (S\sub{m}) genau
1 \isi{Konstituente} umfaßt und Fk\sub{m} (S\sub{m}) = MK (S\sub{m}).
\z
Die S\sub{m} von (\ref{ex:3-148}) zeichnen sich also dadurch aus, daß sie (soweit sie aus mehr als 1 Wort bestehen) eine \isi{Fokusprojektion} vom kleinsten theoretisch möglichen \isi{Fokus}, nämlich 1 Wort, zum größten theoretisch möglichen \isi{Fokus}, nämlich MK (S), haben. Im typischen Fall hat eine EBS\sub{\textit{i}} genau einen solchen S\sub{m}. Manche haben jedoch mehrere: (\ref{ex:3-130a}) und (\ref{ex:3-130b}) sind Elemente derselben EBS\sub{\textit{i}} und sind beide ein S\sub{m}; und nicht alle EBS\sub{\textit{i}} haben einen S\sub{m}: Offenbar aufgrund pragmatischer Gebrauchsrestriktionen haben Sätze mit einer \isi{Negation} anscheinend nie die Menge aller Konstituenten als möglichen \isi{Fokus} (\dash eine \textsq{Satznegation} in einem pragmatischen Sinne gibt es nicht, cf.\ \citet{Givon78}); mindestens bei einigen Typen von Wortfragen (aber nicht bei Satzfragen) und von Exklamativsätzen scheint die \isi{Projektion} zu MK (S) ausgeschlossen; Gradpartikeln verhindern offenbar jede \isi{Fokusprojektion} von dem Fk\sub{\textit{i}} (S\sub{\textit{i}}) = K\sub{\textit{i}}, dem sie zugeordnet sind, zu einem Fk\sub{\textit{i}+1} (S\sub{\textit{i}}) = K\sub{\textit{i}} + K\sub{\textit{j}}, und da sie nie den ganzen Satz als zugeordneten \isi{Fokus} haben, können sie in keinem S\sub{m} vorkommen.

Es scheint eine empirische Tatsache zu sein, daß die Anzahl der Foki
jedes S\sub{m} maximal ist:
\begin{exe}
\ex
\label{ex:3-149}
\textit{Hypothese:} \\
Es gibt keinen S\sub{\textit{i}} in EBS\sub{\textit{i}}, der mehr mögliche Foki als S\sub{m} hat.
\end{exe}
Für Sätze in einer ES\sub{\textit{i}} folgt (\ref{ex:3-149}) aus (\ref{ex:3-143}). Für Sätze mit
verschiedenen Konstituentenbetonungen folgt (\ref{ex:3-149}) (ebenso wie (\ref{ex:3-143}))
vermutlich aus allgemeinen Eigenschaften der Fokusprojektionsregeln;
darüber will ich hier nicht spekulieren.

Wegen (\ref{ex:3-149}) kann S\sub{m} in maximal vielen Kontexttypen vorkommen; dies wollen wir terminologisch hervorheben:
\ea
\label{ex:3-150}
Jeder S\sub{m} in EBS\sub{\textit{i}} ist \textit{kontextuell absolut unmarkiert}.
\z
Aus unseren Definitionen folgt:
\eal
\label{ex:3-151}
\ex
\label{ex:3-151a}
Jeder S\sub{m} ist hinsichtlich der \isi{Betonung} und hinsichtlich der \isi{Wortstellung} kontextuell relativ \isi{unmarkiert}.
\ex
\label{ex:3-151b}
Jeder S\sub{m} weist normale \isi{Wortstellung} und \isi{Normalbetonung} auf.
\ex
\label{ex:3-151c}
Alle S\sub{\textit{i}} in EBS\sub{\textit{i}}, die in derselben EB\sub{\textit{i}} wie ein S\sub{m} von
EBS\sub{\textit{i}} sind, weisen normale \isi{Wortstellung} auf.
\zl


\subsection{Fruchtbarkeit der Explikation}
\label{subsec:3-2-3}

\subsubsection{Adäquatheit}
\label{subsubsec:3-2-3-1}

Es scheint mir deutlich zu sein, daß die -- essentiell pragmatischen --
Formulierungen (\ref{ex:3-147}) und (\ref{ex:3-151}) eine adäquate Explikation des
traditionellen intuitiven Begriffs "`normale \isi{Wortstellung}"' angeben
(soweit er konsistent gebraucht wird). Sie steht in Übereinstimmung
mit den wenigen Explikationsversuchen in der Literatur:
\begin{exe}
\ex\label{ex:3-152}
"`Wenn man einen Sachverhalt schildert oder erfragt, bei dem keine
Einzelheit besonders hervorgehoben wird, stehen die Satzglieder in
einer regelmäßigen Ordnung, die wir als \textsq{Grundstellung}
der Satzglieder bezeichnen wollen."' (\citealt{Griesbach1961a} (IV): 84)
\end{exe}
Es erübrigt sich, auf die Probleme von (\ref{ex:3-152}) einzugehen, die
u.\,a.\ -- ähnlich wie bei den Explikationsversuchen für "`\isi{Normalbetonung}"'
in \ref{subsec:3-1-2} -- mit dem Ausdruck "`besonders hervorgehoben"' zusammenhängen. Es
scheint klar, daß bei adäquater Klärung dieser Probleme kein
inhaltlicher Unterschied zu (\ref{ex:3-147}) besteht. Ähnlich in (\ref{ex:3-153}):
\begin{exe}
\ex\label{ex:3-153}
"`A word order is referred to as basic if it can stand without any
presupposition as to what should be considered as being already known."'
\citep[140]{Kiefer70}
\end{exe}
Wir brauchen nicht näher darauf einzugehen, inwieweit (\ref{ex:3-153}) mit (\ref{ex:3-147})
übereinstimmt und wo die Mängel von (\ref{ex:3-153}) liegen. Immerhin ist
beachtenswert, daß (\ref{ex:3-152}) so, wie die Formulierung ist, vermutlich nur
für S\sub{m} gedacht ist, während (\ref{ex:3-153}) wegen des Modals \textit{can} die
Anwendung auf eine ganze EB\sub{\textit{i}} zuläßt, sofern diese einen S\sub{m} enthält.

\subsubsection{Heuristische Fruchtbarkeit}
\label{subsubsec:3-2-3-2}

Da die Explikation von \textsq{stilistisch normaler Wortstellung} auf
dem Fokusbegriff aufbaut und da wir wissen, daß es Regeln für die
Bestimmung der möglichen Foki eines Satzes geben muß, provoziert diese
Explikation die Frage, welche Faktoren die \isi{Fokusprojektion} bei
nicht-normaler \isi{Wortstellung} blockieren; sie eröffnet damit ein
Forschungsfeld, das~-- mit der Ausnahme von \citet{Contreras76}~-- bisher
weitgehend ignoriert worden ist, möglicherweise aber auch für die
Kenntnis der Regularitäten bei normaler \isi{Wortstellung} ergiebig ist.

Betrachten wir, um etwas mehr über nicht-normale Wortstellungen zu
erfahren, noch einmal (\ref{ex:3-133-2}) und (\ref{ex:3-138-2}). Ein Teil der Fakten folgt aus
der Hypothese (\ref{ex:3-154}). Informell umschrieben besagt sie: Wenn
\begin{exe}
\exi{(133)}{\exalph{133}
\label{ex:3-133-2}
\begin{xlist}
\ex
\label{ex:3-133a-2}
dem Kind hat Karl das \textit{Buch} geschenkt
\ex
\label{ex:3-133b-2}
das Buch hat Karl dem \textit{Kind} geschenkt
\end{xlist}}
\exi{(138)}{\exalph{138}
\label{ex:3-138-2}
\begin{xlist}
\ex
\label{ex:3-138a-2}
es heißt, daß ihn Karl dem \textit{Kind} geschenkt hat
\ex
\label{ex:3-138b-2}
es heißt, daß ihm Karl das \textit{Buch} geschenkt hat
\end{xlist}}
\ex
\label{ex:3-154}
\textit{Hypothese:} \\
Wenn bei zwei Sätzen S\sub{\textit{i}}, S\sub{\textit{j}} die in einer ES\sub{\textit{i}} sind,
\begin{xlist}
\ex
\label{ex:3-154a}
Fk\sub{\textit{n}} (S\sub{\textit{i}}) = Fk\sub{\textit{n}} (S\sub{\textit{j}}) = K\sub{\textit{i}}
\ex
\label{ex:3-154b}
Fk\sub{\textit{n}+1} (S\sub{\textit{i}}) = K\sub{\textit{i}} + K\sub{\textit{j}}, aber nicht Fk\sub{\textit{n}+1} (S\sub{\textit{j}}) = K\sub{\textit{i}} + K\sub{\textit{j}}
\ex
\label{ex:3-154c}
dann kann es Fk\sub{\textit{n}+2} (S\sub{\textit{i}}) = K\sub{\textit{i}} + K\sub{\textit{j}} + K\sub{\textit{k}} geben, aber nicht\\
Fk\sub{\textit{n}+\textit{i}} (S\sub{\textit{j}}) = K\sub{\textit{i}} + K\sub{\textit{j}} + K\sub{\textit{k}}
\end{xlist}
\end{exe}
in S\sub{\textit{j}} auch nur eine \isi{Fokusprojektion} von einem \isi{Fokus} mit \textit{n}
Konstituenten zu einem \isi{Fokus} mit \textit{n} + 1 Konstituenten blockiert ist,
die bei anderer \isi{Wortstellung} (in S\sub{\textit{i}}) möglich ist, dann ist in S\sub{\textit{j}}
jede weitere \isi{Fokusprojektion} zu einem \isi{Fokus} mit \textit{n} + 2 Konstituenten
ebenfalls blockiert.

Aus (\ref{ex:3-154}) folgt, daß (\ref{ex:3-133}a,b) im Gegensatz zu (\ref{ex:3-130}a,b) keine \isi{Fokusprojektion} zur Menge der Konstituenten haben, da bereits eine der Fokusprojektionen zu einem \isi{Fokus} mit geringerem Umfang, wie sie bei (\ref{ex:3-130}a,b) möglich ist (vgl.\ (\ref{ex:3-131d}), (\ref{ex:3-132d})), blockiert ist; das gleiche gilt für (\ref{ex:3-138}) im Verhältnis zu (\ref{ex:3-137}).

\addlines
Aus (\ref{ex:3-154}) geht jedoch nicht hervor; wieso die Sätze (\ref{ex:3-133}a,b) nicht
die Foki von (\ref{ex:3-131d}), (\ref{ex:3-132d}) haben bzw.\ (\ref{ex:3-138}a,b) nicht die von (\ref{ex:3-139d}), (\ref{ex:3-140d}). Eine Vermutung legt sich nahe:
\begin{exe}
\ex\label{ex:3-155}
\textit{Hypothese:} \\
Für Sätze S\sub{\textit{i}}, S\sub{\textit{j}} in EBS\sub{\textit{i}}: \\
Wenn in S\sub{\textit{i}} bei normaler \isi{Wortstellung} die \isi{Konstituentenfolge} K\sub{\textit{j}}
$>$ K\sub{\textit{i}} gilt, dann nimmt, wenn K\sub{\textit{i}} $>$ K\sub{\textit{j}} in S\sub{\textit{j}}, K\sub{\textit{i}} in
S\sub{\textit{j}} an keiner \isi{Fokusprojektion} teil.
\end{exe}
Aus (\ref{ex:3-155}) folgen nicht nur die Verhältnisse in (\ref{ex:3-133}) und (\ref{ex:3-138}),\footnote{\label{fn:3-27}%
	Die Formulierung von (\ref{ex:3-155}) unterscheidet nicht, ob eine der relevanten Konstituenten im Vorfeld oder im Mittelfeld steht. Das bedeutet natürlich nicht, daß es keine grammatisch relevanten Unterschiede zwischen den Wortstellungsregeln fürs Mittelfeld und den Regeln für die Besetzung des Vorfelds gibt. Im Gegenteil: Die \isi{Wortfolge} von (\ref{ex:3-138a}) ist bei gleicher Konstituentenbetonung unmöglich, wenn die erste \isi{NP} im Vorfeld steht, vgl.\ (i); und bei gleicher Konstituentenbetonung ist die \isi{Wortfolge} von (\ref{ex:3-133}) innerhalb des Mittelfelds ausgeschlossen, vgl.\ (ii):
  \ea[*]{ \label{ex:3-fn27i}
ihn hat Karl dem \textit{Kind} geschenkt}
  \zmid
  \eal
  \label{ex:3-fn27ii}
  \ex[*]{
  \label{ex:3-fn27iia}
  weil dem Kind Karl das \textit{Buch} geschenkt hat}
  \ex[*]{
  \label{ex:3-fn27iib}
  weil das Buch Karl dem \textit{Kind} geschenkt hat}
  \zl
  Für (\ref{ex:3-155}) entsteht daraus kein Problem, da eine EBS\sub{\textit{i}}, wie ich
  selbstverständlich vorausgesetzt habe, nur grammatisch wohlgeformte
  Sätze enthält. Die Unakzeptabilität von (i) und (ii) ist innerhalb
  der \isi{Topologie} zu beschreiben und folgt nicht aus pragmatischen
  Regularitäten.

  In dem Maße, wie (\ref{ex:3-155}) empirisch korrekt ist, erfaßt die Hypothese~--
  wie übrigens auch (\ref{ex:3-103})~-- eine wichtige Tatsache: daß bestimmte
  Wortfolgen -- \zb die in (iii)~-- im gleichen Maße normal sind,
  unabhängig davon, ob die erste \isi{Konstituente} im Vorfeld oder im
  Mittelfeld steht:
  \eal
  \label{ex:3-fn27iii}
  \ex
  \label{ex:3-fn27iiia}
  Karl hat dem Kind das Buch geschenkt
  \ex
  \label{ex:3-fn27iiib}
  weil Karl dem Kind das Buch geschenkt hat
  \zl
  Eben diese Tatsache hat so viele Autoren dazu verführt, die Stellung
  des Subjekts nach dem finiten \isi{Verb} wie in (\ref{ex:3-133}) auf eine
  \textsq{Inversion} des Subjekts aus dem Vorfeld ins Mittelfeld
  zurückzuführen. Tatsächlich ist "`Mittelfeld"' ein essentieller Begriff
  der deutschen \isi{Topologie} jedoch gerade deshalb, weil die \zb für
  (iv) relevanten Stellungsregularitäten mittels
  \textsq{Subjektinversion} und ohne Benutzung eines Konstrukts wie
  \textsq{Mittelfeld} überhaupt nicht formulierbar wären.
  \eal
  \label{ex:3-fn27iv}
  \ex
  \label{ex:3-fn27iva}
  das Buch hat ihr seinerzeit ein Priester geschenkt
  \ex
  \label{ex:3-fn27ivb}
  weil ihr das Buch seinerzeit ein Priester geschenkt hat
  \zl
  (Aber wenn \textsq{Mittelfeld} als Domäne von Stellungsregeln gegeben ist,
  ist \textsq{Subjektinversion} ein nutzloses Konstrukt.) Vgl.\ dazu sehr
  klar Griesbach (\citeyear{Griesbach1960a} (I): 105). (Inhaltlich nicht anders schon bei
  \citet[182f, 189, 196]{Erdmann1886}.)%
}
sondern auch die in (\ref{ex:3-156}a,b), wo \textit{das Buch} bzw.\ \textit{dem Kind} natürlich minimale Foki sind, aber im Unterschied zu (\ref{ex:3-130}) jede \isi{Fokusprojektion} ausgeschlossen erscheint. Auch für (\ref{ex:3-156c}) scheint (\ref{ex:3-155}) zuzutreffen, denn im Gegensatz zu (\ref{ex:3-133a}) und (\ref{ex:3-57}) gibt es hier anscheinend nur Fk\sub{1} (\ref{ex:3-156c}) = Fk\sub{m} (\ref{ex:3-156c}) = \textit{dem Kind} + \textit{das Buch}. Ebenso
bei einstelligen Verben wie in (\ref{ex:3-157}): Während in (\ref{ex:3-158}) jeweils eine
\isi{Fokusprojektion} zur Menge der Konstituenten möglich ist, ist dies in
(\ref{ex:3-157}) ausgeschlossen:
\eal
\label{ex:3-156}
\ex
\label{ex:3-156a}
das \textit{Buch} hat Karl dem Kind geschenkt
\ex
\label{ex:3-156b}
dem \textit{Kind} hat Karl das Buch geschenkt
\ex
\label{ex:3-156c}
dem \textit{Kind} hat Karl das \textit{Buch} geschenkt
\zl
\eal
\label{ex:3-157}
\ex
\label{ex:3-157a}
\textit{gebetet} hat Karl
\ex
\label{ex:3-157b}
angekommen ist dein \textit{Vater}
\zl
\eal
\label{ex:3-158}
\ex
\label{ex:3-158a}
Karl hat \textit{gebetet}
\ex
\label{ex:3-158b}
dein \textit{Vater} ist angekommen
\zl
Gar nichts trägt (\ref{ex:3-155}) jedoch zum Verständnis von Fällen wie (\ref{ex:3-159})
bei. Im Widerspruch zu (\ref{ex:3-155}) hat (\ref{ex:3-159b}) den möglichen \isi{Fokus} (\ref{ex:3-161c});
(\ref{ex:3-155}) erklärt nicht, wieso hier~-- anders als in (\ref{ex:3-133}) und (\ref{ex:3-138})~--
eine \isi{Fokusprojektion} unter Einschluß des Subjekts ((\ref{ex:3-162}b,e))
ausgeschlossen ist:
\begin{exe}
\ex
\label{ex:3-159}
\begin{xlist}
\ex
\label{ex:3-159a}
den Hund hat Karl \textit{geschlagen}
\ex
\label{ex:3-159b}
den \textit{Hund} hat Karl geschlagen
\end{xlist}
\ex
\label{ex:3-160}
\begin{xlist}
\ex
\label{ex:3-160a}
Karl hat den Hund \textit{geschlagen}
\ex
\label{ex:3-160b}
Karl hat den \textit{Hund} geschlagen
\end{xlist}
\ex
\label{ex:3-161}
\begin{xlist}
\ex
\label{ex:3-161a}
Fk\sub{1} (\ref{ex:3-159a}) = Fk\sub{m} (\ref{ex:3-159a}) = geschlagen
\ex
\label{ex:3-161b}
Fk\sub{1} (\ref{ex:3-159b}) = den Hund
\ex
\label{ex:3-161c}
Fk\sub{2} (\ref{ex:3-159b}) = Fk\sub{m} (\ref{ex:3-159b}) = den Hund + geschlagen
\end{xlist}
\ex
\label{ex:3-162}
\begin{xlist}
\ex
\label{ex:3-162a}
Fk\sub{1} (\ref{ex:3-160a}) = geschlagen (= Fk\sub{m}  (\ref{ex:3-159a}))
\ex
\label{ex:3-162b}
Fk\sub{2} (\ref{ex:3-160a}) = Karl + geschlagen
\ex
\label{ex:3-162c}
Fk\sub{1} (\ref{ex:3-160b}) = den Hund
\ex
\label{ex:3-162d}
Fk\sub{2} (\ref{ex:3-160b}) = den Hund + geschlagen (= Fk\sub{m} (\ref{ex:3-159b}))
\ex
\label{ex:3-162e}
Fk\sub{3} (\ref{ex:3-160b}) = Karl + den Hund + geschlagen
\end{xlist}
\end{exe}
%\addlines[-1]
Bartsch stellt nicht nur eine ähnliche Regel wie (\ref{ex:3-155}) auf, sondern behauptet \citep[523]{Bartsch76} darüber hinaus, daß für Sätze wie (\ref{ex:3-133}a,b) das \isi{Subjekt} Teil des \isi{Fokus} sein muß. Wir haben gesehen, daß das nicht zutrifft, vgl.\ (\ref{ex:3-135}a,b) und (\ref{ex:3-136}a,b). Was (\ref{ex:3-133}a,b) auszeichnet, ist vielmehr, daß das an erster Stelle stehende Objekt,
im Gegensatz zum \isi{Subjekt} in (\ref{ex:3-130}), nicht Teil des \isi{Fokus} sein
kann. Allgemeiner behauptet Bartsch, daß ein \isi{Subjekt}, das bei normaler
\isi{Wortstellung} vor einem Objekt steht, bei Nachstellung nicht zum \isi{Topik}
gehören könne. Für Fälle wie (\ref{ex:3-159}a,b) ist das Gegenteil richtig:
Dort muß das \isi{Subjekt} (im Gegensatz zu (\ref{ex:3-160}a,b)) zum \isi{Topik} gehören.
\pagebreak

Ich überlasse es künftigen Untersuchungen, die hier wirksamen
Regularitäten aufzudecken.\footnote{\label{fn:3-28}%
  Man beachte auch, daß nach (\ref{ex:3-155}) nur einer der Sätze in (\ref{ex:3-130}) normale \isi{Wortstellung} haben könnte: Wenn die Abfolge \isi{Dativ} $>$ \isi{Akkusativ} in (\ref{ex:3-130a}) normal ist, dürfte bei der Abfolge \isi{Akkusativ} $>$ \isi{Dativ} in (\ref{ex:3-130b}) das \isi{Akkusativobjekt} nicht Teil eines \isi{Fokus} sein, und die \isi{Projektion} zu MK
  (\ref{ex:3-130b}) wäre ausgeschlossen. Das entspräche zwar der Behauptung mancher
  Autoren, wird aber vom Urteil vieler Sprecher widerlegt; vgl.\ die Diskussion zu
  (\ref{ex:3-98}). -- Für die entsprechenden Beispiele mit \textit{zeig-} macht (\ref{ex:3-155})
  allerdings genau die richtigen Voraussagen; vgl.\ Fn.\,\ref{fn:3-21}.%
}

\subsubsection{Explanatorische Fruchtbarkeit}
\label{subsubsec:3-2-3-3}

Die wichtigste Eigenschaft der Explikation von \textsq{stilistisch
  normaler Wortstellung} in (\ref{ex:3-147}) ist~-- ähnlich wie bei der
Explikation von \textsq{stilistisch normaler Betonung} in (\ref{ex:3-79})~--, daß
sie nicht nur anhand empirisch kontrollierbarer Eigenschaften gewisse
Sätze als \textsq{normal} hinsichtlich der \isi{Wortstellung} auszeichnet,
sondern auch klar macht, inwiefern solche Wortstellungen
\textsq{normal} sind. Die von (\ref{ex:3-147}) ausgezeichneten Sätze haben bei
\isi{Normalbetonung} innerhalb ihrer EBS\sub{\textit{i}} maximal viele verschiedene Foki
(vgl.\ die MF (\ref{ex:3-48}) in (\ref{ex:3-50})); bei nicht-normaler \isi{Betonung} der gleichen
\isi{Konstituentenfolge} ergibt sich noch eine Reihe von weiteren
Fokusmöglichkeiten (vgl.\ die Diskussion von (\ref{ex:3-52})--(\ref{ex:3-54}) und
(\ref{ex:3-57})). Vermutlich kann man, sobald ein adäquates System von Hypothesen
entwickelt ist, beweisen, daß die Sätze in einer EB\sub{\textit{i}} mit
stilistisch normaler Konstituentenabfolge zusammengenommen die meisten
möglichen Foki innerhalb ihrer gesamten Menge EBS\sub{\textit{i}} haben, so daß
sie, bei geeigneten Betonungen, unter allen grammatisch möglichen
Permutationen der Konstituenten die geringsten kontextuellen
Restriktionen haben.

Damit sind sie (relativ zu ihrer EBS\sub{\textit{i}}) so etwas wie ein
Allzweckinstrument, und es scheint mir wiederum sehr natürlich, daß
viele Sprecher die Benutzung eines solchen Instruments ganz allgemein
bevorzugen, insbesondere wenn sie sich darauf verlassen (zu können
glauben), daß der RK die bei normaler \isi{Wortstellung} besonders häufigen
Fokusambiguitäten neutralisiert.

Dies klärt auch ein traditionelles Problem der deutschen \isi{Topologie},
das in empirischer wie in methodologischer Hinsicht einiges Interesse
verdient. Nach Ansicht mancher Autoren sind Sätze wie (\ref{ex:3-163}a,b)
1"=deutig, insofern \textit{ein Mädchen} nur Objekt sein könne. (So zu
(\ref{ex:3-163b}) \zb Griesbach (\citeyear{Griesbach1961a} (IV): 89)). Das ist überraschend,
denn im Allgemeinen ist hier die Stellung
\eal
\label{ex:3-163}
\ex
\label{ex:3-163a}
die Frau hat ein Mädchen gebissen
\ex
\label{ex:3-163b}
morgen wird sie ein Mädchen beißen
\zl
des Subjekts nach dem Objekt nicht weniger akzeptabel als die Stellung vor dem
Objekt:\footnote{\label{fn:3-29}%
  Manchmal wird die Behauptung, die für (\ref{ex:3-163}) gelten soll, auch mit
  nominalem \isi{Subjekt} und Objekt im Mittelfeld illustriert (\zb in \citealt{Griesbach1960a} (II): 142); hier ist die Lage aber etwas
  anders. Während (i) für alle Sprecher akzeptabel ist und (\ref{ex:3-fn29iia},c)
  wohl für alle Sprecher unakzeptabel sind, gibt es bei (\ref{ex:3-fn29iib}) mit
  nachgestelltem betonten \isi{Subjekt} einen Idiolektunterschied: Für viele
  Sprecher ist (\ref{ex:3-fn29iib}) unakzeptabel; für viele ist er voll akzeptabel.
  \eal
  \label{ex:3-fn29i}
  \ex
  \label{ex:3-fn29ia}
  weil der Mann ein \textit{Mädchen} gebissen hat
  \ex
  \label{ex:3-fn29ib}
  weil der \textit{Mann} ein Mädchen gebissen hat
  \ex
  \label{ex:3-fn29ic}
  weil der Mann ein Mädchen \textit{gebissen} hat
  \zlmid
  \eal
  \label{ex:3-fn29ii}
  \ex
  \label{ex:3-fn29iia}
  weil das \textit{Mädchen} ein Mann gebissen hat
  \ex
  \label{ex:3-fn29iib}
  weil das Mädchen ein \textit{Mann} gebissen hat
  \ex
  \label{ex:3-fn29iic}
  weil das Mädchen ein Mann \textit{gebissen} hat
  \zl
  Dementsprechend sind (\ref{ex:3-fn29iiia},c) für alle Sprecher 1"=deutig, und für
  die Sprecher; die (\ref{ex:3-fn29iib}) ablehnen, ist auch (\ref{ex:3-fn29iiib}) 1"=deutig. Für
  viele Sprecher, die (\ref{ex:3-fn29iib}) akzeptieren, ist (\ref{ex:3-fn29iiib}) erwartungsgemäß
  2"=deutig.
  \eal
  \label{ex:3-fn29iii}
  \ex
  \label{ex:3-fn29iiia}
  weil die \textit{Frau} ein Mädchen gebissen hat
  \ex
  \label{ex:3-fn29iiib}
  weil die Frau ein \textit{Mädchen} gebissen hat
  \ex
  \label{ex:3-fn29iiic}
  weil die Frau ein Mädchen \textit{gebissen} hat
  \zllast%
}
\eal
\label{ex:3-164}
\ex
\label{ex:3-164a}
den Mann hat ein Mädchen gebissen
\ex
\label{ex:3-164b}
der Mann hat ein Mädchen gebissen
\zl
\eal
\label{ex:3-165}
\ex
\label{ex:3-165a}
morgen wird ihn ein Mädchen beißen
\ex
\label{ex:3-165b}
morgen wird er ein Mädchen beißen
\zl
\addlines[2]\enlargethispage{6pt}% Ohne diese Seitenvergrößerung macht es auf die rechte Seite einen
% Strich .....
Die Autoren, die für (\ref{ex:3-163}) Eindeutigkeit behaupten, tun dies
eigenartigerweise gewöhnlich ohne weiteren Kommentar, als sei das eine
gewissermaßen selbstverständliche und höchst natürliche
Regularität.\footnote{\label{fn:3-30}%
	Für die angebliche Regularität werden zudem verschiedene
  Regelformulierungen vorgeschlagen. Wenn man Sätze wie (i)
  berücksichtigt, kann es in (\ref{ex:3-163}) allenfalls um die Vermeidung von
  Ambiguitäten gehen. Einige behaupten jedoch, die relevante Bedingung
  sei phonologische
  \eal
  \label{ex:3-fn30i}
  \ex
  \label{ex:3-fn30ia}
  solche Theorien haben meistens Germanisten ersonnen
  \ex
  \label{ex:3-fn30ib}
  da sie meistens Germanisten ersonnen haben
  \zl
  Nicht"=Unterscheidbarkeit der \isi{Kasus}. Demnach müßten (ia,b) eindeutig
  synonym mit (iia,b) sein. Da die meisten Sprecher (ii)~-- im
  Gegensatz zu (i)~-- als semantisch abweichend beurteilen, kann
  Kasussynkretismus nicht die relevante (sondern allenfalls eine
  notwendige) Bedingung sein.
  \eal
  \label{ex:3-fn30ii}
  \ex
  \label{ex:3-fn30iia}
  von solchen Theorien sind meistens Germanisten ersonnen worden
  \ex
  \label{ex:3-fn30iib}
  da von ihnen meistens Germanisten ersonnen worden sind
  \zllast%
}
Tatsächlich wirft diese Behauptung aber zwei
schwerwiegende Erlärungsprobleme auf: (a) Wie ist es überhaupt
möglich, daß (\ref{ex:3-163}) 1"=deutig ist, wenn die topologischen Regeln doch
(\ref{ex:3-164}) und (\ref{ex:3-165}) zulassen, und (b) wieso soll gerade die Interpretation
mit \isi{Subjekt} vor Objekt die einzig mögliche sein und nicht die
umgekehrte?

Tatsächlich ist immer schon bestritten worden, daß (\ref{ex:3-163}) 1"=deutig sei;
vgl.\ Erdmann zu (\ref{ex:3-163a}):
\begin{exe}
\ex \label{ex:3-166}
"`Wenn manche Grammatiker empfohlen haben, den \isi{Nominativ} und Accusativ gleicher Form dadurch zu unterscheiden, daß jener vor, dieser hinter das Verbum gestellt werde, so trafen sie damit ebenfalls keine im Sprachgefühle der Deutschen lebendige Richtung. Ein Satz wie: \textit{Cäsar besiegte Pompejus} wird im Deutschen\il{Deutsch} immer zweideutig sein; wer die Zweideutigkeit vermeiden will, der muss zu anderen Mitteln greifen, als zu der Unterscheidung durch die \isi{Wortfolge}. Dazu wird der Stilist raten, aber der Grammatiker kann den Satz nicht für unrichtig erklären."' \citep[183]{Erdmann1886}
\end{exe}
Im Zusammenhang mit dem Typ (\ref{ex:3-163b}) gibt Mentrup einen Hinweis:
\ea
\label{ex:3-167}
	"`Gestern hat \textit{ihn Karl} gesprochen [\ldots]. (Seltener:) Gestern hat
\textit{Karl ihn} gesprochen. (Zur Vermeidung von Mißverständnissen:) Gestern
hat \textit{meine Mutter/""das Mädchen} sie (= Objekt) gesehen."'
\citep[§\,1524]{Duden73}
\z
Mit der Formulierung "`zur Vermeidung von Mißverständnissen"' scheint er
sagen zu wollen: In (\ref{ex:3-168a}) könnte \textit{sie} nach den topologischen Regeln
\isi{Subjekt} oder Objekt sein. In (\ref{ex:3-168b}) kann sie (aufgrund der
topologischen Regeln) nur Objekt sein. Wenn man Mißverständnisse
\begin{exe}
\ex
\label{ex:3-168}
\begin{xlist}
\ex
\label{ex:3-168a}
gestern hat sie meine Mutter/""das Mädchen gesehen
\ex
\label{ex:3-168b}
gestern hat meine Mutter/""das Mädchen sie gesehen
\end{xlist}
\end{exe}
vermeiden will~-- und das sollte man tun~--, dann wird man nur (\ref{ex:3-168b}) wählen, wenn \textit{sie} Objekt ist.

Wenn ein Sprecher kooperativ ist und so weit wie möglich Ambiguitäten
vermeidet, ist das in der Tat eine plausible Strategie. Ein Hörer, der
mit einem derart kooperativen Sprecher rechnet, wird erwarten, daß die
Objektfunktion von \textit{sie} eindeutig kenntlich gemacht ist, also
nur (\ref{ex:3-168b}) erwarten; in (\ref{ex:3-168a}) ist \textit{sie} für ihn dann
eindeutig \isi{Subjekt}. Andererseits läßt diese Strategie genügend
Spielraum, um \textit{sie} auch als Objekt in (\ref{ex:3-168a}) zuzulassen, wenn Hörer
und Sprecher aufgrund des Kontexts nicht mit möglichen
Mißverständnissen rechnen.

Die Annahme einer solchen Strategie könnte erklären, wieso jemand
geneigt ist, (\ref{ex:3-163b}) entgegen den allgemeinen topologischen Regeln in
der besprochenen Weise für 1"=deutig zu halten. Sie hat jedoch
keinerlei Erklärungswert für (\ref{ex:3-163a}). Denn wenn wir zugestehen, daß
(\ref{ex:3-163a}) gemäß den topologischen Regeln~-- und in Übereinstimmung mit
der Intuition sehr vieler (wenn nicht aller) Sprecher~-- 2"=deutig ist,
gibt es nicht wie in (\ref{ex:3-168}) eine 1"=deutige Formulierungsalternative,
die verständlich machen würde, warum (\ref{ex:3-163a}) als 1"=deutig
interpretiert wird. Insbesondere hilft auch der vielleicht
naheliegende Verweis auf das Passiv nicht: Zwar ist (\ref{ex:3-169}) 1"=deutig,
aber warum sollte
\begin{exe}
\ex
\label{ex:3-169}
\begin{xlist}
\ex
\label{ex:3-169a}
die Frau ist von einem Mädchen gebissen worden
\ex
\label{ex:3-169b}
ein Mädchen ist von der Frau gebissen worden
\end{xlist}
\end{exe}
etwa (\ref{ex:3-169a}) als 1-deutige Alternative gewählt werden, wenn \textit{Mädchen} in
(\ref{ex:3-163a}) als \isi{Subjekt} intendiert ist, und nicht vielmehr (\ref{ex:3-169b}), wenn
\textit{Frau} in (\ref{ex:3-163a}) \isi{Subjekt} sein soll? Entsprechend für die Objektinterpretation und (\ref{ex:3-169b}).

Interessanterweise bleibt das Erklärungsproblem auch dann bestehen,
wenn man die Behauptung, (\ref{ex:3-163a}) sei 1"=deutig, zurückweist. Denn es
scheint mir sehr deutlich, daß auch für jene Sprecher, für die (\ref{ex:3-163}a,b) klar 2"=deutig sind, die Interpretation mit \textit{Mädchen} als Objekt
intuitiv näherliegend ist. Über diese Interpretation verfügt man
sofort beim Hören des Satzes, und zwar unabhängig davon, ob \textit{Frau}
bzw.\ \textit{sie} oder \textit{Mädchen} betont ist (\dash auch bei nicht"=normaler
\isi{Betonung}), während es außerhalb eines entsprechenden Kontexts einen
Augenblick des Überlegens braucht, \textit{Mädchen} als mögliches \isi{Subjekt} zu
erkennen.

Nach unseren Erörterungen über stilistisch normale \isi{Wortstellung} drängt
sich die Erklärung für diese Fakten auf: Außerhalb von
desambiguierenden Kontexten bevorzugt der Hörer eine Interpretation,
die der normalen \isi{Wortstellung} entspricht. Und dies ist verständlich,
denn wie wir gesehen haben, ist unter normaler \isi{Wortstellung} die Anzahl
der prinzipiell möglichen Kontexttypen am größten; diese
Interpretation ist daher eine relativ sichere Interpretation. Hinsichtlich des Sprechers bedeutet das, daß von ihm im Zweifelsfall der Gebrauch der normalen \isi{Wortstellung} zu erwarten
ist, da diese \isi{pragmatisch} am vielseitigsten zu verwenden ist.

Unsere Explikation von stilistisch normaler \isi{Wortstellung} leistet damit
eine Erklärung in einem Bereich, dessen Erklärungsbedürftigkeit
traditionell nicht einmal erkannt worden ist.

\section{Strukturell normale Wortstellung}
\label{sec:3-3}

Im vorigen Abschnitt haben wir eine Explikation von
\textsq{stilistisch normaler Wortstellung} besprochen. In der
sprachwissenschaftlichen Tradition findet dieser intuitive Begriff
relativ wenig Beachtung; wo von \textsq{normaler Wortstellung} die
Rede ist, scheint meist ein anderer Begriff gemeint zu sein, der
allerdings kaum je erläutert wird. Der einzige mir bekannte Versuch
dazu stammt von Lenerz. Dank seiner expliziten~-- und für die deutsche
\isi{Topologie} sehr wichtigen~-- Darstellung ist es möglich, im folgenden
die Aspekte zu besprechen, in denen sich dieser andere Begriff von
\textsq{normaler Wortstellung} wesentlich von der Explikation (\ref{ex:3-147})
unterscheidet. Lenerz erklärt:
\begin{exe}
\settowidth\jamwidth{(Altmann 1977: 100)}
\ex{\label{ex:3-171}
"`Wenn zwei Satzglieder A und B sowohl in der Abfolge AB wie auch in
der Abfolge BA auftreten können, und wenn BA nur unter bestimmten,
testbaren Bedingungen auftreten kann, denen AB nicht unterliegt, dann
ist AB die "`unmarkierte Abfolge"' und BA die "`markierte
Abfolge"'."'} \jambox{\citep[27 (14)]{Lenerz77}}
\end{exe}
(Die Prüfung der möglichen Abfolgen für gegebene A und B ist dabei im
wesentlichen~-- aus guten Gründen, vgl.\ Fn.\,\ref{fn:3-27} -- auf das Mittelfeld
beschränkt.)

Ich schlage zunächst in (\ref{ex:3-172}) eine alternative Formulierung für (\ref{ex:3-171})
vor. Anhand der Besprechung der Formulierungsunterschiede wird
deutlicher hervortreten, wie (\ref{ex:3-171}) gemeint ist und welchen Gebrauch
Lenerz von (\ref{ex:3-171}) macht.
\ea
\label{ex:3-172}
\textit{Definition:} \\
Wenn zwei Konstituententypen KT1 und KT2 in der Abfolge KT1 $>$ KT2
unter strukturellen Bedingungen MB1 und in der Abfolge KT2 $>$ KT1 unter
strukturellen Bedingungen MB2 vorkommen können, wobei MB2 $\subset$ MB1, dann
ist KT1 $>$ KT2 die \textit{strukturell normale Abfolge} und KT2 $>$ KT1 eine
\textit{strukturell markierte Abfolge}.
\z
Diese Formulierung weicht in etlichen Einzelheiten von (\ref{ex:3-171}) ab. Der Ersatz von "`AB"' in (\ref{ex:3-171}) durch "`KT1 $>$ KT2"' in (\ref{ex:3-172}) ist nötig, weil
Lenerz für die Normalität von Abfolgen Transitivität annimmt; \dash wenn AB und BC normal sind,
sollen auch ABC und AC normal sein; vgl.\ \zb S.\,86.\footnote{\label{fn:3-31}%
	Mit dieser Annahme
  gibt es jedoch empirische Probleme, vgl.\ \citet[88f]{Lenerz77}.%
} 

Ein unwesentlicher Unterschied ist, daß nach (\ref{ex:3-171}) die Abfolge B $>$ A
auftreten kann, während die Formulierung in (\ref{ex:3-172}) zuläßt, daß MB2 = $\emptyset$, \dash daß B $>$ A unakzeptabel ist. Es scheint mir sinnvoll, die Abfolge A $>$ B auch dann als \textsq{strukturell   normal} zu bezeichnen, wenn sie die einzig mögliche ist; aber das ist bei Bedarf leicht zu ändern.

In (\ref{ex:3-172}) ist MB2 als echte Untermenge von MB1 bestimmt; (\ref{ex:3-171}) dagegen läßt, wörtlich genommen, eine komplementäre Verteilung von A $>$ B und B $>$ A zu, \dash MB1 $\cap$ MB2 = $\emptyset$. Aus seiner Praxis ist klar, daß Lenerz nur den in (\ref{ex:3-172}) formulierten Fall meint.

Nach (\ref{ex:3-171}) und (\ref{ex:3-172}) ist daher \textsq{strukturell normale Abfolge} nur für MB2 $\subset$ MB1, \dash für (MB1 $\cup$ MB2 = MB1) \& (MB1 $\neq$ MB2) erklärt. Es wäre an sich auch möglich und einleuchtend, auf die Bedingung MB1 $\neq$ MB2 zu verzichten, \dash sowohl A $>$ B als auch B $>$ A als \textsq{normale} Abfolgen zu bezeichnen, wenn sie (hinsichtlich struktureller Bedingungen) frei variieren. Einen solchen Fall hat Lenerz nicht vorgesehen.

Aus dieser Tatsache sowie dem Umstand, daß er im Grundsatz Transitivität der Relation \textsq{normale Abfolge} annimmt und komplementäre Verteilung ausschließt, muß man schließen, daß es~-- jedenfalls für sein Untersuchungsgebiet~-- für alle Folgen von
\textsq{Satzgliedern} (so in (\ref{ex:3-171}); in (\ref{ex:3-172}): Konstituententypen) genau eine
vollständig bestimmte \textsq{normale Abfolge} gibt, die unter allen strukturellen Bedingungen möglich ist. Dies könnte~-- was Lenerz
nicht eigens diskutiert~-- eine interessante empirische Annahme sein. Wir
werden jedoch gleich sehen, daß sie das nicht ist.

Da freie Variation nicht zugelassen ist, heißt es in (\ref{ex:3-171}) wie in (\ref{ex:3-172}), daß KT1 $>$ KT2 \textit{die} \isi{normale Abfolge} ist. Dagegen bezeichne ich KT2 $>$ KT1 als \textit{eine} \isi{markierte Abfolge}, während B $>$ A in (\ref{ex:3-171}) als \textit{die} \isi{markierte Abfolge} bezeichnet wird. Grund: Es ist der Fall denkbar, daß es KT1 $>$ KT2 unter MB1, KT2 $>$ KT1 unter MB2 und KT2 $>$ KT1 unter MB3 gibt, wobei MB2 $\subset$ MB1 und MB3 $\subset$ MB1, aber MB2 $\neq$ MB3. Hier tritt ein wichtiges Problem zutage, das weniger in der Formulierung von (\ref{ex:3-171}) selbst als vielmehr in dem Gebrauch liegt, den Lenerz von seiner Definition macht. Die Bedingungen, denen A $>$ B und B $>$ A unterliegen, sind nämlich in seinen Untersuchungen keine kontextuellen Bedingungen (etwa so, daß die Abfolgen C $>$ A $>$ B $>$ D, C $>$ B $>$ A $>$ D und E $>$ A $>$ B $>$ D, aber nicht E $>$ B $>$ A $>$ D möglich wären). Die Bedingungen MB1, MB2 usw.\ sind vielmehr Eigenschaften der
jeweils untersuchten Konstituententypen selbst, ebenso wie es eine
Eigenschaft dieser Konstituenten ist, dem Typ A bzw.\ B anzugehören. Wenn wir die Eigenschaften der Konstituententypen durch Merkmale \textit{A, B, \ldots{} E1, E2, \ldots} mit einem Vorzeichen notieren (wobei
"`+"' heißt, daß der Konstituententyp die fragliche Eigenschaft hat; "`--"', daß er sie nicht hat; und "`o"', daß er sie hat oder nicht hat) dann ist [+\textit{A}] $>$ [+\textit{B}] \zb dann strukturell normal, wenn (\ref{ex:3-173}a,b) akzeptabel und (\ref{ex:3-173c}) unakzeptabel ist:
\eal
\label{ex:3-173}
\ex
\label{ex:3-173a}
[+\textit{A, oE1}] $>$ [+\textit{B, oE1}]
\ex 
\label{ex:3-173b}
[+\textit{B, oE1}]\hspace{0.1em} $>$ [+\textit{A}, +\textit{E1}]
\ex
\label{ex:3-173c}
[+\textit{B, oE1}]\hspace{0.1em} $>$ [+\textit{A}, -\textit{E1}]
\zl
Der oben angenommene Fall, daß MB2 $\neq$ MB3, stellt sich dann so
dar, daß \zb (\ref{ex:3-174}a--c) akzeptabel sind, nicht aber (\ref{ex:3-174}d,e):
\eal
\label{ex:3-174}
\ex
\label{ex:3-174a}
[+\textit{A, oE1}] $>$ [+\textit{B, oE1}]
\ex
\label{ex:3-174b}
[+\textit{B}, +\textit{E1}] $>$ [+\textit{A}, -\textit{E1}]
\ex
\label{ex:3-174c}
[+\textit{B}, -\textit{E1}]\hspace{0.2em} $>$ [+\textit{A}, +\textit{E1}]
\ex
\label{ex:3-174d}
[+\textit{B}, +\textit{E1}] $>$ [+\textit{A}, +\textit{E1}]
\ex
\label{ex:3-174e}
[+\textit{B}, -\textit{E1}]\hspace{0.2em} $>$ [+\textit{A}, -\textit{E1}]
\zl
Um (\ref{ex:3-171}) auf diese Situation anzuwenden, gibt es zwei Möglichkeiten:
Entweder formuliert man die Bedingung: KT2 $>$ KT1 als eine Disjunktion
von Bedingungen: KT2 $>$ KT1 ist möglich wenn KT1 [+\textit{A}], KT2 = [+\textit{B}] \textit{und}
((KT1 = [-\textit{E1}] und KT2 = [+\textit{E1}]) \textit{oder} (KT1 = [+\textit{E1}] und KT2 =
[-\textit{E1}])). Dann ist es etwas irreführend, zu sagen, KT2 $>$ KT1 sei \textit{die}
\isi{markierte Abfolge}, denn die beiden Konstituententypen müssen ja recht
verschiedene Kombinationen von Eigenschaften (\textsq{Bedingungen})
erfüllen, um in der Abfolge KT2 $>$ KT1 vorkommen zu können. Oder aber
man betrachtet nicht mehr [+\textit{A}] und [+\textit{B}] als die für KT1 bzw.\ KT2
charakteristischen Eigenschaften, sondern \zb die Kombination [+\textit{A}, -\textit{E1}] bzw.\ [+\textit{A}, +\textit{E1}]. In diesem Fall erhält man das Resultat, daß [+\textit{A}, -\textit{E1}] $>$ [+\textit{B}] \textit{die} normale und [+\textit{B}] $>$ [+\textit{A}, -\textit{E1}] \textit{die} \isi{markierte Abfolge} ist, und daß [+\textit{A}, +\textit{E1}] $>$ [+\textit{B}] \textit{die} normale und [+\textit{B}] $>$ [+\textit{A}, +\textit{E1}] \textit{die} \isi{markierte Abfolge} ist. Dann ist aber nicht, wie es offenbar eigentlich
angestrebt ist, terminologisch erfaßt, daß [\textit{+B}] $>$ [+\textit{A}] besonderen
Bedingungen unterliegt, denen [+\textit{A}] $>$ [+\textit{B}] nicht unterliegt.

Diese Interpretation von (\ref{ex:3-171}), (\ref{ex:3-172}) läßt also eine beträchtliche
Freiheit, welche Eigenschaften und vor allem Kombinationen von
Eigenschaften man als charakteristisch für KT1 und KT2 ansieht,
welches demnach die für die Abfolge relevanten \textsq{Bedingungen}
sind und demzufolge auch, welche Abfolge \textsq{normal}
bzw.\ \textsq{markiert} ist. Dies macht sich Lenerz an einigen Stellen
zunutze.

Zu den von ihm berücksichtigten Eigenschaften gehören besonders:
welche Relationen KT1 und KT2 zum \isi{Verb} haben (\isi{Subjekt}, indirektes (=
\isi{Dativ}"=) Objekt, direktes (= \isi{Akkusativ}"=) Objekt, \textsq{freie Dative},
\ldots, temporale \isi{Adverbiale}, \ldots); ob sie betont oder unbetont sind; ob
sie \isi{Personalpronomen} oder Substantive sind. Er stellt fest, daß für
unbetonte rein \textsq{kasuelle} (nicht in einer PP enthaltene)
\isi{Personalpronomen} z.\,T.\ andere Regularitäten als für substantivische
NPs gelten. So ist nach seinen Daten aufgrund von (\ref{ex:3-171}) für
Substantive die Abfolge \isi{Dativ} $>$ \isi{Akkusativ} normal und \isi{Akkusativ} $>$ \isi{Dativ}
markiert; dagegen ist \isi{Akkusativ} $>$ \isi{Dativ} die einzig mögliche Abfolge,
wenn der \isi{Akkusativ} ein unbetontes \isi{Personalpronomen} ist. Ebenso ist
nach seinen Kriterien bei gewissen Verben die Abfolge Objekt $>$ \isi{Subjekt}
normal; ist das \isi{Subjekt} ein unbetontes \isi{Personalpronomen}, ist dagegen
nur \isi{Subjekt} $>$ Objekt möglich. Um überhaupt Aussagen über
normale/""markierte Abfolgen im Sinne seiner Definition machen zu
können, muß er also Kombinationen von Eigenschaften als
Charakteristika für KT1 und/""oder KT2 verwenden, \zb [+\textit{Subjekt},
+\textit{Personalpronomen}, -\textit{betont}]. Daher ist die Rede von
"`Satzgliedern A und B"' in (\ref{ex:3-171})~-- darunter versteht er Konstituenten
in bestimmter syntaktischer Relation zum \isi{Verb}, \zb \isi{Subjekt}, Objekt,
\isi{Adverbiale} usw.~-- irreführend und in (\ref{ex:3-172}) durch "`Konstituententypen
KT1 und KT2"' ersetzt: Sie suggeriert, daß die Eigenschaft, dem
Satzgliedtyp A bzw.\ B anzugehören, die charakteristische Eigenschaft
sei, während alle anderen Eigenschaften der betrachteten Konstituenten
zu den relevanten \textsq{Bedingungen} gehören. Dies ist, wie wir
gesehen haben, unzutreffend: Neben dem Satzgliedtyp (oder u.\,U.\ statt
dessen) dienen nach Bedarf auch [± \textit{betont}] und [± \textit{Personalpronomen}]
als charakteristische Eigenschaften.

Mit Hilfe der Kombination von Eigenschaften läßt sich auch das Problem
komplementärer Verteilungen leicht lösen. Wenn \zb (\ref{ex:3-175}a,b)
akzeptabel und (\ref{ex:3-175}c,d) unakzeptabel sind, sind (\ref{ex:3-171}) und (\ref{ex:3-172})
nicht
\begin{exe}
\ex
\label{ex:3-175}
\begin{xlist}
\ex
\label{ex:3-175a}
[+\textit{A}, +\textit{E1}] $>$ [+\textit{B}, -\textit{E1}]
\ex
\label{ex:3-175b}
[+\textit{B}, +\textit{E1}]\hspace{0.1em} $>$ [+\textit{A}, -\textit{E1}]
\ex
\label{ex:3-175c}
[+\textit{A}, -\textit{E1}]\hspace{0.2em} $>$ [+\textit{B}, +\textit{E1}]
\ex
\label{ex:3-175d}
[+\textit{B}, -\textit{E1}]\hspace{0.1em} $>$ [+\textit{A}, +\textit{E1}]
\end{xlist}
\end{exe}
anwendbar, wenn [+\textit{A}] und [+\textit{B}] als charakteristische Eigenschaften der
Konstituententypen betrachtet werden, und weder (\ref{ex:3-175a}) noch (\ref{ex:3-175b}) kann als normal oder markiert ausgezeichnet werden. Wir können aber
einfach die Kombinationen [+\textit{A}, +\textit{E1}] und [+\textit{B}, -\textit{E1}]
bzw.\ [+\textit{A}, -\textit{E1}] und [+\textit{B}, +\textit{E1}] als Charakteristika von KT1
bzw.\ KT2 betrachten. Da für diese Kombination (\ref{ex:3-175a}) bzw.\ (\ref{ex:3-175b}) die einzig
möglichen Abfolgen sind, sind sie dann nach (\ref{ex:3-172}) jeweils
normal.\footnote{\label{fn:3-32}%
	In diesem Fall geht es natürlich noch einfacher, wenn [+\textit{E1}]
  und [-\textit{E1}] als Charakteristika gewählt werden: Dann ist [+\textit{E1}] $>$ [-\textit{E1}]
  die einzig mögliche (und nach (\ref{ex:3-172}) normale) Abfolge. Diese Vereinfachung ist nicht
  möglich, wenn es neben (\ref{ex:3-175}a,b) \zb auch [+\textit{C}, -\textit{E1}] $>$
  (+\textit{A}, +\textit{E1}] gibt.%
} 

Durch solche Freiheiten bei der Bestimmung der charakteristischen Eigenschaften von KT1 und KT2 läßt sich die Forderung, eine \isi{normale Abfolge} für alle Konstituententypen des Satzes festzulegen, ohne Schwierigkeiten erfüllen. Aber im gleichen Maße schwindet natürlich das mögliche empirische Interesse an einem so flexiblen Begriff von
\textsq{normaler Abfolge}.

Auf den wichtigsten Unterschied zwischen (\ref{ex:3-171}) und (\ref{ex:3-172}) sind wir
jedoch noch nicht zu sprechen gekommen.

\addlines
In (\ref{ex:3-171}) ist allgemein von \textsq{Bedingungen}, in (\ref{ex:3-172}) dagegen von
\textsq{strukturellen Bedingungen} die Rede. Darunter verstehe ich
syntaktische, morphologische, logische und phonologische
(incl.\ intonatorische und akzentuelle) Eigenschaften, nicht aber
pragmatische.

Dies ist nötig, wenn man der von Lenerz befolgten Praxis Rechnung
tragen will. Andernfalls könnte man \zb sagen, daß (\ref{ex:3-176b}) eine
markierte \isi{Wortstellung} aufweist und (\ref{ex:3-176a}) eine normale, da (\ref{ex:3-176a}) in demselben \isi{Kontexttyp} vorkommen kann wie (\ref{ex:3-176b}), außerdem aber noch
in vier weiteren; vgl.\ (\ref{ex:3-101}). Dies stände in völligem Widerspruch zu
\begin{exe}
\ex
\label{ex:3-176}
\begin{xlist}
\ex
\label{ex:3-176a}
Karl hat das Buch dem \textit{Mann} gegeben
\ex
\label{ex:3-176b}
Karl hat dem \textit{Mann} das Buch gegeben
\end{xlist}
\end{exe}
Lenerz'~-- in sich konsistenten~-- Ergebnissen, nach denen es gerade
umgekehrt ist. (Wir erinnern uns, daß nach (\ref{ex:3-147}) und (\ref{ex:3-79}) sowohl (\ref{ex:3-176a}) als auch (\ref{ex:3-176b}) stilistisch normale \isi{Wortstellung} haben; (\ref{ex:3-176a}) hat stilistisch normale, (\ref{ex:3-176b}) stilistisch nicht"=normale Betonung\is{Normalbetonung}.)

Die Art, wie Lenerz Beispiele in seine Untersuchungen einführt, könnte
vermuten lassen, daß es ihm, im Gegensatz zu (\ref{ex:3-172}), sogar ganz
ausdrücklich um pragmatische Zusammenhänge geht. So gibt er den
Beispielen, die er diskutiert, gewöhnlich einen \textsq{Fragetest}
bei, und seine Regeln formuliert er unter Verwendung der Ausdrücke
"`Thema"' (entspricht \textsq{Topik}) und "`\isi{Rhema}"' (entspricht \textsq{Fokus});
dies sind nach unserer und seiner \citep[11ff]{Lenerz77} Explikation pragmatische
Begriffe. So z.\,B.
\begin{exe}
\settowidth\jamwidth{(Lenerz 1977: 43)}
\ex\label{ex:3-177}
\begin{enumerate}
		\setlength{\itemindent}{2em}
	\item["`(3)\hphantom{(a)}]Was hast du dem Kassierer gegeben?
		\item[\hphantom{"`}(3) a)] Ich habe dem Kassierer das \textit{Geld} gegeben.
	\item[\hphantom{"`}(3) b)]?* Ich habe das \textit{Geld} dem Kassierer gegeben.
\end{enumerate}
Im Kontext (3), der DO als \isi{Rhema} eindeutig festlegt, ist nur die Antwort (3) a) akzeptabel, also die Abfolge IO DO. Die Abfolge DO IO scheidet also aus, wenn DO das \isi{Rhema} ist!"' \citep[43]{Lenerz77}
\end{exe}
Diese Ausdrucksweise ist jedoch irreführend. Es ist für Lenerz völlig
belanglos, was in (3) a), b) \isi{Fokus} ist: In (3) a) gibt es 5
verschiedene Möglicheiten; in (3) b) für die Sprecher, die den Satz
akzeptieren, genau eine, und zwar mit dem DO als
\isi{Fokus}. Dementsprechend prüft er auch nicht, ob (3) b) in irgendeinem
anderen Kontext als (3) geäußert werden könnte, sondern ihm reicht die
Feststellung, daß (3) b) \textit{mit dieser Betonung} für ihn
unakzeptabel ist. Es geht also gar nicht darum, daß \textit{im
  Kontext} (3) die \textit{Antwort} (3) b) \textsq{unakzeptabel} ist, sondern
der \emph{Satz} (3) b) ist (für ihn) unakzeptabel. Es trifft ja auch
keineswegs zu, daß der Kontext (3) das DO in (3) a), b) als \isi{Fokus}
festlegt. Vielmehr ist durch die \isi{Betonung} in (3) a) festgelegt, daß
\textit{das Geld} ein möglicher \isi{Fokus} ist und daß deshalb (3) a) in einem
Kontext, wie er durch (3) angedeutet wird, ohne Verletzung
konversationeller Maximen geäußert werden kann. In beliebigen anderen
Kontexten, etwa (\ref{ex:3-178}), andert sich an der Akzeptabilität von (3) a),
b) und an den Fokusmöglichkeiten nicht das geringste, nur würde es
gegen die
\ea
\label{ex:3-178}
wem hast du das Geld gegeben?
\z
Maximen verstoßen, ihn dort zu äußern. Hier wie in allen anderen
Fällen macht Lenerz allein die Akzeptabilitätsbeurteilung von Sätzen
mit einer gegebenen \isi{Betonung} zur Grundlage seiner Feststellungen über
markierte und unmarkierte Konstituentenfolgen. Die beigegebenen Fragen
dienen keineswegs dazu, mögliche Foki zu eruieren; sie können
allenfalls, indem sie einen natürlichen Kontext andeuten, die oft sehr
delikate Akzeptabilitätsbeurteilung erleichtern.

In dieser Hilfsfunktion sind die Fragen natürlich nützlich; aber
solche scheinbar auf pragmatische Regularitäten zielenden
Formulierungen wie in (\ref{ex:3-177}) sind doch überraschend. Möglicherweise
sind sie auf ein Mißverständnis zurückzuführen: Lenerz möchte die
Regularitäten, die er untersucht, einerseits zu Dane\v{s}s \textsq{weak
  rules} rechnen (wohl weil er sieht, daß Wortstellungs"= und
Betonungsunterschiede etwas mit Fokusunterschieden zu tun haben, die
Dane\v{s} durch eben solche \textsq{weak rules} beschreiben möchte),
vgl.\ \citet[26f]{Lenerz77}. Andererseits will er aber die Akzeptabilitätsbeurteilung zur Grundlage seiner Klassifikation in \textsq{normale} vs.\ \textsq{markierte} Abfolgen
machen. Dieser Unterschied ist wesentlich: Für Dane\v{s}s \textsq{weak
  rules} ist relevant, wie die intuitive \textsq{Normalitätsbeurteilung} \textit{akzeptabler} Sätze mit
\isi{Topik}/""\isi{Fokus}"=Unterschieden zusammenhängt. Wenn (3) b) in (\ref{ex:3-177}) unakzeptabel ist, kann dies nicht auf die Wirkung einer \textsq{weak rule} zurückgehen; Dane\v{s} führt solche Fälle auf
\textsq{concomitant rules} zurück, vgl.\ das Zitat in \citet[27]{Lenerz77}. Die
Unakzeptabilität dieses Beispiels ist aber nach (\ref{ex:3-171}) eine wesentliche
Grundlage dafür, die Abfolge DO $>$ IO als markiert zu bezeichnen.

In Zusammenhang mit dieser strukturellen, nicht"=pragmatischen
Orientierung von (\ref{ex:3-171}) steht die Tatsache, daß ich in (\ref{ex:3-172}) nicht von
\textsq{Konstituenten} spreche, sondern von \textsq{Konstituententypen}. Die Formulierung in (\ref{ex:3-171}) läßt offen, ob der Vergleich der Abfolgen A $>$ B und B $>$ A jeweils nur innerhalb einer Menge EBS\rlap{\textsuperscript{\small \raisebox{-2pt}{*}}}\textsubscript{\small i} stattfinden soll, die aus Elementen einer Menge EBS\textit{\textsubscript{i}}
wie in (\ref{ex:3-76c}) und allen genau entsprechenden aber unakzeptablen
Sätzen besteht~-- dann stehen primär nur Konstituenten, nicht
Konstituententypen zur Debatte~--, oder ob über Satztypen~-- und damit
Konstituententypen~-- Aussagen gemacht werden sollen. Wenn man sieht,
wie generell Lenerz die Eigenschaften der betrachteten
Konstituenten(typen)~-- ihre charakteristischen Eigenschaften wie die
\textsq{Bedingungen}, unter denen eine Abfolge möglich oder unmöglich
ist~-- formuliert, ist es klar, daß es ihm ganz entschieden auf
Satztypen ankommt. Daher kommt er auch \zb zu dem~-- in sich
konsistenten~-- Schluß, daß in (\ref{ex:3-176a}) eine \isi{markierte Abfolge}
vorliegt. Das ist bemerkenswert, weil der Satz intuitiv nach seinem
Urteil wie nach dem Urteil anderer Sprecher (und unter der Explikation
(\ref{ex:3-147})) stilistisch völlig unauf"|fällig, \textsq{normal} und
\textsq{unmarkiert} ist.

Im Gegensatz dazu ist (\ref{ex:3-147}) eine Explikation des intuitiven Begriffs
der normalen \isi{Wortstellung} und jeweils auf Sätze (als Elemente einer
EBS\sub{\textit{i}}) bezogen, nicht auf Satztypen. Zum einen ist das aufgrund der
pragmatischen Natur der Explikation technisch notwendig: Sie baut,
genau wie die Explikation von \textsq{stilistisch normaler Betonung},
auf der \isi{Topik}/""\isi{Fokus}"=Unterscheidung, \dash auf möglichen Kontexttypen,
auf. Kontexttypen sind primär aber nicht für Satztypen zu bestimmen,
sondern für Sätze. Dieses Vorgehen scheint zum andern auch adäquat,
indem es \zb (\ref{ex:3-176a}) als stilistisch normal auszeichnet. (Das
bedeutet natürlich nicht, daß diese Explikation bei der Betrachtung
von einzelnen Sätzen stehen zu bleiben hätte. Im Gegenteil: Sie
provoziert die Suche nach allgemeinen Regeln der \isi{Fokusprojektion}, und
diese werden natürlicherweise für Klassen von EBS\sub{\textit{i}} formuliert.)

Insofern (\ref{ex:3-147}) die Explikation von unmittelbaren Sprecherintuitionen
ist, ist sie sprachwissenschaftlich relevant, und sie ist essentiell,
insofern sie eben deshalb nicht definitorisch auf unabhängig
begründete Begriffe zurückgeführt werden kann. Allerdings kann man
angeben, unter welchen Bedingungen eine \isi{Wortstellung} als stilistisch
normal empfunden wird, und man kann, wie wir gesehen haben, sogar
erklären, warum gerade diese Abfolge so empfunden wird, was an ihr
also \textsq{normal} ist. Gilt das gleiche für (\ref{ex:3-172})?

Zunächst ist klar, daß \textsq{strukturell normale Abfolge} nach (\ref{ex:3-172})
kein essentieller Begriff ist: Lenerz beginnt seine Abhandlung nicht mit der Untersuchung intuitiver Urteile über normale und markierte Abfolgen, sondern mit einer Definition, die festlegt, daß für das
Vorliegen gewisser Akzeptabilitätsverteilungen bei Konstituententypen mit gegebenen strukturellen Eigenschaften der Ausdruck "`normale (unmarkierte)"' bzw.\ "`\isi{markierte Abfolge}"' gebraucht
wird. Selbstverständlich setzt Lenerz dabei voraus, daß es Aufgabe der
Grammatik ist, akzeptable von unakzeptablen (bzw.\ grammatische von
ungrammatischen) Sätzen zu unterscheiden. Von den dafür nötigen
grammatischen Regeln abstrahiert (\ref{ex:3-172}) insofern, als die Definition
nur noch sagt, \textit{daß} eine gewisse Abfolge nur unter eingeschränkten
Bedingungen vorkommt, während die Regeln der Grammatik angeben, unter
\textit{welchen} Bedingungen sie möglich ist. Hinsichtlich strukturell normaler
Abfolgen sagt sie dasselbe wie die Regeln der Grammatik: daß sie unter
allen strukturellen \textsq{Bedingungen} möglich ist. Es wäre danach
etwas überraschend, wenn \textsq{strukturell normale Abfolge} sich als
ein relevanter Begriff erweisen sollte.\footnote{\label{fn:3-33}%
	Alles, was wir hier über strukturell
  normale Abfolgen ausgeführt haben, trifft mutatis mutandis auch für die strukturell normalen
  Betonungen von Fuchs zu, vgl.\ Fn.\,\ref{fn:3-20}.%
} 

Daß dies ein relevanter Begriff ist, wird offenbar in vielen
traditionellen Grammatiken implizit angenommen. Die Behauptung z.\,B.,
die Folge \isi{Dativ} $>$ \isi{Akkusativ} sei \textsq{die normale Abfolge}, findet
sich in vielen Arbeiten zum Deutschen\il{Deutsch}~-- aber fast ausnahmslos ohne
Erläuterung. Es ist eins von Lenerz' Verdiensten, daß er zum ersten
Mal geklärt hat, in welchem Sinne~-- und mit welcher Berechtigung~--
eine solche Behauptung überhaupt erhoben werden kann; aber dadurch
ist ihre Relevanz noch nicht demonstriert. (Für einen Sprachlerner mag
es nützlich sein, sich einzuprägen, welche Abfolgen keinen
strukturellen Einschränkungen unterliegen; aber ein didaktischer
Gesichtspunkt ist nicht per se von Interesse für die Sprach"= und
Grammatiktheorie.)

Diese traditionelle Annahme ist häufig in transformationelle
Grammatiken übernommen und ohne weitere Reflexion mit der Annahme
verknüpft worden, strukturell normale Abfolgen müßten \textit{als solche} auf
einer \textsq{Ebene} der Syntax repräsentiert werden, es müsse also
eine Ebene geben, auf der nur normale Abfolgen repräsentiert sind. Da
dies offensichtlich nicht die Oberflächenstruktur sein kann, muß es in
transformationellen Grammatiken des klassischen Typs die
\isi{Tiefenstruktur} sein. Aber hier wird wiederum vorausgesetzt, was erst
zu beweisen wäre: daß \textsq{Normalität} im Sinne von (\ref{ex:3-172}) als
solche auf einer syntaktischen Ebene repräsentiert werden muß. Natürlich ist es im Grundsatz möglich, daß eine empirisch adäquate \isi{transformationelle Grammatik} Tiefenstrukturen generiert, die
nur strukturell normale Abfolgen aufweisen. (Streng genommen hieße
das, daß es in einer solchen Grammatik weder obligatorische Bewegungstransformationen noch irgendwelche kontextsensitiven Filter gäbe.) Eine empirisch adäquate Grammatik würde solche Tiefenstrukturen jedoch aus guten empirischen Gründen aufweisen, nicht \textit{weil} man eine
Definition wie (\ref{ex:3-172}) aufstellen kann.

\addlines
Lenerz schließt sich der Auf"|fassung an, daß die (für Substantive)
strukturell \isi{normale Abfolge} IO $>$ DO zugrunde liegt, versucht dafür
jedoch eine Begründung zu geben:
\begin{exe}
\ex \label{ex:3-179}
"`[Es] läßt sich zumindest versuchsweise die \textit{Bedingung}, der die
\isi{markierte Abfolge} DO IO unterliegt, als \textit{Begründung} für eine Umstellung
interpretieren. Unter diesem Gesichtspunkt ist es zumindest eine der
\textit{Funktionen} der Abfolge DO IO, ein rhematisches IO näher an das Ende
des Satzes zu stellen, als es die Abfolge IO DO erlaubt. [\ldots{} D]ie
Umstellung von IO DO zu DO IO hat genau dann \textit{keine Funktion}, wenn
dadurch das \isi{Rhema} DO näher zum Satzanfang gestellt wird, während es
doch in der Abfolge IO DO schon in der \textit{optimalen Stellung} war, nämlich
näher am \isi{Satzende} [\ldots] Dieser \textit{Erklärungsversuch} fußt auf der Annahme,
daß IO DO die unmarkierte Abfolge der Objekte ist, \textit{von der ausgehend}
Umstellungen vorgenommen werden, dessen Funktion es ist, das
rhematische Element \textit{näher zum Satzende} hin zu bewegen. Wenn man
hingegen beide Abfolgen als gleichberechtigt ansieht, kann man die
Asymmetrie nicht erklären, die darin besteht, daß in der Abfolge IO DO
auch das linke Element, nämlich das IO [,] das \isi{Rhema} sein kann,
während in der Abfolge DO IO \textit{nur das rechte Element} \isi{Rhema} sein kann."'
\citep[45; Hervorhebungen von mir]{Lenerz77}
\end{exe}
(Was Lenerz hier \textsq{Rhema} nennt, wäre genauer als \textsq{Teil
  des minimalen Fokus} zu bezeichnen).

Der Erklärungswert der Annahme hängt offenbar davon ab, daß (a)
Umstellungsregeln eine \textsq{thematische Funktion} haben müssen
(sonst spräche nichts dagegen, ein \textsq{rhematisches DO} aus der
Normalfolge in die Abfolge DO $>$ IO umzustellen), und daß (b) die
Position am Ende des Mittelfelds die \textsq{optimale} Stellung für
ein \textsq{Rhema} ist. Gegen beide Annahmen sind jedoch Einwände zu
erheben. Vor allem gibt es unmittelbare empirische Gegengründe. So
sind (\ref{ex:3-180}a,b) (= \citealt[44 (5a,b)]{Lenerz77}) für Le\-nerz voll akzeptabel:

%\pagebreak

\begin{exe}
\ex
\label{ex:3-180}
\begin{xlist}
\ex
\label{ex:3-180a}
ich habe dem \textit{Kassierer} das \textit{Geld} gegeben
\ex
\label{ex:3-180b}
ich habe das \textit{Geld} dem \textit{Kassierer} gegeben
\end{xlist}
\end{exe}
In (\ref{ex:3-180b}) wäre jedoch das DO als Teil des \textsq{Rhemas} in eine
\textsq{nicht"=optimale} Position näher zum Satzanfang umgestellt, was
nach diesen Annahmen nicht sein kann. (Der letzte Satz von (\ref{ex:3-179}) ist
mit diesem Beispiel nicht verträglich.) Und bei Paaren wie (\ref{ex:3-181}a,b) (= \citealt[44 (6e,f)]{Lenerz77})
"`scheinen"' ihm Sätze mit DO $>$ IO nur "`leicht abweichend zu sein"'
(S.\,44), während die Abweichung in (3) b) von (\ref{ex:3-177}) für ihn sehr
deutlich ist:
\begin{exe}
\ex
\label{ex:3-181}
\begin{xlist}
\ex
\label{ex:3-181a}
ich habe dem Kassierer das Geld \textit{gestern} gegeben
\ex
\label{ex:3-181b}
ich habe das Geld dem Kassierer \textit{gestern} gegeben
\end{xlist}
\end{exe}
Nach seinen Annahmen müßte (\ref{ex:3-181b}) jedoch ganz unmöglich sein. Und
warum sollte es nicht eine Umstellung mit \textsq{thematischer Funktion} in eine nicht"=optimale Position geben? Die \textsq{Funktion} der Umstellung eines rhematischen DO vor ein nicht"=rhematisches IO könnte ja vielleicht sein, daß dadurch mögliche Fokusprojektionen verhindert werden (dies ist ein Effekt van (\ref{ex:3-155})), also Fokusdesambiguierung. Sätze wie (\ref{ex:3-182}) werden schließlich allgemein als akzeptabel betrachtet:
\eal
\label{ex:3-182}
\ex
\label{ex:3-182a}
das \textit{Buch} hat Karl dem Kind gegeben
\ex
\label{ex:3-182b}
dem \textit{Kind} hat Karl das Buch gegeben
\zl
Die Abfolge DO $>$ IO auf eine Umstellung aus IO $>$ DO zurückzuführen ist
demnach wenig begründet. Mir scheint die Vermutung nicht abwegig, daß
eigentlich umgekehrt die stillschweigende Annahme, daß es (a) eine
(empirisch signifikante) vollständig bestimmte Normalfolge gibt, die
(b) auf einer syntaktischen \textsq{Ebene} als solche zur
repräsentieren ist, die ursprüngliche Motivation für die Annahme einer
Umstellung und überhaupt für das Interesse an strukturell normalen
Abfolgen ist. Im Kontext von \citet{Lenerz77} ist (\ref{ex:3-179}) nämlich etwas
überraschend, denn die Bedingungen für markierte Abfolgen formuliert
Lenerz im Allgemeinen keineswegs positiv als Bedingungen für Umstellungen. Für
das Verhältnis von (substantivischen) IO und DO gibt er zwei
wesentliche Bedingungen: eine Definitheitsbedingung (S.\,55), auf die
ich hier nicht eingehe, und eine \textsq{Thema"=\isi{Rhema}"=Bedingung}. Diese
lautet in einer ihrer Formulierungen:
\begin{exe}
\ex \label{ex:3-183}
"`DO IO ist nicht möglich, wenn DO das \isi{Rhema} und gleichzeitig IO nicht
das \isi{Rhema} ist."' \citep[44 (4c)]{Lenerz77}
\end{exe}
\addlines[2]
(Diese Formulierung trägt u.\,a.\ (\ref{ex:3-180}), (\ref{ex:3-181}) Rechnung.) Diese
Bedingung hat zwei wichtige Aspekte. (a) Sie setzt voraus, daß alle Abfolgen, die ihr nicht widersprechen, vorkommen können. Bei wörtlichem Verständnis wird man sie daher nicht als Kontextprädikat einer Transformationsregel (\textsq{Umstellung}) interpretieren (was technisch auch nahezu unmöglich wäre, wenn man (\ref{ex:3-176}), (\ref{ex:3-180}) und (\ref{ex:3-181})
berücksichtigt), sondern negativ als Oberflächenrestriktion (Filter), die gewisse von der Grammatik generierte Konstituentenfolgen als ungrammatisch markiert. (b) Dadurch daß alle anderen Abfolgen implizit als möglich gekennzeichnet werden, geht aus (\ref{ex:3-183}) unmittelbar hervor,
daß DO $>$ IO im Sinne von (\ref{ex:3-172}) eine markierte und IO $>$ DO die normale
Abfolge ist. Eine zusätzliche Auszeichnung der normalen Abfolge ist nicht nötig. Sie ist nicht \textsq{als solche} repräsentiert, sondern (wie es auch ihrer Definition entspricht) durch (\ref{ex:3-183}) für die Oberflächenstruktur charakterisiert.\footnote{\label{fn:3-34}%
	Es ist klar~-- (\ref{ex:3-183}) zeigt es deutlich~--, daß die Beschreibung von  topologischen Regularitäten mittels Filtern einer Beschreibung mittels Umstellungsregeln empirisch nicht äquivalent ist. Nach vorläufigen Untersuchungen ergeben sich bei der Benutzung von Filtern für das Mittelfeld größere Generalisierungen.

  Wir haben bereits gesehen, daß Lenerz' pragmatische Formulierungen
  irreführend sind; nach den Untersuchungen, die er wirklich anstellt,
  wäre (\ref{ex:3-183}) korrekter mit Bezug auf Betonungen wie in (i) statt auf
  \isi{Rhema} (\isi{Fokus}) zu formulieren:
  \ea
  \label{ex:3-fn34i}
  *[+\textit{NP}, -\textit{Personalpronomen}, +\textit{Akkusativ}, +\textit{betont}] $>$ [+\textit{NP}, +\textit{Dativ}, -\textit{betont}]
  \z
  Es wäre eine in mancher Hinsicht interessante (empirisch nicht
  äquivalente) Alternative, Formulierungen wie (\ref{ex:3-183}) wörtlich zu nehmen
  und (a) Fokusregeln auch für unakzeptable Sätze gelten zu lassen, (b)
  Filter unter Bezug auf Foki (statt auf \isi{Betonung}) wie in (ii) zu
  formulieren:
  \ea
  \label{ex:3-fn34ii}
  *[+\textit{NP}, -\textit{Personalpronomen}, +\textit{Akkusativ}, +\textit{Fokusteil}] $>$ [+\textit{NP}, +\textit{Dativ}, -\textit{Fokusteil}]
  \z
  Dazu sind noch keine empirischen Untersuchungen unternommen worden.%
}

Ungeklärte Annahmen über Tiefenstrukturen und \textsq{Normalität}
verbinden sich nicht selten mit Annahmen über typologische
Markiertheit. So könnte man -- unter Verwendung eines Arguments von \citet{Maling70}~-- sagen: Typologisch ist es normal, daß in einer Sprache L\sub{\textit{i}} IO $>$ DO $>$ \isi{Verb} normal ist, wenn in L\sub{\textit{i}} IO $>$ DO normal ist, und daß in L\sub{\textit{j}} \isi{Verb} $>$ DO $>$ IO normal ist, wenn in L\sub{\textit{j}} DO$>$ IO normal ist. Im Deutschen\il{Deutsch} ist (a) IO $>$ DO strukturell normal; (b) unter Voraussetzung von (a) ist es typologisch normal, daß im Deutschen\il{Deutsch} IO $>$ DO $>$ \isi{Verb} strukturell normal ist. (c) Die Abfolge IO $>$ DO $>$ \isi{Verb} kommt im Deutschen\il{Deutsch} tatsächlich vor; da sie aus typologischen Gründen als strukturell normal zu betrachten ist, muß sie (d) auf einer syntaktischen Ebene als solche repräsentiert werden; (e) also haben Tiefenstrukturen für deutsche Sätze die Abfolge IO $>$ DO $>$ \isi{Verb}.

Hier treten zu den allgemeinen Problemen mit Annahmen über tiefenstrukturelle Abfolgen noch die
Probleme der Typologie und Probleme des Zusammenhangs zwischen Typologie und einzelsprachlicher
Grammatik. Wir wollen für unsere Zwecke~-- in starker Idealisierung der tatsächlichen
Verhältnisse~-- annehmen, \textsq{normal} und \textsq{markiert} im typologischen Sinne seien klare
und relevante Begriffe.\footnote{\label{fn:3-35}%
  Tatsächlich ist das schon deshalb nicht der Fall,
  weil typologische Untersuchungen in dem hier interessierenden Bereich in hohem Maße Annahmen
  darüber zu machen pflegen, was in den Einzelsprachen jeweils \textsq{normal}
  bzw.\ \textsq{markiert} ist,~-- Annahmen, deren Basis und Relevanz häufig nicht klar ist. 

  Unabhängig davon sind die grundsätzlichen Probleme, nach welchen Kriterien gegebene sprachliche
  Erscheinungen als typologisch normal oder markiert zu beurteilen sind, wohlbekannt.%
}
Dies könnte wichtig sein, wenn sich z.\,B.\ erweisen würde, daß generell Abfolgen, die nach
(\ref{ex:3-172}) in Einzelsprachen markiert sind, auch typologisch markiert sind, und
umgekehrt. Wenn dies der Fall ist, wäre eine gewisse Relevanz von \textsq{markiert} im Sinne von
(\ref{ex:3-172}) demonstriert. 

Ob es tatsächlich Zusammenhänge dieser Art gibt, können wir hier offen
lassen. Für topologische Zwecke ist wichtig, daß selbst dann, wenn es
sie gibt, nicht evident ist, ob sie in der Grammatik einer \isi{Einzelsprache} repräsentiert werden sollten, und wenn ja, wie. A priori"=Annahmen wie die, daß strukturell normale Abfolgen einer
\isi{Einzelsprache}, wenn sie typologisch normalen Abfolgen entsprechen, als solche von der Basiskomponente einer transformationellen Grammatik ausgezeichnet werden müßten, tragen zur Fortentwicklung einer empirischen Sprachtheorie nichts bei.

\section{Zusammenfassung}
\label{sec:3-4}

Der intuitive Begriff \textsq{Normalbetonung} läßt sich unter
Rückgriff auf den Begriff \textsq{Fokus} adäquat explizieren: Ein Satz
S\sub{\textit{i}} hat stilistisch normale Betonung\is{Normalbetonung} gdw.\ er unter allen Sätzen,
die sich von S\sub{\textit{i}} nur hinsichtlich der Konstituentenbetonung
unterscheiden, die meisten möglichen Foki hat.

Drei Dinge sind dabei wesentlich: (1) Insofern nicht von aktuellen
Foki (in einem gegebenen Kontext), sondern von möglichen Foki die Rede
ist, basiert diese Explikation auf einem satzgrammatischen Konzept
(und nicht auf der Betrachtung von Äußerungen). (2) Indem diese
Explikation auf dem Fokusbegriff aufbaut, basiert sie zugleich auf
dem Begriff \textsq{möglicher Kontexttyp}, \dash auf einem
essentiell pragmatischen Begriff. \textsq{Normalbetonung} ist damit
ein inhärent pragmatisches Konzept. (3) Durch eben diesen Rückgriff
auf pragmatische Zusammenhänge hat die vorgeschlagene Explikation
erklärenden Charakter: Eine \isi{Betonung}, die die relativ meisten
möglichen Foki zuläßt, ist \textsq{normal}, weil sie kontextuell am
wenigsten restringiert ist.

Diese Explikation ist in vielfältiger Weise fruchtbar; u.\,a.\ erklärt
sie die Eigenschaften von Sätzen wie
\begin{exe}
\exi{(65)}{
\label{ex:3-65-2}
John called Bill a Republican, and then \textit{he} insulted \textit{him}.}
\end{exe}
Darauf aufbauend läßt sich der Begriff \textsq{stilistisch normale Wortstellung} explizieren: Ein Satz S\sub{\textit{i}} weist \textsq{stilistisch normale Wortstellung} auf gdw.\ er unter allen Sätzen, die sich von
S\sub{\textit{i}} nur hinsichtlich der \isi{Wortstellung} und/""oder der \isi{Betonung}
unterscheiden, bei geeigneter \isi{Betonung} die meisten möglichen Foki
hat, \dash in den meisten Kontexttypen vorkommen kann. Diese
Explikation erweist sich wiederum als explanatorisch adäquat: Ein
Satz mit einer solchen \isi{Wortstellung} ist \textsq{normal}, \textit{weil} er so
etwas wie ein Universalinstrument darstellt, indem diese \isi{Wortstellung}
kontextuell minimal restringiert ist.

Diese Explikation erklärt u.\,a., wieso bei 2"=deutigen Sätzen wie
\begin{exe}
\exi{(163)}{\exalph{163}
\label{ex:3-163-2}
\begin{xlist}
\exi{a.}{\exalph{163a}
\label{ex:3-163a-2}
die Frau hat ein Mädchen gebissen}
\end{xlist}}
\end{exe}
die Interpretation mit \textit{Mädchen} als Objekt intuitiv leichter zugänglich ist als die andere.

Im Unterschied dazu haben wir eine Definition von \textsq{strukturell normaler Wortstellung} betrachtet und gefunden, daß sie (a) sich tiefgehend von der Explikation von \textsq{stilistisch normaler} \isi{Wortstellung} unterscheidet, (b) in hohem Maß willkürliche Anwendungen
erlaubt und (c) von äußerst fraglicher Relevanz ist.

\sloppy
\printbibliography[heading=subbibliography,notkeyword=this]
\refstepcounter{mylastpagecount}\label{chap-normale-Betonung-end}
\end{document}
