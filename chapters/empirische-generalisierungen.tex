\documentclass[output=paper]{langsci/langscibook}
\author{Tilman N. Höhle}
\title{Empirische Generalisierung vs.\ ,Einfachheit‘. Zur Zuordnung\newlineTOC{} zwischen formalen und logischen Eigenschaften von Sätzen im\newlineTOC{} Deutschen}
\abstract{}
\maketitle

\rohead{\thechapter\hspace{0.5em}Empirische Generalisierung} % Display short title
\ChapterDOI{10.5281/zenodo.1169665}
\begin{document}
\label{chap-empirische-generalisierungen}
\selectlanguage{german}

\renewcommand*{\thefootnote}{\fnsymbol{footnote}}
\setcounter{footnote}{4}

\footnotetext{%
	\emph{Anmerkung der Herausgeber:} Die vorliegende Arbeit erschien erstmals in Clément, Danièle (Hrsg.). 1980. \emph{Empirische rechtfertigung von syntaxen. Beiträge zum Wuppertaler kolloquium vom 25.--29. september 1978}, 61--71. Bonn: Bouvier Verlag Grundmann. Editorische Eingriffe in den Wiederabdruck beschränken sich auf Anpassungen an das einheitliche Bandformat. (Dabei war die im Original als Fn.\,1 geführte Autoren"=Fn.\ als „*“ auszuzeichnen; entsprechend verändert sich die weitere Fn."=Zählung.)%
}

\setcounter{footnote}{0}

\setcounter{section}{-1}
\section[Einleitung]{Einleitung\protect\footnote{Dies ist eine leicht erweiterte Fassung des ersten
    Teils des Vortrags, den ich auf dem Kolloquium \textsq{Empirische Rechtfertigung von Syntaxen}
    gehalten habe. Ich danke den Teilnehmern des Kolloquiums, besonders H."=H.\,Lieb und
    A.\,v.\,Stechow, für ihre Anmerkungen. -- Im zweiten Teil des Vortrags, der hier nicht
    repräsentiert ist, ging es -- in Explikation von \citet[17 Fn.\,1]{Hoehle78a} -- darum, daß es
    im Deutschen wegen der Existenz von Ellipsen wie bei \textit{vorles"=} (cf.\ \textit{er hat
      (ihr) (etwas) vorgelesen} keine allgemeine Zuordnungsfunktion zwischen Konstituenten und
    zugeordneten Argumentstellen geben kann.}
}%0
%\il{Deutsch} geht in Sternchenfußnoten nicht
\label{sec:2-0}

\renewcommand*{\thefootnote}{\arabic{footnote}}
\setcounter{footnote}{0}

Eine Grammatik G\textsubscript{i} einer natürlichen Sprache L\textsubscript{i} soll (mindestens) die Sätze von L\textsubscript{i} aufzählen. Jeder Satz S\textsubscript{i} von L\textsubscript{i} soll durch G\textsubscript{i} eine phonologische Charakterisierung und eine logische Charakterisierung (LC) erhalten. Dabei unterliegt G\textsubscript{i} verschiedenen Adäquatheitsbedingungen; u.\,a.\  soll G\textsubscript{i} \textit{deskriptiv adäquat} sein, \dash die in L\textsubscript{i} vorhandenen Regularitäten als solche zum Ausdruck bringen.

Ich möchte an einigen Beispielen demonstrieren, daß in vielen Fällen Konflikte entstehen, wenn man (i) die Regularitäten der syntaktischen \textsq{Oberflächenstruktur} korrekt repräsentiert und zugleich (ii) die LC eines Satzes auf -- im alltäglichen Sinn des Wortes -- möglichst \textsq{einfache} Weise aus der Oberflächenstruktur gewinnen möchte. Während meines Wissens in allen Grammatiken die LC\textsubscript{i} eines Satzes S\textsubscript{i} relativ zu der syntaktischen Charakterisierung SC\textsubscript{i} formuliert wird und daher jede Grammatik in einem \textit{formalen} Sinn eine \textsq{Syntax} enthält, sind die oberflächenstrukturellen Regularitäten, die ich hier diskutieren will, weitgehend unabhängig von logischen Eigenschaften der Sätze; sie konstituieren dadurch einen Phänomenbereich, der in einem \textit{inhaltlichen} Sinn \textsq{syntaktisch} ist. Mir scheint, daß man über die \textsq{empirische Rechtfertigung von Syntaxen} sinnvoll nur sprechen kann, wenn man diesen inhaltlich gefüllten Sinn von Syntax als eigenständigen Phänomenbereich vor Augen hat. Wenn \textsq{Syntax} in dem rein formalen Sinn im Blick steht, daß LCs relativ zu formalen Strukturen formuliert werden, können Probleme der \textit{empirischen} Rechtfertigung der \textit{Syntax}, soweit ich sehe, nicht entstehen, da es dann immer nur um Fragen der \textsq{Einfachheit} und nicht um Fragen der deskriptiven \isi{Adäquatheit} geht (cf.\ \ref{subsec:2-1-1-2}).

Die Probleme, die ich ansprechen möchte, haben alle mit dem Ansatz der Kategorie "`Satz"' zu tun; und zwar bei \isi{Satzadverbialen}, Equi"=Infinitiven und A.c.I."=Konstruktionen.

\section{Sätze mit Satzadverbialen}%1
\label{sec:2-1}
\subsection{Die \textit{S/""S}-Analyse von Satzadverbialen}%1.0.
\label{subsec:2-1-0}

Wir wollen, in Übereinstimmung mit einem großen Teil der Literatur, annehmen, daß uneingebettete Sätze wie (\ref{ex:2-01a}) und eingebettete Sätze verschiedener Art, etwa der \isi{Relativsatz} in (\ref{ex:2-02a}), vom gleichen Kategorientyp sind, sagen wir \textit{S}. In einer Grammatik, die einem Satz direkt eine strukturelle Repräsentation zuordnet, kann das wie in (\ref{ex:2-01b}) bzw.\ (\ref{ex:2-02b}) repräsentiert werden.
\eal
	\ex \label{ex:2-01a} die Männer schlafen
	\ex \label{ex:2-01b} [\textsubscript{S\textsubscript{1}} die Männer schlafen]
\zl
\eal
	\ex \label{ex:2-02a} die Männer schlafen, die du suchst
	\ex \label{ex:2-02b} [\textsubscript{S\textsubscript{1}} die Männer schlafen [\textsubscript{S\textsubscript{2}} die du suchst]]
\zl
\addlines
Es ist nun weiterhin üblich zu sagen, daß sich Satzadverbiale wie \textit{leider}, \textit{angeblich}, \textit{übrigens}, \textit{bedauerlicherweise} usw.\  dadurch auszeichnen, daß sie sich mit einem \textit{S} zu einem \textit{S} verbinden; da solche \isi{Adverbiale} in einer Kategorialgrammatik dementsprechend die Kategorie \textit{S/""S} bilden, bezeichne ich das allgemein als \textit{S/""S}-Analyse. Ein Satz wie (\ref{ex:2-03a}) hat nach dieser Analyse die Form (\ref{ex:2-03b}). (Von solchen Analysen wird \zb in \citet{Eisenberg75} und \citet{Eisenberg76} reicher Gebrauch gemacht.)
\eal
	\ex \label{ex:2-03a} die Männer schlafen leider
	\ex \label{ex:2-03b} [\textsubscript{S\textsubscript{1}} [\textsubscript{S\textsubscript{2}} die Männer schlafen] leider]
\zl
Die Rechtfertigung für die \textit{S/""S}-Analyse ist semantischer Art: Nach üblicher Auf"|fassung enthält die LC von (\ref{ex:2-03a}) als einen Teil die LC von (\ref{ex:2-01a}), \dash mittels \textit{leider} in (\ref{ex:2-03a}) wird über den Rest des Gesamtsatzes eine Aussage gemacht. Wenn wir die LC von (\ref{ex:2-01a}) durch (\ref{ex:2-04a}) andeuten, ist, nach dieser Auf"|fassung, die LC von (\ref{ex:2-03a}) durch (\ref{ex:2-04b}) anzudeuten.
\eal \label{ex:2-04}
	\ex \label{ex:2-04a} SCHLAF (MÄNNER)
	\ex \label{ex:2-04b} LEIDER (SCHLAF (MÄNNER))
\zl
Es liegt auf der Hand, daß der Übergang von (\ref{ex:2-03a}) zu (\ref{ex:2-04b}) -- \dash die Übersetzungsregel für bzw.\ die LC von Adverbialen\is{Adverbiale} wie \textit{leider} -- besonders \textsq{einfach} zu formulieren ist, wenn (\ref{ex:2-01a}) in (\ref{ex:2-03a}) in derselben Weise enthalten ist, wie (\ref{ex:2-01b}) in (\ref{ex:2-03b}) enthalten ist. Wenn die \textit{S/""S}"=Analyse eine empirische Hypothese über die syntaktische Form von Sätzen mit \isi{Satzadverbialen} sein soll, stehen ihr jedoch mindestens 2 syntaktische Regularitäten entgegen: Verbstellung und \isi{Extraposition}.

\subsection{Anfangsstellung des Finitums}%1.1.
\label{subsec:2-1-1}
\subsubsection{Fakten}%1.1.1.
\label{subsec:2-1-1-1}

Man pflegt hinsichtlich der Stellung des regierenden Verbs im Deutschen\il{Deutsch} (mindestens) 3 Satztypen zu unterscheiden: Erststellung wie in (\ref{ex:2-05}), \textsq{Zweitstellung} wie in (\ref{ex:2-06}) (Erst"= und Zweitstellung kann man als \textsq{Anfangsstellung} zusammenfassen) und \textsq{Endstellung} wie in (\ref{ex:2-07}) (relevante Teil"=Sätze sind hervorgehoben):
\eal \label{ex:2-05}
	\ex \label{ex:2-05a} soll Karl den Hund füttern?
	\ex \label{ex:2-05b} fütter doch bitte den Hund!
	\ex \label{ex:2-05c} Karl soll, \textit{ist er erst mal gesund}, die Hunde füttern
\zl
\eal \label{ex:2-06}
	\ex \label{ex:2-06a} den Hund füttert Karl
	\ex \label{ex:2-06b} deshalb habe ich die These, \textit{Karl verstehe diese Behauptung}, noch nie geglaubt
	\ex \label{ex:2-06c} wenn Berta nach Hause kommt und \textit{der Gerichtsvollzieher steht vor der Tür}, dürfte sie einen ziemlichen Schreck kriegen
\zl
\eal \label{ex:2-07}
	\ex \label{ex:2-07a} ich weiß, \textit{daß Berta den Hund füttern soll}
	\ex \label{ex:2-07b} Karl behauptet, \textit{den Hund füttern zu wollen}
\zl
Mit dem Ausdruck "`Zweitstellung"' ist gemeint, daß in Sätzen wie (\ref{ex:2-06}) vor dem finiten \isi{Verb} (V\textsuperscript{f}) genau eine \isi{Konstituente} steht. Diese Erklärung ist im Allgemeinen unproblematisch und fruchtbar. Auf die Probleme, die mit der Bestimmung "`genau eine \isi{Konstituente}"' zusammenhängen, will ich hier nicht eingehen; sie sind für unsere Zusammenhänge irrelevant. Wichtig ist dagegen, was mit der Formulierung "`vor V\textsuperscript{f}"' gemeint ist. In (\ref{ex:2-06a}) ergibt sich kein Problem; da (\ref{ex:2-06a}) die Form (6a') hat, steht das V\textsuperscript{f} \textit{füttert} in S\textsubscript{1} zweifellos an zweiter Stelle nach der ersten \isi{Konstituente}. Ein Problem ergibt sich jedoch bei Sätzen wie (\ref{ex:2-08a}).
\begin{exe}
	\exi{(6)}
	\begin{xlist}[a'.]
	\exi{a'.}\exalph{6a'}\label{ex:2-06a'} [\textsubscript{S\textsubscript{1}} [\textsubscript{NP} den Hund] [\textsubscript{V\textsuperscript{f}} füttert] [\textsubscript{NP} Karl]] %a' ist noch nicht wirklich hübsch
	\end{xlist}
\end{exe}
\eal
	\ex leider schlafen die Männer \label{ex:2-08a} 
	\ex{[\textsubscript{S\textsubscript{1}} leider [\textsubscript{S\textsubscript{2}} [\textsubscript{V\textsuperscript{f}} schlafen] die Männer]]} \label{ex:2-08b}
\zl 
Nach der \textit{S/""S}"=Analyse hat (\ref{ex:2-08a}) die Form (\ref{ex:2-08b}). Hier steht V\textsuperscript{f} ebenfalls an zweiter Stelle in \textit{S}\textsubscript{1}, jedoch an erster Stelle im eigenen Satz \textit{S}\textsubscript{2}. Heißt "`vor V\textsuperscript{f}"' soviel wie "`innerhalb des Gesamtsatzes (\textit{S}\textsubscript{1}) vor V\textsuperscript{f}"', wie (\ref{ex:2-08b}) es nahelegt? Oder heißt es soviel wie "`innerhalb des minimalen Satzes \textit{S\textsubscript{n}}, in dem V\textsuperscript{f} vorkommt"'? Der minimale \textit{S\textsubscript{n}} in (\ref{ex:2-08b}) wäre \textit{S}\textsubscript{2}; in \textit{S}\textsubscript{2} hat \textit{schlafen} aber die Erststellung inne. In (\ref{ex:2-06b}) andererseits steht das V\textsuperscript{f} \textit{verstehe} nur in \textit{S}\textsubscript{2} nach genau einer \isi{Konstituente} (\textit{Karl}), während es in \textit{S}\textsubscript{1} mindestens 5 Konstituenten vor sich hat. Beim Einbettungstyp (\ref{ex:2-06b}) ist es belanglos für die Wohlgeformtheit von \textit{S}\textsubscript{2}, wieviele Konstituenten in \textit{S}\textsubscript{1} vorausgehen, vgl.\ (\ref{ex:2-09}). Belangvoll ist nur, ob in \textit{S}\textsubscript{2} weniger (\ref{ex:2-10a}) oder mehr (\ref{ex:2-10b}) als eine \isi{Konstituente} vor V\textsuperscript{f} stehen:
\eal \label{ex:2-09}
	\ex \label{ex:2-09a} die These, \textit{Karl verstehe diese Behauptung}, habe ich noch nie geglaubt
	\ex \label{ex:2-09b} ich habe die These, \textit{Karl verstehe diese Behauptung}, noch nie geglaubt
	\ex \label{ex:2-09c} geglaubt habe ich die These, \textit{Karl verstehe diese Behauptung}, noch nie
	\ex \label{ex:2-09d} noch nie habe ich die These, \textit{Karl verstehe diese Behauptung}, geglaubt
\zl
\eal \label{ex:2-10}
	\ex[*]{die These, \textit{verstehe Karl diese Behauptung}, habe ich noch nie geglaubt} \label{ex:2-10a}
	\ex[*]{die These, \textit{Karl diese Behauptung verstehe}, habe ich noch nie geglaubt} \label{ex:2-10b}
\zl 
Daraus muß man schließen, daß für die Interpretation von "`vor V\textsuperscript{f}"' der minimale \textit{S\textsubscript{n}} relevant ist, der V\textsuperscript{f} dominiert.

Dasselbe gilt, wenn der fragliche Satz nicht zu einem \isi{Substantiv} (\textit{These} in (\ref{ex:2-09}), (\ref{ex:2-10})), sondern zu einem \isi{Verb} gehört (cf.\ (\ref{ex:2-11})).
\eal \label{ex:2-11}
	\ex[]{wenn er glaubt, er \textit{versteht diese Behauptung}, irrt er sich}\label{ex:2-11a} 
	\ex[*]{wenn er glaubt, \textit{versteht er diese Behauptung}, irrt er sich} \label{ex:2-11b}
	\ex[*]{wenn er glaubt, \textit{er diese Behauptung versteht}, irrt er sich} \label{ex:2-11c}
\zl 
Ebenso ist bei eingebetteten Sätzen vom Typ (\ref{ex:2-05c}) mit \textsq{Erststellung} des V\textsuperscript{f} nicht \textit{S}\textsubscript{1}, sondern \textit{S\textsubscript{n}} relevant (cf.\ (\ref{ex:2-12})). Wieviele Konstituenten dem V\textsuperscript{f} \textit{hat} in \textit{S}\textsubscript{1} von (\ref{ex:2-12a}) vorausgehen, ist belanglos; entscheidend ist, daß in \textit{S}\textsubscript{2} V\textsuperscript{f} an erster Stelle steht.
\eal \label{ex:2-12}
	\ex[]{gleichwohl soll Karl, \textit{hat der Damm auch bedauerlicherweise der Flut nicht widerstanden}, einen Orden erhalten} \label{ex:2-12a} 
	\ex[*]{gleichwohl soll Karl, \textit{bedauerlicherweise hat der Damm auch der Flut nicht widerstanden}, einen Orden erhalten} \label{ex:2-12b}
\zl 
(\ref{ex:2-12b}) ist offensichtlich genau deshalb unakzeptabel, weil diese Regularität nicht beachtet ist. Dies ist jedoch mit der \textit{S/""S}"=Analyse von \textit{bedauerlicherweise} nicht vereinbar, denn nach ihr stände \textit{hat} in (\ref{ex:2-12b}) an erster Stelle eines \textit{S}\textsubscript{3}. Wie (\ref{ex:2-12b}) nach der \textit{S/""S}-Analyse akzeptabel sein sollte, wäre auch zu erwarten, daß (\ref{ex:2-13a}) akzeptabel ist, da das V\textsuperscript{f} \textit{schlafen} an zweiter Stelle in \textit{S}\textsubscript{2} stände; statt dessen ist (\ref{ex:2-13b}) akzeptabel, wo V\textsuperscript{f} an erster Stelle steht, wenn \textit{bedauerlicherweise} zur Kategorie \textit{S/""S} gehört.
\eal
	\ex[*]{bedauerlicherweise die Männer schlafen} \label{ex:2-13a}
	\ex[]{bedauerlicherweise schlafen die Männer}\label{ex:2-13b} 
\zl 
Wenn die \textit{S/""S}-Analyse andererseits behauptet, daß für die Stellung des V\textsubscript{f} nicht der eigene \textit{S\textsubscript{n}}, sondern der Gesamtsatz \textit{S}\textsubscript{1} ausschlaggebend ist, hätte sie zu klären, wie der Akzeptabilitätsunterschied zwischen (\ref{ex:2-12a}) und (\ref{ex:2-12b}) zustande kommt und wieso in (\ref{ex:2-09})--(\ref{ex:2-11}) genau dieselben Verhältnisse wie in uneingebetteten Sätzen bestehen, cf.\ (\ref{ex:2-14}).
\eal \label{ex:2-14}
	\ex Karl versteht diese Behauptung \label{ex:2-14a}
	\ex[*]{versteht Karl diese Behauptung} \label{ex:2-14b}
	\ex[*]{Karl diese Behauptung versteht} \label{ex:2-14c}
\zl
((\ref{ex:2-14b}) ist als Aussage wie (\ref{ex:2-14a}) gemeint, nicht als Frage.) Dies scheint aussichtslos.

Aus diesen Beobachtungen ergibt sich die in (\ref{reg15}) formulierte Regularität:
\begin{exe}
	\ex \label{reg15} Für Sätze mit Erst- oder Zweitstellung von V\textsuperscript{f} ist relevant, wieviele Konstituenten vor V\textsuperscript{f} im eigenen Satz \textit{S\textsubscript{n}} stehen. Wieviele Konstituenten innerhalb des Gesamtsatzes \textit{S}\textsubscript{1} vor V\textsuperscript{f} stehen, ist gleichgültig.
\end{exe}

\subsubsection{Folgerung}%1.1.2.
\label{subsec:2-1-1-2}

Da die \textit{S/""S}-Analyse mit diesen Fakten nicht verträglich ist, muß sie fallengelassen
werden. Da Satzadverbiale sich hinsichtlich der Stellung des V\textsuperscript{f} genauso verhalten
wie \textit{bona fide}"=Satzgenossen des V\textsuperscript{f} -- etwa Objekte und Subjekte -- sind
sie als Satzgenossen von V\textsuperscript{f} zu analysieren; dies sei (willkürlich) als
\textsq{\textit{ADV}"=Analyse} bezeichnet.\footnote{%
	Der Fall eines Verstoßes gegen die in
  (\ref{reg15}) angenommene Zweitstellungsregularität, der gewissermaßen komplementär zur
  \textit{S/""S}-Analyse ist, findet sich in \citet[131]{Eisenberg76}. Dort bilden koordinierte NPs
  wie \textit{Karl und Kurt} in \textit{Karl und Kurt sind müde} nicht eine \isi{Konstituente}, sondern
  vor dem V\textsuperscript{f} \textit{sind} sollen hier 3 Konstituenten stehen. Auch hier kann für
  den Verstoß gegen die Regularität allenfalls der (irrelevante) Grund angeführt werden, daß unter
  dieser Analyse die Zuordnung zwischen SC und LC des Satzes -- in dem gewählten Übersetzungssystem!
  -- besonders \textsq{einfach} ist.%
}

Es ist wichtig zu erkennen, daß die Begründungen für die beiden konkurrierenden Analysen völlig verschiedenen Status haben. Die in der \textit{ADV}"=Analyse postulierte Konstituentenstruktur stützt sich auf eine empirische \isi{Generalisierung}, die eine Aussage über den formalen Auf"|bau von Sätzen macht. Die in der \textit{S/""S}"=Analyse postulierte Struktur dagegen hat mit empirischen Generalisierungen, gar über die Form von Sätzen, nicht das geringste zu tun; ihre einzige Begründung ist der Wunsch, die Zuordnung zwischen der logischen und der syntaktischen Charakterisierung der Sätze möglichst \textsq{einfach} zu formulieren. Davon abgesehen, daß der untechnische Einfachheitsbegriff höchst problematisch und möglicherweise gar nicht in nützlicher Weise explizierbar ist, kann er nur beim Vergleich mehrerer Theorien angewendet werden, die empirisch -- \dash vor allem auch: hinsichtlich ihrer deskriptiven \isi{Adäquatheit} -- gleichwertig sind. Das ist bei der \textit{S/""S}"=Analyse gegenüber der \textit{ADV}"=Analyse offensichtlich nicht der Fall.

Ganz allgemein ist damit zu rechnen, daß die Zuordnung zwischen LC und SC beliebig komplex sein
kann, da der Ansatz der SC auf empirischen Generalisierungen vom Typ (\ref{reg15}) aufbaut, während
die Form der LC -- die im übrigen weitgehend beliebig eingerichtet werden kann -- für den Auf"|bau der
SC allenfalls Einfachheitsargumente anbieten kann (die im Zweifelsfall irrelevant sind).\footnote{%
	In
  \citet[158--160]{Eisenberg76} wird \zb \textit{zweihundert Meter} in \textit{der Zug ist
    zweihundert Meter lang} als eine \isi{Konstituente} analysiert, die nicht als ganze übersetzt wird;
  vielmehr entsprechen ihre Bestandteile völlig verschiedenen Teilen der LC.%
}

\subsection{Extraposition}%1.2.
\label{subsec:2-1-2}

Ein Phänomen, das für die Analyse von \isi{Satzadverbialen} wie für die Analyse von Equi"= und A.c.I."=Konstruktionen besonders interessant ist, ist die sog. \isi{Extraposition}. Damit ist die Erscheinung gemeint, daß ein Element \textit{E\textsubscript{i}} -- besonders ein Satz --, das in einem gewissen Sinn zu einem Satz \textit{S\textsubscript{j}} oder zu einem Element \textit{E\textsubscript{j}} von \textit{S\textsubscript{j}} gehört, im \isi{Nachfeld} von \textit{S\textsubscript{j}} steht.

Relativsätze etwa stehen typischerweise entweder am Ende der \isi{NP}, auf die sie sich \textsq{beziehen} (cf.\ (\ref{ex:2-16a})), oder am Ende (= im \isi{Nachfeld}) des Satzes, in dem sich ihre Bezugs-\isi{NP} befindet (cf.\ (\ref{ex:2-16b})), nicht aber etwa, dissoziiert von der Bezugs"=\isi{NP}, im Mittelfeld (cf.\ (\ref{ex:2-16c})).
\eal
	\ex[]{[\textsubscript{S\textsubscript{1}} die Männer [\textsubscript{S\textsubscript{2}} \textit{von denen du gesprochen hast}] haben geschlafen]} \label{ex:2-16a}
	\ex[]{[\textsubscript{S\textsubscript{1}} die Männer haben geschlafen [\textsubscript{S\textsubscript{2}} \textit{von denen du gesprochen hast}]]} \label{ex:2-16b}
	\ex[*]{[\textsubscript{S\textsubscript{1}} die Männer haben [\textsubscript{S\textsubscript{2}} \textit{von denen du gesprochen hast}] geschlafen]} \label{ex:2-16c}
\zl
Es ist bekannt, daß die \isi{Extraposition} (in den klaren Fällen) \textit{rightward bounded} ist, \dash das extraponierte Element \textit{E\textsubscript{i}} kann nur am Ende des Satzes \textit{S\textsubscript{j}} stehen, \textsq{in den es gehört}, und nicht ans Ende eines übergeordneten Satzes \textit{S\textsubscript{j-m}} treten (vgl.\ (\ref{ex:2-17a}) vs.\  (\ref{ex:2-17b}) und (\ref{ex:2-18a}) vs.\  (\ref{ex:2-18b})):
\eal \label{ex:2-17}
	\ex[]{[\textsubscript{S\textsubscript{1}}[\textsubscript{S\textsubscript{2}} daß die Männer schlafen[\textsubscript{S\textsubscript{3}} \textit{von denen du gesprochen hast}]] ist bekannt]} \label{ex:2-17a}
	\ex[*]{[\textsubscript{S\textsubscript{1}}[\textsubscript{S\textsubscript{2}} daß die Männer schlafen] ist bekannt [\textsubscript{S\textsubscript{3}} \textit{von denen du gesprochen hast}]]} \label{ex:2-17b}
\zl
\eal \label{ex:2-18}
	\ex[]{[\textsubscript{S\textsubscript{1}} mir ist [\textsubscript{S\textsubscript{2}} daß die Männer schlafen [\textsubscript{S\textsubscript{3}} \textit{von denen du gesprochen hast}] bekannt]]} \label{ex:2-18a}
	\ex[*]{[\textsubscript{S\textsubscript{1}} mir ist [\textsubscript{S\textsubscript{2}} daß die Männer schlafen] bekannt [\textsubscript{S\textsubscript{3}} \textit{von denen du gesprochen hast}]]} \label{ex:2-18b}
\zl 
Dies ist wiederum, wie die Stellungsregularitäten für regierende Verben, ein Phänomen, das in einem inhaltlichen Sinn genuin syntaktischer Art ist, da es auf logische (oder morphologische, phonologische,  \ldots{}) Eigenschaften der Sätze nicht zurückgeführt werden kann. Das heißt nicht, daß die Extrapositionsregularität durch eine formal ausgezeichnete \textsq{Syntaxregel} beschrieben werden muß; sie könnte auch durch stellungs"= und struktursensitive Übersetzungsregeln erfaßt werden.%\small?

Das Extrapositionsphänomen hat unmittelbare Auswirkungen auf die Analyse von \isi{Satzadverbialen}. Nach der \textit{S/""S}-Analyse hat (\ref{ex:2-19a}) die Form (\ref{ex:2-19b});
\eal
	\ex die Männer schlafen leider \label{ex:2-19a}
	\ex {[\textsubscript{S\textsubscript{1}} [\textsubscript{S\textsubscript{2}} die Männer schlafen] leider]} \label{ex:2-19b}
\zl
ein auf \textit{die Männer} bezogener \isi{Relativsatz} sollte daher am Ende von \textit{S}\textsubscript{2} stehen können und am Ende von \textit{S}\textsubscript{1} unmöglich sein. Die Daten sind jedoch genau umgekehrt: (\ref{ex:2-20a}), das nach der \textit{S/""S}-Analyse die Form (\ref{ex:2-20b}) hat, ist unmöglich, während (\ref{ex:2-21a}), trotz der behaupteten Form (\ref{ex:2-21b}), akzeptabel ist:
\eal
	\ex[*] {die Männer schlafen, \textit{von denen du gesprochen hast}, leider} \label{ex:2-20a}
	\ex[]{[\textsubscript{S\textsubscript{1}} [\textsubscript{S\textsubscript{2}} die Männer schlafen [\textsubscript{S\textsubscript{3}} \textit{von denen du gesprochen hast}]] leider]} \label{ex:2-20b}
\zl 
\eal
	\ex die Männer schlafen leider, \textit{von denen du gesprochen hast} \label{ex:2-21a}
	\ex {[\textsubscript{S\textsubscript{1}} [\textsubscript{S\textsubscript{2}} die Männer schlafen] leider [\textsubscript{S\textsubscript{3}} \textit{von denen du gesprochen hast}]]} \label{ex:2-21b}
\zl 
Die Verteilung der Akzeptabilitätsurteile in (\ref{ex:2-20a}) vs.\  (\ref{ex:2-21a}) ist hingegen genau das, was unter der \textit{ADV}"=Analyse von \isi{Satzadverbialen} zu erwarten ist.

Für alle Grammatiktypen, die syntaktische Regularitäten wie die \isi{Extraposition} mittels der \textsq{oberflächenstrukturellen} Form der Sätze zu erfassen versuchen, entstehen unter der \textit{S/""S}"=Analyse kaum lösbare Probleme. Ich sehe nicht, wie auf motivierte Weise (\ref{ex:2-20a}) vs.\ (\ref{ex:2-21a}) erfaßt werden kann, ohne daß zugleich die Fakten von (\ref{ex:2-17})--(\ref{ex:2-18}) unerklärbar werden.

Bei Grammatiken in der Montague-Tradition ist die Lage anders. Während eine Konstituentenstruktur, die einem Satz \textit{S\textsubscript{i}} zugeordnet wird, etwas über die \textsq{Form} von \textit{S\textsubscript{i}} aussagen soll und syntaktische Regularitäten relativ zu dieser Form formuliert werden, ist ein Montaguescher Analysebaum zu \textit{S\textsubscript{i}} lediglich eine Darstellung der Derivationsgeschichte von \textit{S\textsubscript{i}}, einschließlich der Reihenfolge, in der die -- formal weitgehend unrestringierten -- Regeln der Syntax angewendet werden, um \textit{S\textsubscript{i}} zu generieren. In einem derart flexiblen System ist es kein prinzipielles Problem, eine Regel R\textsubscript{\textit{i}} zu formulieren, die einen \textit{S} wie (\ref{ex:2-22}) mit dem \textit{S/""S leider} derart kombiniert, daß der \textit{S} (\ref{ex:2-23}) resultiert (mündlicher Vorschlag von v.\ Stechow):
\ea die Männer schlafen, von denen du gesprochen hast \label{ex:2-22}
\z 
\ea die Männer schlafen leider, von denen du gesprochen hast \label{ex:2-23}
\z
Allerdings müßte R\textsubscript{\textit{i}} über ziemlich komplexe Informationen hinsichtlich der Abgrenzung von Mittel- und \isi{Nachfeld} verfügen. Dafür sind verschiedene Möglichkeiten denkbar, aber soweit ich sehe, ist das nur dadurch auf relativ generelle Weise zu erreichen, daß quasi"=transformationelle Regeln in die Derivation aufgenommen werden.

Weitgehend unklar ist mir, ob eine Grammatik vom Typ \citet{Eisenberg76}, \citet{Eisenberg75} die diskutierten Fakten mit einer \textit{S/""S}-Analyse vereinbaren könnte. Zum einen scheint mir, daß sie zu den Grammatiktypen gehört, die syntaktische Regularitäten -- mindestens u.\,a.\  -- mittels der oberflächensyntaktischen Form der Sätze zu erfassen versuchen, und die Diskussionsbeiträge von Lieb zu diesem Vortrag haben diesen Eindruck bestätigt. Unter dieser Voraussetzung halte ich die vorgetragenen Gründe gegen die \textit{S/""S}"=Analyse für zwingend. Andererseits findet sich in der gesamten mir bekannten Literatur in der Tradition von \citet{Eisenberg76}, \citet{Eisenberg75} keine einzige explizit formulierte Syntaxregel, und an keiner Stelle wird erklärt, welcher Zusammenhang zwischen Konstituentenstrukturen, Syntaxregeln und syntaktischen Regularitäten besteht. Es ist daher, streng genommen, ganz ungewiß, wie innerhalb dieser Tradition überhaupt empirische Gründe für oder gegen den Ansatz gewisser Konstituentenstrukturen sinnvoll diskutiert werden können.

\section{Equi-Konstruktionen}%2
\label{sec:2-2}

Es ist weithin üblich, die Infinitivkonstruktionen (IKs) in (\ref{ex:2-24a}) und (\ref{ex:2-25a}) in gleicher Weise als \textit{S} zu analysieren, so daß sie die Strukturen (\ref{ex:2-24b}) bzw.\ (\ref{ex:2-25b}) erhalten.
\eal \label{ex:2-24}
	\ex Karl versuchte, \textit{Heinz zu helfen} \label{ex:2-24a}
	\ex {[\textsubscript{S\textsubscript{1}} Karl versuchte [\textsubscript{S\textsubscript{2}} \textit{Heinz zu helfen}]]} \label{ex:2-24b}
\zl
\eal \label{ex:2-25}
	\ex Karl wollte \textit{Heinz helfen} \label{ex:2-25a}
	\ex {[\textsubscript{S\textsubscript{1}} Karl wollte [\textsubscript{S\textsubscript{2}} \textit{Heinz helfen}]]} \label{ex:2-25b}
\zl
Die IK in (\ref{ex:2-24a}) ist ein \textsq{satzwertiger Infinitiv}, der, jedenfalls in dieser Konstruktion, mit dem regierenden \isi{Verb} \textit{versuchte} kein \isi{Kohärenzfeld} (cf.\ \citealt{Bech1955}) bildet; für die meisten Zwecke ist es durchaus sinnvoll, ihr die Kategorie \textit{S} zuzuweisen. Die IK in (\ref{ex:2-25b}) dagegen bildet mit \textit{wollte} ein \isi{Kohärenzfeld}. Im folgenden wende ich mich dagegen, (\ref{ex:2-25b}) für (\ref{ex:2-25a}) anzusetzen. Zuvor muß aber erklärt werden, was für (\ref{ex:2-25b}) sprechen könnte.

Der wesentliche Grund scheint semantischer Art zu sein: \textit{woll"=} wird logisch gewöhnlich als eine Relation zwischen einem Gegenstand und einer Proposition repräsentiert, derart daß etwa die LC von (\ref{ex:2-25a}) die LC von (\ref{ex:2-26a}) als Teil enthält. Wenn wir -- von \isi{Tempus} usw.\  abgesehen -- (\ref{ex:2-26b}) als Übersetzung von (\ref{ex:2-26a}) betrachten, wäre (\ref{ex:2-25a}) daher etwa als (25a') zu übersetzen:
\eal
	\ex Karl hilft Heinz \label{ex:2-26a}
	\ex HELF (KARL, HEINZ) \label{ex:2-26b}
\zl
\begin{exe}
	\exi{(25)}
	\begin{xlist}[a'.]
		\exi{a'.} \label{ex:2-25a'} WOLL (KARL, HELF (KARL, HEINZ))
	\end{xlist}
\end{exe}
Darüber hinaus muß sichergestellt werden, daß \textit{woll"=} im Sinn von \textit{wünsch"=} mit einem \isi{Verb} im einfachen \isi{Infinitiv} (V\textsuperscript{ei}) nur dann vorkommt, wenn das V\textsuperscript{ei} ein semantisch nichtleeres \isi{Subjekt} selegiert (cf.\ (\ref{ex:2-27}) vs.\ (\ref{ex:2-29})). (Das gleiche gilt für das \textsq{inferentielle} \textit{woll"=} im Sinn von \textit{angeb"=}, \textit{behaupt"=}, das sich obligatorisch mit einem V\textsuperscript{ei} verbindet; cf.\ (\ref{ex:2-28}) vs.\ (\ref{ex:2-29}).)
\eal \label{ex:2-27}
	\ex[*]{Karl will ihr übel werden} \label{ex:2-27a}
	\ex[*]{Karl will (es) regnen} \label{ex:2-27b}
\zl
\eal \label{ex:2-28}
	\ex[]{Karl will einen Marsmenschen kennen} \label{ex:2-28a}
	\ex[]{Karl will krank geworden sein} \label{ex:2-28b}
	\ex[*]{Karl will ihr übel geworden sein} \label{ex:2-28c}
	\ex[*]{Karl will (es) geregnet haben} \label{ex:2-28d}
\zl
\eal \label{ex:2-29}
	\ex ihr wurde übel \label{ex:2-29a}
	\ex es hat geregnet \label{ex:2-29b}
\zl
Unter diesen Voraussetzungen ist es verständlich, daß in transformationalistischen wie in primär logisch orientierten Grammatiken Strukturen wie (\ref{ex:2-25b}) angesetzt werden: Wenn man das latente \isi{Subjekt} von \textit{S}\textsubscript{2}, das als identisch mit dem \isi{Subjekt} von \textit{S}\textsubscript{1} gelten soll, durch eine Equi-NP-Tilgung oder durch einen äquivalenten interpretativen Mechanismus erfaßt (wodurch, unter gewissen Voraussetzungen, zugleich die in
(\ref{ex:2-27})--(\ref{ex:2-28}) gezeigten Fakten erfaßt werden können), ist \textit{S}\textsubscript{2} in (\ref{ex:2-25b}) in relevanter Hinsicht identisch mit (\ref{ex:2-26a}), und die Übersetzung (25a') läßt sich auf sehr \textsq{einfache} Weise aus (\ref{ex:2-25b}) gewinnen.

Gegen den Ansatz eines eingebetteten \textit{S} in (\ref{ex:2-25a}) spricht jedoch, daß dies eine
kohärente \isi{Infinitivkonstruktion} (cf.\ \citealt{Bech1955}) ist. Als eines unter mehreren Indizien
soll uns hier die Beobachtung der \isi{Extraposition} genügen. Wenn wir den \isi{Relativsatz} in
(\ref{ex:2-30a}) extraponieren, erhalten wir (\ref{ex:2-30c}) als akzeptables Resultat. Der Satz
(\ref{ex:2-30b}), der nach der Analyse (\ref{ex:2-25b}) zu erwarten wäre, ist
unakzeptabel. (\ref{ex:2-30c}) dagegen hätte gemäß (\ref{ex:2-25b}) die Form (\ref{ex:2-30d}), die
ein Verstoß gegen die Extrapositionsregularität wäre. Mit der Regularität vereinbar ist die Struktur
(\ref{ex:2-30e}), so daß auch (\ref{ex:2-25a}) als einfacher Satz ohne eingebetteten \textit{S} wie
in (\ref{ex:2-30e}) zu analysieren ist.\footnote{%
	Der Unterschied zwischen satzwertigen IKs wie in
  (\ref{ex:2-24}) und kohärenten IKs kann nicht darauf zurückgeführt werden, daß in (\ref{ex:2-24})
  ein \isi{Verb} im \isi{Infinitiv} mit \textit{zu} (V\textsuperscript{zi}) vorliegt und in (\ref{ex:2-25}) ein
  V\textsuperscript{ei}. Denn kohärente IKs, die sich syntaktisch in keiner relevanten Weise von der
  Konstruktion bei \textit{woll-} unterscheiden, finden sich auch mit V\textsuperscript{zi},
  \zb bei \textit{pfleg-}, \textit{droh-}, \textit{schein-}, \textit{hab-}, \textit{bleib-} und
  besonders bei \textit{wiss-} (\textit{Karl weiß eine gute Suppe zu kochen}), das wie
  \textit{woll-} subjektselegierend ist und vermutlich als logisch (mindestens) 2-stellig analysiert
  werden muß.%
}
\eal \label{ex:2-30}
	\ex[]{daß Karl dem Jungen, \textit{von dem wir sprachen}, helfen will, \ldots{}} \label{ex:2-30a}
	\ex[*]{[\textsubscript{S\textsubscript{1}} daß Karl [\textsubscript{S\textsubscript{2}} dem Jungen helfen [\textsubscript{S\textsubscript{3}} \textit{von dem wir sprachen}]] will] \ldots{}} \label{ex:2-30b}
	\ex[]{daß Karl dem Jungen helfen will, \textit{von dem wir sprachen}, \ldots{}} \label{ex:2-30c}
	\ex[]{[\textsubscript{S\textsubscript{1}} daß Karl [\textsubscript{S\textsubscript{2}} dem Jungen helfen] will [\textsubscript{S\textsubscript{3}} \textit{von dem wir sprachen}]] \ldots{}} \label{ex:2-30d}
	\ex[]{[\textsubscript{S\textsubscript{1}} daß Karl dem Jungen helfen will [\textsubscript{S\textsubscript{2}} \textit{von dem wir sprachen}]] \ldots{}} \label{ex:2-30e}
\zl 
\citet{Evers75} hat das hier vorliegende Problem bemerkt und dadurch gelöst, daß er auf eine Struktur wie (\ref{ex:2-25b}) (i) eine \textsq{\isi{Verb}"=Raising}"=Regel anwendet, deren Effekt u.\,a.\ (ii) ein \textsq{Pruning} von \textit{S}\textsubscript{2} ist. Davon abgesehen, daß (ii) völlig willkürlich ist, bestehen Probleme, die in \citet[86f.]{Hoehle78a} genannt sind.

Lieb (mündliche Mitteilung) akzeptiert Strukturen wie (\ref{ex:2-25b}), schlägt jedoch vor, daß eingebettete Sätze wie \textit{S}\textsubscript{2} in (\ref{ex:2-25b}), (\ref{ex:2-30d}) in Abhängigkeit von Matrixverben wie \textit{woll-} hinsichtlich der \isi{Extraposition} \textsq{durchlässig} sind. Auf den ersten Blick ist dieser Vorschlag attraktiv, da er formale Ähnlichkeit mit den wohlmotivierten \textsq{bridge conditions} von \citet{Erteschik73} zu haben scheint, nach denen unter gewissen semantisch/""pragmatischen Bedingungen Extraktionen aus Sätzen zulässig sind, die sonst unmöglich sind. Dagegen ist einzuwenden, daß die \textsq{bridge conditions} von Erteschik-Shir inhaltlich wohldefiniert sind, während Liebs \textsq{Durchlässigkeitsbedingungen} für Konstruktionen mit \textit{woll-} und A.c.I."=Konstruktionen (cf.\ Abschnitt~\ref{sec:2-3}) gelten müßten, die sich als Klasse von satzwertigen IKs wie in (\ref{ex:2-24b}) nicht in unabhängig relevanter Weise unterscheiden; zum anderen erklärt dieser Vorschlag nicht, warum (\ref{ex:2-30d}) unmöglich ist. Im übrigen gibt es eine Reihe weiterer Unterschiede zwischen satzwertigen IKs und kohärenten Konstruktionen (cf.\ \citealt{Bech1955, Evers75}), die durch diesen Vorschlag in keiner Weise abgedeckt werden, aber alle unter der Annahme, daß es sich um einfache Sätze handelt, eine Erklärung finden.

Manche Probleme dieser Konstruktionen lassen sich im \textsq{Conditions"=framework} (cf.\ \citealt{Chomsky77, Chomsky78, Koster78}) leicht behandeln; der Unterschied zwischen kohärenten und inkohärenten Infinitkonstruktionen ist dort aber nicht ohne weiteres rekonstruierbar.

Wie die distributionellen und logischen Fakten auch bei einer Oberflächenstruktur wie (\ref{ex:2-30e}) ohne Rückgriff auf \textsq{Tiefenstrukturen} beschrieben werden können, ist in \citet[84f.\ und 173ff.]{Hoehle78a} dargestellt.

\section{A.c.I.-Konstruktionen}%3
\label{sec:2-3}

\largerpage
Bei A.c.I."=Konstruktionen läßt sich dasselbe Argumentationsmuster wie bei kohärenten Equi"=Konstruktionen anwenden; wir können die Diskussion deshalb kurz halten.

Nehmen wir an, daß Sätze wie (\ref{ex:2-31}) im einfachsten Fall wie (\ref{ex:2-32}) zu übersetzen sind. Im Interesse einer möglichst \textsq{einfachen} Übersetzung möchte man (\ref{ex:2-31}) daher die Struktur (\ref{ex:2-33}) geben (so \zb \citealt[123]{Eisenberg75}, \citealt{Chomsky77, Chomsky78}).
\eal \label{ex:2-31}
	\ex daß Karl es zu dem Streit kommen sah, \ldots{} \label{ex:2-31a}
	\ex daß Karl ihr schlecht werden sah, \ldots{} \label{ex:2-31b}
\zl
\eal \label{ex:2-32}
	\ex SEH (KARL, INGRESS (STREIT)) \label{ex:2-32a}
	\ex SEH (KARL, SCHLECHT--WERD (SIE)) \label{ex:2-32b}
\zl
\eal \label{ex:2-33}
	\ex {[\textsubscript{S\textsubscript{1}} daß Karl [\textsubscript{S\textsubscript{2}} es zu dem Streit kommen] sah]} \label{ex:2-33a}
	\ex {[\textsubscript{S\textsubscript{1}} daß Karl [\textsubscript{S\textsubscript{2}} ihr schlecht werden] sah]} \label{ex:2-33b}
\zl
Wiederum sprechen jedoch u.\,a.\ die Extrapositionsfakten gegen diese Strukturierung. Wenn wir den \isi{Relativsatz} in (\ref{ex:2-34a}) extraponieren, erhalten wir nicht, wie nach (\ref{ex:2-33}) anzunehmen, (\ref{ex:2-34b}), sondern (\ref{ex:2-34c}), was nur mit einer einfachen Satzstruktur wie (\ref{ex:2-34d}) zu vereinbaren ist.
\eal \label{ex:2-34}
	\ex[]{daß er die Steine, \textit{die er suchte}, dort liegen sah, \ldots{}} \label{ex:2-34a}
	\ex[*] {[\textsubscript{S\textsubscript{1}} daß er \label{ex:2-34b} [\textsubscript{S\textsubscript{2}} die Steine dort liegen [\textsubscript{S\textsubscript{3}} \textit{die er suchte}]] sah]}
	\ex[]{daß er die Steine dort liegen sah, \textit{die er suchte}, \ldots{}} \label{ex:2-34c}
	\ex[]{[\textsubscript{S\textsubscript{1}} daß er die Steine dort liegen sah [\textsubscript{S\textsubscript{2}} \textit{die er suchte}]]} \label{ex:2-34d}
\zl
Zur weiteren Demonstration, daß A.c.I."=Konstruktionen weitgehend nicht die Eigenschaften eingebetteter \textit{S} haben, cf.\ \citet{Bech1955, Reis76, Evers75}. Allerdings haben solche Konstruktionen eine Reihe überraschender Eigentümlichkeiten, zu deren Beschreibung und Verständnis jedoch -- entgegen den Voraussetzungen von \citet{Chomsky78, Chomsky77} -- ihre Zuweisung zu einer Kategorie \textit{S} in keiner Weise hilfreich ist (cf.\ \citealt[53--61]{Hoehle78a}). -- Zur Beschreibung und Übersetzung von A.c.I."=Konstruktionen auf der Grundlage einfacher Satzstrukturen wie (\ref{ex:2-34d}) cf.\ \citet[83f.]{Hoehle78a}.%\small?

\section{Zusammenfassung}%4
\label{sec:2-4}

Wir haben 3 Phänomenbereiche betrachtet, in denen die formalen und die logischen Eigenschaften von Sätzen gewissermaßen auseinanderfallen. In dreifacher Hinsicht scheint mir die Diskussion solcher Phänomene wichtig.

(i) Sie zeigen, daß Sätze mindestens zum Teil formale Eigenschaften haben, die aus ihren logischen (oder sonstigen) Eigenschaften nicht deduzierbar sind; gegenüber \textsq{Syntax} in dem trivialen Sinn, daß die LC\textsubscript{i} eines Satzes S\textsubscript{i} in irgendeiner Weise auf eine formale Charakterisierung SC\textsubscript{i} Bezug nehmen muß, konstituiert sich dadurch ein Bereich \textsq{autonomer} Syntax.

(ii) Die wichtige Unterscheidung zwischen dem zentralen Begriff der sprachlichen
Regularität einerseits -- und den damit zusammenhängenden Begriffen \textsq{empirische Generalisierung} und \textsq{deskriptive Adäquatheit} -- und dem empirisch uninteressanten intuitiven Begriff der \textsq{Einfachheit} andererseits tritt bei der Diskussion solcher Fälle,
wie mir scheint, besonders deutlich hervor.

(iii) Es ist signifikant, daß die Generalisierungen, um die es hier ging, im wesentlichen solche der \isi{Topologie}, \dash der \textsq{Wortstellungs}"=Regularitäten sind. Insofern sich \textsq{autonome} Syntax besonders innerhalb der \isi{Topologie} manifestiert, scheint\label{page-besonderes-theoretisches-interesse} mir dieses Teilgebiet der Grammatik ganz besonders theoretisches Interesse zu verdienen.
%
\sloppy
\printbibliography[heading=subbibliography,notkeyword=this]
\refstepcounter{mylastpagecount}\label{chap-empirische-generalisierungen-end}
\end{document}

