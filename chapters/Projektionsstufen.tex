\documentclass[output=paper]{LSP/langsci}
\author{Tilman N. Höhle}
\title{Projektionsstufen bei V-Projektionen: Bemerkungen zu Frey/Tappe\newlineTOC{}  (1991)}

%\epigram{Change epigram in chapters/01.tex or remove it there}
\abstract{}
\maketitle
\rohead{\thechapter\hspace{0.5em}Projektionsstufen bei V-Projektionen} % Display short title
\ChapterDOI{10.5281/zenodo.1169679}
\begin{document}
\label{chap-projektionsstufen}
\selectlanguage{german}

\renewcommand*{\thefootnote}{\fnsymbol{footnote}}
\setcounter{footnote}{4}

\footnotetext{%
	\emph{Anmerkung der Herausgeber:} In dieser bisher unveröffentlichten Arbeit (vom 03.04.1991) kommentiert Höhle ein Papier von Werner Frey und Hans-Thilo Tappe (vom 30.01.1991), das als Grundlage eines -- im Rahmen des SFB 340 ,,Sprachtheoretische Grundlagen für die Computerlinguistik'', 1989ff.\ zu entwickelnden -- GB"=Fragments gedacht war. Die kommentierte 1991er Version des letzteren Papiers ist nach Auskunft der Verfasser nicht mehr verfügbar; eine überarbeitete Version (vom September 1992) befindet sich auf der Homepage von Werner Frey (\url{http://www.zas.gwz-berlin.de/fileadmin/mitarbeiter/frey/SyntaxMF.pdf}, letzter Zugriff: 06.\,Februar 2018).

Der hier vorliegende Abdruck ist an das Standardformat dieses Bandes angepasst; textuelle Eingriffe beschränken sich auf die Spezifizierung der S.\,\pageref{proj-eckig1}, \pageref{proj-eckig2} nur durch Namen angedeuteten (mutmaßlichen) Literaturverweise (in eckigen Klammern), die dem Literaturverzeichnis hinzugefügt sind. Zu den hier verwendeten Begriffen (S"=Feld, E"=Satz, F"=Satz, etc.) vgl.\ \textit{Topologische Felder}, in diesem Band Kapitel~\ref{chap-topo}, S.\,\pageref{subsec:1-1.1}--\pageref{chap-topo-end}) und \textit{Der Begriff ,Mittelfeld'} (Kapitel~\ref{chap-mittelfeld} in diesem Band, S.\,\pageref{chap-mittelfeld}--\pageref{chap-mittelfeld-end}).}

\renewcommand*{\thefootnote}{\arabic{footnote}}
\setcounter{footnote}{0}

\setcounter{section}{-1}

\section{F/Ts Vorschläge}%0.
\label{sec:5-0}

\largerpage
Tappe und Frey schlagen in F[rey]/""T[appe] (\citeyear{FreyTappe1991}) vor, daß sich Objekte, \isi{Adverbiale} und \isi{Subjekt} im S-Feld von E-Sätzen des Deutschen\il{Deutsch} und des Niederländischen\il{Niederländisch} im Allgemeinen in einer \textsq{basisgenerierten \isi{Adjunkt}"=Position} befinden. Wenn es Phrasenstrukturregeln geben würde, könnte man diese Struktur durch eine \isi{rekursiv} anzuwendende Regel wie (\ref{ex:5-1}) beschreiben:
\begin{exe}
\ex%1
\label{ex:5-1}
V\textsuperscript{i} $\rightarrow$ X\textsuperscript{max}  V\textsuperscript{i} wobei X\textsuperscript{max} Objekt, Adverbial oder \isi{Subjekt} sein kann
\end{exe}
Diese Annahme hat nützliche Konsequenzen, und ich will sie als korrekt betrachten. Nach F/T ist in (\ref{ex:5-1}) i = 2. Diese Annahme halte ich für zweifelhaft; vgl.\ \sectgref{sec:5-2}.

Tappe und Frey schlagen weiter vor, daß es im S-Feld Verbprojektionen gibt, die nicht durch (\ref{ex:5-1}) beschrieben werden können, sondern durch eine~-- nicht \isi{rekursiv} anzuwendende~-- Regel wie (\ref{ex:5-2}) zu erfassen sind:

\ea%2
\label{ex:5-2}
V\textsuperscript{1} ${\rightarrow}$ (Y\textsuperscript{max})  V\textsuperscript{e}, \\
wobei Y\textsuperscript{max} andere Charakteristika als die X\textsuperscript{max} von (\ref{ex:5-1}) hat
\z
Diese Annahme will ich hier nicht problematisieren. Nach F/""T ist V\textsuperscript{e} = V\textsuperscript{0}. Das halte ich für zweifelhaft.


\section{V\textsuperscript{e} ${\neq}$ V\textsuperscript{0}}%1.
\label{sec:5-1}

Am Ende des S-Felds finden sich Kollokationen von \isi{VZ} (\textsq{Verbzusatz}, Konverb) und \isi{Verb}. Diese können (im Ndl.) oder müssen (im Dt.) eine \isi{Konstituente} bilden, die massiv verschieden von dem V\textsuperscript{1} in (\ref{ex:5-2}) zu sein scheint. Man kann sie sinnvoll durch V\textsuperscript{e} in (\ref{ex:5-2}) repräsentieren. V\textsuperscript{e} hat zwar z.\,T.\ ähnliche Eigenschaften wie ein Wort (genauer: wie ein Kompositum); \zb kann V\textsuperscript{e} an der Verbinversion im Ndl.\ teilnehmen, vgl.\ (\ref{ex:5-3a}). Aber V\textsuperscript{e} kann nicht mit V\textsuperscript{0} identifiziert werden, wenn man unter V\textsuperscript{0} ein Wort wie \zb \textit{ruft} in (\ref{ex:5-4a}) versteht; denn sonst wäre (\ref{ex:5-4b}) statt (\ref{ex:5-4a}) zu erwarten, und (\ref{ex:5-3b}) sollte unmöglich sein.

\begin{exe}
\ex%3
\label{ex:5-3}
\begin{xlist}
\ex%3a
\label{ex:5-3a}
dat hij me zal opbellen

\ex%3b
\label{ex:5-3b}
dat hij me op zal bellen
\end{xlist}

\ex%4
\label{ex:5-4}
\begin{xlist}
\ex[]{%4a
\label{ex:5-4a}
er ruft mich an}

\ex[*]{%4b
\label{ex:5-4b}
er anruft mich}
\end{xlist}
\end{exe}
Generell sind am Ende des S-Felds~-- also an einer Stelle, von der eine V"=\isi{Projektion} ausgeht~-- derartige wortähnliche V\textsuperscript{e}{}"=Konstituenten möglich, während sie in der FINIT"=Position von F"=Sätzen~-- also in der Landeposition der \textsq{Verbbewegung} – nicht möglich sind. Ganz deutlich ist das bei den sog.\ \textsq{verbalen Pseudokomposita} wie \textit{urauf"|führ}{}-, \textit{zwischenfinanzier}{}-, \textit{wettruder}{}-, \textit{rückfrag}{}- usw. Die kommen (bei beträchtlichen idiolektalen Bewertungsunterschieden im Einzelfall) finit zwar z.\,T.\ in E"=Sätzen vor, nicht aber in F"=Sätzen:

\begin{exe}
\ex%5
\label{ex:5-5}
\begin{xlist}
\ex%5a
\label{ex:5-5a}
wenn die das Stück urauf"|führen

\ex%5b
\label{ex:5-5b}
wenn wir euch den Bau zwischenfinanzieren

\ex%5c
\label{ex:5-5c}
wenn die morgen wettrudern

\ex%5d
\label{ex:5-5d}
falls die dann rückfragen
\end{xlist}

\ex%6
\label{ex:5-6}
\begin{xlisti}
\ex%6i
\label{ex:5-6i}
\begin{xlista}
\ex[*]{%6ia
\label{ex:5-6ia}
die urauf"|führen das Stück}
\ex[*]{%6ib
\label{ex:5-6ib}
wir zwischenfinanzieren euch den Bau}
\ex[*]{%6ic
\label{ex:5-6ic}
die wettrudern morgen}
\ex[*]{%6id
\label{ex:5-6id}
die rückfragen dann}
\end{xlista}
\ex%6ii
\label{ex:5-6ii}
\begin{xlista}
\ex[*]{%6iia
\label{ex:5-6iia}
die führen das Stück urauf}
\ex[*]{%6iib
\label{ex:5-6iib}
wir finanzieren euch den Bau zwischen}
\ex[*]{%6iic
\label{ex:5-6iic}
die rudern morgen wett}
\ex[*]{%6iid
\label{ex:5-6iid}
die fragen dann rück}
\end{xlista}
\ex[]{%6iii
\label{ex:5-6iii}
[\textsubscript{V\textsuperscript{k}} urauf"|führen]\textsubscript{i} werden die das Stück wohl kaum t\textsubscript{i}}
\end{xlisti}
\end{exe}
Daraus ergibt sich nebenbei, daß die~-- beliebte, aber durch nichts begründete~-- Annahme (\ref{ex:5-7}) in Widersprüche führt:

\begin{exe}
\ex%7
\label{ex:5-7}
In E-Sätzen ist das finite \isi{Verb} mit einer I\textsuperscript{0}{}"=Position assoziiert und bindet von dort aus eine V\textsuperscript{0}{}"=Spur in einer V-\isi{Projektion}.
\end{exe}
Wenn die finiten Verben von (\ref{ex:5-5}) eine Spur von der Kategorie V\textsuperscript{0} binden könnten (nach dem Motto: Der Verbstamm bewegt sich nach I\textsuperscript{0}, um dort mit den Finitheitsmerkmalen assoziiert zu werden), wäre nicht zu verstehen, wieso sowohl (\ref{ex:5-6i}) als auch (\ref{ex:5-6ii}) von vielen Sprechern abgelehnt wird, die (\ref{ex:5-5}) akzeptieren. Dabei können die \textsq{Pseudokomposita} durchaus eine Spur binden, solange die von höherer \isi{Projektionsstufe} ist wie in (\ref{ex:5-6iii}), wo 0 {\textless} k ist. Wenn dagegen \textit{urauf"|führen} usw.\ in (\ref{ex:5-5}) die V\textsuperscript{e}{}"=Position einnehmen, hat man einen Ausgangspunkt, von dem aus man die Datenkonfiguration deuten kann.

Besonders auf"|fällig sind \textsq{Pseudokomposita} wie \textit{bauspar}{}- und \textit{bauchred}{}-. Viele Sprecher akzeptieren die finit in E-Sätzen (\ref{ex:5-8}), aber nur mit größtem Zögern in F-Sätzen wie (\ref{ex:5-9i}) und überhaupt nicht in solchen wie (\ref{ex:5-9ii}):

\begin{exe}
\ex%8
\label{ex:5-8}
\begin{xlist}
\ex%8a
\label{ex:5-8a}
wieviele Leute in Zukunft noch bausparen
\ex%8b
\label{ex:5-8b}
daß da mal wieder etliche Leute bauchreden
\end{xlist}


\ex%9
\label{ex:5-9}
\begin{xlisti}
\ex%9i
\label{ex:5-9i}
\begin{xlista}
\ex[??]{%9ia
\label{ex:5-9ia}
wieviele Leute bausparen in Zukunft noch?}
\ex[??]{%9ib
\label{ex:5-9ib}
da bauchreden mal wieder etliche Leute}
\end{xlista}
\ex%9ii
\label{ex:5-9ii}
\begin{xlista}
\ex[**]{%9iia
\label{ex:5-9iia}
wieviele Leute sparen in Zukunft noch bau?}
\ex[**]{%9iib
\label{ex:5-9iib}
da reden mal wieder etliche Leute bauch}
\end{xlista}
\end{xlisti}
\end{exe}
Der Kontrast (\ref{ex:5-9i}) vs.\ (\ref{ex:5-9ii}) zeigt, daß eine Zerlegung dieser
Pseudokomposita überhaupt nicht in Frage kommt. Daß viele Sprecher
trotzdem (\ref{ex:5-9i}) weitaus weniger leicht als (\ref{ex:5-8}) akzeptieren, kann man
wieder so deuten, daß \textit{bausparen}/""\textit{bauchreden} in (\ref{ex:5-8})
eine V\textsuperscript{e}{}"=Position einnehmen, gegen deren
Struktureigenschaften sie nicht verstoßen, während sie in (\ref{ex:5-9i}) eine
V\textsuperscript{0}{}"=Position einnehmen würden (und dabei gegen die
Struktureigenschaften von V\textsuperscript{0} verstoßen). Daraus
ergibt sich nebenbei, daß die~-- unter gewissen Annahmen denkbar
erscheinende~-- Hypothese (\ref{ex:5-10}) nicht korrekt sein kann:

\begin{exe}
\ex%10
\label{ex:5-10}
Einfache Ketten der Form \textsq{\isi{NP}~-- finites Verb} bilden in E"=Sätzen die gleiche V"=\isi{Projektion} wie in F2"=Sätzen.
\end{exe}
Denn viele Sprecher unterscheiden zwischen einem indirekten \isi{Interrogativsatz} wie (\ref{ex:5-11a}) mit \textit{bausparen} als V\textsuperscript{e} und dem entsprechenden direkten W"=\isi{Interrogativsatz} (\ref{ex:5-11b}) mit \textit{bausparen} als V\textsuperscript{0}:

\begin{exe}
\ex%11
\label{ex:5-11}
\begin{xlist}
\ex[]{%11a
\label{ex:5-11a}
(es ist egal) wieviele Leute bausparen}
\ex[??]{%11b
\label{ex:5-11b}
wieviele Leute bausparen?}
\end{xlist}
\end{exe}
Demgemäß ist \textit{schnarcht} in (\ref{ex:5-12a}) ein V\textsuperscript{e}, von dem eine V-\isi{Projektion} ausgeht, während es in (\ref{ex:5-12b}) ein V\textsuperscript{0} ist, von dem keine V-\isi{Projektion} ausgeht:

\begin{exe}
\ex%12
\label{ex:5-12}
\begin{xlist}
\ex%12a
\label{ex:5-12a}
daß Karl schnarcht
\ex%12b
\label{ex:5-12b}
Karl schnarcht
\end{xlist}
\end{exe}
Zusammengefaßt ergibt sich (\ref{ex:5-13}):

\begin{exe}
\ex%13
\label{ex:5-13}
\begin{xlisti}
\ex%13i
\label{ex:5-13i}
Höherstufige V"=Projektionen haben V\textsuperscript{e} als Kopf, wobei V\textsuperscript{e} ${\neq}$ V\textsuperscript{0}.
\ex%13ii
\label{ex:5-13ii}
Es gibt keine \textsq{Bewegung} von V\textsuperscript{0} nach I\textsuperscript{0} in E"=Sätzen.
\ex%13iii
\label{ex:5-13iii}
Von dem finiten \isi{Verb} V\textsuperscript{0} in F"=Sätzen geht keine V"=\isi{Projektion} aus.
\end{xlisti}
\end{exe}
Wie die Struktureigenschaften von V\textsuperscript{e} und
V\textsuperscript{0} (besonders im Blick auf die \textsq{Pseudokomposita})
genau zu formulieren sind, liegt nicht auf der Hand. Es dürfte
vorteilhaft sein, diese Frage nicht jetzt im Entwurf eines \textsq{Fragments}
eilig lösen zu wollen. Aber bei allen inhaltlichen Erwägungen über
V"=Projektionen muß man den Unterschied zwischen V\textsuperscript{e}
und V\textsuperscript{0} im Auge behalten.

\section{V\textsuperscript{i} = V\textsuperscript{1}}%2.
\label{sec:5-2}

\ssubsection{}%2.1.
\label{subsec:5-2-1}
Tappe und Frey nehmen an, daß die durch (\ref{ex:5-1}) beschriebenen \textsq{Adjunktstrukturen} V"=Projektionen der Stufe 2 sind. In \citet{FreyTappe1991} finden sich zwei Aussagen als Begründung für diese Annahme. Die erste ist in (\ref{ex:5-14}) zitiert:

\begin{exe}
\ex%14
\label{ex:5-14}
"`Zu diesen [sc.\ etablierten Vorstellungen] zählen wir u.\,a.\ die Annahme [\ldots] des Ausschlusses der X\textsuperscript{1}{}"=Rekursion."' \citep[3]{FreyTappe1991}
\end{exe}
Es ist in empirischer wie in theoretischer Hinsicht unklar, worauf sich diese Annahme stützt. Rekursion über X\textsuperscript{1} ist mindestens für den Bereich der \isi{NP} die Standardannahme in der Literatur, und weder in \citet{FreyTappe1991} noch sonstwo wird deutlich gemacht, was gegen diese Annahme sprechen würde. Bis zum Beweis des Gegenteils halte ich (\ref{ex:5-14}) für willkürlich.

Die zweite Begründung ist in (\ref{ex:5-15}) zitiert:

\begin{exe}
\ex%15
\label{ex:5-15}
"`Wir weisen weiter darauf hin, daß es die V\textsuperscript{2}{}"=Adjunktionsidee erlaubt, hinsichtlich der Vorfeldbesetzung deren Beschränkung auf X\textsuperscript{2}{}"=Konstituenten mit der Möglichkeit [\ldots] der Vorfeldbesetzung durch eine V"=\isi{Projektion} bei gleichzeitiger Strandung gewissen V\textsuperscript{2}{}"=internen Materials zu verbinden, weil sie die notwendigen V\textsuperscript{2}{}"=Konstituenten bereitstellen kann [\ldots]."' \citep[5]{FreyTappe1991}
\end{exe}
Diese Überlegung (die ursprünglich~-- in etwas anderer Form~-- von Thiersch\label{proj-eckig1} [=~\citet{Thiersch1985}] stammt und später von den Besten und Webelhuth [=~\citet{WdB87a}] populär gemacht worden ist) hat die Struktur (\ref{ex:5-16}):


\begin{exe}
\ex%16
\label{ex:5-16}
\begin{xlisti}
\ex%16i
\label{ex:5-16i}
Infinite mehrstellige Verben kommen im Vorfeld mit beliebigem
Sättigungsgrad (\dash alleine, mit 1 Objekt, mit 2 Objekten oder
eventuell auch mit \isi{Subjekt} und Objekten) vor.
\ex%16ii
\label{ex:5-16ii}
Das Vorfeld ist eine Landestelle für Bewegungen des klassischen Typs.
\ex%16iii
\label{ex:5-16iii}
Bei Bewegungen des klassischen Typs können nur maximale Projektionen involviert sein.
\ex%16iv
\label{ex:5-16iv}
Also sind sämtliche V"=Projektionen im Vorfeld V\textsuperscript{max}.
\ex%16v
\label{ex:5-16v}
Maximale Projektionen sind X\textsuperscript{2}.
\ex%16vi
\label{ex:5-16vi}
Also sind auch entsprechende Ketten im S-Feld sämtlich V\textsuperscript{2}.
\end{xlisti}
\end{exe}
Die Annahmen (\ref{ex:5-16i}) und (\ref{ex:5-16ii}) sind
unstrittig. Annahme (\ref{ex:5-16iii}) will ich hier nicht in Zweifel
ziehen, mithin will ich auch (\ref{ex:5-16iv}) akzeptieren. Der
Begriff der \textsq{maximalen Projektion} enthält jedoch eine
Unklarheit. Bevor wir darauf und auf die kritische Annahme (\ref{ex:5-16v})
eingehen, ist noch eine Bemerkung zu (\ref{ex:5-16vi}) zu machen.

\ssubsection{}%2.2.
\label{subsec:5-2-2}
Nach den bisherigen Überlegungen muß man annehmen, daß in Beispielen wie (\ref{ex:5-17}) jeweils α = V\textsuperscript{max} ist. Es ist völlig plausibel, anzunehmen, daß in den E"=Sätzen von (\ref{ex:5-18}) bei den entsprechenden Ketten mit finitem \isi{Verb} ebenfalls jeweils β = V\textsuperscript{max} ist.

\begin{exe}
\ex%17
\label{ex:5-17}
\begin{xlist}
\ex%17a
\label{ex:5-17a}
[\textsubscript{α} gezeigt] hat Karl den Kindern einen Brief
\ex%17b
\label{ex:5-17b}
 [\textsubscript{α} einen Brief gezeigt] hat Karl den Kindern
\ex%17c
\label{ex:5-17c}
[\textsubscript{α} den Kindern einen Brief gezeigt] hat Karl
\end{xlist}
\ex%18
\label{ex:5-18}
\begin{xlist}
\ex%18a
\label{ex:5-18a}
daß Karl den Kindern einen Brief [\textsubscript{β} zeigt]
\ex%18b
\label{ex:5-18b}
daß Karl den Kindern [\textsubscript{β} einen Brief zeigt]
\ex%18c
\label{ex:5-18c}
daß Karl [\textsubscript{β} den Kindern einen Brief zeigt]
\end{xlist}
\end{exe}
(Entsprechend \zb in (\ref{ex:5-6iii}) und (\ref{ex:5-5a}).) Eben solche Strukturen werden von (\ref{ex:5-1}) und (\ref{ex:5-2}) zugelassen. Dabei setzt die Annahme von F/T, daß in (\ref{ex:5-1}) i = 2 ist, natürlich eine zusätzliche Strukturanweisung wie (\ref{ex:5-19}) voraus:

\begin{exe}
\ex%19
\label{ex:5-19}
V\textsuperscript{2} ${\rightarrow}$ V\textsuperscript{1}
\end{exe}
Diese Annahmen lassen jedoch offen, wie die Struktur des S-Felds bei (\ref{ex:5-17}) ist. Man kann jeweils 2 Spuren annehmen; es ist aber u.\,a.\ nicht klar, ob der Aufbau wie in (\ref{ex:5-20i}) oder wie in (\ref{ex:5-20ii}) ist (und es gibt weitere denkbare Alternativen):
\begin{exe}
\ex%20
\label{ex:5-20}
\begin{xlisti}
\ex%20i
\label{ex:5-20i}
\begin{xlista}
\ex%20ia
\label{ex:5-20ia}
[\textsubscript{α} gezeigt]\textsubscript{i}
hat\textsubscript{j} [\textsubscript{α}
Karl [\textsubscript{α} den Kindern
[\textsubscript{α} einen Brief
t\textsubscript{i}]]] t\textsubscript{j}
\ex%20ib
\label{ex:5-20ib}
[\textsubscript{α} einen Brief
  gezeigt]\textsubscript{i} hat\textsubscript{j}
[\textsubscript{α} Karl [\textsubscript{α}
den Kindern t\textsubscript{i}]]
t\textsubscript{j}
\ex%20ic
\label{ex:5-20ic}
[\textsubscript{α} den Kindern einen Brief
  gezeigt]\textsubscript{i} hat\textsubscript{j}
[\textsubscript{α} Karl t\textsubscript{i}]
t\textsubscript{j}
\end{xlista}

\ex%20ii
\label{ex:5-20ii}
\begin{xlista}
\ex%20iia
{\label{ex:5-20iia}
[\textsubscript{α} gezeigt]\textsubscript{i}
hat\textsubscript{j} [\textsubscript{α}
Karl [\textsubscript{α} den Kindern
[\textsubscript{α} einen Brief
[\textsubscript{γ} t\textsubscript{i}
t\textsubscript{j}]]]]}

\ex%20iib
\label{ex:5-20iib}
[\textsubscript{α} einen Brief
  gezeigt]\textsubscript{i} hat\textsubscript{j}
[\textsubscript{α} Karl [\textsubscript{α}
den Kindern [\textsubscript{γ}
t\textsubscript{i} t\textsubscript{j}]]]

\ex%20iic
\label{ex:5-20iic}
[\textsubscript{α} den Kindern einen Brief
  gezeigt]\textsubscript{i} hat\textsubscript{j}
[\textsubscript{α} Karl [\textsubscript{γ}
t\textsubscript{i} t\textsubscript{j}]]
\end{xlista}
\end{xlisti}
\end{exe}
Konstituenten wie γ in (\ref{ex:5-20ii}a) werden \zb für das Vorfeld von (\ref{ex:5-21}) benötigt:

\begin{exe}
\ex%21
\label{ex:5-21}
[\textsubscript{α} gezeigt haben] soll Karl den Kindern einen Brief
\end{exe}
Dementsprechend ist es nicht klar, wie die Struktur eines E-Satzes wie (\ref{ex:5-22}) ist:
\begin{exe}
\ex%22
\label{ex:5-22}
daß Karl den Kindern einen Brief gezeigt hat
\ex%23
\label{ex:5-23}
\begin{xlist}
\ex%23a
\label{ex:5-23a}
daß [Karl [den Kindern [einen Brief [gezeigt hat]]]]
\ex%23b
\label{ex:5-23b}
 daß [Karl [den Kindern [einen Brief gezeigt]]] hat
\ex%23c
\label{ex:5-23c}
daß [Karl [den Kindern [[einen Brief gezeigt] hat]]]
\end{xlist}
\end{exe}
Alle Strukturen in (\ref{ex:5-23}) (und mehr) kommen in Frage, wenn man sich in
erster Linie an der Vorfeldbesetzung orientiert.

%\largerpage[2]
Es ist keineswegs sicher, daß die strukturellen Ambiguitäten von
(\ref{ex:5-20}i,ii) und (\ref{ex:5-23}) wirklich existieren. Im Ndl.\ hat man Fälle wie
(\ref{ex:5-24b}):
\begin{exe}
\ex%24
\label{ex:5-24}
\begin{xlist}
\ex%24a
\label{ex:5-24a}
hij zou het boek gelezen kunnen hebben
\ex%24b
\label{ex:5-24b}
 [\textsubscript{δ} gelezen kunnen hebben] zou hij het boek
\end{xlist}
\end{exe}
In einem entsprechenden E-Satz kann die \isi{Konstituente} δ nicht
realisiert werden:
\begin{exe}
\judgewidth{?*}
\ex%25
\label{ex:5-25}
\begin{xlist}
\ex[*]{%25a
\label{ex:5-25a}
dat hij het boek [\textsubscript{δ} gelezen kunnen hebben] zou}
\ex[?*]{%25b
\label{ex:5-25b}
dat hij het boek zou [\textsubscript{δ} gelezen kunnen hebben]}
\end{xlist}
\end{exe}
Das Pendant zu (\ref{ex:5-24}) ist vielmehr (\ref{ex:5-26}):
\begin{exe}
\ex%26
\label{ex:5-26}
dat hij het boek gelezen zou kunnen hebben
\end{exe}
Es gibt also Fälle, wo eine Spur nicht durch ihr Antezedens ersetzt
werden kann. Offenbar muß, wenn die Verben wie in (\ref{ex:5-25}) und
(\ref{ex:5-26}) in der Endposition versammelt sind, ein
\textsq{Verbkomplex} gebildet werden (eine \isi{Konstituente} mit ähnlichen
Eigenschaften wie V\textsuperscript{e}). Im Ndl.\ sind dabei
z.\,T.\ bestimmte Inversionen der Verben obligatorisch; deshalb ist
(\ref{ex:5-25}) nicht möglich. Im Standarddeutschen\il{Deutsch} gibt es derartige
Inversionen nicht. Trotzdem könnte auch hier die Verbkomplexbildung
obligatorisch sein. Falls sie es ist, scheiden Strukturen wie (23b,c)
und (\ref{ex:5-20i}) aus. Insofern darf von der Annahme
(\ref{ex:5-16vi}) nur eingeschränkt und mit Vorsicht Gebrauch gemacht
werden.

\ssubsection{}%2.3.
\label{subsec:5-2-3}
Die Annahme (\ref{ex:5-16iii}) lautet: Bei Bewegungen des klassischen Typs können nur maximale Projektionen (X\textsuperscript{max}) involviert sein. Aber was sind maximale Projektionen?

Nach der klassischen Interpretation beinhaltet (\ref{ex:5-16iii}) vermutlich auf jeden Fall (\ref{ex:5-27}):

\begin{exe}
\ex%27
\label{ex:5-27}
Die Spur einer Bewegung des klassischen Typs kann nicht den Kopf einer \isi{Projektion} bilden.
\end{exe}
%\largerpage[2]\enlargethispage{3pt}
Die Vorfeldbesetzung im Deutschen\il{Deutsch} erfüllt diese Bedingung in der Tat, wenn man die Strukturen (\ref{ex:5-20ii}) annimmt. Die Spur \textit{t\textsubscript{i}} bildet dort jeweils nicht den Kopf einer \isi{Projektion}; vielmehr werden die nicht gesättigten Teile des Thetarasters von \textit{t\textsubscript{i}} durch Externalisierung (Funktionale Komposition) an γ weitergegeben, wobei γ eine \isi{Projektion} von \textit{t\textsubscript{j}} ist. In entsprechender Weise bildet \textit{gezeigt }in der Struktur (\ref{ex:5-23a}) nicht den Kopf einer \isi{Projektion}, sondern das \isi{Thetaraster} von \textit{gezeigt} geht durch Externalisierung an die erste \isi{Projektionsstufe} von \textit{hat} über. Für diese Überlegung ist es gleichgültig, welche \isi{Projektionsstufe} man für \textit{t\textsubscript{i}} bzw.\ für \textit{gezeigt} annimmt. 

\pagebreak
In vielen älteren Untersuchungen herrscht die Annahme (\ref{ex:5-28}):
\begin{exe}
\ex%28
\label{ex:5-28}
Eine \isi{Konstituente} von der Kategorie X\textsuperscript{j} in \isi{Basisposition} ist nur dann nicht Kopf einer \isi{Projektion}, wenn j = 2 ist.
\end{exe}
Wenn (\ref{ex:5-28}) richtig ist, müssen die Spuren \textit{t\textsubscript{i}} von α und mithin auch α selbst in (\ref{ex:5-20ii}) V\textsuperscript{2} sein, und damit ist kompatibel, daß die β in (\ref{ex:5-18}) ebenfalls V\textsuperscript{2} sind. Aber die Forderung (\ref{ex:5-28}) ist ohne Berücksichtigung von Externalisierungsvorgängen aufgestellt worden, wie sie im Verbkomplex stattfinden, und allgemein ist die Frage, wie (\ref{ex:5-28}) begründet ist. Es gibt sofort Probleme, wenn man die gängige Annahme (\ref{ex:5-29}) hinzufügt:

\begin{exe}
\ex%29
\label{ex:5-29}
Eine \isi{Konstituente} von der Kategorie X\textsuperscript{2} (eine \textsq{maximale Projektion}) kann einen Spezifikator enthalten, \dash eine \isi{Konstituente}, die Tochter von X\textsuperscript{2} und Schwester von X\textsuperscript{1} ist.
\end{exe}
Es ist nie gelungen, für die klassische \isi{VP} des Englischen\il{Englisch} einen Spezifikator im Sinne von (\ref{ex:5-29}) zu identifizieren. (Eine Chance für einen Spezifikator hat man nur dann, wenn man die \isi{Basisposition} des Subjekts in der V"=\isi{Projektion} lokalisiert. Ob dieser Spezifikator dann nützliche Dienste tut, ist offen.) Das ist einer der Gründe dafür, daß Fukui\label{proj-eckig2} [=~\citet{Fukui1986}] (\ref{ex:5-28}) aufgegeben hat und Spezifikatoren (und damit Projektionen der Stufe 2) nur in gewissen besonderen Fällen annimmt. Dann kann man folgenden Wortgebrauch einführen:

\begin{exe}
\ex%30
\label{ex:5-30}
\begin{xlisti}
\ex%30i
\label{ex:5-30i}
Eine \isi{Konstituente} K ist in der Struktur S eine \textsq{maximale Projektion}
gdw.\ K von der Kategorie X\textsuperscript{j} ist und in S nicht Kopf
einer \isi{Projektion} von X\textsuperscript{j} ist.
\ex%30ii
\label{ex:5-30ii}
Eine Kategorie X\textsuperscript{j} ist in der Sprache
L\textsubscript{i} ein \textsq{maximaler Projektionstyp}
gdw.\ X\textsuperscript{j} in L\textsubscript{i} niemals als Kopf einer
\isi{Projektion} X\textsuperscript{k} mit k {\textgreater} j auftritt.
\end{xlisti}
\end{exe}
Den vermutlich intendierten vollen Gehalt von (\ref{ex:5-16iii}) kann man dann durch (\ref{ex:5-31}) verdeutlichen:
\begin{exe}
\ex%31
\label{ex:5-31}
\begin{xlisti}
\ex%31i
\label{ex:5-31i}
Die Spur einer Bewegung des klassischen Typs muß eine maximale
\isi{Projektion} im Sinne von (\ref{ex:5-30i}) sein. (${\approx}$ (\ref{ex:5-27}))
\ex%31ii
\label{ex:5-31ii}
Eine solche Spur muß zugleich von maximalem Projektionstyp
im Sinne von (\ref{ex:5-30ii}) sein.
\end{xlisti}
\end{exe}
Auf diesem Hintergrund kann man ohne Probleme annehmen, daß im
Deutschen\il{Deutsch} V\textsuperscript{1} ein maximaler Projektionstyp ist. In
(\ref{ex:5-1}) ist dann i = 1; Regel (\ref{ex:5-19}) ist überflüssig. Die α in (\ref{ex:5-17})
und die β in (\ref{ex:5-18}) sind dann V\textsuperscript{1}. Man beachte:
Wenn man zusätzlich zu einer rekursiven Strukturanweisung (\ref{ex:5-32i}) noch
eine Strukturanweisung (\ref{ex:5-32ii}) einführen würde, würde (\ref{ex:5-17}) gegen (\ref{ex:5-31ii})
verstoßen.

\begin{exe}
\ex%32
\label{ex:5-32}
\begin{xlisti}
\ex%32i
\label{ex:5-32i}
V\textsuperscript{1} ${\rightarrow}$
X\textsuperscript{max} V\textsuperscript{1}, wobei
X\textsuperscript{max} Objekt oder Adverbial sein kann
\ex%32ii
\label{ex:5-32ii}
V\textsuperscript{2} ${\rightarrow}$
X\textsuperscript{max} V\textsuperscript{1}, wobei
X\textsuperscript{max} \isi{Subjekt} ist
\end{xlisti}
\end{exe}

\section{Noch einmal: V\textsuperscript{1} ${\rightarrow}$ (Y\textsuperscript{max}) V\textsuperscript{e}}%3.
\label{sec:5-3}

Frey und Tappe erörtern, daß infinite Verben nicht immer allein im Vorfeld stehen können:

\begin{exe}
\ex%33
\label{ex:5-33}
\begin{xlist}
\ex[*]{%33a
\label{ex:5-33a}
gemacht hat Paul die Kassiererin wütend \citep[8]{FreyTappe1991}
}
\ex[]{%33b
\label{ex:5-33b}
wütend gemacht hat Paul die Kassiererin}
\end{xlist}
\end{exe}
Sie schlagen vor, \textit{wütend} hier zu den Y\textsuperscript{max}
von (\ref{ex:5-2}) zu rechnen. Da \textit{gemacht} dann kein V\textsuperscript{1}
(und schon gar nicht ein V\textsuperscript{2}) darstellt, verstößt
(\ref{ex:5-33a}) gegen F/""Ts Annahme, daß das Vorfeld von einer
X\textsuperscript{2}{}"=\isi{Konstituente} gefüllt sein muß. Man sieht
jedoch, daß diese Annahme überflüssig ist: Wenn \textit{gemacht} eine
Spur von der Kategorie V\textsuperscript{e} bindet, verstößt (\ref{ex:5-33a}) auf
jeden Fall gegen (\ref{ex:5-31ii}). (\ref{ex:5-33b}) dagegen ist voll kompatibel mit (\ref{ex:5-31}).

Tappe und Frey machen die weitere Annahme, daß die
Y\textsuperscript{max} von (\ref{ex:5-2}) in dem Sinn \textsq{ortsfest} sind, daß sie
nicht \textsq{gescrambelt} werden können. (Diese Annahme modifizieren sie auf
S.\,10f.\ etwas. Auf die dort diskutierten Daten gehe ich nicht ein.) Es
ist nicht ganz deutlich, wie sie dieser empirischen Annahme formal
Rechnung zu tragen gedenken. Man könnte folgendes Prinzip postulieren:

\begin{exe}
\ex%34
\label{ex:5-34}
Die Landeposition einer \textsq{Scrambling"=Bewegung} ist konfigurationell nicht unterscheidbar von der Ausgangsposition der Bewegung.
\end{exe}
Nach dieser Annahme kann ein X\textsuperscript{max} von (\ref{ex:5-1}) gescrambelt werden, da sein Landeplatz eine Adjunktion an V\textsuperscript{i} sein müßte; sowohl die Ausgangsposition als auch die Landeposition ist dann nach F/T eine Tochter von V\textsuperscript{2}. Wenn ein Y\textsuperscript{max} von (\ref{ex:5-2}) gescrambelt würde, wäre dagegen die Ausgangsposition eine Tochter von V\textsuperscript{1} und die Landeposition eine Tochter von V\textsuperscript{2}.

Möglicherweise~-- ganz klar ist das nicht~-- soll man die Äußerung (\ref{ex:5-35}) im Sinn von (\ref{ex:5-34}) verstehen:

\begin{exe}
\ex%35
\label{ex:5-35}
"`Gleichzeitig verfügen diese Sprachen [sc.\ das Englische\il{Englisch} und die skandinavischen\il{Skandinavisch} Sprachen] nicht über Grundpositionen des deutschen Typs (sind also keine \textit{scrambling}{}"=Sprachen)."'  \citep[5f.]{FreyTappe1991}
\end{exe}
Aus (\ref{ex:5-34}) ergibt sich ja, daß eine Sprache nur dann Scrambling haben kann, wenn sie über Grundpositionen gemäß (\ref{ex:5-1}) verfügt. Allerdings kann man die rekursiven V\textsuperscript{i}{}"=Strukturen in Übereinstimmung mit (\ref{ex:5-34}) durchaus als V\textsuperscript{1}{}"=Strukturen analysieren: Die Grundposition von Y\textsuperscript{max} ist nach (\ref{ex:5-2}) eine Schwester von V\textsuperscript{e}; die Landeposition bei Scrambling wäre eine Schwester von V\textsuperscript{1}.

\section{X\textsuperscript{2} als abgeschlossene Kategorie}%4.
\label{sec:5-4}

Offenbar spricht nichts dafür, rekursive V\textsuperscript{2}{}"=Strukturen im Deutschen\il{Deutsch} anzunehmen. Möglicherweise spricht etwas dagegen. \isi{SLF}"=Konjunkte wie α in (\ref{ex:5-36a}) sind prädikative Kategorien. α ist die erste \isi{Projektionsstufe} einer funktionalen Kategorie, sagen wir α = I\textsuperscript{1}. Durch Externalisierung erhält I\textsuperscript{1 }von V\textsuperscript{max} ein \isi{Thetaraster} mit einer ungesättigten Subjekts"=Thetarolle und weist diese Thetarolle dem \isi{Subjekt} \textit{sie} zu.

\begin{exe}
\ex%36
\label{ex:5-36}
\begin{xlist}
\ex[]{%36a
\label{ex:5-36a}
wenn sie nach Hause kommt und [\textsubscript{α}
sieht [\textsubscript{V}\textsuperscript{max} da den
  Gerichtsvollzieher]]}
\ex[]{%36b
\label{ex:5-36b}
wenn sie nach Hause kommt und [\textsubscript{β}
da steht [\textsubscript{V}\textsuperscript{max} der
  Gerichtsvollzieher vor der Tür]]}
\ex[*]{%36c
\label{ex:5-36c}
wenn sie nach Hause kommt und [\textsubscript{β} da sieht [\textsubscript{V}\textsuperscript{max} den Gerichtsvollzieher]]}
\end{xlist}
\end{exe}
In der gleichen Position kann auch ein F2-\isi{Konjunkt} stehen, wie in
(\ref{ex:5-36b}); dabei sei β = I\textsuperscript{2}. (\ref{ex:5-36c}) zeigt, daß
ein solches I\textsuperscript{2} nicht \isi{prädikativ} sein kann. Diese
Tatsache kann man durch verschiedene Stipulationen erfassen. Eine
besonders einfache~-- d.\,h.: besonders generelle und damit starke~-- Annahme ist in (\ref{ex:5-37}) formuliert:
\begin{exe}
\ex%37
\label{ex:5-37}
X\textsuperscript{2}{}"=Kategorien sind in dem Sinn abgeschlossen, daß sie kein \isi{Thetaraster} aufnehmen können.
\end{exe}
Aus (\ref{ex:5-37}) folgt, daß β in (\ref{ex:5-36c}) im Unterschied zu α in (\ref{ex:5-36a}) kein \isi{Thetaraster} aufnehmen, also auch keine Thetarolle zuweisen kann. Aus (\ref{ex:5-37}) folgt zugleich, daß in (\ref{ex:5-36a}) V\textsuperscript{max} ${\neq}$ V\textsuperscript{2} sein muß, denn V\textsuperscript{2} könnte kein \isi{Thetaraster} aufnehmen, folglich auch keins an α weitergeben.

\printbibliography[heading=subbibliography,notkeyword=this]
\refstepcounter{mylastpagecount}\label{chap-projektionsstufen-end}
\end{document}
